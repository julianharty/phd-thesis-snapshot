\usepackage{silence}
\WarningFilter{latex}{Command \InputIfFileExists}
\WarningFilter{nag}{Command \sf}

\usepackage[table]{xcolor}
\usepackage{array}

% The following is to remove a warning: Package fmtcount Warning: \ordinal already defined use \FCordinal instead. on input line 484. 
\let\ordinal\relax
\usepackage{datetime}
\usepackage{ifpdf}

\usepackage{subfig}
\usepackage{algorithm}

\usepackage[newfloat]{minted} %Helps format sourcecode samples
\usepackage{listings}
\usepackage{multicol, multirow}

\usepackage{morewrites}
\extrafloats{100}

\usepackage{caption}
\DeclareCaptionType{prerequisites}[Prerequisites][List of Prerequisites]
\DeclareCaptionType{methodology-map}[Methodologies][Methodology Map]
\usepackage{tocloft}

\usepackage[noend]{algpseudocode}
% \newsubfloat{figure}
% \newsubfloat{table}
\usepackage{outlines}  %for multilevel lists
\usepackage{pgfgantt}
\usepackage[para]{threeparttable}


\setlength{\marginparwidth}{2cm}
\usepackage{todonotes}
\usepackage[super]{nth}
\usepackage{soul}
\usepackage{longtable,booktabs,threeparttablex}
\usepackage{svg}
\usepackage{copyrightbox}
\usepackage[normalem]{ulem}
\useunder{\uline}{\ul}{}

\usepackage{fancyhdr}
\usepackage[en-GB]{datetime2}


% comment these for final
\newcommand{\yijun}[1]{\textcolor{red}{[YY: #1 ?]}}
% \def\yy#1#2{\textcolor{red}{#1}\footnote{YY:{#2}\textcolor{black}}}
\newcommand{\yy}[2]{{#1}\footnote{YY:{#2}}}

\newcommand{\akb}[1]{\textcolor{purple}{[AKB: #1]}}
\newcommand{\arosha}[1]{\textcolor{purple}{[AKB: #1 ?]}}

\newcommand{\marian}[1]{\textcolor{blue}{[MP: #1 ?]}}
\newcommand{\julian}[1]{\textcolor{olive}{[JH: #1 ]}}
\newcommand{\isabel}[1]{\textcolor{brown}{[IE: #1]}}


%%%%%% 
% \newfontfamily\bengalifont[Script=Bengali]{Lohit Bengali}
% \newcommand\palatino{\fontfamily{ppl}\selectfont}
% \newcommand\bangla{\fontfamily{ppl}\selectfont}

% Colours thanks to http://www.maths.adelaide.edu.au/anthony.roberts/LaTeX/ltxusecol.php
% uncomment these for final
% \newcommand{\yijun}[1]{\textcolor{red}{}}
% \newcommand{\arosha}[1]{\textcolor{red}{}}
%-----------------------

% https://tex.stackexchange.com/questions/99809/box-or-sidebar-for-additional-text for the following code
\usepackage{wrapfig}
\usepackage[many]{tcolorbox}
\usepackage{pdflscape}  % So I can rotate specific pages in the PDF file to landscape, instead of using the rotating package which messed up other seemingly unrelated aspects of the generated document e.g. the chapter headings dropped when I used rotating. 
% Thanks to https://stackoverflow.com/questions/44078121/two-columns-in-pdflscape-latex for the MWE in https://stackoverflow.com/a/44106137/340175
\usepackage{lipsum}

\newenvironment{WrapText}[1][r]
  {\wrapfigure{#1}{0.5\textwidth}\tcolorbox}
  {\endtcolorbox\endwrapfigure}

% end of https://tex.stackexchange.com/questions/99809/box-or-sidebar-for-additional-text

\providecommand{\e}[1]{\ensuremath{\times 10^{#1}}}

\setlength{\arrayrulewidth}{1mm}
\setlength{\tabcolsep}{18pt}
\renewcommand{\arraystretch}{1.5}

\newcolumntype{s}{>{\columncolor[HTML]{AAACED}} p{3cm}}

% Better page layout for A4 paper, see memoir manual.
\settrimmedsize{297mm}{210mm}{*}
\setlength{\trimtop}{0pt} 
\setlength{\trimedge}{\stockwidth} 
\addtolength{\trimedge}{-\paperwidth} 
\settypeblocksize{634pt}{448.13pt}{*} 
\setulmargins{4cm}{*}{*} 
\setlrmargins{*}{*}{1.5} 
\setmarginnotes{17pt}{51pt}{\onelineskip} 
\setheadfoot{\onelineskip}{2\onelineskip} 
\setheaderspaces{*}{2\onelineskip}{*} 
\checkandfixthelayout
%
\frenchspacing
% Font with math support: New Century Schoolbook
\usepackage{fouriernc}
\usepackage[T1]{fontenc}
\usepackage{ marvosym }
\usepackage{ latexsym }

\OnehalfSpacing 
%
% Sets numbering division level
\setsecnumdepth{subsection} 
\maxsecnumdepth{subsubsection}
%
\usepackage{minitoc}

\usepackage{calc,soul,fourier}
\makeatletter 
\newlength\dlf@normtxtw 
\setlength\dlf@normtxtw{\textwidth} 
\newsavebox{\feline@chapter} 
\newcommand\feline@chapter@marker[1][4cm]{%
	\sbox\feline@chapter{% 
		\resizebox{!}{#1}{\fboxsep=1pt%
			\colorbox{gray}{\color{white}\thechapter}% 
		}}%
		\rotatebox{90}{% 
			\resizebox{%
				\heightof{\usebox{\feline@chapter}}+\depthof{\usebox{\feline@chapter}}}% 
			{!}{\scshape\so\@chapapp}}\quad%
		\raisebox{\depthof{\usebox{\feline@chapter}}}{\usebox{\feline@chapter}}%
} 
\newcommand\feline@chm[1][4cm]{%
	\sbox\feline@chapter{\feline@chapter@marker[#1]}% 
	\makebox[0pt][c]{% aka \rlap
		\makebox[1cm][r]{\usebox\feline@chapter}%
	}}
\makechapterstyle{daleifmodif}{
	\renewcommand\chapnamefont{\normalfont\Large\scshape\raggedleft\so} 
	\renewcommand\chaptitlefont{\normalfont\Large\bfseries\scshape} 
	\renewcommand\chapternamenum{} \renewcommand\printchaptername{} 
	\renewcommand\printchapternum{\null\hfill\feline@chm[2.5cm]\par} 
	\renewcommand\afterchapternum{\par\vskip\midchapskip} 
	\renewcommand\printchaptertitle[1]{\color{gray}\chaptitlefont\raggedleft ##1\par}

} 
\makeatother 
\chapterstyle{daleifmodif}
%
% UoB guidelines:
%
% The pages should be numbered consecutively at the bottom centre of the
% page.
\makepagestyle{myvf} 
\makeoddfoot{myvf}{}{\thepage}{\tiny{generated on: \DTMnow}}
\makeevenfoot{myvf}{}{\thepage}{} 
\makeheadrule{myvf}{\textwidth}{\normalrulethickness} 
\makeevenhead{myvf}{\small\textsc{\leftmark}}{}{} 
\makeoddhead{myvf}{}{}{\small\textsc{\rightmark}}
\pagestyle{myvf}

\newcommand{\clearemptydoublepage}{\newpage{\thispagestyle{empty}\cleardoublepage}}

\makeindex

\usepackage{import}

\usepackage{lipsum}					%Needed to create dummy text

\usepackage{graphicx}					%Calls figure environment
%\usepackage[figuresleft]{longtable,rotating}			%Long tab environments including rotation. 
\usepackage[utf8]{inputenc}			%Needed to encode non-english characters 
									%directly for mac
\usepackage{colortbl}					%Makes coloured tables
\usepackage{wasysym}					%More math symbols
\usepackage{mathrsfs}					%Even more math symbols
\usepackage{float}						%Helps to place figures, tables, etc. 
\usepackage{placeins}                   %Helps ensure figures are displayed in the same section.
\usepackage{verbatim}					%Permits pre-formated text insertion
\usepackage{upgreek }					%Calls other kind of greek alphabet
\usepackage{latexsym}					%Extra symbols
\usepackage[authoryear,
		     sort&compress]{natbib}		%Calls bibliography commands % square,numbers,
\usepackage{url}
\def\UrlBreaks{\do\/\do-}               %Allows long URLs to be split on the / and - characters. See https://norwied.wordpress.com/2012/07/10/how-to-break-long-urls-in-bibtex/ 
% \Urlmuskip=0mu plus 1mu               % This produces large white space gaps between sections of a URL so not ideal, comment out.
% If I need to improve the formatting of DOI's then revisit https://tex.stackexchange.com/questions/68506/bibtex-url-and-doi-line-breaks-on-different-characters 

\urlstyle{sf}
\usepackage[british]{babel}	            %For languages characters and hyphenation
\babelprovide[import]{bengali}

\usepackage{color}                    	%Creates coloured text and background
\usepackage[colorlinks=true, % allcolors=black
		     ]{hyperref} %Creates hyperlinks in cross references 
		     
		     % Useful tips for colour in hyperrefs https://tex.stackexchange.com/questions/50747/options-for-appearance-of-links-in-hyperref
		     

% Colours mainly from http://latexcolor.com/
\definecolor{arsenic}{rgb}{0.23, 0.27, 0.29}  
\definecolor{darkjunglegreen}{rgb}{0.1, 0.14, 0.13}
\definecolor{dimgray}{rgb}{0.41, 0.41, 0.41}
\definecolor{mypurple}{rgb}{0.2, 0.2, 0.2}
\definecolor{persianindigo}{rgb}{0.2, 0.07, 0.48}
\definecolor{slategray}{rgb}{0.44, 0.5, 0.56}
       
\hypersetup{
  linkcolor= persianindigo,
  citecolor= persianindigo,
  urlcolor=darkjunglegreen,
  pdftitle={Improving Application Quality using Mobile Analytics},
  pdfauthor={Julian Harty},
  pdfpagemode={UseThumbs},
}		     
		     
\usepackage{memhfixc}					%Must be used on memoir document 
									%class after hyperref
\usepackage{enumerate}					%For enumeration counter
\usepackage{footnote}					%For footnotes
\usepackage{boxhandler}                 %For the \bxtable in the Catrobat case study.
\usepackage{microtype}					%Makes pdf look better.
\usepackage{rotfloat}					%For rotating and float environments as tables, 
									%figures, etc. 
\usepackage{alltt}						%LaTeX commands are not disabled in 
									%verbatim-like environment
%\usepackage[version=0.96]{pgf}	
				
\usepackage{adjustbox}
\usepackage{mdframed}               % Put a border around text that spans pages https://tex.stackexchange.com/a/36528/88466 
\mdfdefinestyle{MyFrame}{%
    linecolor=blue,
    outerlinewidth=2pt,
    roundcorner=20pt,
    innertopmargin=\baselineskip,
    innerbottommargin=\baselineskip,
    innerrightmargin=20pt,
    innerleftmargin=20pt,
    userdefinedwidth=14cm,
    backgroundcolor=gray!50!white}


\usepackage{array,tabularx,longtable,tabu}
\usepackage{colortbl,tikz}
\usetikzlibrary{calc}
\usetikzlibrary{tikzmark}

\widowpenalty=1000
\clubpenalty=1000
\renewcommand\bf{\bfseries}
%
% New command definitions for my thesis
%
\newcommand{\keywords}[1]{\par\noindent{\small{\textbf Keywords:} #1}} %Defines keywords small section
\newcommand{\mycolumnheading}[2]{\multicolumn{1}{>{\centering\arraybackslash}p{#1}}{\textbf{#2}}}  % From https://latex.org/forum/viewtopic.php?t=20931
\newcommand{\parcial}[2]{\frac{\partial#1}{\partial#2}}                             %Defines a partial operator
\newcommand{\vectorr}[1]{\mathbf{#1}}                                                        %Defines a bold vector
\newcommand{\vecol}[2]{\left(                                                                         %Defines a column vector
	\begin{array}{c} 
		\displaystyle#1 \\
		\displaystyle#2
	\end{array}\right)}
\newcommand{\mados}[4]{\left(                                                                       %Defines a 2x2 matrix
	\begin{array}{cc}
		\displaystyle#1 &\displaystyle #2 \\
		\displaystyle#3 & \displaystyle#4
	\end{array}\right)}
\newcommand{\pgftextcircled}[1]{                                                                    %Defines encircled text
    \setbox0=\hbox{#1}%
    \dimen0\wd0%
    \divide\dimen0 by 2%
    \begin{tikzpicture}[baseline=(a.base)]%
        \useasboundingbox (-\the\dimen0,0pt) rectangle (\the\dimen0,1pt);
        \node[circle,draw,outer sep=0pt,inner sep=0.1ex] (a) {#1};
    \end{tikzpicture}
}
\newcommand{\range}[1]{\textnormal{range }#1}                                             %Defines range operator
\newcommand{\innerp}[2]{\left\langle#1,#2\right\rangle}                                 %Defines inner product
\newcommand{\prom}[1]{\left\langle#1\right\rangle}                                         %Defines average operator
\newcommand{\tra}[1]{\textnormal{tra} \: #1}                                                       %Defines trace operator
\newcommand{\sign}[1]{\textnormal{sign\,}#1}                                                   %Defines sign operator
%\newcommand{\sech}[1]{\textnormal{sech} #1}                                                  %Defines sech
\newcommand{\diag}[1]{\textnormal{diag} #1}                                                    %Defines diag operator
\newcommand{\arcsech}[1]{\textnormal{arcsech} #1}                                       %Defines arcsech
\newcommand{\arctanh}[1]{\textnormal{arctanh} #1}                                         %Defines arctanh
%Change tombstone symbol
\newcommand{\blackged}{\hfill$\blacksquare$}
\newcommand{\whiteged}{\hfill$\square$}


\let\oldsqrt\sqrt

\def\sqrt{\mathpalette\DHLhksqrt}
\def\DHLhksqrt#1#2{%
\setbox0=\hbox{$#1\oldsqrt{#2\,}$}\dimen0=\ht0
\advance\dimen0-0.2\ht0
\setbox2=\hbox{\vrule height\ht0 depth -\dimen0}%
{\box0\lower0.4pt\box2}}
%
% My caption style
\newcommand{\mycaption}[2][\@empty]{
	\captionnamefont{\scshape} 
	\changecaptionwidth
	\captionwidth{0.9\linewidth}
	\captiondelim{.\:} 
	\indentcaption{0.75cm}
	\captionstyle[\centering]{}
	\setlength{\belowcaptionskip}{10pt}
	\ifx \@empty#1 \caption{#2}\else \caption[#1]{#2}
}
%
% My subcaption style
\newcommand{\mysubcaption}[2][\@empty]{
	\subcaptionsize{\small}
	\hangsubcaption
	\subcaptionlabelfont{\rmfamily}
	\sidecapstyle{\raggedright}
	\setlength{\belowcaptionskip}{10pt}
	\ifx \@empty#1 \subcaption{#2}\else \subcaption[#1]{#2}
}

\usepackage{lettrine}
\newcommand{\initial}[1]{%
	\lettrine[lines=3,lhang=0.33,nindent=0em]{
		\color{gray}
     		{\textsc{#1}}}{}}

\usepackage{memhfixc} % https://tex.stackexchange.com/questions/323924/page-breaking-problem-with-memoir-and-hyperref

%%%%%%%%%%%%%%% An attempt to reproduce the \newthought{} command from the tufte-book document.
%%%% Sources:
%%%%   https://github.com/Tufte-LaTeX/tufte-latex/blob/efb8c8e836890bad71fb5834acdd316ebde6db12/tufte-common.def#L779
%%%%   https://github.com/Tufte-LaTeX/tufte-latex/blob/efb8c8e836890bad71fb5834acdd316ebde6db12/tufte-common.def#L1527-L1535
%%%%   Improvement to vertical formatting https://tex.stackexchange.com/a/291748/88466 (via https://tex.stackexchange.com/questions/291746/tufte-latex-newthought-after-section)
%%%%   https://tex.stackexchange.com/questions/288472/bold-first-letter-combined-with-smallcaps 
\newskip\tufteskipamount
\tufteskipamount=1.0\baselineskip plus 0.5ex minus 0.2ex

\newcommand{\tuftebreak}{\par\ifdim\lastskip<\tufteskipamount
  \removelastskip\penalty-100\tufteskip\fi}

\newcommand{\tufteskip}{\vspace\tufteskipamount}

\makeatletter
\def\tuftebreak{%
  \if@nobreak\else
    \par
    \ifdim\lastskip<\tufteskipamount
      \removelastskip \penalty -100
      \tufteskip
    \fi
  \fi
}
\makeatother

%%%%%%% From https://tex.stackexchange.com/a/288476/88466
\newlength\ltempa
\newlength\ltempb
\newcommand\newthought[1]{%
   %\addvspace{1.0\baselineskip plus 0.5ex minus 0.2ex}%
   \noindent\expandafter\formatnewthought#1\relax}
\def\formatnewthought#1#2\relax{%
  \tuftebreak
  \settowidth\ltempa{\textsc{#1#2}}%
  \settowidth\ltempb{\textsc{#1\mbox{}#2}}%
  \addtolength\ltempa{-\ltempb}%
  \textbf{#1}\kern\ltempa\textsc{#2}%
}

%%%%%% Improve the aesthetics of the thesis
% Colour the epigraphs for each chapter (which I've yet to add)
% From https://tex.stackexchange.com/a/251157/88466
\newcommand{\epigraphrulecolor}{blue}
\makeatletter
\addtodef\@epirule{\color{\epigraphrulecolor}}{}
\makeatother

\newcommand\buzzwords[1]{\textcolor{purple}{Buzzwords: \newthought{#1}}\\ \vspace{1pt}{\color{purple}\hrulefill}}

\newcommand{\secref}[1]{\autoref{#1}. \nameref{#1}} % \secref{section:my} %from https://stackoverflow.com/a/30844515/340175

%%%%% A vain attempt to support various non-European based languages
\begin{comment}
\usepackage{CJKutf8} % For CJK support see https://www.overleaf.com/learn/latex/Chinese
\usepackage{arabtex} % https://tex.stackexchange.com/a/557159/88466


\usepackage{utf8}
\end{comment}

%%%%%% Finally able to add a custom content type (prerequisites) to the toc in a memoir project. 
%
% Start simple,
% https://texblog.org/2008/07/13/define-your-own-list-of/ and 
% https://texblog.org/2011/09/09/10-ways-to-customize-tocloflot/ and
% https://tex.stackexchange.com/questions/61086/how-to-create-my-own-list-of-things  Then crash and burn in the thesis project...
% The memoir package caused a basic solution to fail. Here's a good reproduction of the sort of errors I encountered https://tex.stackexchange.com/questions/388489/custom-list-throw-latex-error-command-mycustomfiction-already-defined 
% \usepackage{morewrites} needed because of the complexity of the contents of the thesis. Fix found in  https://tex.stackexchange.com/questions/385263/toc-not-generated-with-imakeidx-and-reledpar
% https://tex.stackexchange.com/a/289964/88466 A really helpful answer I wish I'd seen hours ago on debugging the writes.
% https://tex.stackexchange.com/questions/89743/poorly-formatted-newlistof-with-memoir similar to the issues I've encountered.
% A diligent user of memoir and tex.stackexchange describes the many places they also had to research to find a workaround. It includes a good set of checklists for me and others to use https://tex.stackexchange.com/a/388531/88466 
%
% I got in deeper than necessary partly as I found promising clues in discussions that included custom lists AND something else e.g.
% Here's where I went in well over my head and spent several hours https://tex.stackexchange.com/questions/270598/creating-list-of-examples-using-tocloft-in-memoir-class 
% custom lists and theorums https://tex.stackexchange.com/questions/151442/multiple-lists-of-theorems-with-different-titles-any-tex-in-my-case-xelatex
%
% There are some nice techniques to include or create new Lists of... , particularly in 
% https://tex.stackexchange.com/questions/58397/how-to-add-custom-floats-to-listoffigures
% https://tex.stackexchange.com/questions/247038/change-lstlistoflistings-numbering for simple custom formatting of a list
% https://tex.stackexchange.com/questions/159986/redefining-toc-lof-and-lot 
%

% Prerequisite command
\newcommand{\listcommentname}{List of Prerequisites}
\newlistof{listofprerequisites}{lop}{\listcommentname}
\newcounter{Prerequisite}[chapter]
\renewcommand{\thePrerequisite}{\arabic{Prerequisite}}
\newcommand{\Prerequisite}[1] {
    \refstepcounter{Prerequisite}    
    \par\noindent\textbf{Prerequisite \thePrerequisite.  #1.}
    \addcontentsline{lop}{Prerequisite}{\protect\numberline{\thePrerequisite}#1}\par} 

%%%% Madness https://tex.stackexchange.com/a/89744/88466
\makeatletter
\let\l@Prerequisite\l@figure
\makeatother
\renewcommand{\thePrerequisite}{\thechapter.\arabic{Prerequisite}}

%%%%%% Improving the appearance of the thesis
%
% Simple improvements: use epigraphs and make them attractive: https://tex.stackexchange.com/questions/251148/how-to-change-the-color-of-the-memoir-epigraph-rule
% https://tex.stackexchange.com/a/251157/88466 works well, I've tested it as a MWE in https://www.overleaf.com/2898222459grzvqpbnnjxc
%
% The following uses colours very well and I'd love to do something similar when I've time to improve the appearance of my thesis SHOULD-DO https://tex.stackexchange.com/questions/606098/list-of-definitions-list-of-theorems-list-of-examples-and-list-of-activities I'm holding off partly as I expect kaobook and memoir will have many, often subtle and time-consuming, differences in how the latex needs to be written.
% https://tex.stackexchange.com/search?q=%5Bmemoir%5D+color might provide some useful clues to using colour
% https://tex.stackexchange.com/questions/294661/how-do-i-create-a-new-float-environment-with-bigger-and-colored-contents provides examples of setting colours for particular floats in the article document class. It might be some help but looks a bit crude and basic.
% https://tex.stackexchange.com/questions/56004/customizing-section-formatting-using-memoir-class-color-and-numbering for minor improvements, more useful for learning a bitabout how memoir and latex hang together though.

%%%%%%%%%%% Formatting JSON extracts
%\usepackage{bera}
\colorlet{punct}{red!60!black}
\definecolor{background}{HTML}{EEEEEE}
\definecolor{delim}{RGB}{20,105,176}
\colorlet{numb}{magenta!60!black}

\lstdefinelanguage{json}{
    basicstyle=\normalfont\ttfamily,
    basicstyle=\footnotesize,
    numbers=left,
    numberstyle=\scriptsize,
    stepnumber=1,
    numbersep=8pt,
    showstringspaces=false,
    breaklines=true,
    frame=lines,
    backgroundcolor=\color{background},
    literate=
     *{0}{{{\color{numb}0}}}{1}
      {1}{{{\color{numb}1}}}{1}
      {2}{{{\color{numb}2}}}{1}
      {3}{{{\color{numb}3}}}{1}
      {4}{{{\color{numb}4}}}{1}
      {5}{{{\color{numb}5}}}{1}
      {6}{{{\color{numb}6}}}{1}
      {7}{{{\color{numb}7}}}{1}
      {8}{{{\color{numb}8}}}{1}
      {9}{{{\color{numb}9}}}{1}
      {:}{{{\color{punct}{:}}}}{1}
      {,}{{{\color{punct}{,}}}}{1}
      {\{}{{{\color{delim}{\{}}}}{1}
      {\}}{{{\color{delim}{\}}}}}{1}
      {[}{{{\color{delim}{[}}}}{1}
      {]}{{{\color{delim}{]}}}}{1},
}
%%% Joint sources for the above https://www.latex4technics.com/?note=3QTU and https://tex.stackexchange.com/a/83100/88466