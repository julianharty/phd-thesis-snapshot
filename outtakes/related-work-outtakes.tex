\section{Related Work Methodology Outtakes}
Marian suggests I aim for writing brief, one-paragraph summaries and apply the T tactic of the broad literature, the top bar of the T are the many and various papers on a topic, and I'm picking these ones that are most directly germane to my research which become the vertical bar of the T.\todo{I'm still currently more verbose than this :(} 

Marian also suggested I might end up with two or a maximum of 3 levels of Ts. The higher level would be on Software Quality Improvement. The lower level would be on mobile analytics.

\section{Introduction to the related work}
% Reinstate the following once I've completed the first complete draft of the following sections.
% starting with research into app stores and their effects on software development and engineering \ref{rw-app-stores-and-their-effects-on-software-development-and-engineering}...

\section{Software Development}
% For the purposes of this research there are at least two camps in research. The first camp is where the research comes from the field and is applied in the field of production, shipping software and the second camp seeks ways to improve the tools, techniques, and results \emph{without} dealing with the practical aspects. The work of the second camp remains unused in reality and oft only reviewed in-passing by other researchers who want to claim their research generates better `results' than that of the other camp members. The second camp's work overall is on an orthogonal path to the work and world of practitioners. The gulf seems wide between these two camps.

\subsection{From: Assessment criteria}

The specialist areas include: 
\begin{enumerate}
    \item \textbf{Software development practices}: 
    \item \textbf{Software Quality [Improvements] for mobile app developers}: Software Quality has been a contested topic for decades with no single accepted coherent agreement on what form(s) it takes, how software quality is measured, etc. Then comes the similarly vexed challenge of determining the concept and application of improvement in the quality/qualities of software. 
    \item \textbf{Mobile Analytics}: Research into \textbf{Processes, artefacts, and tools} necessary when using mobile analytics for improving software quality/qualities. These groupings emerged during the analysis of the literature and through understanding the practices of app developers.
    \begin{itemize}
        \item \textbf{Processes}: a.k.a. Analytics in Use - research into the processes developers use when they use mobile analytics
        \item \textbf{Artefacts}: things the developers create and maintain as part of their development work. Some of these are generated, in particular the app binary that is destined for end users once delivered by the app store.
        \item \textbf{Mobile Analytics Tools}: that the app developers use are worth researching in order to learn about their characteristics.
    \end{itemize}
\end{enumerate}

    
The \emph{``Data analytics for decision support in software release management"}~\sidecite{didar2018data_analytics_phd_thesis}, a PhD thesis, introduces a proposed Plan-Monitor-Improve Framework for release management.