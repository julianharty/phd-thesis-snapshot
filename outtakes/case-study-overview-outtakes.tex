\chapter{Case Study Overview Outtakes}
This file contains material that may still be relevant to the case studies but that doesn't need to be in the generated thesis. The first set of outtakes is the wish lists.

\section{Wish Lists}
Each case study has a temporary ``wish list" to enhance the case-study where practical. These will be followed up where it's ``essential" for the thesis. The entire wish list will be removed from the generated thesis pre-submission. This wasn't a commission for me to do all the items I've got on the lists in the various case studies.

\subsection*{GTAF: wish list}
Here's my wish list so we can discuss it and decide what's appropriate to do of these and what to park.
{\small
\begin{itemize}
    \itemsep0em
    \item Contact the team again once the thesis has been drafted to review the findings and ask the additional open questions here. 
    \item Ask them about their development practices 12+ months on from the active engagement.
    \item Ask clarifying and follow-up questions based on their 2020 blog post e.g. the source of the graph data and their assessment of their progress in 2021, and on their plan to increase the use of mobile analytics in the apps.
    \item Ask if they would like any further help or involvement. If so, request access to their Firebase Analytics console and any other developer centric mobile analytics tools.
    \item Explore if they'd be willing to change the relationship to being more of an action research case study where I help them (free-of-charge) to identify and address major stability issues in their Android apps.
\end{itemize}

\begin{itemize}
    \itemsep0em
    \item I could usefully do at least a day's worth of analysis on their Google Play Console and Android Vitals reports for their apps based on what I've observed. Whether that'd be sensible to do pre-submission is an open question.
    \item Similarly there's probably quite a bit of information that can be mined from their issue database on gitlab.
    \item ... and from using exodus-privacy e.g. \url{https://reports.exodus-privacy.eu.org/en/reports/search/com.greentech.hadith/}. There's an option to obtain reports for all the Greentech apps (around 10) and all the Kiwix apps (close to 20) and all the Catrobat apps (around 7), etc. and perform some analysis and reporting. Then it's also possible to scan the app binary for localhalo and other apps that use react-native to see if it's possible to establish a signature for Sentry (and similar mobile analytics designed for React Native apps), etc. etc. 
    \item Add results from analysing the binaries of the apps with exodus-privacy.
\end{itemize}
}  % end \small


\subsection*{Local Halo: Wish list}
{\small
\begin{itemize}
    \itemsep0em
    \item I wish we could find out how actively the development team were reading, reviewing and addressing crashes being reported. However, as the project no longer appears to be active that's unlikely to happen.
    \item I'd also appreciate ongoing access to both Sentry and Google Play Console with Android Vitals for the app, again I doubt this is viable given the project is defunct.
    \item NB: the material in this section needs editing and condensing there's a bit too much info and too much repetition. Once I add content to the next 3 chapters for Local Halo I should have a better idea of what doesn't belong here.
    \item Find a place to discuss the penetration of react-native in Android apps on Google Play, see \url{https://www.appbrain.com/stats/libraries/details/react_native/react-native}.
\end{itemize}
}


\subsection*{Moodspace: Wish list}
{\small
\begin{itemize}
    \item I emailed Ian in April 2021, it'd be useful to receive a reply, not least so he can cross-check what I have included in the thesis.
    \item Add results from analysing the binaries of the apps with exodus-privacy.
\end{itemize}
}


\subsection*{Moonpig: Wish list}
{\small
\begin{itemize}
    \item The chapter and related material will be reviewed by the lead contributor to this case study.
    \item Add results from analysing the binaries of the apps with exodus-privacy.
\end{itemize}
}

\subsection*{Kiwix: Wish list}
{\small
\begin{itemize}
    \itemsep0em
    \item I've requested some information on who had access to Google Play Console for the Kiwix apps. This would be useful for the analytics-in-use chapter. Ditto the same info for Catrobat, now I'm able to contact them again.
    \item Well there are many things I'd wish we'd done during the case study however the past is impractical to change. At some point I'll aim to reengage with the project and see whether we can address the stability issues that have emerged in 2021.
    \item Add results from analysing the binaries of the apps with exodus-privacy. \url{https://reports.exodus-privacy.eu.org/en/reports/search/org.kiwix.kiwixmobile/} 0 trackers, 9 permissions. Wikimed in English \url{https://reports.exodus-privacy.eu.org/en/reports/org.kiwix.kiwixcustomwikimed/latest/} 0 trackers, 6 permissions. PHeT \url{https://reports.exodus-privacy.eu.org/en/reports/org.kiwix.kiwixcustomphet/latest/} 0 trackers and 6 permissions.
\end{itemize}
}

\subsection*{Catrobat: Wish list}
{\small
\begin{itemize}
    \itemsep0em
    \item Ask why Pocket Paint is on F-Droid~\url{https://f-droid.org/en/packages/org.catrobat.paintroid/} but not Pocket Code.
    \item It'd be helpful to re-establish communications with the project team in order to follow up on the results of the hackathon and on their current practices. I can also analyse their issues database to see whether they're actively using mobile analytics.
    \item Check whether they use code coverage measures and if so, what the numbers are.
    \item Extension work: it'd be interesting to apply \url{https://luiscruz.github.io/android_test_inspector/} to the project's apps.
    \item \textbf{Where should I write up the hackathon?} here? or in the next few chapters?
    \item Add results from analysing the binaries of the apps with exodus-privacy. Pocket Code \url{https://reports.exodus-privacy.eu.org/en/reports/search/org.catrobat.catroid/} 2 trackers and 16 permissions. Pocket Paint \url{https://reports.exodus-privacy.eu.org/en/reports/org.catrobat.paintroid/latest/} 0 trackers and 4 permissions.
\end{itemize}
}  % end \small


\subsection*{C1: Wish list}
{\small

\begin{itemize}
    \item NB: it is not practical to publish the results of analysing the app binary with exodus-privacy as doing so may leak additional clues about the app and therefore the project. Check the contract.
    \item Seek a pertinent example of using automated tests for OkHttp.
\end{itemize}
}

\subsection{Source code analysis: wish list}
{\small
\begin{itemize}
    \item Of the 107 projects, find out which also used any form of crash analytics. This information would help provide insights into how many combined in-app analytics with crash reporting and therefore analysis of reliability/stability issues (the core of this thesis in terms of analytics).
\end{itemize}
}


\subsection*{Crashlytics: Wish list}
{\small

\begin{itemize}
    \item I'd like to be able to compare the outputs between Crashlytics and Android Vitals, in greater depth than I've managed so far.
\end{itemize}
}

\subsection*{Firebase Analytics: Wish list}
{\small

\begin{itemize}
    \item Well there's so much scope to experiment further with Firebase Analytics when working with real apps that have significant activity. I'll leave the details of my wishlist until a later date so I can continue with finishing off my thesis.
\end{itemize}
}


\subsection*{Google Play Console with Android Vitals: Wish list}
{\small
\begin{itemize}
    \item Reconnect with the current development team for Android Vitals and apprise them of my recent research findings.
\end{itemize}
}

\subsection*{Iteratively with Amplitude: Wish list}
{\small
\begin{itemize}
    \item Consider asking Patrick, the ex-CEO, for summary data re their customer and the deployment base. 
    \item Ditto, ask for any customers who may be willing to share their experiences of using the iteratively tooling. (requested on \nth{7} Jan 2022)
    \item Clean up the project to remove authentication details then prepare and opensource it.
\end{itemize}
}


\subsection*{Microsoft App Center: Wish list}
{\small
\begin{itemize}
    \item Ideally permission to directly use material from the relevant case study.
    \item Alternately, examples from a similarly large-scale project and app.
    \item To compare the results in depth between Android Vitals and App Center.
    \item To investigate their core analytics offering in addition to the crash reporting.
    \item To discover if they've added support for tracking ANRs.
\end{itemize}
}

\subsection*{Sentry: Wish list}
{\small
\begin{itemize}
    \item Ongoing access to the Local Halo data (not urgent or important, just useful).
\end{itemize}
}
