\section{Synthesis and Recommendations}
\label{section-synthesis-recommendations}

This section synthesises the overall findings to provide some recommendations for how development teams can apply analytics productively in their ongoing work. Several cadences have emerged so far that align with a) new releases of an app b) operational, `business-as-usual', once a release has been rolled out and reached the majority of uses who will install it. Note: There will also be times when the development team needs to adapt and respond to emerging events where applying analytics is also relevant, these are outside the scope of this thesis.

\subsection{Mobile Analytics Policy}
A~\href{glossary-mobile-analytics-policy}{Mobile Analytics Policy} helps the team distinguish the boundaries of their use of mobile analytics, for instance whether they are willing to include analytics as part of their mobile app, and if so the limits of the data the analytics would be allowed to collect and who would, and who would not, have access to the contents of the underlying data. In the case studies, the non-for-profit projects (Kiwix, Catrobat, and Greentech) have all chosen not to use in-app analytics libraries. Catrobat chose to do so when Google removed access to the Fabric Crashlytics service as the replacement Firebase Crashlytics service collected and reported on data considered sensitive and unnecessary by the project team.

\subsection{Mobile Analytics Strategy}
The~\href{glossary-mobile-analytics-strategy}{Mobile Analytics Strategy} ...

\subsubsection{Combining policy and strategy}
There can be areas that blur between policy and strategy, for instance on whether to use a third-party provider of analytics - the reasons may be for a mix of both policy and strategy aspects. On example is a team may have a policy of only using external offerings when they do not cost money and have a strategy of using and reusing pre-existing offerings rather than creating custom software. External analytics offerings could potentially meet both the policy and strategy, however if the cost is measured in external data access and external use of that data e.g. to profile users then that service offering may be excluded from a policy perspective.

\subsection{Design and Implementation Approach}

\subsection{Health Checks}

\subsubsection{Pre-release Health Checks}


\subsubsection{Release Health Checks}

Hypothesis - does Google Play rollout to the users suffering most first?

\subsubsection{Daily Health Checks}
These are daily health checks for mature releases, typically at least a week old. A good approximation for the actual time is the time taken to reach around 70\% of the peak of the eventual steady state peak/maximum for this release~\footnote{Note: this is a heuristic based on observing many releases and the realisation that some of the userbase will not install a particular release}.

% 0.707 is the RMS of a sine wave, and perhaps coincidentally seems to be about the right number to use for the boundary value for the transition from new release rollout to steady state where the behaviour tends to be more predictable and stable in terms of quality metrics such as reliability, stability, and performance.

\begin{enumerate}
    \item For those using internal (in-app) analytics, check for any new issues especially in the most recent release, rapidly accelerating issues, emerging trends, outages, gaps in data, and so on.
    \item Check the external (platform) analytics if they are available.
    \item Check the progress for active tickets that have been raised for recent releases (including the current release). Similar check for any regressions in the current release where fixes of bugs in previous releases have supposedly been addressed.
\end{enumerate}

These checks may be combined with other daily project and team health checks depending on the scope of the individual's role, responsibility, and so on. For instance, developers may also check the state of branches, pull requests, and recent commits in the codebase, or codebases~\footnote{For instance of server-side systems, APIs provided by others within and outside the organisation, and so on.}, pertaining to the app.
