\chapter{Extended discussions}
These topics extend the core discussions in my thesis. They are often based on my experiences beyond and before the PhD research.


\section{Software development practices}
Jez Humble's work could be part of a preamble for software development practices, to set the scene. I know Jez through meeting him on at a conference in India. We became friends and met again a couple of times in the Bay Area and in India. His work is credible and widely influential in software development.

\begin{itemize}
    \item \emph{Continuous delivery sounds great, but will it work here?}~\citep{humble2018_continuous_delivery_sounds_great}. 
    \begin{itemize}
        \item ``Continuous delivery is about reducing the risk and transaction cost of taking changes from version control to production. Achieving this goal means implementing a series of patterns and practices that enable developers to create fast feedback loops and work in small batches. This, in turn, increases the quality of products, allows developers to react more rapidly to incidents and changing requirements and, in turn, build more stable and higher-quality products and services at lower costs."
        \item ``If this sounds too good to be true, bear in mind: continuous delivery is not magic. It's about continuous, daily improvement at all levels of the organization—the constant discipline of pursuing higher performance. As presented in this article, however, these ideas can be implemented in any domain; this requires thoroughgoing, disciplined, and ongoing work at all levels of the organization. Particularly hard, though essential, are the cultural and architectural changes required."
    \end{itemize}
    
\end{itemize}

To establish and maintain healthy mobile apps the apps need tending to so they continue to perform well as new devices, new OS releases, and so on continue to change the operating context/environment for the app. As flaws are discovered in the app they can also be addressed using continuous delivery techniques and concepts. The app store ecosystem and how releases reach end users means a staccato or pizzicato delivery is inevitable and that there'll be numerous versions of the app in use across the userbase. 
