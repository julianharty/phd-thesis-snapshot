\chapter{Outtakes from the Introduction}
\subsection{Additional research considerations}
\julian{I've reoriented my previous sub-RQ's here rather than delete them at this stage. MUST-DO consider whether to keep them in the thesis.}

This leads to several related questions that provide additional context to this research: these are grouped in three categories: \href{sec:sources}{\textbf{\nameref{sec:sources}}}, \href{sec:value}{\textbf{\nameref{sec:value}}}, and \href{sec:impact}{\textbf{\nameref{sec:impact}}}.

\subsubsection{Sources}~\label{section-sources}~\label{sec:sources}
At a superficial level, there seem to be mobile analytics offerings that operate within the app and those that are external to the app, particularly those that gather data at the platform level. Several widespread analytics offerings are evaluated as part of this research. Several more were considered to provide an understanding of the overall context.
\begin{itemize}
    \item \emph{What sources of analytics are available? And how do the sources that were investigated compare in terms of the data they collect and how they are used?}
    % \item \emph{Hypothesis}: The various sources have distinct characteristics and purposes. Developers may wish to choose particular analytics to best suit their context and objectives.
\end{itemize}

\subsubsection{Value}~\label{section-value}~\label{sec:value}
Does using analytics provide quantitative and/or qualitative value that can be measured? Could it provide value in terms of assessing the quality of our work that was invested into developing, testing and preparing software before it was launched?
\begin{itemize}
    \item \emph{How much does the fidelity of the analytics offerings matter?} Can the results be used productively even if they are flawed? %The research discovered numerous errors in various analytics offerings. These results may be of interest to the research community given the endemic nature of mobile analytics in real-world apps in the major app stores.

    \item \emph{How can analytics help with bug investigation?} Isolated instances of bugs may be hard to assess in terms of the likely scope and impact on a user-base; \emph{how, where and when can analytics help with bug investigation?} We might also consider practical limits,~\emph{e.g.,} constraints that are enforced by the real-world analytics being used.
\end{itemize}

\subsubsection{Impact}~\label{section-impact}~\label{sec:impact}
Here the focus is on whether the value has sufficient impact for anyone else to be interested in using and applying analytics. Given the nature of the research the main measures are practical, \emph{i.e.}, in the real-world.
\begin{itemize}
    \item \emph{Do development teams use analytics in their practice?}  And if so, how? % Hypothesis: If development teams find practices useful they will generally try to use them intrinsically. Do they?
\end{itemize}

\begin{comment}


    %\section{My research methodology, and my choices}
    Needs full rework.
    
    Merge into the strategy
    Headings:
    \begin{itemize}
        \item case study approach
        \item data collection
        \item (move 'action research' from the introduction as it's not the strategy, instead mention it in the relevant case study: Kiwix).
    \end{itemize}
    
    
    draft wording for a high-level scene setting: Within these case studies there were particular needs that led to choices in the specific case studies:
    
    
    MUST-DO Map RQ's to the case studies and explain how they're mapped, and the methods I'll be introducing later on. High level choices of the key choices I made in order to answer the RQs. Map in Chapter 3.
    
    \end{comment}

%Multiple ecosystems. Data types and informant types. 


%Move to the case study that used the action research:
%Although I had prior experience in industry of the efficacy and potency of applying usage analytics to improve software development and testing of mobile apps, that experience was generally covered by confidentially agreements, and also the analytics tools have changed and developed markedly since those experiences. Therefore, action research seemed appropriate, particularly as one of the long-term opensource projects had extremely high failure rates according to the \emph{de facto} Android analytics tool. I decided it was appropriate and necessary to see if I could help that project directly to improve their mobile apps -- \emph{``physician heal thyself"}\sidenote{\href{https://en.wikipedia.org/wiki/Physician,\_heal\_thyself}{wikipedia.org/wiki/Physician,\_heal\_thyself}.}.


\begin{comment}
    - Could I try phrasing my RQs as OKRs to see if doing so helps me to improve the clarity and relevance of the RQs.
    - Also, how about creating annotated editions of my RQs where the annotations include context, commentary, connections to other RQs, notes on twitter-style answers to each, etc.
    - We want to know more about 'this' topic. Then provide Operational questions - to be addressed by the research, which will help us learn more about the topic.
    - What's a RQ and what's an analytical lens (to be used to help with the RQ)?
\end{comment}

\section{Research Strategy}
The research aims to provide practical and applicable insights to developers of mobile apps. As such, the strategy was to work predominantly with development teams for mobile apps to ensure -- as far as practical -- that the research is externally validated through their experiences, practices, and feedback. As developers need to make choices appropriate to their context, the research includes a variety of apps including commercial and not-for-profit, small, medium and large development teams, and user bases. The research also has an international flavour with developers situated on various continents including Asia, Europe, and the USA.

\subsection{Potential sources of evidence}
Broadly we can learn from people, use software analysis tools (including static analysis tools such as \myindex{Lint}~\sidenote{Aptly described as software that checks software~\cite{gimpel2014_software_that_checks_software}.}), and use \textbf{data} to answer the research questions. In some cases the data comes from a single source, in others cases from several. The research strategy combines aspects of all these sources of evidence. %In Chapter 3, the sources of evidence are discussed in more detail in \href{sec:potential-sources-of-evidence}{\nameref{sec:potential-sources-of-evidence}}. For now, the key consideration is how to effectively perform the research.

\subsection{Case Studies}
Case studies have been established by various researchers as an effective method to understand software qualities, for instance in various research by Khalid~\emph{et al.}, in (\citeyear{khalid2014_prioritizing_the_devices_to_test_your_app_on_casestudy_android_games, khalid2015_what_do_mobile_app_users_complain_about, khalid2016_examining_the_relationship_between_findbugs_warnings_and_app_ratings}), and the work of \sidecite{martinez_fernandez2019_continuously_assessing_and_improving_software_quality_with_software_analytics_tools}. And case studies are well suited to exploratory research, for which they may offer insights not available with other approaches~\sidecite{rowley2002_using_case_studies_in_research}, especially when events are contemporary and where the investigator has little or no control in~\sidecite[][Chapter 1]{yin2018_case_study_research_and_applications_6th_edition}. They are also recommended where the research includes multiple data sources~\sidecite{rowley2002_using_case_studies_in_research}. This research therefore uses case studies, as the research investigates real-world use of mobile analytics where many factors are outside the control of the researcher.

The choice of case studies is intended to provide variety and the opportunity for comparison and to provide a multi-layered approach.


\subsection{Analytics}
As the main research question considers the application of analytics, the research needs to include a combination of usage and analytics data, where the analytics data is then applied with the intent of improving quality of the product. The developers may not be successful in achieving improvements, although we hope they will be. They may also be able to improve their practices, so again their current and revised processes are also of interest.

[Almost] anything can be measured sufficiently to be useful. In his book~\emph{How to Measure Anything, \nth{3} edition}~\sidecite{hubbard2014measure} argues that anything can be measured and proposes an approach to do so. His framework, called \emph{Applied Information Economics: A Universal Approach to Measurement}, is summarised in five steps:

\begin{enumerate}
    %\setlength\itemsep{-0.5em} %\itemsep0em
    \item Define the decision.
    \item Determine what you know now.
    \item Compute the value of additional information. (If none, go to step 5.)
    \item Measure where the information value is high. (Return to steps 2 and 3 until further measurement is not needed.)
    \item Make a decision and act on it. (Return to step 1 and repeat as each action creates new decisions.)
\end{enumerate} ~\cite[p.9]{hubbard2014measure}.

This approach helps to frame this research into applying analytics to development practices, especially in terms of the practical aspects and nature of real-world apps, development teams, and user-bases.


\subsection{Perspectives of the developers of the apps and the tools}
As the analytics tools also influence the results, the research includes discussions and collaborations with several organisations who create these tools.

This research is not immune from also being improved, and similarly the process is likely to have plenty of scope for improvement as we learn more from the various projects, teams, apps, and analytics tools. Similarly, there is scope to find flaws, limitations, and weaknesses in the analytics tools; therefore there was scope in the research to share findings with the teams responsible for these, and related, software tools and to use the experiences and insights from any such sharing as part of this research.


\subsection{Triangulation}
\index{triangulation}
Where available and practical, triangulation %~\sidenote{c.f. colligation}
of data and analytics reports will be used to help increase the confidence in the analytics and in the efficacy of using mobile analytics. As~\sidecite{marr2015bigdatabook} recommends:~\emph{``Measure metrics and data backed up or triangulated with other data sources."} and~\emph{``Where possible use a combination of data sets and triangulate the data. In other words, see if each data set delivers the same result so you can confirm and validate the answers."}. Triangulation of research methods is extensively covered in research. For example, \sidecite{fielding2012_triangulation_and_mixed_methods_designs} provides a rich discussion of triangulation and mixed methods design for various research areas. % To the best of my knowledge...  %SHOULD-DO decide how much to focus on triangulation, whether to discuss data triangulation as a technique for comparing analytics results (which doesn't seem to quite apply - at least in what I've read).

The next level of validity is the role of the researcher: even if this work achieves desirable results for projects where the researcher is directly involved, could other development teams achieve similar results by applying a similar approach? To help gain evidence, the research engages a separate opensource development project and development team with the researcher as a guide and mentor rather than a hands-on contributor.

Cross validation through multiple case studies: the research compares opensource and commercial projects for their approaches and results. It also compares various in-app analytics offerings to discern if they have distinctive characteristics that may help the development teams.

