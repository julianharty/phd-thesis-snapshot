\chapter{Outtakes from the Introduction}
\subsection{Additional research considerations}
\julian{I've reoriented my previous sub-RQ's here rather than delete them at this stage. MUST-DO consider whether to keep them in the thesis.}

This leads to several related questions that provide additional context to this research: these are grouped in three categories: \href{sec:sources}{\textbf{\nameref{sec:sources}}}, \href{sec:value}{\textbf{\nameref{sec:value}}}, and \href{sec:impact}{\textbf{\nameref{sec:impact}}}.

\subsubsection{Sources}~\label{section-sources}~\label{sec:sources}
At a superficial level, there seem to be mobile analytics offerings that operate within the app and those that are external to the app, particularly those that gather data at the platform level. Several widespread analytics offerings are evaluated as part of this research. Several more were considered to provide an understanding of the overall context.
\begin{itemize}
    \item \emph{What sources of analytics are available? And how do the sources that were investigated compare in terms of the data they collect and how they are used?}
    % \item \emph{Hypothesis}: The various sources have distinct characteristics and purposes. Developers may wish to choose particular analytics to best suit their context and objectives.
\end{itemize}

\subsubsection{Value}~\label{section-value}~\label{sec:value}
Does using analytics provide quantitative and/or qualitative value that can be measured? Could it provide value in terms of assessing the quality of our work that was invested into developing, testing and preparing software before it was launched?
\begin{itemize}
    \item \emph{How much does the fidelity of the analytics offerings matter?} Can the results be used productively even if they are flawed? %The research discovered numerous errors in various analytics offerings. These results may be of interest to the research community given the endemic nature of mobile analytics in real-world apps in the major app stores.

    \item \emph{How can analytics help with bug investigation?} Isolated instances of bugs may be hard to assess in terms of the likely scope and impact on a user-base; \emph{how, where and when can analytics help with bug investigation?} We might also consider practical limits,~\emph{e.g.,} constraints that are enforced by the real-world analytics being used.
\end{itemize}

\subsubsection{Impact}~\label{section-impact}~\label{sec:impact}
Here the focus is on whether the value has sufficient impact for anyone else to be interested in using and applying analytics. Given the nature of the research the main measures are practical, \emph{i.e.}, in the real-world.
\begin{itemize}
    \item \emph{Do development teams use analytics in their practice?}  And if so, how? % Hypothesis: If development teams find practices useful they will generally try to use them intrinsically. Do they?
\end{itemize}

\begin{comment}


    %\section{My research methodology, and my choices}
    Needs full rework.
    
    Merge into the strategy
    Headings:
    \begin{itemize}
        \item case study approach
        \item data collection
        \item (move 'action research' from the introduction as it's not the strategy, instead mention it in the relevant case study: Kiwix).
    \end{itemize}
    
    
    draft wording for a high-level scene setting: Within these case studies there were particular needs that led to choices in the specific case studies:
    
    
    MUST-DO Map RQ's to the case studies and explain how they're mapped, and the methods I'll be introducing later on. High level choices of the key choices I made in order to answer the RQs. Map in Chapter 3.
    
    \end{comment}

%Multiple ecosystems. Data types and informant types. 


%Move to the case study that used the action research:
%Although I had prior experience in industry of the efficacy and potency of applying usage analytics to improve software development and testing of mobile apps, that experience was generally covered by confidentially agreements, and also the analytics tools have changed and developed markedly since those experiences. Therefore, action research seemed appropriate, particularly as one of the long-term opensource projects had extremely high failure rates according to the \emph{de facto} Android analytics tool. I decided it was appropriate and necessary to see if I could help that project directly to improve their mobile apps -- \emph{``physician heal thyself"}\sidenote{\href{https://en.wikipedia.org/wiki/Physician,\_heal\_thyself}{wikipedia.org/wiki/Physician,\_heal\_thyself}.}.


\begin{comment}
    - Could I try phrasing my RQs as OKRs to see if doing so helps me to improve the clarity and relevance of the RQs.
    - Also, how about creating annotated editions of my RQs where the annotations include context, commentary, connections to other RQs, notes on twitter-style answers to each, etc.
    - We want to know more about 'this' topic. Then provide Operational questions - to be addressed by the research, which will help us learn more about the topic.
    - What's a RQ and what's an analytical lens (to be used to help with the RQ)?
\end{comment}