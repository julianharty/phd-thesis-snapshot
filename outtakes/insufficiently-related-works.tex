\chapter{Insufficiently related works}
These are outtakes from the related works chapter as they're insufficiently related to the core thesis to be included there. And yet, these topics are potentially relevant so could be re-included as needed in the main thesis or in future publications on related topics.

\section{In-app advertising}
While advertising in apps has been extensively researched the connection with mobile analytics is indirect. What they may share are:
\begin{itemize}
    \item Use of SDKs that have one of more libraries integrated into the app.
    \item Some form of tracking and reporting based on usage, however for advertising it's more likely to be tracking how many ads were served and the revenue the developer received rather than any other aspect of the app's behaviour or usage.
    \item Any adverse effects on the UX for end-users.
\end{itemize}


\section{Heatmapping}
User experience (UX) can be assessed using a wide variety of tools and techniques, such as heatmapping (which uses screen and/or interaction recording), A/B testing frameworks, funnel and journey analytics, and so on. By their very nature they're user-focused and - in practice - seldom incorporated into mobile apps or development practices for mobile apps. Similarly, based on my investigations, they are seldom researched although I have co-written work on this topic including examples of using heatmapping to improve usability of mobile apps~\cite{harty_aymer_playbook_2016}. Superficial research has been published to assess the events recorded by one of the heatmapping analytics tools - Appsee~\citep{yildirim2019_ux_analytics_for_android_platforms}. For those interested in developing a d-i-y approach to heatmapping there are a couple of options available including, the Caret-HM project~\footnote{\url{https://github.com/stlab-unt/Caret-HM}} that uses a web interface to interact with an Android emulator to record touchscreen interactions for apps installed on the emulator. The research is described in~\citep{nurmuradov2017_caret-hm-heatmapping-android-emulator}. It may be useful for small scale heatmapping however the combination of using a web interface with an emulator rather than actual Android devices limits the realism and the scale of the approach.
