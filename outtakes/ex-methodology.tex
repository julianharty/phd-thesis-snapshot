%The data needs to be rich and from real-world apps and projects which justify the use of cases. The research also had to engage with Industry and take a case-based approach situated in real projects. 

\julian{Interim actions:
\begin{enumerate}
    \itemsep0em
    \item (done) Map the methods to the six perspectives in Figure~\ref{fig:six-perspectives-in-the-methodology}. 
    \item Mapping of perspectives into the data that drives them - justify the evidence that each perspective would need.
    \item Data collection methods that will deliver the evidence: I now have the foundation for the rich data I want to collect.
\end{enumerate}}

% Data needed - justify the use of cases
% Therefore...
% Data I am seeking
% Map the data onto the analysis


%This approach is consistent with the traits of qualitative research elegantly summarised in \citealp[pp 113-114]{zieris2020_phd_qualitative_analysis_of_knowledge_transfer_in_pair_programming} that also aligns with and reflects many of the 10 features in \citealt[p.150]{ball2000_putting_ethnography_to_work_cognitive_ethnography}. (Trying to understand things in context)

%To a certain extent, experiments were natural as they arose \textit{in situ} where and when situations arose that were reported by mobile analytics that the development team chose to respond to. \textit{``A key feature of natural experiments is that they offer insight into causal processes, which is one reason why they have an established role in developmental science."}~\citep[p.877]{salkind_encyclopedia_2021_natural_experiment}



%(\textit{c.f.} targeted observation / vectored research~\footnote{MUST-DO ask Marian for her reference on using vectors in questioning. Source: `Petre 2008 Targeted observation of expert software designers. NSF workshop on studying design creativity (Aix-en-Provence, France, February)'})

%`\textbf{Sensebuilding}' methods built on insights found through sense-making, both investigating detail through micro-experiments (i.e., local app experiments) and identifying characteristics and patterns (i.e., macro-discoveries) that were not present in individual case studies (through across-case comparisons).

%\textit{Micro-experiments} (i.e., local app experiments) were created by developing small mobile apps intended to exercise particular aspects of mobile analytics (similar to the `invent the future' adage~\footnote{For example: \url{https://quoteinvestigator.com/2012/09/27/invent-the-future/}}). The inputs to an app were directed in order to determine the outputs from mobile analytics. These helped to answer questions and gaps observed as part of sensemaking.  The local app experiments gave insight into the relationships between tools, quality of analytics, and potential impact of analytics use on apps (i.e., perspectives \uartefacts, \utools, \iartefacts, \itools). 

%\textit{Across-case comparisons} were concerned largely with understanding current practice and identifying potential improvements to practice (i.e., perspectives \uuse and \iuse), as well as the influence of the quality of tools in practice (\itools).


\begin{comment}
Commented out after adding the grouping suggested with Marian on \nth{22} Oct 2021. Preserved until this chapter is in better shape.

\julian{Conflations: 1) data collection methods and analysis 2) contributors and goals also conflated. Action: Group rows by what I was targeting: I need to provide the rationale for the methods I used.} Two suggested groupings to try:
\begin{itemize}
    \item 3 Groupings: Tool outputs, validation, action research. 
    \item 4 Groupings: beacons-and-drill-downs, comparisons, contextualisation and clarification, evaluation through action research.
\end{itemize}
\end{comment}


\section{Text-based methodology maps (a w-i-p)}
\emph{The following hierarchical lists are a work-in-progress while the details are established. They will be replaced and retired.}


{\footnotesize
\begin{description}
    \itemsep0em
    \item[Research Method] :
        \begin{itemize}
            \item Case Study
        \end{itemize}
    \item[Research Tools] :
    \begin{itemize}
        \itemsep0em
        %\item[]
        \item Action Research
        \item[$\bullet$] [Semi-] Controlled Experiment (Field experiments)
        \item Hackathon
    \end{itemize}
    \item[Analytics Provider] :
        \begin{itemize}
            \item Google Play Console with Android Vitals
        \end{itemize}
    \captionof{methodology-map}{Kiwix case study methodology map}
    \label{methodology-map-kiwix-case-study}    
\end{description}
}
% Thanks to https://fedoramagazine.org/latex-typesetting-part-1/ for helping me discover {description} and for enabling me to find workarounds when it didn't render as I'd expected e.g. where I needed to add [$\bullet$] to stop it interpreting [Semi-] as a style.

{\footnotesize
\begin{description}
    \itemsep0em
    \item[Research  Method] :
        \begin{itemize}
            \item Inductive analysis of source code
        \end{itemize}
    \item[Research Tool] TBC
    \item[Data Analysis Method] TBC
    \captionof{methodology-map}{Logging using Mobile Analytics}
    \label{methodology-map-logging-using-mobile-analytics}
\end{description}
}


\begin{comment}
Therefore the research methodology has three main prongs:

\begin{enumerate}
    \itemsep0em
    \item Case-studies: to engage effectively with the organisation, projects, and project teams. The case studies are the source of much of the evidence used in how mobile analytics is used by development teams in practice. See page \pageref{section-case-study-procedure} for details of the case study procedure; \textit{i.e.} the structure of a case study within the context of this research. 
    \item Data-collection: to collect data in practice given the nature of the engagements between research and the projects.
    \item Analysis: to make sense of the manifold rich data sources and the findings that emerge. 
\end{enumerate}

The methodology also includes validation findings and analysis with the development teams who participate in the research. 


\subsection{Rationale}
The rationale for the choice of methods

\newthought{How the methods fit in a framework of theories and complement each other}
\isabel{I would say that the methodology is a description not only of the methods, but also of the rationale for the choice of methods and also how they fit together in a framework of theories, and how they complement and add to each other to provide triangulation and to mitigate threats to validity. So quite a lot to do here.}

\end{comment}