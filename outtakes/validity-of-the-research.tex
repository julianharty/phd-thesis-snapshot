\subsection{Validity of my research}

\subsection{Validity of the tools being used}

\subsubsection{Validity of platform analytics}

\subsubsection{Validity of Fabric Crashlytics}

\subsubsection{Validity of various mobile apps}

Our simple opensource Android app, Zipternet~\footnote{\url{https://github.com/ISNIT0/zipternet}}, included HTML-based content. In the initial releases the app had a flaw that caused a crash to occur when any external web links were selected in the content~\footnote{See issue 6 on Github for this project:~\href{https://github.com/ISNIT0/zipternet/issues/6}{Fix crash of the app when external URL selected in WebView}.}. This crash occurred consistently whenever any external web link was selected and was detected and reported by Google Play Console's pre-launch report and it was also reported in Android Vitals, albeit less often than when the crash occurred on devices we used for local testing of the app.


In February 2020 the iOS version of the Pocket Code app was enhanced to include a mechanism to cause crashes deterministically~\footnote{See JIRA ticket:~\href{https://jira.catrob.at/browse/CATTY-161}{CATTY-161 - INTERMEDIATE TICKET: Add ``easter egg" for Crashlytics} for the iOS feature.}. This ability to cause crashes deterministically was designed and implemented to determine whether the crashes would be reported correctly in Firebase Crashlytics (they were). The iOS team removed the ``easter egg" immediately after the workshop in Poland~\footnote{See JIRA ticket:~\href{https://jira.catrob.at/browse/CATTY-162}{CATTY-162 - TRAINING TICKET: Remove "easter egg" for Crashlytics}.}. The Android version of Pocket Code already used the older Fabric Crashlytics API and crashes were being reported in the Fabric Crashlytics dashboard.