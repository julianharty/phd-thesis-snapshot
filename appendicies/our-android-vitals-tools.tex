\chapter{Software we developed for Google Play Console}
Developers are only able to read Android Vitals reports in real-time in the user interface Google provides in Google Play Console. This limits their ability to perform trend analysis beyond the period Google allots. Also they are restricted to taking screenshots, etc. to save pertinent information. There are several limits, a hard limit of 60 (sixty) days and seemingly limits based on volumes of crash and ANR data. 

Our contributions enable them to preserve various relevant reports and also individual crash and ANR trace data by using our Vitals Scraper software. They can then also use our software to group incompletely grouped crash clusters which enables them to improve the prioritisation of their work to investigate and improve quality aspects of their apps.

The developers can also choose to make the reports and data available to others, for instance for analysis and research. Our software is extensible and freely available as permissively-licensed opensource software tools. 

\section{Background}
Google Play Console is a developer's view of the Google Play app store and includes data and reports for the apps the person has access to. Typically people have access to data for apps they develop and/or support.

At least some of the data is collected from users who opt-in to share their usage and diagnostics data with Google, and Google - in turn - shares some of this data with developers~\cite{google_play_share_usage_and_diagnostics_info_with_google}.

Google Play Console provides various summary data as monthly reports in a downloadable format, and these files are available historically so the data can be downloaded for previous periods. 

\subsection{Summary Data is available for download}
The monthly summary reports are available to download\footnote{\url{https://support.google.com/googleplay/android-developer/answer/6135870}} either interactively or using a command line tool \texttt{gsutil}\footnote{\url{https://support.google.com/googleplay/android-developer/?p=stats\_export}}. The reports include crashes and ANRs, ratings, reviews, installations, and financial information. 

From what we can tell the data is available from when it was first generated by Google Play Console.

Google used to provide detailed crash and ANR reports until May 2018~\cite{google_play_download_and_export_monthly_reports} before removing this facility. They also removed the ability to download these reports for prior periods which left a gap in terms of being able to analyse either of these quality issues.

\section{Design and Development}
\begin{itemize}
    \item Add figure from discussion with Yijun on scrapers.
    \item Explain about the use of the embedded browser and using TypeScript
    \item Explain how the software was developed and by whom
    \item Discuss some of the technical challenges
    \item Discuss some of the softer challenges and risks of Google blocking the user account and those of those it deems "related"
    \item How did we address these various challenges, and what remains to address.
\end{itemize}

\textbf{Note} Consider my materials in a draft paper on Scraper APIs.