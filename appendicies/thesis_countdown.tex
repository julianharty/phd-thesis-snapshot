\label{section-thesis-countdown}
Arosha estimates there's about 5 weeks for a typical full-time PhD student to complete the entire thesis as a viable draft. To help me track and motivate myself, here's a countdown of the approximately 200 hours equivalent (5 weeks * 5 days * 8 hours).

The thesis will still need revising which is likely to increase the total work before it's fit to submit.
\newcommand\reverselabel[1]{%
  \def\theenumi{}%
  \renewcommand\makelabel{\makebox[\dimexpr\labelwidth-3pt\relax][r]{%
    \the\numexpr#1-\value{enumi}+1\relax}}}%
% Thanks to: https://tex.stackexchange.com/questions/231248/how-can-i-enumerate-backwards

\begin{enumerate}
    \reverselabel{200}
    \item Work on literature review chapter. Added several papers related to software testing for Android apps. Learning more about applying mutation testing to Android apps.
    \item Added this section and the reverse counter. Read several papers from QRS 2017, selected one of those read - on software reliability as user perception... Interestingly (to me) I spent more time searching, reading, and formatting the references than writing. Let's see how the time's distributed as I continue to make progress.
    \item Worked on filling in missing details for the `my contributions' section. Also spent about 20 mins looking through various ICST 2021 workshops to see if any were worth aiming to submit some of my work to. Some recent non-thesis work on test tools for Android protocol analysis and testing might be a good fit.
    \item Wrote up the PRADA paper in the related works chapter. 
    \item Added a couple more papers with brief notes. Reading these papers so I can write them up. Distracted by the OU admin deciding to close my account prematurely - that took about an hour in addition to my work on my thesis.
    \item Wrote some notes on logging.
    \item Revisiting my paper on improving logging (from 2017) combined with some additional current investigation into what's available and might be particularly relevant for a case study. Discovered some links have long since disappeared e.g. where a business has been acquired. Discussions with Marian and Isabel today.
    \item Spent another hour reading and writing about logging practices. Added several examples from the K-9 Android email client app. I also incorporated a revised introduction from my old draft paper on improving logging.- good to be able to re-purpose and reuse that work!
    \item Watching one of the Twitter Flight 2015 videos on the backend design of Answers, mobile analytics. Also reading up on two of the key algorithms that underpinned their work (Hyper Log Log, and Bloom Filters).
    \item Some non-thesis tidying up of my paper notes, sketches, and printed materials.
    \item Reading some of Li Li's recent papers. Wrote up one of them so far.
    \item Also investigated the project reproduction materials from one of Li Li's papers \url{https://github.com/CraftDroid} and how they might be used for real-world application by commercial developers (\textit{i.e.} beyond the research community).
    \item Read the second of the papers Li Li co-authored for ISSTA 2019. Emailed the lead author as the reproduction package lacks test or installation scripts. Underwhelmed with the paper as it stands, wrote it up nonetheless.
    \item Updated and extended the Iteratively case study. Also wrote two additional paragraphs in the summary for the case studies.
    \item Read up on a unit testing library for log messages and compared it with the \texttt{logAssert} library we wrote. Writing this up now.
    \item Revising the logging case study section and adding various notes to help me remember what to include from my existing research and writing related to this topic.
    \item An extended journey into the sizes of native all development teams, started by~\url{https://twitter.com/GergelyOrosz/status/1345288831029956610}.
    \item Hour long call with Gergely Orosz ex-Uber, \href{https://www.mobileatscale.com/}{mobileatscale.com}
    \item Wrote some material on dynamics of development teams, requested more info from the source.
    \item Researching and writing notes on developing for Kindle Fire devices and for the Amazon appstore as part of the discussion chapter. Created developer account as part of the process.
    \item Wrote up the first paper in my Zotero database related to my thesis, on automating UX oriented performance testing.
    \item Sketched a figure representing the user-population for an installed app during a new release of the app where there are both existing and new users. Note: I've yet to add this figure in the thesis. In parallel I've been working with a colleague on investigating native crashes for an Android app and learned a great deal which also applies to the thesis. A subset will need to be written up here.
    \item Read~\emph{`Communication in Testing: Improvements for Testing Management'}~\citep{paakkonen2009_communication_in_testing} in PG Forum and then subsequently. Spent at least 30 minutes tracking down and crafting the reference.
    \item Interviewed Gergely Orosz on his team's use of mobile analytics.
    \item Started writing up notes on the above paper and from Gergely. Realised the case studies need to be restructured to support the inclusion of the interview, did so, and added some initial additional material to start to illustrate common strands from various sources.
    \item Re-read and wrote notes on a relatively dated paper on model based testing for Android apps.
    \item Added two figures, one revised from a rough sketch into a rough powerpoint slide. Lost an extra 20 mins when the laptop spontaneously rebooted.
    \item Wrote about install bases.
    \item Extended the install base topic to add cohorts. Added a reference from HP/dimensional research on app abandonment.
    \item Started writing about the networking example. 15\% of the 200 hours i.e. 30 hours completed, and 100 commits committed. This will be commit 101.
    \item Worked on the related works chapter and added material based on several sources.
    \item First pass through ~\citep{avizienis2004_basic_concepts_and_taxonomy} - this is a key paper, I'm glad to have found it.
    \item Discovered Pfleeger's work on risk management which is paraphrased in Amland's work.
    \item Wrote up DBS based on TBS. Email discussion with the author and his brother (I know both as friends and colleagues) to share my idea with them and ask permission to base it on their work.
    \item Thanks to Yijun's encouragement - sketched out a fresh layout to set the thesis in context in order to resolve the related works chapter.
    \item Used PowerPoint to create a simplified version of the sketch and then variations of this simplified version to use in the related works chapter. 
    \item Wrote fresh materials at the start of the related works chapter. Also did some reading.
    \item Continued work on the Makers section of the related works chapter. Looked at several relevant items in Zotero, realised I've lots and lots of references so need to be more selective than using what I've collected and collated so far. Added a photo of the hysteresis sketch I did at Shonan.
    \item Researched work on this thesis after a gap of over 3 weeks. I spent between 15 and 20 hours revising the paper we had submitted to NIER 2021 so it was fit for MOBILESoft 2021 as a short paper (4+1). The work included recreating and debugging the software source code used to perform the automated analysis we wrote about. I'm a lot more confident in the work and approach now. 
    \item The work and approach needs to be written up in the logging case study, starting doing so on \nth{18} Feb 2021.
    \item Wrote some introductory material to the logging case study.
    \item More writing and research in logging for mobile platforms and UNIX. I've still got to write up and cite various examples that proves the fundamentals of processing output streams in Android and iOS as logs, and on remote logging via approaches such as those in CocoaLumberjack.
    \item External discussions into chunking logfiles on devices and transmitting them, nothing written here yet.
    \item Added new introduction to the Case Studies chapter. Also using print\_a\_chapter to help me focus. Note: in the last month or two I've also spent some time (5+ hours I'm guessing) trying to use one of the Tufte latex templates for my thesis. This work has been parked as it's not on the critical path and I'm now returning to the content of the thesis rather than the presentation - both are important, however it's impractical for me to invest a lot of time on both concurrently given my other commitments.  
    \item Spent overall about 5 hours working on reading and making notes related to a paper Yijun suggested:~\emph{``Trends in Software Engineering Research for Mobile Apps"}~\citep{nagappan2016_future_trends_in_sw_eng_for_mobile_apps}.
    \item Realised this was a useful exercise and that my research is not covered in this survey paper.
    \item Discovered it was probably written at a previous Shonan workshop \#70 where many of the participants for \#152 were previously together with other highly competent researchers in app stores, etc.
    \item Annotated 2 of the figures from this paper to situate my research.
    \item Read one of several relevant references cited in that survey paper.
    \item Catching up on industrial/commercial mobile analytics startups - several acquired relatively quickly and only AppAnnie still exists as an entity (it was also one of the early players that did well in the early days so may be the market leader so harder to dislodge or acquire. Updated the 10 dots/coins exercise for Iteratively with a new entry.
    \item Reading 2 more papers and about to write then both up~\citep{linares2015_mining_android_app_execution_traces_etc, martin2017_survey_in_app_store_analysis_for_software_engineering_IEEE_edition}
    \item Spent most of an hour reading through the 4 Crowd RE workshops (International Workshop on Crowd-Based Requirements Engineering) partly as they mentioned event logs as a source of information. However none of the papers were very relevant to my research :( There are some on Requirements Engineering and one on GoPED (Data Preprocessing for Goal-Oriented Process Discovery) which might be of interest in future.
    \item Re-reading~\citep{maalej2016_towards_data_driven_requirements_engineering} and added some of the terms to the glossary.
    \item Ongoing reading of ~\citep{maalej2016_towards_data_driven_requirements_engineering}.
    \item This hour may have been relatively well spent, the next couple weren't so I'll discount them. I decided to improve the glossary of terms so it can extend beyond a page. I spent sooooo much time (probably another 2 to 3 hours) trying to sort out the formatting of the heading of the table. In the end I've left some notes in the file and abandoned the exercise now I've got something basic working ish that was better than what I had previously. 
    \item A journey into \url{arxiv.org} partly to register myself as an author, and then to provide a fresh perspective on seeking relevant research.
    \item Started adding material to the related work chapter on ANRs. Note: I'm also adding fieldstones to the draftmaterials/fieldstones file.
    \item Started working on the red thread (which is currently a meta-chapter). Reading and assessing~\emph{``Revisiting Prior Empirical Findings for Mobile Apps: An Empirical Case Study on the 15 Most Popular Open-Source Android Apps"}~\citep{syer2013_empirical_findings_for_mobile_apps} (which studies bugs reported and fixed in these codebases to compare and contrast who fixes them and the time taken between bug report and fix.
    \item Reviewed my progress with Arosha and Yijun on Fri 09 Apr 2021. Made some notes on what to cover. V3 of the thesis due.
    \item Reviewed much of the email correspondence with the Product Manager Nandan Pujar of Appachhi. 
    \item Rediscovered various ideas and materials I'd created and suggested at the time (2015 and 2016)
    \item Ongoing review of my notes and materials related to AppPulse Mobile. Interesting to rediscover various code and project details.
    \item Further reading related to AppPulse Mobile.
    \item Joined a couple of sessions for workshops affiliated with ICST 2021.
    \item Extended the Red Thread chapter.
    \item Calls with Joe and later Marian re my progress. With Marian agreed to create 2 inventories.
    \item Working on the first inventory
    \item Completed the first and then the second inventory. This was a useful exercise in helping me assess the state of affairs in my thesis and various areas I need to tackle.
    \item Added a few minor contributions to this body of writing.
    \item Call with Arosha and Yijun on Fri \nth{16} April 2021 - lots of interesting ideas came from discussing the conceptual pre-requisites tree in the red thread section. Also received great suggests from Joe Reeve on \nth{18}.
    \item Very helpful discussion with Marian about the results of me preparing the inventories. This led to discussing how to reorder the content to make the message much clearer. I've had a first go at changing the order accordingly.
    \item Created two new files: a \href{section-case-studies-red-thread}{red-thread for the case studies}, and a closing section~\href{section-synthesis-recommendations}{\nameref{section-synthesis-recommendations}}.
    \item Follow ups with Joe and Marian, separately, on the structure of case studies. 
    \item Worked mainly on the case studies, and created a first table on the app case studies.
    \item Some follow up and ongoing work for the large industrial case study tracking the effects of a new release rollout, etc.
    \item Further work on the case studies.
    \item Ditto.
    \item Reviewing and updating GTAF materials.
    \item Reviewed materials and recent info on LocalHalo and Moodspace.
    \item Emailed my contacts for Moodspace and for Moonpig asking for any recent insights.
    \item Quick, temporary, ad-hoc summary for the secondary research via Gergely Orosz.
    \item Focusing on the catrobat case study
    \item Working through old evidence I've collected for the Catrobat case study
    \item continued...
    \item Adding material to the catrobat case study section
    \item Call with Joe on the case studies generally and catrobat in particular. He suggested several key focusing questions (noted in the file)
    \item Experimented with minted and YAML which may be a great way to provide a consistent structure for my case studies.
    \item Wrote an abstract for the Catrobat (Pocket Code) case study, this was really helpful to do.
    \item Ongoing work on this case study, in particular on the evidence meta-section.
\end{enumerate}
