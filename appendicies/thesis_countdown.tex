\label{section-thesis-countdown}
Arosha estimates there's about 5 weeks for a typical full-time PhD student to complete the entire thesis as a viable draft. To help me track and motivate myself, here's a countdown of the approximately 200 hours equivalent (5 weeks * 5 days * 8 hours).

The thesis will still need revising which is likely to increase the total work before it's fit to submit.
\newcommand\reverselabel[1]{%
  \def\theenumi{}%
  \renewcommand\makelabel{\makebox[\dimexpr\labelwidth-3pt\relax][r]{%
    \the\numexpr#1-\value{enumi}+1\relax}}}%
% Thanks to: https://tex.stackexchange.com/questions/231248/how-can-i-enumerate-backwards

\begin{enumerate}
    \reverselabel{200}
    \item Work on literature review chapter. Added several papers related to software testing for Android apps. Learning more about applying mutation testing to Android apps.
    \item Added this section and the reverse counter. Read several papers from QRS 2017, selected one of those read - on software reliability as user perception... Interestingly (to me) I spent more time searching, reading, and formatting the references than writing. Let's see how the time's distributed as I continue to make progress.
    \item Worked on filling in missing details for the `my contributions' section. Also spent about 20 mins looking through various ICST 2021 workshops to see if any were worth aiming to submit some of my work to. Some recent non-thesis work on test tools for Android protocol analysis and testing might be a good fit.
    \item Wrote up the PRADA paper in the related works chapter. 
    \item Added a couple more papers with brief notes. Reading these papers so I can write them up. Distracted by the OU admin deciding to close my account prematurely - that took about an hour in addition to my work on my thesis.
    \item Wrote some notes on logging.
    \item Revisiting my paper on improving logging (from 2017) combined with some additional current investigation into what's available and might be particularly relevant for a case study. Discovered some links have long since disappeared e.g. where a business has been acquired. Discussions with Marian and Isabel today.
    \item Spent another hour reading and writing about logging practices. Added several examples from the K-9 Android email client app. I also incorporated a revised introduction from my old draft paper on improving logging.- good to be able to re-purpose and reuse that work!
    \item Watching one of the Twitter Flight 2015 videos on the backend design of Answers, mobile analytics. Also reading up on two of the key algorithms that underpinned their work (Hyper Log Log, and Bloom Filters).
    \item Some non-thesis tidying up of my paper notes, sketches, and printed materials.
    \item Reading some of Li Li's recent papers. Wrote up one of them so far.
    \item Also investigated the project reproduction materials from one of Li Li's papers \url{https://github.com/CraftDroid} and how they might be used for real-world application by commercial developers (\textit{i.e.} beyond the research community).
    \item Read the second of the papers Li Li co-authored for ISSTA 2019. Emailed the lead author as the reproduction package lacks test or installation scripts. Underwhelmed with the paper as it stands, wrote it up nonetheless.
    \item Updated and extended the Iteratively case study. Also wrote two additional paragraphs in the summary for the case studies.
    \item Read up on a unit testing library for log messages and compared it with the \texttt{logAssert} library we wrote. Writing this up now.
    \item Revising the logging case study section and adding various notes to help me remember what to include from my existing research and writing related to this topic.
    \item An extended journey into the sizes of native all development teams, started by~\url{https://twitter.com/GergelyOrosz/status/1345288831029956610}.
    \item Hour long call with Gergely Orosz ex-Uber, \href{https://www.mobileatscale.com/}{mobileatscale.com}
    \item Wrote some material on dynamics of development teams, requested more info from the source.
    \item Researching and writing notes on developing for Kindle Fire devices and for the Amazon appstore as part of the discussion chapter. Created developer account as part of the process.
    \item Wrote up the first paper in my Zotero database related to my thesis, on automating UX oriented performance testing.
    \item Sketched a figure representing the user-population for an installed app during a new release of the app where there are both existing and new users. Note: I've yet to add this figure in the thesis. In parallel I've been working with a colleague on investigating native crashes for an Android app and learned a great deal which also applies to the thesis. A subset will need to be written up here.
    \item Read~\emph{`Communication in Testing: Improvements for Testing Management'}~\citep{paakkonen2009_communication_in_testing} in PG Forum and then subsequently. Spent at least 30 minutes tracking down and crafting the reference.
    \item Interviewed Gergely Orosz on his team's use of mobile analytics.
    \item Started writing up notes on the above paper and from Gergely. Realised the case studies need to be restructured to support the inclusion of the interview, did so, and added some initial additional material to start to illustrate common strands from various sources.
    \item Re-read and wrote notes on a relatively dated paper on model based testing for Android apps.
    \item Added two figures, one revised from a rough sketch into a rough powerpoint slide. Lost an extra 20 mins when the laptop spontaneously rebooted.
    \item Wrote about install bases.
    \item Extended the install base topic to add cohorts. Added a reference from HP/dimensional research on app abandonment.
    \item Started writing about the networking example. 15\% of the 200 hours i.e. 30 hours completed, and 100 commits committed. This will be commit 101.
    \item Worked on the related works chapter and added material based on several sources.
    \item First pass through ~\citep{avizienis2004_basic_concepts_and_taxonomy} - this is a key paper, I'm glad to have found it.
    \item Discovered Pfleeger's work on risk management which is paraphrased in Amland's work.
    \item Wrote up DBS based on TBS. Email discussion with the author and his brother (I know both as friends and colleagues) to share my idea with them and ask permission to base it on their work.
    \item Thanks to Yijun's encouragement - sketched out a fresh layout to set the thesis in context in order to resolve the related works chapter.
    \item Used PowerPoint to create a simplified version of the sketch and then variations of this simplified version to use in the related works chapter. 
    \item Wrote fresh materials at the start of the related works chapter. Also did some reading.
    \item Continued work on the Makers section of the related works chapter. Looked at several relevant items in Zotero, realised I've lots and lots of references so need to be more selective than using what I've collected and collated so far. Added a photo of the hysteresis sketch I did at Shonan.
    \item Researched work on this thesis after a gap of over 3 weeks. I spent between 15 and 20 hours revising the paper we had submitted to NIER 2021 so it was fit for MOBILESoft 2021 as a short paper (4+1). The work included recreating and debugging the software source code used to perform the automated analysis we wrote about. I'm a lot more confident in the work and approach now. 
    \item The work and approach needs to be written up in the logging case study, starting doing so on \nth{18} Feb 2021.
    \item Wrote some introductory material to the logging case study.
    \item More writing and research in logging for mobile platforms and UNIX. I've still got to write up and cite various examples that proves the fundamentals of processing output streams in Android and iOS as logs, and on remote logging via approaches such as those in CocoaLumberjack.
    \item External discussions into chunking logfiles on devices and transmitting them, nothing written here yet.
    \item Added new introduction to the Case Studies chapter. Also using print\_a\_chapter to help me focus. Note: in the last month or two I've also spent some time (5+ hours I'm guessing) trying to use one of the Tufte latex templates for my thesis. This work has been parked as it's not on the critical path and I'm now returning to the content of the thesis rather than the presentation - both are important, however it's impractical for me to invest a lot of time on both concurrently given my other commitments.  
    \item Spent overall about 5 hours working on reading and making notes related to a paper Yijun suggested:~\emph{``Trends in Software Engineering Research for Mobile Apps"}~\citep{nagappan2016_future_trends_in_sw_eng_for_mobile_apps}.
    \item Realised this was a useful exercise and that my research is not covered in this survey paper.
    \item Discovered it was probably written at a previous Shonan workshop \#70 where many of the participants for \#152 were previously together with other highly competent researchers in app stores, etc.
    \item Annotated 2 of the figures from this paper to situate my research.
    \item Read one of several relevant references cited in that survey paper.
    \item Catching up on industrial/commercial mobile analytics startups - several acquired relatively quickly and only AppAnnie still exists as an entity (it was also one of the early players that did well in the early days so may be the market leader so harder to dislodge or acquire. Updated the 10 dots/coins exercise for Iteratively with a new entry.
    \item Reading 2 more papers and about to write then both up~\citep{linares2015_mining_android_app_execution_traces_etc, martin2017_survey_in_app_store_analysis_for_software_engineering_IEEE_edition}
    \item Spent most of an hour reading through the 4 Crowd RE workshops (International Workshop on Crowd-Based Requirements Engineering) partly as they mentioned event logs as a source of information. However none of the papers were very relevant to my research :( There are some on Requirements Engineering and one on GoPED (Data Preprocessing for Goal-Oriented Process Discovery) which might be of interest in future.
    \item Re-reading~\citep{maalej2016_towards_data_driven_requirements_engineering} and added some of the terms to the glossary.
    \item Ongoing reading of ~\citep{maalej2016_towards_data_driven_requirements_engineering}.
    \item This hour may have been relatively well spent, the next couple weren't so I'll discount them. I decided to improve the glossary of terms so it can extend beyond a page. I spent sooooo much time (probably another 2 to 3 hours) trying to sort out the formatting of the heading of the table. In the end I've left some notes in the file and abandoned the exercise now I've got something basic working ish that was better than what I had previously. 
    \item A journey into \url{arxiv.org} partly to register myself as an author, and then to provide a fresh perspective on seeking relevant research.
    \item Started adding material to the related work chapter on ANRs. Note: I'm also adding fieldstones to the draftmaterials/fieldstones file.
    \item Started working on the red thread (which is currently a meta-chapter). Reading and assessing~\emph{``Revisiting Prior Empirical Findings for Mobile Apps: An Empirical Case Study on the 15 Most Popular Open-Source Android Apps"}~\citep{syer2013_empirical_findings_for_mobile_apps} (which studies bugs reported and fixed in these codebases to compare and contrast who fixes them and the time taken between bug report and fix.
    \item Reviewed my progress with Arosha and Yijun on Fri 09 Apr 2021. Made some notes on what to cover. V3 of the thesis due.
    \item Reviewed much of the email correspondence with the Product Manager Nandan Pujar of Appachhi. 
    \item Rediscovered various ideas and materials I'd created and suggested at the time (2015 and 2016)
    \item Ongoing review of my notes and materials related to AppPulse Mobile. Interesting to rediscover various code and project details.
    \item Further reading related to AppPulse Mobile.
    \item Joined a couple of sessions for workshops affiliated with ICST 2021.
    \item Extended the Red Thread chapter.
    \item Calls with Joe and later Marian re my progress. With Marian agreed to create 2 inventories.
    \item Working on the first inventory
    \item Completed the first and then the second inventory. This was a useful exercise in helping me assess the state of affairs in my thesis and various areas I need to tackle.
    \item Added a few minor contributions to this body of writing.
    \item Call with Arosha and Yijun on Fri \nth{16} April 2021 - lots of interesting ideas came from discussing the conceptual pre-requisites tree in the red thread section. Also received great suggests from Joe Reeve on \nth{18}.
    \item Very helpful discussion with Marian about the results of me preparing the inventories. This led to discussing how to reorder the content to make the message much clearer. I've had a first go at changing the order accordingly.
    \item Created two new files: a \href{section-case-studies-red-thread}{red-thread for the case studies}, and a closing section~\href{section-synthesis-recommendations}{\nameref{section-synthesis-recommendations}}.
    \item Follow ups with Joe and Marian, separately, on the structure of case studies. 
    \item Worked mainly on the case studies, and created a first table on the app case studies.
    \item Some follow up and ongoing work for the large industrial case study tracking the effects of a new release rollout, etc.
    \item Further work on the case studies.
    \item Ditto.
    \item Reviewing and updating GTAF materials.
    \item Reviewed materials and recent info on LocalHalo and Moodspace.
    \item Emailed my contacts for Moodspace and for Moonpig asking for any recent insights.
    \item Quick, temporary, ad-hoc summary for the secondary research via Gergely Orosz.
    \item Focusing on the catrobat case study
    \item Working through old evidence I've collected for the Catrobat case study
    \item continued...
    \item Adding material to the catrobat case study section
    \item Call with Joe on the case studies generally and catrobat in particular. He suggested several key focusing questions (noted in the file)
    \item Experimented with minted and YAML which may be a great way to provide a consistent structure for my case studies.
    \item Wrote an abstract for the Catrobat (Pocket Code) case study, this was really helpful to do.
    \item Ongoing work on this case study, in particular on the evidence meta-section.
    \item More work on the Catrobat case study.
    \item Catrobat case study again.
    \item ...and again.
    \item Discussed a red-thread with PG Forum led by Marian. Added notes to the fieldstones material which is one of the unpublished sections of this thesis.
    \item Had a follow up discussion with Marian based on my progress with the Catrobat case study so far. She's suggested more frequent sync ups will help me to establish the necessary contents, and once this has been done then the rest of the case studies and ultimately the thesis will be easier to align. I agree on with her.
    \item Added another two preliminary subsections to the Catrobat case study based on her observations and started working on these.
    \item Spent time applying various daily checks, filed a bug and revised several others for the industrial case study. This isn't strictly work on my PhD thesis, however the practice helps me to think of how to distil the story/message of the thesis as well as the efficacy and relevance of applying the practices.
    \item Updated the my publications section of my thesis. Started cross-referencing the relevant case studies that were introduced in my publications. This didn't take an hour, the rest was time spent on my case studies.
    \item First pass readings of 5 papers Google Scholar recommended: 
    \begin{itemize}
        \item \href{https://10.1109/TR.2021.3066170}{`Effort-Aware Just-in-Time Bug Prediction for Mobile Apps Via Cross-Triplet Deep Feature Embedding'} - out of scope.
        \item \href{https://10.1109/ACCESS.2021.3074266}{`Code Complexity and Version History for Enhancing Hybrid Bug Localization'} - out of scope.
        \item \href{https://10.1109/TSE.2021.3071473}{`Pathidea: Improving Information Retrieval-Based Bug Localization by Re-Constructing Execution Paths Using Logs'} - shows promise potentially for future work, it's not part of the current story of my core thesis. It may be relevant to the case study on using firebase for logging. Certainly it has some interesting and relevant references on a) bug localisation b) logging practices.
        \item \href{http://www.ece.ubc.ca/~mjulia/publications/AndroR2_2021.pdf}{`ANDROR2: A Dataset of Manually-Reproduced Bug Reports for Android apps'}. Certainly well worth cross-checking the papers they cite. I'll hold off until I've made more progress with writing up several case studies. TODO.
        \item \href{https://10.1109/TSE.2021.3071193}{`A Survey of Performance Optimization for Mobile Applications'} - several of the topics it covers are also reported by Google Play Console's Android Vitals. Worth a longer read.
    \end{itemize}
    \item The last of those papers led me to several more written by one of the authors: Maria Kechagia, and reminded me to cross check \href{http://www0.cs.ucl.ac.uk/staff/F.Sarro/publications.html}{Federica Sarro}'s work. 
    \begin{itemize}
        \item \href{https://doi.org/10.1145/3183399.3183416}{`Enabling real-time feedback in software engineering'} - Great concepts. Aspects of realtime feedback are available in various software analytics tools e.g. on crashes and ANRs, and on usage. Worth reading. Also there are several interesting looking papers that cite this one:~\url{https://scholar.google.com/scholar?cites=1231011581933816035&as_sdt=2005&sciodt=0,5&hl=en} TODO.
        \item \href{https://doi.org/10.1007/s10664-014-9343-7}{`Charting the API minefield using software telemetry data'} Similar to SafeDK's commercial tracking service - now defunct. The research sourced data from BugSense, which is minuscule~\href{https://www.appbrain.com/stats/libraries/details/bugsense/bugsense}{BugSense statistics on AppBrain} and its successor combined~\href{https://www.appbrain.com/stats/libraries/details/splunkmint/splunk-mint}{Splunk MINT statistics on AppBrain} compared to the data Google Play Store collects.
        \item \href{https://doi.org/10.4230/OASIcs.ICCSW.2013.57}{`Improving the quality of APIs through the analysis of software crash reports'}
    \end{itemize}
    \item Many hours 10+ working on slides and test recordings for my MOBILESoft 2021 paper.
    \item Another 10+ hours spent on the slides and test recordings. Innumerable challenges where connections would fail while recording, I made mistakes I don't know how to edit out, exceeding the time limit of 10 minutes allowed, the usual technical/software snafus, batteries expiring, and so on.
    \item And the final batch of 10+ hours before I finally had two recordings that met the criteria and were tolerable. I uploaded the later one together with the slides. Now correcting the automatically generated sub-titles/captions.
    \item As part of the above I also revisited many of the papers cited in the presentation to check the appropriate ones were being cited correctly.
    \item Created a private repo for my private reproduction of a skeletal version of the mobile analytics logging project. Emailed my co-authors about adding a license to the reproduction package.
    \item Following up on the early collaborative work on mobile analytics logging to back fill aspects of the analysis to see which of the 107 projects also incorporate crash reporting.
    \item Writing a fresh version of the case study on logging based on applying Marian's and Arosha's advice.
    \item This is at times mind-numbing, does anyone \textit{really} care about this research? Let's hope the darkest hour is just before dawn.
    \item Worked on the logging case study. There are various gaps in what I understand of the process we followed e.g. where did Haonan switch from using GHTorrent to querying github.com? I've emailed him to ask for his insights.
    \item Took various images from the presentation we created for MOBILESoft 2021 and uploaded them here.
    \item Worked out how to add a listoflistings, debugged issues where the documented approach to rename the caption wasn't recognised.
    \item More writing and adding of contents to the logging case study.
    \item Ditto.
    \item Ditto, and additional correspondence with Haonan which has been helpful in clarifying the early steps in our work before he had been introduced to the group.
    \item Ditto, added some notes on future work to the case study and re-reading my correspondence and notes related to the smartnavi project and app.
    \item Call with Arosha and Yijun to discuss the work on the above case study and generally. Agreed it was OK to self-plagiarise material from the logging paper provided I clearly explain this has been done (which I have done).
    \item Added 4 abstracts into comments in my Introduction for the research questions. Also added some additional paper references to the related work chapter.
    \item More time spent reading and researching materials pertaining to the logging case study. 
    \item Ditto. (Also joined a couple of presentations from MOBILESoft 2021 until my Internet connection collapsed again).
    \item Slightly out of order as I did it earlier today, nonetheless I spent about an hour working through materials related to HP and AppPulse Mobile.
    \item Created a Word document using Arosha's suggested structure for the case studies so I could work offline. Started writing up the Kiwix one.
    \item Writing the Kiwix case study.
    \item Ditto. (I may well have spent another hour writing and researching which I've not tracked).
    \item Ditto. Also had a chat with Marian first and the next day with Arosha. Migrated the contents of the Word document here now I've a reliable Internet connection.
    \item Spent a good hour tidying up citations and adding references in the Kiwix case study and also tried (as yet unsuccessfully) to improve the bibliography for GitHub projects, e.g. see~\url{https://academia.stackexchange.com/a/14015/10327} for the approach I tried. \url{https://github.blog/2014-05-14-improving-github-for-science/} and \url{https://guides.github.com/activities/citable-code/} might be better in the longer term anyway.
    \item Expanded the section on Analytics Intervention and related materials for the Kiwix Case Study.
    \item Started migrating existing material into the revised Kiwix case study.
    \item Added a couple of terms to the glossary, ongoing work on the Kiwix case study.
    \item Added some notes on the Android WebView that need to be expanded shortly.
    \item Another couple of hours spent on the Kiwix case study.
    \item As above.
    \item Added material to the introduction to the case studies.
    \item Also added material to the Kiwix case study and spent time tracking down relevant results.
    \item Ditto. Looks like the thesis needs lots more time to complete than previously estimated.
    \item Added screenshots and explanation of why Vitals Scraper was a relevant and necessary part of the research.
    \item Torn on how much of the evaluation to leave in the Evaluation chapter vs migrating the contents to the relevant case study (here Kiwix).
    \item Ongoing work on the Kiwix case study.
    \item Added more material, checked the current state of Kiwix and WikiMed in GPC and Android Vitals and written up the Kiwix aspect.
    \item Doing some investigation on a Huawei Android device and its WebView.
    \item Spent several hours learning about Huawei's ecosystem in terms of their app store, ...
    \item Analytics and crash reporting, ...
    \item Devices and OS replacements for many of their current devices, ...
    \item And also wrote some background notes about experimenting with WebViews, which inadvertently started the investigations into Huawei's ecosystem. I've a good 20+ tabs still open on Huawei, however it's soon time to wrap up this side-channel for the main research.
    \item Successfully signed up as an organisation for Huawei's development platform as part of learning about their relevant services and SDKs.
    \item Further research and note taking on Huawei's crash reporting, etc.
    \item Wrapping up the Huawei topic for now, also added some notes on trusting the custodians with examples from Google and Huawei.
    \item Following old trails on crash reporting including Google's first implementation in Android Market. See~\href{{chapter-crash-recording-and-reporting-in-android}}{\nameref{{chapter-crash-recording-and-reporting-in-android}}}.
    \item Around 4 to 5 hours spent working on the Kiwix-Android case study. Writing about the mini case-studies, reading various old emails and related material.
    \item See above - more work on the mini case-studies. Reading about CrashScope
    \item Tried CrashScope again, also sent another message to one of the project owners in the hope they're able to provide access to a suitable, working instance.
    \item Moved post case-study material from Kiwix to the discussion chapter, added several notes.
    \item Added a so-what to the end of the mini case studies for Kiwix. Trimmed one of the screenshots.
    \item Friday \nth{2} July 2021 was mainly spent working on the Kiwix case study.
    \item Working in parallel with Arosha adding comments and notes to address them...
    \item Did some more background research of prior materials.
    \item It's about time to change focus from the Kiwix case study which now has many of the gaps replaced by content. There are still 3 `MUST-DO' comments to address, however I've agreed with my supervisors to:
    \begin{itemize}
        \item park the Kiwix case study for a while,
        \item reduce duplication in the rest of the case studies - instead put a brief note to explain where the additional case study confirms work described in previous case studies,
        \item remove some of the supporting case studies if they don't contribute materially to answering my research questions, and
        \item to focus on my Evaluation chapter where I need to evaluate my research questions in light of the evidence I've provided in my current case studies.
        \item \textbf{NB} there are still various improvements needed in the Kiwix case study so I'll need to revisit it later on.
    \end{itemize}
    \item Reading and digesting one of the many key papers from Microsoft~\citep{buse2012_information_needs_for_software_development_analytics}. Making some notes in the related works chapter as I progress.
    \item The previous paper led to three more, including two highly relevant ones on \href{glossary-wer}{WER}. More reading, note taking and writing.
    \item ditto. Been great to learn more about how WER works and the effects of using it.
    \item Spent another hour reading and making notes about~\citep{kinshuman2009_debugging_in_the_very_large}. Time to wrap up for the evening.
    \item Compared WER with Android Vitals and created a table for the comparisons.
    \item Added material on functional correctness and segment.io's request for fakes for developer-oriented testing. Wrote notes on debugging in the large after reading the relevant Microsoft research papers on the topic.
    \item Filled in the section on fail early. Provided notes on error reporting with an example from Microsoft App Center.
    \item Filled in the section on RxJava. Also esplained about the effects of app stores assessing apps and their developers in terms of fail early.
    \item Migrated and expanded material on approaches to reduce failure rates from my rejected ICST 2020 Industry Paper. (This a contentious topic for some in research). Also mentioned game theory to help readers understand why there's more to life than developers doing what seems to be good software hygiene. 
    \item Started rewording the Introduction chapter to reduce the use of first person and to revise it to reflect the research being presented in my thesis. Added 7 references, one not yet referenced (UiLog).
    \begin{itemize}
        \item Outside the work on the thesis I also spent about an hour reviewing App Center data for a large commercial Android app.
    \end{itemize}
    \item Started revising the evaluation chapter. Moved various content around accordingly. Discovered I've two chapters with challenges to validity so need to merge the contents.
    \begin{itemize}
        \item Weekly call with first Arosha, then Yijun: Fri \nth{9} July 2021. Arosha will review the Introduction, Yijun will create a table of the RQs and how they're answered in the case studies, and how the Discussion chapter intersects with the RQs. Meanwhile I'll continue working on the evaluation chaper and and the RQs.
    \end{itemize}
    \item Expanding the coverage of heatmapping, spent some time making sense of one of the papers I discovered. Spidering several topics via~\url{https://www.connectedpapers.com/search?q=mobile\%20analytics}.
    \item Wrote notes of my albeit limited forrays into heatmapping.
    \item Spent quite a long time finding out about the current state of heatmapping software and various analytics providers who claim to provide it or equivalent offerings.
    \item Writing about Apteligent, thankfully found some of their reports were available on the Internet Archive.
    \item Some updates from the recent industry project where the crash and ANR rates have increased for various reasons. Further reading and writing on heatmapping and Apteligent's findings particularly where it intersected with the recent industry case study on networking IO.
    \item Worked on the introduction to help establish why Stability/Reliability was chosen as the area for the research. This involved reading what Apple's App Store says on the topic and checking the policies of Google Android and Huawei's AppGallery.
\end{enumerate}
