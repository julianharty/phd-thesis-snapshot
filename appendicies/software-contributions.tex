\chapter{Software Contributions}\label{software-contributions-chapter}
This research is not intended to live in a vacuum! One of the aims throughout my research has been to contribute potentially useful software and materials and make it freely available for whoever wishes to use and improve it. This chapter itemises various projects on GitHub that I contributed to. Many of the projects I instigated and often co-developed them with Joseph Reeve a friend and part-time colleague on an occasional basis. He is a prolific, fast and responsive software developer (see~\url{https://github.com/isnit0} for his activities on GitHub) and enabled me to expand the research significantly rather than relying on my limited software development skills. 

\section{Co-developed GitHub projects}
\begin{itemize}
    \item AndroidCrashDummy\\ \url{https://github.com/ISNIT0/AndroidCrashDummy}\\ A small Android app to test logging and exceptions.
    \item AndroidLogAssert\\ \url{https://github.com/ISNIT0/AndroidLogAssert}\\ A library to facilitate and test log messages for Android apps.
    \item Android Monkey Test with Login \\ \url{https://github.com/commercetest/android-monkey-test-with-login}\\ An experiment to help increase the effectiveness of Android Monkey when it needs to interact with an Android app that uses a login page (since monkeys seldom enter a valid username and password. (Android Monkey is a very popular test automation tool used both by developers and researchers.) 
    \item Android Stability Analysis\\ \url{https://github.com/commercetest/android-stability-analysis}\\ A utility script to pattern match clusters of errors reported in Android Vitals. Android Vitals does not group clusters completely which leads to flaws in their rankings of the errors. This scripts helps teams identify common clusters, the totals for these clusters can then be recalculated. The recalculated totals then enable a corrected league table to be produced. Teams can then choose the order to address reported errors based on the corrected ranking rather than the flawed ranking Google generates.
    \item (Android) Vitals Scraper\\ \url{https://github.com/commercetest/vitals-scraper}\\ one of the major contributions of my research. Released as an \texttt{npm} package \url{https://www.npmjs.com/package/vitals-scraper}. The npm site reports seven downloads per week, on average.
    \item Log Searcher \url{https://github.com/ISNIT0/log-searcher} A tool for searching Android codebases and analysing usage of \texttt{``Log.*"}. Includes a discussion in the README on the rich potential of designing and using logging.
    \item Log-complexity-analysis\\ \url{https://github.com/ISNIT0/log-complexity-comparison}\\ An experimental early-stage project to try and identify complex code that lacks logging.
    \item Logcat-filter and analysis tool\\ \url{https://github.com/ISNIT0/logcat-filter}\\ A utility that runs on a developer's computer to filter and analyse log messages for specified Android apps. Generates a JSON format output on request to facilitate additional processing.
    
    % \item Goldfinch Android \emph{[private]} \url{https://github.com/commercetest/goldfinch-android}
    
    \item Zipternet\\ \url{https://github.com/ISNIT0/zipternet}\\ A small Android application generator, written in Kotlin, to provide a basic app with sufficient functionality to be potentially realistic and useful. It is similar in concept to the custom apps created and maintained by the Kiwix Android project team.
\end{itemize}

\section{Solo projects}
\begin{itemize}
    \item Android analytics testing\\ \url{https://github.com/commercetest/android-analytics-testing}\\ Notes to help prepare for testing of analytics for Android devices. The notes may also suit testing of analytics on other platforms.

    \item Software quality hackathons\\ \url{https://github.com/commercetest/software-quality-hackathons}\\ An introductory set of notes to help run a software quality hackathon. Created to support the hackathon with the Catrobat team in Graz, Austria in November 2019. 
    \item GPC Reports Analysis\\ \url{https://github.com/julianharty/gpc-report-analysis}\\ Various small scripts written in \texttt{R} to analyse various Google Play Monthly reports.
    
    \item Testing with analytics workshop\\ \url{https://github.com/julianharty/testing-with-analytics-workshop}\\ Material to support my workshop presented at the test:fest 2020 conference in Poland on \nth{28} February 2020.
\end{itemize}

\section{Contributions to external projects}
Contributions to external projects include:
\begin{itemize}
    \item Analytics Testing [using] Puppeteer\\ \url{https://github.com/dumkydewilde/analytics-testing-puppeteer}\\ Simply created a small README for the project. The project automates testing of Google Web Analytics.
    
    \item EduVPN\\
    \url{https://github.com/eduvpn/android}\\
    I created a proof-of-concept Continuous Build for the project \url{https://github.com/commercetest/android}, however the underlying projects were particularly complicated to build and problematic so the work has stalled pending decisions on the direction for the overall codebase.
    
    \item Pocket Code (Android)\\
    \url{https://github.com/Catrobat/Catroid}\\
    I work with the project team, including the developers, to help them understand the principles and tools for using analytics as part of their development and bug investigation practices. I have also been involved in helping instigate and design several small projects in parallel which are not on GitHub.
    
    \item Pocket Code (iOS)\\
    \url{https://github.com/Catrobat/Catty}\\
    Worked with the lead developer to design and implement crash and mobile analytics using Firebase.
    
    \item Kiwix Android\\
    \url{https://github.com/kiwix/kiwix-android/}\\
    The first of the case studies. I have been involved with this project since 2014.

    \item Release Engineering for Mobile Apps\\ \url{https://github.com/shonan-releng-mobile/shonan-releng-mobile}\\ Contributed material on an ongoing basis for international collaborative research on this topic. 
\end{itemize}

\section{Summary of software contributions}
Software development is an essential aspect of my research, albeit one I practice on an occasional basis. I aim to be a well-behaved and effective citizen and participant working with and across various software teams. This seems to be working out OK given the many and various collaborations and contributions related to my PhD research.
