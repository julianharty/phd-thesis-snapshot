\chapter{Software Contributions}
\label{app:software-contributions}

This research is not intended to live in a vacuum! One of the aims throughout my research has been to contribute potentially useful software and materials and make it freely available for whoever wishes to use and improve it. This chapter itemises various projects on GitHub that I contributed to. Many of the projects I instigated and often co-developed them with Joseph Reeve a friend and part-time colleague on an occasional basis. He is a prolific, fast and responsive software developer (see~\url{https://github.com/isnit0} for his activities on GitHub) and enabled me to expand the research significantly rather than relying on my limited software development skills. 

\section{Co-developed GitHub projects}~\label{subsec:co-developed-github-projects}
\begin{itemize}
    \item AndroidCrashDummy\\ \url{https://github.com/ISNIT0/AndroidCrashDummy}\\ A small Android app to test logging and exceptions.
    
    \item AndroidLogAssert\\ \url{https://github.com/ISNIT0/AndroidLogAssert}\\ A library to facilitate and test log messages for Android apps.
    
    \item Android Monkey Test with Login \\ \url{https://github.com/commercetest/android-monkey-test-with-login}\\ An experiment to help increase the effectiveness of Android Monkey when it needs to interact with an Android app that uses a login page (since monkeys seldom enter a valid username and password. (Android Monkey is a very popular test automation tool used both by developers and researchers.) 
    
    \item Android Stability Analysis\\ \url{https://github.com/commercetest/android-stability-analysis}\\ A utility script to pattern match clusters of errors reported in Android Vitals. Android Vitals does not group clusters completely which leads to flaws in their rankings of the errors. This scripts helps teams identify common clusters, the totals for these clusters can then be recalculated. The recalculated totals then enable a corrected league table to be produced. Teams can then choose the order to address reported errors based on the corrected ranking rather than the flawed ranking Google generates.
    
    \item (Android) Vitals Scraper\\ \url{https://github.com/commercetest/vitals-scraper}\\ one of the major contributions of my research. Released as an \texttt{npm} package \url{https://www.npmjs.com/package/vitals-scraper}. The npm site reports seven downloads per week, on average.
    
    \item IDot (The source code is available on request at \url{https://github.com/commercetest/IDot/}; it currently includes some private configuration details so is not yet opensourced). It is based on AndroidCrashDummy and intended to facilitate testing and evaluation of mobile analytics services. 
    
    \item Log Searcher \url{https://github.com/ISNIT0/log-searcher} A tool for searching Android codebases and analysing usage of \texttt{``Log.*"}. Includes a discussion in the README on the rich potential of designing and using logging.
    
    \item Log-complexity-analysis\\ \url{https://github.com/ISNIT0/log-complexity-comparison}\\ An experimental early-stage project to try and identify complex code that lacks logging.
    
    \item Logcat-filter and analysis tool\\ \url{https://github.com/ISNIT0/logcat-filter}\\ A utility that runs on a developer's computer to filter and analyse log messages for specified Android apps. Generates a JSON format output on request to facilitate additional processing.
    
    \item RACHEL Status app \url{https://github.com/ISNIT0/RachelStatusApp} A very simple CORDOVA app used to create an internal test release in Google Play Console. 
    
    % \item Goldfinch Android \emph{[private]} \url{https://github.com/commercetest/goldfinch-android}
    
    \item Zipternet\\ \url{https://github.com/ISNIT0/zipternet}\\ A small Android application generator, written in Kotlin, to provide a basic app with sufficient functionality to be potentially realistic and useful. It is similar in concept to the custom apps created and maintained by the Kiwix Android project team.
\end{itemize}

\begin{comment}
RACHEL Status app uses Cordova. Needed updating. Also needs to use JAVA8 compiler, see \url{https://kodejava.org/how-do-i-set-the-default-java-jdk-version-on-mac-os-x/} to set this for the CLI.
\end{comment}

\subsection{Android Stability Analysis}~\label{subsec:android-stability-analysis}
In August 2019 we started another opensource project \href{https://github.com/commercetest/android-stability-analysis}{android-stability-analysis}~\citep{android-stability-analysis} to enable scattered crash clusters reported in Android Vitals to be regrouped. Currently, the analysis tool simply matches reported crash clusters with a series of lines that should be common to related crashes, see Listing \ref{listing:codelines}  for an example.

\begin{lstlisting}[caption=Example of lines to match,label=listing:codelines]
at org.kiwix.kiwixmobile.downloader.
 DownloadService.pauseDownload(DownloadService.java:266)
 at org.kiwix.kiwixmobile.downloader.
    DownloadFragment$DownloadAdapter.
    setPlayState(DownloadFragment.java:227)
 at org.kiwix.kiwixmobile.downloader.
    DownloadFragment$DownloadAdapter.
    lambda$getView$5(DownloadFragment.java:286)
\end{lstlisting}


\section{Solo projects}~\label{sec:solo-projects}
\begin{itemize}
    \item Android analytics testing\\ \url{https://github.com/commercetest/android-analytics-testing}\\ Notes to help prepare for testing of analytics for Android devices. The notes may also suit testing of analytics on other platforms.

    \item Software quality hackathons\\ \url{https://github.com/commercetest/software-quality-hackathons}\\ An introductory set of notes to help run a software quality hackathon. Created to support the hackathon with the Catrobat team in Graz, Austria in November 2019. 
    \item GPC Reports Analysis\\ \url{https://github.com/julianharty/gpc-report-analysis}\\ Various small scripts written in \texttt{R} to analyse various Google Play Monthly reports.
    
    \item Testing with analytics workshop\\ \url{https://github.com/julianharty/testing-with-analytics-workshop}\\ Material to support my workshop presented at the test:fest 2020 conference in Poland on \nth{28} February 2020.
\end{itemize}

\section{Contributions to external projects}~\label{sec:contributions-to-external-projects}
Contributions to external projects include:
\begin{itemize}
    \item Analytics Testing [using] Puppeteer\\ \url{https://github.com/dumkydewilde/analytics-testing-puppeteer}\\ Simply created a small README for the project. The project automates testing of Google Web Analytics.
    
    \item EduVPN\\
    \url{https://github.com/eduvpn/android}\\
    I created a proof-of-concept Continuous Build for the project \url{https://github.com/commercetest/android}, however the underlying projects were particularly complicated to build and problematic so the work has stalled pending decisions on the direction for the overall codebase.
    
    \item Pocket Code (Android)\\
    \url{https://github.com/Catrobat/Catroid}\\
    I work with the project team, including the developers, to help them understand the principles and tools for using analytics as part of their development and bug investigation practices. I have also been involved in helping instigate and design several small projects in parallel which are not on GitHub.
    
    \item Pocket Code (iOS)\\
    \url{https://github.com/Catrobat/Catty}\\
    Worked with the lead developer to design and implement crash and mobile analytics using Firebase.
    
    \item Kiwix Android\\
    \url{https://github.com/kiwix/kiwix-android/}\\
    The first of the case studies. I have been involved with this project since 2014.

    \item Release Engineering for Mobile Apps\\ \url{https://github.com/shonan-releng-mobile/shonan-releng-mobile}\\ Contributed material on an ongoing basis for international collaborative research on this topic. 
\end{itemize}

\section{Software we developed for Google Play Console}~\label{sec:software-we-developed-for-google-play-console}
Developers are only able to read Android Vitals reports in real-time in the user interface Google provides in Google Play Console. This limits their ability to perform trend analysis beyond the period Google allots. Also they are restricted to taking screenshots, etc. to save pertinent information. There are several limits, a hard limit of 60 (sixty) days and seemingly limits based on volumes of crash and ANR data. 

Our contributions enable them to preserve various relevant reports and also individual crash and ANR trace data by using our Vitals Scraper software. They can then also use our software to group incompletely grouped crash clusters which enables them to improve the prioritisation of their work to investigate and improve quality aspects of their apps.

The developers can also choose to make the reports and data available to others, for instance for analysis and research. Our software is extensible and freely available as permissively-licensed opensource software tools. 

\subsection{Background}
Google Play Console is a developer's view of the Google Play app store and includes data and reports for the apps the person has access to. Typically people have access to data for apps they develop and/or support.

At least some of the data is collected from users who opt-in to share their usage and diagnostics data with Google, and Google - in turn - shares some of this data with developers~\cite{google_play_share_usage_and_diagnostics_info_with_google}.

Google Play Console provides various summary data as monthly reports in a downloadable format, and these files are available historically so the data can be downloaded for previous periods. 

\subsection{Summary Data is available for download}
The monthly summary reports are available to download\footnote{\url{https://support.google.com/googleplay/android-developer/answer/6135870}} either interactively or using a command line tool \texttt{gsutil}\footnote{\url{https://support.google.com/googleplay/android-developer/?p=stats\_export}}. The reports include crashes and ANRs, ratings, reviews, installations, and financial information. 

From what we can tell the data is available from when it was first generated by Google Play Console.

Google used to provide detailed crash and ANR reports until May 2018~\cite{google_play_download_and_export_monthly_reports} before removing this facility. They also removed the ability to download these reports for prior periods which left a gap in terms of being able to analyse either of these quality issues.

\subsection{Design and Development}
\begin{itemize}
    \item Add figure from discussion with Yijun on scrapers.
    \item Explain about the use of the embedded browser and using TypeScript
    \item Explain how the software was developed and by whom
    \item Discuss some of the technical challenges
    \item Discuss some of the softer challenges and risks of Google blocking the user account and those of those it deems "related"
    \item How did we address these various challenges, and what remains to address.
\end{itemize}

\textbf{Note} Consider my materials in a draft paper on Scraper APIs.

\section{Summary of software contributions}~\label{sec:summary-of-software-contributions}
Software development is an essential aspect of my research, albeit one I practice on an occasional basis. I aim to be a well-behaved and effective citizen and participant working with and across various software teams. This seems to be working out OK given the many and various collaborations and contributions related to my PhD research.
