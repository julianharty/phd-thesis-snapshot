\section{Manifesto for Agile Software Development}~\label{sec:manifesto-for-agile-software-development}

\begin{enumerate}
    \item We are uncovering better ways of developing software by doing it and helping others do it.
    \item Through this work we have come to value:

    \item Individuals and interactions over processes and tools
    \item Working software over comprehensive documentation
    \item Customer collaboration over contract negotiation
    \item Responding to change over following a plan

    \item That is, while there is value in the items on the right, we value the items on the left more.
\end{enumerate}

\begin{table}[]
    \centering
    \begin{tabular}{ccc}
       Kent Beck         & James Grenning & Robert C. Martin \\
       Mike Beedle       & Jim Highsmith  & \\
       Arie van Bennekum & Andrew Hunt    & \\
       Alistair Cockburn & Ron Jeffries   & \\
       Ward Cunningham   & Jon Kern       & \\
       Martin Fowler     & Brian Marick   & \\
    \end{tabular}
    \caption{Caption}
    \label{tab:agile-manifesto-contributors}
\end{table}




Steve Mellor, 
Ken Schwaber, 
Jeff Sutherland, and 
Dave Thomas.


© 2001, the above authors

this declaration may be freely copied in any form,
but only in its entirety through this notice.

Principles behind the Agile Manifesto

We follow these principles:
\begin{enumerate}
    \item Our highest priority is to satisfy the customer through early and continuous delivery of valuable software.
    \item Welcome changing requirements, even late in development. Agile processes harness change for the customer's competitive advantage.
    \item Deliver working software frequently, from a couple of weeks to a couple of months, with a preference to the shorter timescale.
    \item Business people and developers must work together daily throughout the project.
    \item Build projects around motivated individuals. Give them the environment and support they need, and trust them to get the job done.
    \item The most efficient and effective method of conveying information to and within a development team is face-to-face conversation.
    \item Working software is the primary measure of progress.
    \item Agile processes promote sustainable development. The sponsors, developers, and users should be able to maintain a constant pace indefinitely.
    \item Continuous attention to technical excellence and good design enhances agility.
    \item Simplicity--the art of maximizing the amount of work not done--is essential.
    \item The best architectures, requirements, and designs emerge from self-organizing teams.
    \item At regular intervals, the team reflects on how to become more effective, then tunes and adjusts its behavior accordingly.
\end{enumerate}
