\chapter{Commentary on my PhD experience}
I don't yet know whether any of the following will appear in my thesis, nonetheless I've decided it's worth writing down various thoughts and notes related to \textit{doing} my PhD.

\section{Why this PhD?}
I started out with several motivations, these included:
\begin{itemize}
    \item Alignment with my consultancy work in testing of mobile apps.
    \item Helping improve mobile apps, including better and more fullfilling testing of those apps.
    \item Providing a vessel to collaborate with Arosha Bandara in particular, and various others in academia, particularly at the Open University.
    \item The challenge of such a major, long-term undertaking - would I stay the course and reach a successful outcome?
    \item Using the PhD to drive myself to provide better evidence for my ideas, etc.
    \item And, finally, to a lesser extent, it'd be nice to have the kudos of receiving a PhD. Also, when I started, Elena was still at Oxford on her combined Master's Degree course and I wanted to see if I could beat her to a PhD! As she decided to finish her academic studies then there's only me in this non-existent race... and I've taken far longer than I'd hoped or dreamed I'd do, so never mind...
\end{itemize}

Taking stock in early May 2019, I'm still motivated to complete my PhD, partly because to quit now would mean defeat and to a certain extent that I'd have wasted so much time and mental energy to little effect. I realise there's still a great deal of work to do, both in the primary research and in writing, nonetheless I'm getting more confident I'll be able to write an adequate thesis \& some of the materials and results my underpin a revised edition of my Mobile Analytics Playbook, which would be useful.

\section{My Research Aims and Objectives}
Effective research needs to address aims and objectives, initially set by the originator(s) of the research. These originators may include the authors, funders, academic supervisors and heads, \emph{etc.}

My aims and objectives for my PhD research include:
\begin{itemize}
    \item Sharing key ideas with relevant, supporting, and credible evidence.
    \item Making much of the work usable by others, including the concepts, tools and techniques I discovered and used.
    \item TBC?
\end{itemize}

\subsection{Enabling reusability of my work}

\begin{itemize}
    \item \href{https://opensource.com/article/19/9/how-open-source-academic-work}{"How to open source your academic work in 7 steps"}
    \item Reproducible Research with R and RStudio~\cite{gandrud2020reproducible}.
\end{itemize}


\section{My journey}
\begin{itemize}
    \item Dreams and the literature review (2014). Wrote Analytics chapters for 2 collaborative books: The Mobile Developer's Guide to the Galaxy and the "Mobile Developer's Guide to the Parallel Universe, 3rd Edition".
    \item Ongoing presentations at industry events, workshops, etc.
    \item Distractions and alternative topics (later discounted).
    \item Writing "The Mobile Analytics Playbook" with Antoine Aymer and HPE. 5 000 copies printed + online distribution. Various updates to the Mobile Developer's Guide to the Galaxy collaborative book (ISBN 978-0-9970694-0-2 PDF edition).
    \item RatingReviews Android app (not finished or published).
    \item 2017, Co-author on Mobile Twin Peaks paper \url{http://oro.open.ac.uk/48833/1/mobile-twin-peaks.pdf}.
    \item KiwixAndroid false starts.
    \item Android Vitals.
    \item Goldfinch Android app, early forays into Integrating Mobile Analytics.
    \item Authored short paper on Android Vitals for MOBILESoft 2019.
    \item Coauthored short paper for WAMA 2019 and presented the paper at the workshop.
    \item Wrote full paper for A-Mobile 2019 which was rejected as the work was still premature. Encouraged to continue. 
    \item Kiwix Hackathon, Stockholm, August 2019: Investigated ways to create the custom apps reliably (the process was ill documented and broken). Also worked with developers to create a custom bug-fix release (2.5.3). This release addressed 2 of the most frequent crashes.
    \item October 2019 instigated the design competition for the proposed dictionary powered by Kiwix app on \url{https://99designs.ch}
    \item October 2019: wrote a full paper for the Industry Track at ICST2020.
    \item \nth{23} October 2019 started incorporating some of the materials from my ICST2020 paper in my thesis.
    \item \nth{4} November 2019: read 2 papers related to academic reading and writing this weekend. Worked hard on addressing various run-time issues of vitals-scraper and managed to successfully collect overview results for but my RACHEL status app (Joe's probably fixed the problem) and also crash clusters for Kiwix apps and Catroid apps. Worked remotely with Joe today. Documented ideas on improving the interactions with the app being scraped in issue \#46 (\url{https://github.com/commercetest/vitals-scraper/issues/46}).
    \item \nth{13} November 2019: Prague, Czech Republic. I did \emph{not} present at EuroSTAR 2019 as I was sick. Nonetheless the topic was \emph{"The Wise \& Foolish Builders - A Critical Approach to Using Analytics for Testing"} and there was significant interest in the topic from participants, exhibitors and other speakers. The organisers found a substitute speaker (who did a good job of presenting their work) and we've agreed we'd like me to present the work as a webinar in a few months. Note: I started drafting a researched whitepaper to support this presentation and continue to contribute to it.
    \item November 2019: Started contributing to various \texttt{eduVPN} opensource projects focusing on creating Continuous Integration (CI) and suitable automated tests for the various sub-projects. 
    \item Reviewed part of a chapter on Optimization for Guenther Ruhe (the keynote speaker at WAMA 2019). He asked if he could acknowledge my contributions in the chapter/book which I agreed to.
    \item November 2019: \nth{3} author for HUCAPP 2020 Position Paper on TX - Lived Experience for Software Testers using Tools during Software Testing. My contributes were both historical (I helped initiate the research topic and encouraged the primary author to embark on her PhD as well as reviewing drafts, making various modest contributions, and assisting with the preparation especially using Latex.
    \item \nth{16} November 2019: Graz, Austria. I presented a revised edition of my proposed EuroSTAR material to around 15 - 20 people at Graz University of Technology\footnote{\url{https://www.tugraz.at/en/home/}} in the morning. The audience was a mix of students and post-graduate students and staff from the university who all contribute or contributed to the Catrobat project. The session overran by about 60\% due to the interesting discussions. We then went to the Graz office of Dynatrace to meet one of the employees. During an informal discussion on testing analytics 2/3rds of the entire staff joined in as audience to ask questions, etc. There's clearly lots of interest in the topic. We learned a bit about their product offering, had a beer with them, and left with the encouragement to try out their software. They also are considering me as a speaker at their 700 - 900 person conference in Austria.
    \item \nth{17} November 2019: Graz, Austria. Co-led a hackathon with Joseph Reeve and professor Wolfgang Slany who leads the Catrobat project and leads the Computer Science Department. We made good progress, the number of people at the event varied from 6 (3 leads + 3) to around 10. The event ended shortly after 8pm followed by an informal dinner listening to live football between Austria and Macedonia.
    \item \nth{18} November 2019: Graz, Austria. Started doing post event analysis and write-up. Noticed yesterday's hackathon has led to developers creating a ticket and implementing code to catch 'soft-errors' within the app rather than letting them leak to Google's Android Vitals \cite{CATDROID-426-JIRA}.
    \item \nth{8} to \nth{12} December 2019: NII Shonan Center, Japan. Workshop on Release Engineering for Mobile Applications. Very useful workshop that helped me better understand research practices by a range of academics in my field. See \url{https://github.com/shonan-releng-mobile/shonan-releng-mobile} for some of the outputs of the workshop.
    \item \nth{10} December 2019: Isabel submitted the Illusions of Usability paper to ICST. I am the \nth{4} author.
    \item \nth{11} December 2019: received rejection for my ICST 2019 Industry paper. Overall the reviewers indicated I am working in an interesting and relevant topic, however the contents of the paper were unclear and I hadn't provided several key details they expected e.g. about feedback from Google. This paper primed the writing of a full paper for MobileSoft 2020 in Jan 2020.
    \item 20 Dec 2020 Isabel's HUCAPP 2020 paper was accepted, "Stuck in Limbo With Magical Solutions: The Testers' Lived Experience of Tools And Automation". I am a co-author. I helped revise the paper in the next few weeks.
    \item Jan 2020 wrote Doctoral Symposium paper for ICST 2020. This was accepted on \nth{22} Jan.
    \item Jan 2020 Prof Wolfgang Slany agreed to help support my workshop in Poland; we ended up with three of the developers coming to support and collaborate in the workshop; and our first iOS app to study.
    \item \nth{11} Jan 2020: Isabel's paper "Illusions of Usability..." was rejected for the ICST 2020 Tools Track. We revised and submitted a shorter edition for a related workshop, TAIC-PART 2020 which was accepted.
    \item \nth{20} Jan 2020: Wrote a fast abstract quickly for TAIC-PART 2020 on Data Dynamics. This was accepted.
    \item Jan 2020: Wrote most of a full paper for MobileSoft 2020. Managed to invite 2 co-authors in addition to both supervisors. However a combination of timing and lack of time meant we withdrew the paper. The work was heading in the right direction and will form part of my thesis. I wrote a mini-retrospective on the experience. 
    \item \nth{25} Jan 2020: submitted extended abstract on onboarding to ICGSE'20, partly to learn about the shepherding process they offered to authors of Industry Experience Report papers.
    \item Feb 2020: corresponded with the conference organisers on various details related to the camera-ready submission for ICST. Helped Isabel through the experience for our joint paper.
    \item \nth{12} Feb 2020: ICGSE'20 paper provisionally accepted. Actual shepherding process started about 2 weeks later with Tony Clear.
    \item \nth{15} Feb 2020: submitted Student Research Competition paper for MobileSoft 2020. Very tight constraint of 2 pages all in. They reopened submission updates which enabled me to improve the paper. It was accepted later in Feb. 
    \item late Jan, several weeks in Feb 2020: underwent a full Ethics approval process to get permission to use feedback from participants at my workshop in Poland. HREC/3497 was approved a week before the workshop.
    \item \nth{6} Feb 2020: scheduled to perform a dry-run of my workshop in Poland at Post-Graduate (PG) Forum at the Open University. However, there were too few people who attended and those who did attend were not sufficiently representative to perform any testing. So, instead we discussed the impact of flaws in analytics and then research tips as a group.
    \item Feb 2020: drafted an extended version of my fast abstract for TAIC-PART 2020 as a possible submission to the Industry Innovation Track for RE'20. Submitted the abstract on \nth{24} Feb.
    \item Feb 2020: worked with Rahul, a Master's student at TU Graz to design and implement Firebase Analytics in the Pocket Code Android app. Then also worked with the iOS development lead, Michael who implemented more complete support in the iOS app. We also migrated the Android app's crashlytics service from Google's Fabric service to Google's Firebase service. We encountered major problems with the migration of other apps in Firebase and also lost the ability to configure Firebase for new apps. Google's been informed and providing suggestions that have yet to be effective.
    \item \nth{28} Feb 2020: Testing using Analytics workshop in Wroclaw, Poland. Worries about Corona-virus by participants, volunteers and organisers of the workshop and conference meant we had under half the planned participants. And of the 12 who did participate in the workshop, only around 4 had experience testing software. The net results were that we did some useful testing and learned about crash analytics for the Pocket Code Android and iOS apps, however we didn't manage to do practical testing using analytics and therefore no-one provided feedback on this topic so the Ethics Approval turned out not to be relevant.
    \item \nth{29} Feb 2020: I gave the opening keynote on \emph{A critical approach to using analytics for testing} to around 300 people at testfest.pl~\footnote{\url{http://testfest.pl/speaker/julian-harty/}}.
    \item \nth{1} March 2020: MOBILESoft 2020 SRC formal acceptance. Details for camera-ready still not provided by \nth{11} March, Coronavirus stopped play...
    \item \nth{3} March 2020: withdrew the paper for RE'20 as it was nowhere near ready. The material have garnered lots of interest and support, including with Isabel Evans' supervisors (at Malta University). 
    \item \nth{5} March 2020: ICGSE'20 paper formally accepted. However it's not clear what happens between now and the official camera-ready date of \nth{15} March 2020 AoE. Coronavirus has also affected the planning and communication for the conference!
    \item \nth{11} March 2020: Started structuring this material so I can focus on three key aspects: 1) my research questions (RQs), 2) the introduction based on the RQs, 3) the literature review based on my RQs.
    \item \nth{26} March 2020: Regained access to analytics data for localhalo after a 2 week gap. Started investigating and reporting issues to the dev team.
    \item \nth{26} March 2020: Joined the remote training for R, had issues trying to process Google's reports from Google Play Console which have a Byte Order Mark (BOM) and are formatted as UTF-16. Created the Data Sources meta-chapter so I can identify what data I have and use to support my research. This led to me experimenting with R to see if I could use it to process the Google Play Console reports.
    \item \nth{26} March 2020: Discussed the ETA with Arosha and Yijun as the recent events and challenges tracking down examiners means the proposed target for the thesis of the end of May 2020 isn't viable or useful. I'll still aim to complete the thesis in the coming months with the aim of being examined in early Autumn 2020.
    \item \nth{27} March 2020: So much has happened in the last few weeks. Mum died on \nth{15}, I've probably had coronavirus where I wasn't up to writing here at all, despite planning to spend the last 10 days ish writing furiously, etc.
    \item \nth{27} March 2020: Managed to work out how to load the Google Play Console CSV files into R. Now I need to learn how to analyse their contents using R. Started another appendix: Things I'm learning through the PhD. 

\end{itemize}

\section{Personal impediments}
I've been the main limiting factor in my progress, where I've not believed I was able to develop the requisite software (I still have severe doubts and have yet to flourish in developing apps or other software). I also spent some time mulling over whether to switch fields for my PhD, into research into technology to augment teaching and learning. 

I've also been easy to distract, there are times when my heart wasn't in the PhD. Studying remotely and part-time has been a struggle. For several years I didn't explain to Jessica what I was doing (in terms of studying a PhD at all) so didn't really have much emotional support either.

Also, I've ended up taking various suspensions, two because of deaths in my close family.

There are lesser impediments too, e.g. my weakness in analytical techniques for data. 

\section{Practical Impediments}
Practically, I've been hindered by not finding a project team and organisation who are keen to embrace my ideas and the effects they'd have on their app or their working practices.

Obtaining data from Google Dev Console has been an ongoing challenge, especially as they withdrew access to some of the more detailed data. A friend and I created a web page scraper (software that navigates web pages and extracts content) which needed to be updated periodically to cope with changes and bugs in the Google web site \url{https://github.com/commercetest/vitals-scraper/commits/master}

\section{Indications of progress}
Perhaps it'll be interesting for me at least to track aspects of my progress such as my word and page counts?

\begin{table}[htpb]
    \centering
    \footnotesize
    \begin{tabular}{r|r|l|p{7cm}}
     Words &Pages  &On &Remarks\\
         \hline
         7K7 &49   & 20-Jun-2019 & \\
         10K7 &nn &02-Sep-2019 & \\
         11K1 &>70 &23-Sep-2019 & \\
         12K6 &83 &23-Oct-2019 & \\
         13K1 &88 &04-Nov-2019 & \\
         14K0 &91 &17-Nov-2019 &Excl. \textasciitilde 800 words for EuroSTAR 2019 paper.\\
         14K0 &89 &17-Nov-2019 &Removed a \texttt{\textbackslash cleardoublepage}. \\
         14K7 &93 &18-Nov-2019 &Writing up hackathon and install testing. \\
         15K3 &95 &25-Nov-2019 &Adding some more notes for the hackathon. \\
         15K9 &97 &01-Dec-2019 &Completed write-up of the JIRA tickets PocketCode hackathon. \\
         15K9 &97 &11-Dec-2019 &ICST 2020 Industry Track paper rejected. \\
         15K9 &97 &20-Dec-2019 &HUCAPP 2020 Isabel's paper accepted, I'm \nth{2} author. \\
         16K0 &99 &29-Dec-2019 &Added fieldstones; also drafting PhD symposium for ICST 2020. \\
         16K4 &100 &30-Dec-2019 &Further fieldstones; added the following sections. Working on my PhD symposium paper for ICST 2020 \\
    \end{tabular}
    \caption{Progress on this thesis in 2019}
    \label{tab:my_progress_on_this_thesis_in_2019}
\end{table}
    
\begin{table}[htpb]
    \centering
    \footnotesize
    \begin{tabular}{r|r|l|p{7cm}}
     Words &Pages  &On &Remarks\\
         \hline
         17K2 &102 &03-Mar-2020 &I've done lots of writing in various papers, not much here until today. Rough totals are 17K2 + 0K8 (EuroStar paper) + 3K1 for the onboarding paper (so far) + 2K for RE'20 + 1K1 for SRC MobileSoft + 6K5 for the full MobileSoft paper + 1K6 for ICST PhD Symposium + 7K3 for my full ICST paper + 1K6 for my Fast Abstract TAIC-PART + other draft writing = >41K2 words!\\
         17K2 &104 &07-Mar-2020 &I've rebased the wordcounts to only include the PhD as I'm adding more details below of other writings instead. \textbf{NB} words in tables seem to be excluded fron the word-counts by Overleaf.\\
         17K3 &106 &09-Mar-2020 &I added a new section in the fieldstones chapter. I've spent much of my time reading various papers (6 or so) and various chapters related to research, thesis and writing a thesis. I've also worked on revising and fleshing out my onboarding paper.\\
         12K9 &80 &24-May-2020 &My first update here for a long time. I temporarily removed lots of materials so I could focus on the early chapters. That said I ended up writing several more draft chapters instead. So much has happened in the interim too; the COVID-19 crisis and lockdown, the death and funeral of my mother, etc. etc. Anyway, it's time to try and get this thesis written! Onwards...\\
         14K0 &86 &24-May-2020 &Summary at end of the writing day: wrote additional notes and started to fill in some of the skeleton structures. Started the conclusion chapter. Also moved material around, the thesis is starting to finally emerge in terms of structure and content, albeit it's still a murky swamp of materials. I managed to remove all latex errors and warnings too! :)\\
         14K8 &99 &26-May-2020 &Added several figures and some additional stuff from my unfinished paper originally for MOBILESoft 2020.\\
         16K0 &111 &29-May-2020 &(@14:20) Adding introductions and summaries to various chapters together with more material in various chapters. The shape continues to form, still much to do!\\
         16K9 &115 &01-Jun-2020 &Revised the introduction based on feedback from Yijun Yu. Plenty more to do, as ever.\\
         
    \end{tabular}
    \caption{My progress on this thesis in 2020}
    \label{tab:my_progress_on_this_thesis_2020}
\end{table}

I wrote / co-wrote various papers and similar documents about my research. Some were accepted, others rejected. Here's as complete a list as I can recall. TODO complete this list, add what I learned and applied from each.

\begin{table}[htpb]
    \centering
    \footnotesize
    \begin{tabular}{p{4.3cm}|p{1.9cm}|r|p{4.7cm}}
     Title &Venue &Words &Remarks\\
     
     \hline
     Improving Mobile Apps Using Embedded Analytics &TAIC\_PART 2013 &1644 &Co-author akbandara \\
     
     \hline
     Towards Analytics-Driven Automated Test Generation for Mobile Applications &MobiCOM 2017 &3474 &Co-authored with akbandara and yijunyu \\
     
     \hline
     Google Play Console: Insightful Development using Android Vitals and Pre-Launch Reports &MobileSoft 2019 &2368 &Acknowledges Google's Fergus Hurley.\\
     
     Mobile Application Testing and Quality Improvement using Software Analytics &OU PhD Conference 2019 &507 &A short 1-page abstract.\\
     
     Mobile Application Testing and Quality Improvement using Software Analytics &OU PhD Conference 2019 &839 &Another format for the CRC 2019 conference.\\
     
     Symbiosis between Google Play Console and Testing Android Apps &(none yet) &2424 &Fairly complete in terms of my draft papers.\\
     
     Quality Analysis Using App-Centric Usage Data &A-Mobile 2019 &3679 &Draft, unsubmitted alternate paper. Some overlap with others.\\
     
     Hives, Guards, Trust, algorithms, decisions &A-Mobile 2019 &375 &Another alternate, unsubmitted paper.\\
     
     Better Android Apps using Android Vitals &A-Mobile 2019 &4392 &Co-written with Matthias Mueller.\\
     
     Report for Google Engineering Team &N/A &TBA &30 page report for Google Engineering team for Google Play Console and Android Vitals to help them understand and diagnose various issues I discovered.\\
     
     The Wise and Foolish Builders, A Critical Approach to Using Analytics for Testing &EuroSTAR 2019 & &Unpublished, count excluded as testfest.pl is a more recent edition of this.\\
     
     Testing Veracity and Fidelity of Software Analytics &ICST 2020 &7278 &Rejected, insufficiently clear to be accepted. Useful feedback. (Industry Track.)\\
     
     Various material for NII Shonan Meeting. &NII Shonan & &Co-written with various participants.\\

    \end{tabular}
    \caption{Written materials to end of 2019}
    \label{tab:written_materials_to_end_2019}
\end{table}     
\begin{table}[htpb]
    \centering
    \footnotesize
    \begin{tabular}{p{4.3cm}|p{1.9cm}|r|p{4.7cm}}
     Title &Venue &Words &Remarks\\     
     \hline
     How Can Software Testing be Improved by Analytics to Deliver Better Apps? &ICST-2020 &1664 &Doctoral Symposium.\\ 
     
     Stuck In Limbo With Magical Solutions: The Testers’ Lived Experiences of Tools and Automation &HUCCAP 2020 &4777 &Position paper, I was \nth{4} author. Co-written with: Isabel Evans, Chris Porter, Mark Micallef.\\
     
     Test Tools: an illusion of usability? &TAIC-PART 2020 &6132 &Co-authored with: Isabel Evans, Chris Porter, Mark Micallef. I was \nth{4} author.\\
     
     Fast Abstract: Data Dynamics for Testing Systems &TAIC-PART 2020 &1627 &Written quickly :) Further developed in the draft RE'20 paper.\\
     
     Can Quality be Measured with Fidelity? An Empirical Evaluation of Using Software Analytics for Android Apps &MobileSoft 2020 &6447 &Withdrawn as it had flaws, ran out of time. Co-authors: Arosha K. Bandara, Yijun Yu, Li Li, Wolfgang Slany\\
     
     Requirements for Data Dynamics in Business Software Systems (Industry Innovation Track) &RE'20 &2021 &Withdrawn as the contents were far from complete and this was a speculative article at this stage.\\
     
     Improving App Quality Despite Flawed Mobile Analytics (Student Research Competition) &MobileSoft 2020 &1145 &Accepted, not yet revised for camera-ready edition.\\
     
     A Critical Approach to Using Analytics for Testing &\href{http://testfest.pl/}{testfest.pl} 2020 &855 &Unpublished paper, supports my keynote presentation to 300+ people in Wroclaw, Poland. It's based on what I wrote for EuroSTAR 2019.\\
     
     Designing Engineering Onboarding for 60+ Nationalities &ICGSE'20 &3864 &Not directly related to my thesis. Shepherded by Tony Clear. \\
     
     Polish Workshop: Testing Using Analytics &(none yet) &765 &An early work-in-progress.\\

    \end{tabular}
    \caption{Written materials in 2020}
    \label{tab:written_materials_in_2020}
\end{table}

\begin{table}[htpb]
    \centering
    \footnotesize
    \begin{tabular}{r|r|l|l}
    Project &Commits  &When &Remarks\\
     \hline
    \href{https://github.com/commercetest/vitals-scraper/}{Vitals-Scraper} &72 &2019 &Co-author Joe Reeve \\
    
    \end{tabular}
    \caption{Software Projects}
    \label{tab:software}
\end{table}
