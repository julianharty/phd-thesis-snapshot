\chapter{Learning R}
The R language, and the popular IDE R Studio, have provided a productive environment for data analysis and graphing. In terms of my research, they are part of my objective to share the code and analysis in conjunction with the thesis and the underlying data.

\section{Learning R}
Automated Data Collection with R~\cite{munzert2014automated} is an interesting, relevant book with several useful examples\footnote{\url{http://www.r-datacollection.com/}}. However the world has moved on since the book was written and published so some of the examples need updating and enhancing.

Other resources include the RStudio cheatsheets~\footnote{\url{https://rstudio.com/resources/cheatsheets/}} and online courses on the datacamp.com website~\footnote{\url{https://learn.datacamp.com/}}.

\section{Using R for reproducible research}
An aptly titled book, \emph{Reproducible research with R and R studio}, now in its third edition~\cite{gandrud2020reproducible} provides a useful model for my research and analysis. Accordingly, the source code for my analysis using R is available in an opensource project on GitHub \url{https://github.com/julianharty/gpc-report-analysis}. Publishing the code facilitates reproducibility and enables others to use, critique, and improve the code.

\begin{lstlisting}[caption=Parsing dates from Wikipedia content,label=listing:parse_dates]
> yend_clean <- str_extract_all(danger_table$yend, "^[[:digit:]]{4}")
> danger_table\$yend <- as.numeric(yend_clean)
> danger_table$yend

  [1] 2001 1992 2013 2013 2013 2013 2016 2016 2016 2003 1986 2014
 [13] 2005 2013 2003 2013 1993 2012 1984 2015 2017 2016 2017 2000 
 [25] 2019 1997 2018 2012 1997 2002 2006 1992 2016 2007 1997 1982
 [37] 2015 2016 2016 2014 2015 2010 1996 2016 1999 2007 2014 2013
 [49] 2012 2012 2010 2011 1994
\end{lstlisting}

\section{Quality Control Tools: Implemented in R}
There are various visually-oriented quality tools, including Ishikawa diagrams. The \texttt{SixSigma} package can be used to create these diagrams using R code, as discussed in an interesting article \emph{7 Basic Quality Tools with R}~\cite{7_basic_quality_tools_with_R}.

Another relevant reference is \emph{Quality Control in R}~\cite{quality_control_in_R_book}. Chapter 3 \emph{The Seven Quality Control Tools in a Nutshell: R and ISO Approaches} discusses these same tools in greater depth and rigour. The authors also provide a website with an R package~\url{http://www.qualitycontrolwithr.com/package.html} and their example appears to predate and inspire the one references in~\cite{7_basic_quality_tools_with_R}.

\subsection{Ishikawa Diagrams}
As discussed in the Related Work chapter, Ishikawa diagrams can hep identify causes and effects visually so they can be considered and the issue addressed.

\begin{lstlisting}
# Import the SixSigma package
library(SixSigma)

# Specify the effect to be analyzed
b.effect <- "Delay"

# Create a vector with the names of the causes classification groups
b.groups <- c("Personnel", "Weather", "Suppliers", "Planning")

# Create a vector that contains the causes
b.causes <- c(vector(mode = "list", length = length(b.groups)))

# Create lists corresponding to the causes for each corresponding group
b.causes[1] <- list(c("Training", "Inadequate"))
b.causes[2] <- list(c("Rain", "Temperature", "Wind"))
b.causes[3] <- list(c("Materials", "Delays", "Rework"))
b.causes[4] <- list(c("Customer", "Permissions", "Errors"))

# Create the cause-and-effect diagram
ss.ceDiag(b.effect,
          b.groups,
          b.causes,
          main = "Cause-and-Effect Diagram (SixSigma package)",
          sub = "Construction Example")
\end{lstlisting}
Source code reproduced from GitHub, and \href{https://gist.github.com/rsalaza4/eff4a0a7e8e4894e4b82152fcf66e847/raw/cfbfe93c784989ee8fd6b686dd26fb88f4cec26d/Cause-and-effect\%20diagram.R}{available online}. This generates the functional diagram shown in Figure ~\ref{fig:ishikawa-example-in-R}.

\begin{figure}[ht]
    \centering
    \includegraphics[width=0.8\textwidth]{images/cause-effect-diagram.png}
    \caption{Cause and Effect diagram generated by R code}
    \label{fig:ishikawa-example-in-R}
\end{figure}

\hypertarget{pareto.diagrams.in.r}{}
\subsection{Pareto Diagrams}
Some issues have greater effects than others, and potentially those with the greatest effects are worth addressing first in order to maximise the measured improvement \emph{e.g.} in software quality.

The \texttt{qcc} R library enables Pareto figures, such as Figure~\ref{fig:pareto_chart_example}, to be generated in R~\cite{7_basic_quality_tools_with_R}. 

\begin{lstlisting}
# Import the qcc package
library(qcc)                  

# Create a vector with the number of defects per defect type
defects <- c(27, 389, 65, 9, 15, 30, 12, 109, 45, 321)            

# Create a vector with the names of the defects 
names(defects) <- c("Defect 1", "Defect 2", "Defect 3", "Defect 4",
                    "Defect 5", "Defect 6", "Defect 7", "Defect 8",
                    "Defect 9", "Defect 10")   

# Create the Pareto chart
pareto.chart(defects,
             ylab = "Frequency",
             y2lab ="Cumulative Percentage",
             main = "Pareto Chart",
             cumperc = seq(0, 100, by = 20))
\end{lstlisting}
\href{https://gist.githubusercontent.com/rsalaza4/a69615daba7c7c56290838b46cc121cc/raw/974774009f38d6af534168f2a324ce2b3c61d632/Pareto\%20chart.R}{Code Gist for Pareto chart.R}
\begin{figure}[ht]
    \centering
    \includegraphics[width=0.8\textwidth]{images/Pareto_chart_example.png}
    \caption{Pareto Chart example}
    \label{fig:pareto_chart_example}
\end{figure}

