\chapter{Research Methods}

Hybrid approach: 

\section{Analysing papers}
There are various recommendations on ways to read and analyse research papers that seem relevant to my research. These include the well established "How to Read a Paper"\cite{keshav2007read} which recommends a three-pass approach; "Doing Postgraduate Research"\cite{potter2006doing} which recommends the CARS model for both reading and writing, and The Art of Reading Research Papers

\subsection{CARS model}
As described by \cite{potter2006doing}, Creating a research space (CARS) describes three moves
\begin{enumerate}
    \item Establishing a territory: by introducing and reviewing items of previous research in the area.
    \item Establishing a niche: by indicating a gap in previous research, raising a question about it or extending previous knowledge in some way.
    \item Occupying the niche: by outlining purposes or stating the nature of the present contribution.
\end{enumerate}.
These moves can be applied when reading the abstract of papers relevant to one's research.

There are several papers with advice on how to read research papers; these include The Art of Reading Research Papers\cite{madooei_art_of_reading_research_papers}, which is undated and perhaps not peer-reviewed, and How to Read a Paper\cite{keshav2007read}. Both recommend a multi-pass approach of reading the paper in phases, each gleaning additional insights by investing more time.

\section{MVT (Minimal Viable Testing) of Analytics}
Comprehensive testing is beyond practical reach for the majority of people who will use analytics systems. There may, however, be significant value in establishing a MVT (Minimal Viable Testing) approach and using it to test and evaluate Analytics offerings. 

\emph{c.f.} Don't make me think!'s approach to usability testing~\cite{krug2000dont_make_me_think}\footnote{\url{http://sensible.com/dmmt.html}} (3 chapters were made available online on usability testing as part of the second edition of the book\footnote{\url{https://web.archive.org/web/20100609163635/sensible.com/secondedition}}) 

The importance of zeros and spaces? 