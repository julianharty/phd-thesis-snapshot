\setchapterpreamble[u]{\margintoc}
\chapter{Conclusions and Future Work}\label{chapter-conclusions-and-future-work}

\section{Conclusions}
When developers apply mobile analytics to find failures they are able to significantly improve the stability/reliability of their mobile app. When they stop paying attention and don't apply mobile analytics entropy returns eventually for a variety of reasons including new releases of the operating system, updates to third-party software used by the app, such as the Android System WebView, and flaws in ongoing software development and maintenance of the app. The developers do not need to address all the issues that are reported to effect material improvements (and indeed some issues are impractical for developers to fix in the time and resources they have available).

When developers choose to include in-app mobile analytics they generally prefer to use those mobile analytics as their primary source of information, even if the platform analytics also finds a common subset of the same failures. The meta-data collected and reported by the in-app mobile analytics combined with the availability of reports even at low usage volumes of the app are key drivers in this preference to use the in-app mobile analytics reporting. For two of the app-centric case studies the data collected by in-app mobile analytics \Glspl{sdk} was sufficient that they chose not to use in-app mobile analytics in their apps. Some projects choose to have several in-app \Glspl{sdk} embedded in their apps, often for distinct purposes.

There appears to be a general, perhaps misplaced, trust that the mobile analytics tools are accurate - both by the app developers and in the literature. Eighteen distinct flaws were found in the mobile analytics services, and there are probably many more flaws waiting to be discovered given the practical limitations placed by both industrial collaboration and the scope of a PhD. 

The field of mobile analytics evolved throughout the research period and it is likely to continue to evolve for years to come. The reporting and analysis appears to be fairly perfunctory and there may be significant scope to improve the analysis and reporting.

The human motivations that determine the use of mobile analytics are at least as interesting as the quantitative results of the effects of whatever use occurs. With the exception of the \myindex{Moonpig} development team, none of the teams used mobile analytics on an ongoing, proactive basis. 

\section{Future Work}



