\setchapterpreamble[u]{\margintoc}
\chapter{Findings: Mobile Analytics Tools and their Artefacts}

\section{Fieldstones}
\julian{These need integrating or removing pre-submission.}

An interesting phenomenon observed during the Catrobat hackathon where some of the crashes that appeared in Android Vitals were believed to come from `soft errors' in the Pocket Code app. The issue, CATROID-426, was logged during the hackathon~\sidecite{catroid_426_soft_crashes_should_not_be_reported_to_the_play_console} and the developers wrote two sets of code changes (also known as `commits'). These were merged into the app's codebase on \nth{21} Nov 2019 and released in the Pocket Code app several weeks later.

The intent was laudable, however, at least some of the soft crashes continued to occur over a month later, as documented in \url{https://jira.catrob.at/browse/CATROID-422}. This issue was raised in the hackathon and closed as a duplicate by one of the developers involved in trying to stop the soft errors from appearing in Android Vitals \sidecite{catroid_426_soft_crashes_should_not_be_reported_to_the_play_console}.

TODO Mention the Pocket Code experience when migrating from Fabric to Firebase and the additional, unexpected analytics that appeared. Forward reference to the discussion on intrusiveness.

\itools \myindex{iTools} simple facilities such as the ability to search through the failures to find any failure clusters that match. A recent example is searching for instances of an \texttt{IndexOutOfBoundsException} in the \myindex{Kiwix} custom apps\sidenote{\href{https://github.com/kiwix/kiwix-android/issues/2542}{Index Out of Bounds Exception on Custom App \#2542}} where \myindex{Android Vitals} had to be checked page by page for each app to see if the crash was still happening.

Aggregation and mining across the matching clusters would also be useful. Tagging/labelling might also help, \emph{ditto} facilities to cross-reference within and across systems (\emph{c.f.} hyperlinking and reference links.