\chapter{Tools and their artefacts}~\label{chapter-apps-and-their-artefacts}
\julian{This chapter covers \utools and \itools.}

\section{Some limits of what can be measured}

Here's a placeholder list, the points will need integrating.
\begin{itemize}
    \item React Native runtime - within runtime crashes vs. application crashes. (LocalHalo and Taskinator apps).
    \item Crashes at startup c.f. private correspondence with Google.
\end{itemize}

\section{Pre-launch reports}
The GTAF project uses pre-launch reports (an intrinsic part of Google Play Console), and the pre-launch report includes automated testing of pre-release apps. The crashes reported in pre-launch reports do not necessarily affect end users. Conversely the pre-launch report automated testing does not find all the failures that affect end users. (Dua \& Zikr app).

Why some projects stopped using pre-launch reports: c.f. the Google bug. TODO add link to the issue on Google and add supporting text.


\section{Differences between mobile analytics tools}
Pocket Code incorporated in-app mobile analytics that recorded both crashes and errors (generally these errors are exceptions that \textit{are} caught and handled by the app) the case study provided the opportunity to study Fabric Crashlytics and to enable its outputs to be compared and contrasted with those from Google Play Console with Android Vitals. \textbf{TODO} discuss the differences.


\section{Summary of tools and their artefacts}
TBC