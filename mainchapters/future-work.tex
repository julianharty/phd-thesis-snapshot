\chapter{Future work}
This PhD provides a snapshot of research I have done in recent years. This work fits within a larger body of research with similar overall objectives, to improve both the software and how that software is developed. This chapter introduces various research I am actively involved in which builds on the work described in my thesis.

\section{Evaluating Google Play Console Reports}
With discovering the many and various flaws and quirks in Google's analytics products I decided a set of experiments would help increase the trust and transparency of Android Vitals and Google Play Console. Aspects of these experiments may also be highly relevant to evaluate other analytics tools, products, and services. Some of the experiments could trigger defensive mechanisms used to 'protect' large-scale online services such as Google Play. For instance App Stores, including Google Play, are known to take protective and corrective activities related to fake reviews. Some experiments might want to assess the effectiveness of protective defenses, others may aim to avoid triggering them. 

Google is known to take punitive actions against developers whom it deems to be related to undesirable behaviours. %MUST-DO add references to examples. 
As these actions can include a lifetime ban for the account deemed to have transgressed, and for other accounts it considers are related, the effects of their actions can be far reaching and remove all revenues for an organisation's Android apps where the onus is on the people in that organisation being able to convince Google of the merits of reinstatement often without knowing what the reasons were, what the evidence was, and with no other practical way of restoring their account.

The Google Engineering team have been defensive and guarded in terms of suggestions to help evaluate the behaviours of their system. SHOULD-DO add some examples. 

A cautious set of basic tests to help establish characteristics of analytics from no (zero) to tens of users is available online at 
\url{https://joedocs.com/julianharty/evaluating-gpc-reports}

\textbf{Placeholders to extend on this topic}: Holding ourselves and each other to account. Humility, openness, comparisons with financial auditing of businesses. \emph{c.f.} Sentry.io's work, opensourcing of client libraries, etc. 

\begin{itemize}
    \item Due diligence. \emph{`Trust and Confirm')~\footnote{\url{https://www.leadergrow.com/articles/443-trust-but-verify}}}. The risks of overtrusting and not doing due diligence, especially where money is involved. The author encourages verification (as I do). Beware of end to end systems that are not verifiable by an outside party~\footnote{\url{https://www.forbes.com/sites/frankarmstrong/2019/10/21/trust-but-verify/}}.
    \item \emph{``The Only Thing Necessary for the Triumph of Evil is that Good Men Do Nothing"}~\footnote{\url{https://quoteinvestigator.com/2010/12/04/good-men-do/}}
\end{itemize}

\section{Enhancing quality vs. enhancing UX}
Success of software is multifactorial and development teams need to consider where, when and how to focus their energies to achieve satisfactory outcomes for themselves and their patrons (often their management and funders). Where does improving and achieving high-quality software fit amongst other demands on their time and energies?

\begin{figure}[ht]
    \centering
    \includegraphics[width=10cm]{images/Firebase-pocketcode-android-7-day-new-user-retention-29-may-2020.png}
    \caption{Firebase>Catrobat>New User Retention}
    \label{fig:Firebase-pocketcode-android-7-day-new-user-retention-29-may-2020}
\end{figure}

An area of my ongoing research is to evaluate the impacts of enhancing software quality in terms of stability compared to enhancing the UX of an app. The Pocket Code mobile Android app has low retention rates for new users according to Firebase, the 7 day retention of new users is illustrated in Figure \ref{fig:Firebase-pocketcode-android-7-day-new-user-retention-29-may-2020}. Both the iOS and Android Pocket Code apps have similar screens, especially the initial screens seen by new users of either app. Through our research and the focus on identifying and fixing various stability metrics we plan to focus on improving the design of the user experience for new users. One measure will be the `new user retention' report.

\section{Adding in-app analytics to PocketCode}
The Pocket Code Android app included the legacy Fabric Crashlytics analytics library for several years predating my involvement with the project (the earliest commit to add Fabric Crashlytics is on \nth{27} June 2017~\footnote{\href{https://github.com/Catrobat/Catroid/commit/95aa37ff5263402b41b63f50296aabc8c354433e}{\texttt{CAT-2420 Replace Firebase with Crashlytics}}}). This recorded both caught and uncaught exceptions but nothing else about how the app was used, or by whom. As part of my collaboration with the Catrobat team we agreed we would design and add in-app analytics in addition to the crash reporting. This work coincided with migrating from the legacy Fabric reports to the replacement Firebase reports so Firebase Analytics was selected. I helped lead the initial design and testing of the in-app analytics~\footnote{Various details are available in a Google hosted document~\url{https://docs.google.com/document/d/1cqCP2aqcx8mpTTHWe9ooGygtXh7eTvYzM3f9FbexUNM/edit?usp=sharing}}. The work has been delayed because of various impacts of the COVID-19 pandemic, the aim is to resume it as various countries and regions emerge from their respective lockdowns.

\section{Developer-centric logging in mobile apps}
My research in mobile analytics overlaps and closely aligns with research on logging - both are used by software development teams to learn about how their software behaves. I am part of a group of distributed, international researchers investigating aspects of how and why developers add logging to their mobile applications. One of our current research areas investigates the use of Google Firebase Analytics for logging. \emph{Ad-hoc} research notes are available at \url{https://joedocs.com/julianharty/apm-logging-research}.

\section{Improving the design, implementation and engineering of analytics}
\url{http://iterative.ly}
\begin{itemize}
    \item Guiding developers to a) actively use data already available to them b) design how to use optional analytics tools efficaciously 
\end{itemize}

\section{Using Mobile Analytics to improve testing of mobile apps}
% MUST-DO expand this section.
\begin{itemize}
    \item Using Mobile Analytics to improve the testing of mobile apps. This was work I had hoped to perform as part of this research, various practical constraints meant this was not practical during the PhD.
\end{itemize}

\section{Summary of Future Work}
The thesis is a snapshot of work done to date. The PhD journey and a combination of the relationships and experiences that have been established on the journey have led to various interesting and related areas of ongoing and future work which I am actively engaged in. Please get in contact if you would like to collaborate with any of these and/or with other topics that relate to my research.
