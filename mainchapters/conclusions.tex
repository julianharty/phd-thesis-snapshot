\chapter{Conclusions}
Add opening paragraph - see Isabel's comments. 

My research has shown how mobile analytics is already part of the Google Android ecosystem and assesses perceived qualities 
% At least 4 different perceptions (Isabel verbal comment)
of Android apps whether developers are aware of this or not. For developers who choose to pay attention to analytics related to these perceived qualities of their apps they can positively influence the reliability of their apps and also the scores Google assigns to their apps. 

The vast majority of mobile apps already include at least one mobile analytics library, again developers can choose whether they wish to use these analytics libraries to help them improve the quality of their apps and how they create and maintain those apps. 

The qualities of the analytics tools, perhaps unsurprisingly, also matter and are material. My research discovered numerous flaws and inconsistencies in the analytics tools Google provide. Google acknowledge some of these issues and asked for a comprehensive report so they can analyse these and fix those they deem sufficiently relevant. They state they will not provide information about their plans in terms of changes and improvements to their tools which, sadly, leaves the feedback loop incomplete in terms of the effects of the findings.

Add 3 circles 

1> incomplete circle: where Googel don't feedback what they're doing as a result of the issues I reported (into a black hole?)
2> vicious circle: if the analytics can't be trusted, the apps will get worse, and there won;t be the will in Google to improve the analytics either.
3> virtuous circle: where Google do accept the feedback and improve the analytics, etc. 

MUST-DO Discuss with Stuart about creating an influence diagram, says Isabel :) 

Both opensource project teams have changed their working practices to actively address some of the issues reported in Google Play Console and Android Vitals. Their apps are much more reliable through applying simple changes to their development practices. The Catrobat team are extending their development practices to use mobile analytics to improve their understanding of how their key Pocket Code app is performing.

My research has already led to collaborations with researchers on various projects and at various universities internationally~\footnote{In: Austria, Canada, Hong Kong, and the Netherlands currently.}, and to improving product design of an analytics infrastructure using mobile analytics for development teams with \href{https://iterative.ly/}{Iteratively}. 

%My findings have led me to devise an approach to testing analytics tools, this research is ongoing. 
There is plenty of scope for further research, analysis, design and testing of mobile analytics tools. My ongoing research areas are covered in the Future Work chapter.


