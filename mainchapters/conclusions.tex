\chapter{Conclusions}
My research has shown how mobile analytics is already part of the Google Android ecosystem and assesses perceived qualities of Android apps whether developers are aware of this or not. For developers who choose to pay attention to analytics related to these perceived qualities of their apps they can positively influence the reliability of their apps and also the scores Google assigns to their apps. 

The vast majority of mobile apps already include at least one mobile analytics library, again developers can choose whether they wish to use these analytics libraries to help them improve the quality of their apps and how they create and maintain those apps. 

The qualities of the analytics tools, perhaps unsurprisingly, also matter and are material. I discovered numerous flaws and inconsistencies in the analytics tools Google provide. Google acknowledge some of these issues and asked for a comprehensive report so they can analyse these and fix those they deem sufficiently relevant. They state they will not provide information about their plans in terms of changes and improvements to their tools which leaves the feedback loop incomplete.

My findings have led me to devise an approach to testing analytics tools, this research is ongoing. There is plenty of scope for further research, analysis, design and testing of mobile analytics tools. Some suggested areas of research include:

\begin{itemize}
    \item Using Mobile Analytics to improve the testing of mobile apps. This was work I had hoped to perform as part of this research, various practical constraints meant this was not practical during the PhD.
    \item TBC
\end{itemize}
