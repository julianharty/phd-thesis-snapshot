\chapter{Conclusions}
\label{chapter-conclusions}

%MUST-DO \akb{\nth{16} Jul 2021: Include external (non-personal) motivation for the research, or this could be part of the summary section of this chapter.}


If the mobile app and every supporting component in the system were reliable and complete we might not need or benefit from mobile analytics. Indeed they would be a distraction and wasteful. However, in the real-world every component be it software, service, system, person, or device has flaws and limitations. We are not omnipresent or omniscient. 
This raises three important questions that build from each other: 1) Could mobile analytics provide insights into a small yet vital subset of software quality? 2) Would this help development teams understand how their apps are performing in terms of their (un-) reliability? and 3) can that understanding help the developers address at least some of the underlying causes of the failures?


Pertinent information needs to flow from users' devices and reach the developers somehow in order for developers to understand the usage, performance, and behaviour of their apps. 

Developer can plan for the future operations~\footnote{\emph{i.e.} the operations within the context of DevOps.} of the app in several ways, such as embedding code within an app to record events, context, and other data. They generally use an API to do so. Some of these sources of feedback are instigated by developers using logging and/or mobile analytics APIs, others are instigated by library and/or platform providers.
%
They can also plan various processes, ~\emph{e.g.} their release, bug investigation, and triage processes. Others may choose not to plan ahead; As many have observed, including the business guru Peter Drucker,~\emph{Not making a decision is a decision}, as summed up in Truth 31 in the book by~\cite{gunther2013truth_about_better_decision_making}.

Implicit feedback mechanisms provided by various forms of usage analytics (mobile analytics) can improve developers' awareness of how the software is being used and how it is behaving.  
The feedback mechanisms are used to identify probable issues in mobile apps where the development team then choose to address at least a subset of those issues with the aim of improving the quality-in-use~\citep{bevan1999_89_quality_in_use_meeting_user_needs_for_quality} of future releases of that app. 

\hrulefill

This research was inspired by learning about the power and capabilities of applying usage data collected by mobile apps in extremely popular real-world mobile apps in the early days of mobile apps (before Android, iOS and other modern mobile platforms were released). At the time I worked for Google and was responsible for testing all Google's mobile apps, including: Search, GMail, Maps, and many others.

Several years later, when consulting with another company with over ten million active users I discovered they had included two analytics tools in their apps where there were numerous discrepancies in the data collected, the ways the counted and the resulting reports, yet they were both considered necessary. The engineers then added a third analytics library in the hope it would correlate with one or other of the existing libraries - it didn't, instead it had distinct characteristics and counts. And yet the developers were able to discover how their apps were used in incredible detail and by applying what they learned their apps became increasingly popular and financially successful. Meanwhile research emerged on insights gleaned from long term studies of usage related analytics applied to two mobile apps~\citep{patro2013_capturing_mobile_experience_in_the_wild} which increased my motivation to dig deeper into research in this fertile domain.

These experiences led me to starting my PhD in order to research the potential of mobile analytics, and to understand some of their flaws and the effects of those flaws. During my research there have been incredible changes in the mobile landscape (for instance major manufacturers, operating systems, etc. have appeared, mushroomed, and disappeared). Similarly many test automation tools and frameworks have been and gone. Meanwhile, apps and app stores have spread beyond smartphones and tablets to desktop operating systems, cloud-based product offerings such as Salesforce, etc. Google's Android platform includes platform-level data collection, reporting and analytics intended to help developers learn about ways they can improve their apps. Additionally, regulation has started to emphasise and highlight some of the many risks and concerns with gathering data wantonly. 

Despite all these changes, and my limited inroads into a subset of the entire landscape, the research seems to indicate the potential of applying usage analytics to improve both the product (the software) and the process (how the software is developed and tested). The research also identified flaws within analytics tools and also between analytics tools. Both the potential and the flaws appear worth sharing with researchers and with practitioners to help them chose and use analytics wisely.



Add opening paragraph - see Isabel's comments. 

My research has shown how mobile analytics is already part of the Google Android ecosystem and assesses perceived qualities 
% At least 4 different perceptions (Isabel verbal comment)
of Android apps whether developers are aware of this or not. For developers who choose to pay attention to analytics related to these perceived qualities of their apps they can positively influence the reliability of their apps and also the scores Google assigns to their apps. 

The vast majority of mobile apps already include at least one mobile analytics library, again developers can choose whether they wish to use these analytics libraries to help them improve the quality of their apps and how they create and maintain those apps. 

The qualities of the analytics tools, perhaps unsurprisingly, also matter and are material. My research discovered numerous flaws and inconsistencies in the analytics tools Google provide. Google acknowledge some of these issues and asked for a comprehensive report so they can analyse these and fix those they deem sufficiently relevant. They state they will not provide information about their plans in terms of changes and improvements to their tools which, sadly, leaves the feedback loop incomplete in terms of the effects of the findings.

Add 3 circles 

1> incomplete circle: where Google don't feedback what they're doing as a result of the issues I reported (into a black hole?)
2> vicious circle: if the analytics can't be trusted, the apps will get worse, and there won;t be the will in Google to improve the analytics either.
3> virtuous circle: where Google do accept the feedback and improve the analytics, etc. 

MUST-DO Discuss with Stuart about creating an influence diagram, says Isabel :) 

Both opensource project teams have changed their working practices to actively address some of the issues reported in Google Play Console and Android Vitals. Their apps are much more reliable through applying simple changes to their development practices. The Catrobat team are extending their development practices to use mobile analytics to improve their understanding of how their key Pocket Code app is performing.

My research has already led to collaborations with researchers on various projects and at various universities internationally~\footnote{In: Austria, Canada, Hong Kong, and the Netherlands currently.}, and to improving product design of an analytics infrastructure using mobile analytics for development teams with \href{https://iterative.ly/}{Iteratively}. 

%My findings have led me to devise an approach to testing analytics tools, this research is ongoing. 
There is plenty of scope for further research, analysis, design and testing of mobile analytics tools. My ongoing research areas are covered in the Future Work chapter.

\section{software contributions}
I instigated and led the development of several opensource software utilities to help record and preserve reports and underlying failure data from Google Play Console and Android Vitals. We have successfully extended the capabilities of the utilities and further extensions are practical. Research benefits:
\begin{itemize}
    \item Evidence preserved for analysis:
    \item Data can now be compared and further assessed:
    \item Data and contents can be shared by developers of Android apps with other researchers - extending the body of knowledge about crashes (un-reliability) and ANRs (run-time unresponsiveness). \emph{bringing previously unknown data, practices and tools into the open so others can understand them, make more informed decisions, and perform further research.}
\end{itemize}

I made data available~\cite{harty_wama_dataset_examples} for further research based on data originally only made available to app developers. Larger volumes of data are available upon request.

The research has been presented at various conferences and workshops, including at NII Shonan~\footnote{NII Shonan hosts highly collaborative workshops on various topics with participants from multiple continents~\citep{acm_shonan_workshops_2020}.} in 2019 at meeting 152 on "Release Engineering for Mobile Applications"~\footnote{~\url{https://shonan.nii.ac.jp/seminars/152/}}.

This thesis ends with an acceptance and understanding that developers benefit from help to solve bugs that happen in the field.

Even top development teams are likely to learn of bugs they were not able to find, and cannot reproduce. For instance, Google's Android Auto Team have asked for help from end users to identify patterns that may help the team find and address a long-running and frustrating bug in Android Auto~\footnote{\url{https://support.google.com/androidauto/thread/2865341?msgid=44437416}}. As reported by \texttt{autoevolution} in May 2020:  
\emph{"As it turns out, the Android Auto team wasn’t able to reproduce the whole thing, so it’s now asking users to send additional reports with more information to help fix the problem."}~\footnote{~\url{https://www.autoevolution.com/news/google-wants-users-to-help-fix-widespread-android-auto-bug-143760.html}}.

Mobile analytics has been shown to help multiple projects and teams improve the stability of their apps. This thesis provides various examples, there is much more useful and relevant research to do in this area. I hope you'll contribute and I would be happy to work with you in this area too. 