\chapter{Evaluation}
This chapter evaluates the research performed during the PhD.

\section{Evaluation of the case studies}
The case studies both led to significant improvements to the measured reliability of the treatment app. For the Kiwix project they then updated the codebase for the majority of their apps and achieved similar improvements in the measured reliability despite only using the default analytics provided by Google Play Console. For the Catrobat project the development teams chose to add crash and mobile analytics to their iOS app (which had neither previously) and mobile analytics to their Android app.

\section{Evaluation of the Analytics Tools}
The research primarily used and compared two analytics tools, both developed and maintained by Google. The research identified a wide range of flaws, summarised here.

\begin{enumerate}
    \item Testing discouraged: being encouraged and able to test the workings of the algorithm can increase the confidence in the analytics being generated. Flaws can be identified, discussed and either addressed or workarounds found.
    \item Negative user populations: Google Play Console provides two complementary graphs in addition to a third value useful for triangulation and for understanding the user-base for an app. For some apps the combination of the data in the first two graphs adds up to a negative volume of users.
    \item Gaps in the data: On numerous occasions Android Vitals graphs had gaps in the charts.
    \item No updates for 10+ days: in Autumn 2019. No explanation or information was provided to developers about the problem or the causes.
    \item No service dashboard: Google Play Console does not provide anywhere to check the current status or previous outages (unlike many other similar tools and other Google services).
    \item Repeated graphs in dashboard report.
    \item Inconsistent data ranges for some of the graphs in the dashboard report.
    \item Incorrect data ranges for Crash and ANR statistics reports.
    \item Unexplained and unfounded headline warning alerts.
    \item Missing URL parameters for links to drill-down reports in Android Vitals for some Android Releases.
    \item Second most popular country's details conflated with the the most popular and used for the title of one of the acquisition reports.
    \item Poor grouping of crash clusters.
    \item Lack of reports for low to medium volume user-bases.
    \item 10x difference in calculated crash rate between the two analytics tools.
\end{enumerate}

\iffalse 
\begin{table}[!htbp]
    \begin{threeparttable}[t]
    \footnotesize
    \centering

    \begin{tabular}{p{0.15\textwidth}p{0.3\textwidth}p{0.4\textwidth}}
    Flaw &Example &Impact \\
    \hline

    Negative populations &
    Pocket Code (see lifetime report L) &
    Lack of trust in Google’s analytics.\\
    
    Gaps in the data &
    Data missing for 1+ days (see below for examples) &
    Calculations affected, reduction in value of the reports, incomplete information. \\
    
    Missing URL parameters &
    Links to drill down on crash clusters for particular Android versions\tnote{1} &
    Confusion, risk of misattributing the reported crashes to a particular Android version when they are actually for the overall, current range of Android releases an app is running on. \\
    
    Incorrect ranges (a possible off-by-one calculation?) &
    Mismatch between overview and timeseries reports for Crashes and ANRs (see figures W,X,Y,Z). \\
    
    Poor groups of crash clusters &
    Kiwix examples dated \nth{4} July 2020\tnote{2} &
    Flaws in bug identification (mistakenly believing the scope of a crash cluster), potential suboptimal prioritisation of bugs to address. \\
    
    Lack of reports for low-volumes of usage &
    e.g. for WikiMed in Japanese with 1842 active users &
    Some key reports not available during early growth stages of an app, those apps may fail to thrive if they’re failing and developers don’t know the causes are poor reliability. \\
    
    \nth{2} country’s data conflated with \nth{1} &
    Acquisition Reports in app dashboard (see screenshot C)\tnote{3} &
    Confusion and lack of confidence in the reports. \\
    
    No facility for testing and testing discouraged &
    Implicit from the Terms of Service, section\tnote{4}. Other analytics tools, including those provided by Google, do encourage testing. &
    Developers dissuaded from testing the functionality or reporting. They have to [decide whether to] take on trust what Google chooses to tell them. \\


    \end{tabular}
    \begin{tablenotes}
    \item[1]e.g. the URL includes the relevant numeric value for most but not all the Android Versions \texttt{androidVersion=28} is correct and present for Android 8.1
    
    \item[2] a) \texttt{java.lang.IllegalStateException org.kiwix.kiwixmobile.core.base.BaseActivity.onCreate} \\
    b) \texttt{tgkill}
    
    \item[3] Note: partly addressed in the June 2020 revamp of Google Play Console.
    
    \item[4] \emph{``Broken Functionality We don’t allow apps that crash, force close, freeze, or otherwise function abnormally.”}~\cite{google_play_developer_policy_center}
    \end{tablenotes}
    
    \caption{Issues in Google Play Console Reports}
    \label{tab:issues-in-google-play-console-reports}
    \end{threeparttable}
\end{table}
\fi

% Please add the following required packages to your document preamble:
% \usepackage{booktabs}
% \usepackage[normalem]{ulem}
% \useunder{\uline}{\ul}{}
\begin{table}[!htbp]
\scriptsize
\renewcommand\TPTminimum{\textwidth}
%% Arrange for "longtable" to take up full width of text block
\setlength\LTleft{0pt}
\setlength\LTright{0pt}
\setlength\tabcolsep{6pt}
\begin{tabular}{p{0.18\textwidth}p{0.38\textwidth}p{0.4\textwidth}} %{@{}lll@{}}

\toprule
Flaw &
  Example &
  Impact \\ \midrule
Negative populations &
  Pocket Code (see lifetime report below) &
  Lack of trust in Google’s analytics \\
Gaps in the data &
  Data missing for 1+ days (see below for examples) &
  Calculations affected, reduction in value of the reports, incomplete information \\
Missing URL parameters &
  Links to drill down on crash clusters for particular Android versions e.g. Android 8.0 vs. Android 8.1 &
  Confusion, risk of misattributing the reported crashes to a particular Android version when they are actually for the overall, current range of Android releases an app is running on. \\
Incorrect ranges (a possible off-by-one calculation) &
  Mismatch between overview and timeseries reports for Crashs and ANRs (see below for examples - 4 figures) &
  One day’s data ‘missing’ from the timeseries reports. Confusion, lack of trust in the reports. \\
Poor groups of crash clusters &
  Kiwix example 04 Jul 2020\\ java.lang.IllegalStateException\\ org.kiwix.kiwixmobile.core.base.BaseActivity.onCreate\\ And tgkill &
  Flaws in bug identification (mistakenly believing the scope of a crash cluster), potential suboptimal prioritisation of bugs to address. \\
Lack of reports for low-volumes of usage &
  E.g. for WikiMed in Japanese with 1842 active users. &
  Some key reports not available during early growth stages of an app, those apps may fail to thrive if they’re failing and developers don’t know the causes are poor reliability \\
2nd country’s data conflated with 1st. &
  Acquisition Reports in app dashboard (see below for screenshots). Note partly addressed in the June 2020 revamp of Google Play Console. &
  Confusion and lack of confidence in the reports \\
No facility for testing and testing discouraged &
  Implicit from the Terms of Service, section “Broken Functionality We don’t allow apps that crash, force close, freeze, or otherwise function abnormally.” Other analytics tools, including those provided by Google, do encourage testing. &
  Developers dissuaded from testing the functionality or reporting. They have to {[}decide whether to{]} take on trust what Google chooses to tell them. \\
Repeated graphs &
  In dashboard report for apps 2 graphs are repeated: New users acquired, and Users lost. &
  Breaks ‘don’t repeat yourself’ (DRY). Waste of attention. \\
Inconsistent date ranges for some dashboard reports &
  The audience growth reports: new users acquired by country, and top countries graphs are for fixed periods - the period is not explained or visible on the report. In contrast the Android Vitals graphs are explicitly for the last 30 days, not necessarily ideal but at least documented that the period is fixed. &
  Confusion, inconsistent behaviour of the reports, potential for misinterpretation. \\
No updates for several days &
  In September 2019 the Android Vitals graphs and data were not updated for around 10 days &
  Inability to see or respond to stability issues, loss of confidence in the service. \\
Unexplained negative headline rate &
  Crash rate of 1.66\% - urgent warning, yet drilling down into the report none of the details reach 1.66\%, they are all significantly lower values. &
  Reduction in confidence in their alerts, and similarly a lack of trust in their calculations \\
No Service Problem Reporting &
  Google, and many others, provide a service status page, and many also include a history. For instance Salesforce provides https://trust.salesforce.com/en/ and status pages e.g. https://status.salesforce.com/incidents/5800 &
  A lack of transparency which leads to a lack of trust. \\ \bottomrule
\end{tabular}
    \caption{Issues in Google Play Console Reports}
    \label{tab:issues-in-google-play-console-reports}
\end{table}


\section{Evaluation of the software created for the research}



\section{Validity considerations}
In absolute terms, my research covers a minuscule percentage of all the apps available in Google Play, roughly 1 in 100,000. So these results may not apply to all the apps, or potentially even a majority of them. And yet, the results have consistently indicated that when development teams pay attention to stability metrics they are able to materially improve the reliability of their mobile apps even though their apps range across several app store categories and range in userbase from under 1,000 active users to over 160,000. These apps are spread across 6 of the 7 groups of downloads identified in AppBrain's `Download distribution of Android apps'~\cite{appbrain_download_statistics_june_2019} and similarly 5 of the 7 groupings representing over 94\% of the distribution of downloads in Google Play according to Wang ~\emph{et al} (2018)~\cite{wang2018beyond}.

\textbf{MUST-DO} answer the following question: What exists in the literature, common practices, vs what I was able to achieve. \emph{From a question raised by Alistair Willis, OU, 30 April 2020.}

\subsection{How many developers are enough to ask?}
On of the key considerations for research is adequacy in terms of coverage. For my research there are several types of coverage, including: development teams, user-bases for the various apps, software tech stacks used (in terms of programming languages, analytics libraries, etc.), application domains, and so on. 

c.f. Krug is a well respected Usability guru whose work is inherently practical in nature. In the first edition of his~\emph{``Don't make me Think"} book he discusses ways to obtain practical results even with short timescales and few resources. In terms of obtaining value the author indicated that 3 to 4 people were capable of delivering more relevant feedback by involving them over time, (Chapter 9 in ~\cite{krug2000dont_make_me_think}). In terms of selecting the candidates his recommendation was to worry less about selecting 'representative users', instead\emph{``Recruit loosely, and grade on a curve."} (Chapter 9 in ~\cite{krug2000dont_make_me_think})~\footnote{Note: Krug made several chapters, including this one available online when the second edition of the book was published. I have copies of all three versions of the book and of these chapters as PDF files.}.

My research included working with two mature project teams and developers of three commercial apps. It is also based on work I did in industry that predates the PhD research, unfortunately I am not able to provide details of those projects in my thesis. 

\section{Summary of evaluation of case studies}

