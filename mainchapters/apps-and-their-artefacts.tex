\chapter{Apps and their artefacts}~\label{chapter-apps-and-their-artefacts}
\julian{This chapter covers \uartefacts and \iartefacts.}

\section{Normal and acceptable ranges of failures}
For the GTAF case study, the crash rates, as measured by Android Vitals, vary massively by release of the app (Al Quran (Tafsir \& by Word).\pending{A great deal more analysis is possible given the number of active apps and the ongoing access to Google Play Console.} 

Android Vitals does not report any failures in the GTAF taskinator app, which is created using React Native according to the development team. This is pertinent as similar behaviours in Android Vitals were observed for another app-centric case study, LocalHalo. 


\section{Improvements in crash rates}
For both Catrobat and Kiwix the hackathon and the post-hackathon bug fixes were highly effective in terms of cumulatively reducing the crash rate over several subsequent releases. % TODO Check whether the releases were contiguous.
The Catrobat case study replicated the improvements seen in the Kiwix case study and also the efficacy of using a hackathon as an immediate intervention. TODO discuss limitations of hackathons and forward link to the section on the useful half-life of hackathons. 


For Pocket Code, the improvements in the stability of the app were particularly encouraging as the project had already implemented many of `good' development practices.


\section{Summary of apps and their artefacts}
TBC