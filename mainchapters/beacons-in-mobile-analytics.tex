%\section{Beacons in Mobile Analytics}~\label{beacons-in-mobile-analytics}
%\julian{This is where I'm gathering notes on beacons found in mobile analytics outputs. It may eventually be incorporated into my thesis pre-submission if it's sufficiently `cooked'. For now, it's where raw content and ideas will be collected and collated. I've included it temporarily to see if it would add or detract from the thesis.}

% Copied from the Methodology chapter on 9th Dec 2021
In this research the beacons include: the shape of graphs in mobile analytics report, the contents of one or more failure clusters, a method call in stack traces, and so on.

Candidates for beacons include (numbered for ease of reference):
\begin{enumerate}
    \item An aggregate increase in error rate.
    \item Early adverse trends during a phased release rollout.
    \item Correlations with one to a few predetermined factors (e.g. OS release, device model, ...).
\end{enumerate}

%\marian{Now specify how you spotted beacons... I looked for things on this basis, how I kept track of things, what selection criteria were used and why?}

The beacons were spotted through a mix of pattern recognition, visual anomalies, adverse trends, \emph{etc.} Often they were grounded through experience of being familiar with working with the artefacts of mobile apps.

Beacons emerge in various ways. For instance in reports they include: anomalies within a report, mismatches and inconsistencies between two sibling reports or between a master report and the linked detailed report.

Inspirations for this work include: Scott Barber's work on modelling software performance [testing]~\sidecite{barber2004_ucml_v1_1}.
