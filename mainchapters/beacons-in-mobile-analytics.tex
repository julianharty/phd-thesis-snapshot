\section{Beacons in Mobile Analytics}~\label{beacons-in-mobile-analytics}
\textit{This is where I'll gather notes on beacons found in mobile analytics outputs. It may eventually be incorporated into my thesis. For now, it's where raw content and ideas will be collected and later collated.}

Inspirations include:
\begin{itemize}
    \item Scott Barber's work on modelling software performance [testing].
\end{itemize}

Candidates for beacons include (numbered for ease of reference):
\begin{enumerate}
    \item An aggregate increase in error rate
    \item Early adverse trends during a phased release rollout
    \item Correlations with one to a few predetermined factors (e.g. OS release, device model, ...)
    \item 
\end{enumerate}

% Copied from the Methodology chapter on 9th Dec 2021
In this research the beacons include: the shape of graphs in mobile analytics report, failure clusters, a method call in stack traces, and so on.

\marian{Now specify how you spotted beacons... I looked for things on this basis, how I kept track of things, what selection criteria were used and why?}

Beacons emerge in various ways. For instance in reports they include: anomalies within a report, mismatches and inconsistencies between two sibling reports or between a master report and the linked detailed report.


Drill-down can include many activities, such as:
 
 \begin{itemize}
    \itemsep0em
    \item Collating similar failures: 
    \item Bug identification and localisation: Establishing potentially pertinent patterns in the reports, and characterising when a failure \emph{does and does not} occur are part of this work. Obtaining an identifying definitive boundaries may be impractical, the work is often iterative and exploratory in nature and lossy. 
    \item Ordering and ranking clusters of failures:
    \item Bug investigation:
    \item Checking whether there is likely to be sufficient evidence for any triage process: 
    \item Comparing information sources:
    \item ...
\end{itemize}
