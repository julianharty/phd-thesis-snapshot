\chapter{Overview of the case studies}~\label{chapter-case-studies-overview}

In the previous chapter app-centric and tool-centric case studies were introduced, together with the procedure used for the app-centric case studies. This chapter builds on their work and introduces each case study using a consistent structure to make them easy to comprehend and to facilitate their comparisons. Three subsequent, contiguous chapters will each address the temporal dimensions (understanding and improvement) of an object of analysis (use, artefacts, and tools). These will in turn be followed by the discussion chapter.

A consistent structure is used to present each case study covering the following topics:
\begin{itemize}
    \itemsep0em
    \item Background: how the case study came about.
    \item App-centric case study procedure: specifies details of the procedure used for this case study. This applies Section~\ref{methodology-app-centric-case-study-procedure}.
    \item Development microcosm: an overview of their development team and practices.
    \item Experiences of using mobile analytics: a brief summary of the team's use of mobile analytics for their project.
    \item Data collected and methods used for collection: specific summary of the data collected and methods used for the data collection based on Table~\ref{tab:mapping-datasources-to-six-perspectives}. The Exodus Privacy project~\footnote{https://reports.exodus-privacy.eu.org/en/} has been used to check the mobile analytics libraries detected in the apps from the app-centric case studies. 
    \item Intervention: a summary of the intervention performed. \textit{This only applies for case studies with an intervention.}
    \item Outcomes for the organisation/company: the outcomes achieved by the company, or organisation for non-profits, through their use of mobile analytics.
    \item Contributions to the research and where they are located in the rest of this thesis: links to where this case study contributes to the thesis.
    \item \textit{A temporary ``wish list" to enhance the case-study where practical. These will be followed up where it's ``essential" for the thesis. The entire wish list will be removed from the generated thesis pre-submission.}
\end{itemize}



\section*{Open questions for this chapter}
{\small
\begin{itemize}
    \itemsep0em
    \item Should I include a summary of applying the app-centric case study procedure in each case study with the specifics for that case study? - Yes I need to (as discussed with Arosha, Marian and Yijun)
    % \item I think it's worth adding an `\textbf{intervention}' subsection to each of the three case studies with interventions. - Agreed and done.
    \item TODO connect the tables with their cases.
\end{itemize}
}  % end \small

The case studies are presented in the same order as Table~\ref{tab:app-centric-studies-research-perspective} for the app-centric cases and then Table ~\ref{tab:tool-centric-studies-research-perspective} for the tool-centric cases. The app-centric case studies are split into two groups, the first do not have interventions (Sections \ref{case-study-overview-gtaf}, \ref{case-study-overview-localhalo}, \ref{case-study-overview-moodspace}, \ref{case-study-overview-moonpig}, and \ref{case-study-overview-smartnavi}) the second group do (Sections \ref{case-study-overview-kiwix}, \ref{case-study-overview-catrobat}, \ref{case-study-overview-C1}).

\clearpage




%===================================================================

\section{App-centric: GTAF}~\label{case-study-overview-gtaf}
% A couple of sentences to introduce them
Greentech Apps Foundation (GTAF) is a UK based charity who provide Islamic apps free of charge and without in-app advertising. The project started in 2016 with the aim of enabling people to learn the Quran in the local language - Bangla - in Bangladesh. The project was started by a self-taught Android developer and his cousin Yemin, at the time an undergraduate student in computer science, who is now employed by the project in a hybrid role of software developer and project manager. Table \ref{tab:gtaf_anaytics_overview} summarises the key facts for this case study.

{\renewcommand{\arraystretch}{0.8}% Tighter
\begin{table}[htbp!]
    \centering
    \small
    \setlength{\tabcolsep}{1pt}
    \begin{tabular}{lp{11cm}}
       % Question &Answer  \\
       \toprule
       Website &\url{https://gtaf.org/} \\
       Google Play Home & \url{https://play.google.com/store/apps/dev?id=7665838187257770408} \\
       Founded & 2016 \\
       Business Domain & Not-for-profit.  \\
       Business type & Educational foundation. \\
       Technologies  & Android apps\footnotemark \\
       & React Native \\
       Source code  & Closed and not available for research \\
       Analytics used by team & Firebase, OneSignal, Google Crashlytics, Google Play Console \\
       Development Practices & Small hybrid development team \\
       \midrule
       User base & 1,000,000'+ for their 10 Android apps \\
       Installations & 1,000,000's for their 10 Android apps \\
       \midrule
       Research methods &Online interview and email discussions, etc. \\
       Analytics collected &Google Play Console with Android Vitals \\
       Research software & None applicable? \\
       Additional data collected &Direct access to Google Play Console with Android Vitals, and to the public, issue database. Interview notes and emails. \\
       Active period & June 2020 to September 2020 \\
       \bottomrule
    \end{tabular}
    \caption{Case Study key facts: GTAF}
    \label{tab:gtaf_anaytics_overview}
\end{table}
}

\footnotetext{The project have subsequently released several of their apps on other platforms, see \url{https://gtaf.org/apps}.}

\subsection{GTAF: Background - How the case study came about}
A fellow PhD researcher contributes voluntarily as a developer as part of the extended project team and introduced me to the core project team who agreed my research was of interest to them and something they were willing to support.

\subsection{App-centric case study procedure}
The app-centric case study procedure is as follows:
{\small
\begin{enumerate}
    \itemsep0em
    \item Exploration and selection: the project had various popular Android apps in Google Play store. They were already using several mobile analytics services so were already familiar with the concepts presented in this research.
    \item Engagement: the developers had had to address excessively high crash rates so they were interesting in the research and willing to help. They were happy to provide read access to their Google Play Account for their apps, they were cautious about providing access to additional analytics services or their source code so we agreed the research would start with Google Play Console with Android Vitals and access to their issue tracking system, which is public anyway.
    \item The action research stage: consisted of three elements: a discussion on their use of mobile analytics, ongoing read only access to their Google Play Console, and a report based on the analysis of the Google Play Console and Android Vitals reports and data for their top apps. Ill-health and some other complications effectively limited any additional action research. 
    \item \textit{Post-hoc} analysis: as the research has ongoing access to Google Play Console the analysis continues.
    \item Wrap-up: either once the viva has completed; or TBD \textit{e.g.} the project team may wish to disconnect from the research or they may wish to increase the engagement beyond the end of the PhD.
    \item Publication: Given the relatively limited involvement there is little scope to publish findings in their own right, however the findings will be published in this thesis and may also contribute to additional publications.
\end{enumerate}
}


\subsection{GTAF: development microcosm} 
The project hosts their development artefacts on gitlab.com, where they maintain their issues in a publicly available online location \url{https://gitlab.com/greentech/}, and the source code is private. There are various developers, some are volunteers, several are paid for (through donations to the charity). From the interview I got the impression developers of some of the less active apps are fairly autonomous, which includes their choice and any use of mobile analytics. 

Three of the apps (which effectively become four as one app is released as two distinct binaries) were in ongoing active development (\href{https://play.google.com/store/apps/details?id=com.greentech.quran}{Al Quran},~\href{https://play.google.com/store/apps/details?id=com.greentech.hadith}{Hadith Collection}, and~\href{https://play.google.com/store/apps/details?id=com.greentech.hisnulmuslim}{Dua \& Zikr}, which is also released separately in Bangla~\href{https://play.google.com/store/apps/details?id=com.greentech.hisnulmuslimbn}{{Dua and Zikr (Hisnul Muslim)}}~\emph{in Bengali}); and they planned to revamp two more of the apps (\href{https://play.google.com/store/apps/details?id=com.greentech.islamicquiz}{(Islamic Quiz)} and~\href{https://play.google.com/store/apps/details?id=com.greentech.salatbn}{Meaningful prayers (salat)}~\textit{in Bengali}, which was called salat in our interview).

The team occasionally used Firebase TestLab~\footnote{\url{https://firebase.google.com/docs/test-lab}} to test some of the apps and autonomous `Robo testing'~\footnote{\url{https://firebase.google.com/docs/test-lab/android/robo-ux-test}} performed automatically by the test lab has triggered various crashes in the apps being tested. One such example was where an app was missing a `resource'. The team fixed the build by adding the missing resource but did not explicitly retest the app afterwards in Firebase.  

\subsection{GTAF: Experiences of using mobile analytics}
The development team check Android Vitals approximately once a week, and Firebase more frequently as the team decided the crash reports in Firebase are more actionable. Perhaps unsurprisingly they check more often after new releases of their apps looking for any new bugs arising in the new release as it rolls out across the user population.

They noticed differences noticed in the reports of Firebase compared to Android Vitals, however they were not overly concerned about the differences as their focus was on the crashes reported in Firebase as they contain more contextual detail. ANRs seldom checked, considered to be less impactful on users and lower frequencies. % TODO ask for access to their Firebase stats?

At the time of the case study, the team's development priorities for the rest of 2020 and until April 2021 the team were focusing on bug-fixes which included fixing the causes of crashes being reported by mobile analytics for their apps. In March 2021, they published a blog post which confirms this focus and includes a chart of their average daily crashes for 2020 which shows their progress in addressing peaks in the crash rate~\citep{gtafblog2021_gtaf_accomplishment_2020}. The chart does not provide any additional information \emph{e.g.} of which app(s) the chart was plotted for or the source of the data. However, from the appearance of the chart it can be inferred that the source is probably Firebase Analytics.

The same blog post~\citep{gtafblog2021_gtaf_accomplishment_2020} explains one of their goals for 2021 was to `integrate analytics features in our application' to improve the user experience for the people who use the GTAF apps.

\subsection{GTAF: data collected and methods used for collection}
The data was collected from four primary sources: 1) an online, \textit{pre-study} interview, recorded in handwritten notes, 2) ongoing read access to Google Play Console with Android Vitals, both automated and interactive snapshots were captured, 3) email correspondence, maintained in a GMail account, and 4) the project's public issues database, which was searched interactively. These are illustrated in Table~\ref{tab:gtaf-data-sources}.


\begin{table}
    \centering
    \footnotesize
    \tabcolsep=0.12cm
    \begin{tabular}{p{2.4cm}p{2.4cm}rp{2.4cm}p{3cm}p{2.5cm}}
        Data Source & Records & Volumes & Analysis method &Contribution & Remarks \\
        \toprule
         Interview with core developer & contemporaneous notes\footnotemark & 1 & Ask the app devs & Set scope \& direction & Online call \\
         Google Play Console with Android Vitals &Interactive screenshots \& Vitals-scraper outputs &10\textsuperscript{1} & Beacon finding, drill down, across case comparisons, observation \& analysis. & Indications of the development team's attention to the crash rate, insights into the performance of their apps, corroboration of findings across various case studies. &  \\         
         email conversations & GMail & 10\textsuperscript{1} & Ask the app devs & Feedback, and sense-making.  &  \\
         Public GitLab & Issues  & 10\textsuperscript{2} & Observation and Analysis, analysis of development artefacts. & Corroboration of what the development team say they do in terms of using mobile analytics. &  \\

         \bottomrule
    \end{tabular}
    \caption{GTAF: data sources}
    \label{tab:gtaf-data-sources}
\end{table}



As the project did not provide access to Firebase or the other in-app analytics it was not feasible to compare their outputs, and similarly they did not provide access to the source code of their apps so that could not be studied.

\subsection{GTAF: Intervention}
None: the GTAF case study did not include a formal intervention.

\subsection{GTAF: Outcomes for the company}
The organisation found mobile analytics helpful and addressed the crashes they believed were tractable and productive to fix in terms of improving the user experience.


\subsection{GTAF: Contributions to the research and where they are located in the rest of this thesis}
TBC\pending{To be added when I write the subsequent chapters.}

\julian{There is scope to do ongoing analysis of the Google Play Console and Android Vitals reports for the project's 10+ Android apps. They help indicate some foibles in the Dashboard page for several apps - at least, where the combined ANR and crash rate report does not agree with the separate Crash and ANR reports from Android Vitals.}

\subsection*{GTAF: wish list}
Here's my wish list so we can discuss it and decide what's appropriate to do of these and what to park.
{\small
\begin{itemize}
    \itemsep0em
    \item Contact the team again once the thesis has been drafted to review the findings and ask the additional open questions here. 
    \item Ask them about their development practices 12+ months on from the active engagement.
    \item Ask clarifying and follow-up questions based on their 2020 blog post e.g. the source of the graph data and their assessment of their progress in 2021, and on their plan to increase the use of mobile analytics in the apps.
    \item Ask if they would like any further help or involvement. If so, request access to their Firebase Analytics console and any other developer centric mobile analytics tools.
    \item Explore if they'd be willing to change the relationship to being more of an action research case study where I help them (free-of-charge) to identify and address major stability issues in their Android apps.
\end{itemize}

\begin{itemize}
    \itemsep0em
    \item I could usefully do at least a day's worth of analysis on their Google Play Console and Android Vitals reports for their apps based on what I've observed. Whether that'd be sensible to do pre-submission is an open question.
    \item Similarly there's probably quite a bit of information that can be mined from their issue database on gitlab.
    \item Add results from analysing the binaries of the apps with exodus-privacy.
\end{itemize}
}  % end \small

\clearpage

%===================================================================

\section{App-centric: Local Halo}~\label{case-study-overview-localhalo}
LocalHalo was a startup based in London who made a social network for neighbours~\footnote{\url{https://ain.ua/en/2019/10/18/localhalo-raises-500k/}} with developers in Ukraine, London, and Kazakhstan.~\citep{karpenko2019_localhalo_a_social_network_for_neighbors}. 
Table~\ref{tab:local_halo_anaytics_overview} provides an overview of this case study.

{\renewcommand{\arraystretch}{0.8}% Tighter
\begin{table}[htbp!]
    \centering
    \small
    \setlength{\tabcolsep}{1pt}
    \begin{tabular}{lp{11cm}}
       % Question &Answer  \\
       \toprule
       Website &\url{https://www.localhalo.com/} \\
       Google Play Home & \url{https://play.google.com/store/apps/developer?id=NAY+PROTECT+LTD} \\
       Founded &2018 \\
       Business Domain &Digital neighbourhood groups in UK.\\
       Business type &Startup, two co-founders: CEO and CTO \\
       Technologies  &React Native for cross-platform Android and iOS development \\
       &Expo development framework \\
       Source code  &Closed and not available for research \\
       Analytics used by team &Sentry, Mixpanel, Google Play Console \\
       Development Practices &Not explicit, a small distributed team \\
       \midrule
       User base &7,000 registered users, 1k to 2k monthly active (Jan 2020) \\
       Installations &1,000's for the Android app \\
       \midrule
       Research methods &Interview, email discussions, bug analysis, use of mobile analytics \\
       Analytics collected &Live access to: Sentry, Google Play Console with Android Vitals \\
       Research software &Vitals-Scraper used to preserve results \\
       Additional data collected &Interview notes, emails with the CTO, and automated emails from their mobile analytics services. \\
       Active period &Jan 2020 to June 2020 \\
       \bottomrule
    \end{tabular}
    \caption{Case Study key facts: Local Halo}
    \label{tab:local_halo_anaytics_overview}
\end{table}
}

\subsection{Local Halo: Background - How the case study came about}
I was introduced to the CEO of Local Halo by one of their team who knew of my work in October 2019. An online call was followed by an in-person meeting with the CEO in November 2019 where he offered to be a case study for my research. After an online call in January 2020 with the CTO they provided access to Google Play Console with Android Vitals for their Local Halo Android app~\footnote{The underlying app is cross-platform which generates both Android and iOS binaries.}, and to their Sentry~\footnote{\url{https://sentry.io/welcome/}}, which was used to track technical issues with their app and website. 

We agreed the research would not have access to the third analytics service, Mixpanel~\footnote{\url{https://mixpanel.com/}}, which they used for user behaviour and related analytics. This service included personally identifiable information of their users and would not be useful for this research. We also agreed no access would be provided to the source code as that was deemed sensitive. 

\subsection{Local Halo: Development microcosm}
The development team used the Expo development platform \url{https://expo.dev/} to create native apps that ran on Android and iOS apps. The mobile app was written in React Native~\footnote{\url{https://reactnative.dev/}} (the associated website is likely to have been written using React js~\footnote{\url{https://reactjs.org/}} and also instrumented using Sentry Analytics which were also available for research purposes). These apps were released on Google Play and Apple's App Store respectively. The CTO was actively involved in writing and maintaining the source code and was supported by developers in three locations, in the Ukraine, London, and Kazakhstan.~\citep{karpenko2019_localhalo_a_social_network_for_neighbors}.  Data in the Sentry mobile analytics reports indicate there was at least one distinct developer in addition to the CTO.

Little additional information was available during conversations in terms of their development or release practices for their mobile apps. One observation is Expo claims to automate the release process to the app stores so the LocalHalo development team may have relied and used the Expo service. And Sentry provided reports on the release numbers and their rollout.

\subsection{Local Halo: Experiences of using mobile analytics}
The development team used at least three distinct mobile analytics services: Mixpanel for behavioural analytics, Sentry for error analytics, and Google Play Console with Android Vitals (these were the ones discussed with the CTO). 

Local Halo incorporated two analytics libraries into their cross-platform mobile application: Sentry for crash reporting and Mixpanel for business-oriented usage analytics. For their Android app they also had access to Google Play Console. Interestingly, the Exodus Privacy project detects additional trackers and does not detect Sentry, as their report for Local Halo confirms: \url{https://reports.exodus-privacy.eu.org/en/reports/225323/}.

They seldom used Google Play Console with Android Vitals. According to the CTO it did not suit apps developed in React Native as the crashes did not actually crash the shell app which wraps the React Native app~\footnote{Note, other developers have asked about this behaviour, for instance on StackOverflow \href{https://stackoverflow.com/questions/66166824/native-crash-reporting-for-expo-deployed-to-android/}{Native crash reporting for Expo deployed to Android?}} 
(which is what appears to be monitored by Android). The shell app restarts the React Native app automatically.

The development team chose to use separate analytics services for user behaviour analytics (Mixpanel) and issues such as crashes and anomalies (Sentry). Presumably the respective SDKs were incorporated into the React Native source code~\footnote{Some of the errors reported by Sentry indicate the integration used TypeScript (which is supported in React Native, see \url{https://reactnative.dev/docs/typescript}. Mixpanel also provides an opensource wrapper that supports React Native \url{https://github.com/mixpanel/mixpanel-react-native}.}~\footnote{Note: Local Halo also incorporated Sentry into their website and provided access to Sentry for their website, however the website is out of scope for this research and will not be considered further here.}. Only two people have access to Sentry: the CTO and me, therefore any other members of the development team would have indirect access, for instance via screenshots and/or bug reports raised by the CTO. 

They saw value in using analytics to improve business results, for instance for App Store Optimisation to improve the ranking of their app in the app stores. The CTO made a key observation during the interview \emph{``If you have lots of crashes you have zero chance of being promoted [by the app store]."}
%
In terms of user experience analytics the CTO also observed \emph{``To improve user retention you need to do both, eliminate the bad stuff [and] improve the good stuff [to] increase value"}.

The overall impression was the team had decided to incorporate mobile analytics to help them provide a reliable and valuable service for their current and hoped-for users where the team would address errors on an \emph{ad-hoc} basis.


\subsection{Local Halo: Data collected and methods used for collection}
This case study combines three immediate sources of data with automatically provided reports. The interactive sources are: 1) separate meetings with the founders, 2) email conversations with the founders, and 3) online access to two mobile analytics tools. The automatically provided reports include 100's of automated emails from Sentry's hosted mobile analytics service, and Vital Scraper was also used to collect reports and crash details from Google Play Console with Android Vitals. Table~\ref{tab:localhalo-data-sources} provides more details.

\begin{table}
    \centering
    \footnotesize
    \tabcolsep=0.12cm
    \begin{tabular}{p{2.4cm}p{2.4cm}rp{2.4cm}p{3cm}p{2.5cm}}
        Data Source & Records & Volumes & Analysis method & Contribution & Remarks \\
        \toprule
         1-to-1 meetings with founders & contemporaneous notes\footnotemark & 3 & Ask the app devs. & Set scope and direction & 1 was in-person, 2 were online \\
         email conversations & GMail & 10\textsuperscript{1} & Ask the app devs. & Insight into the Expo bug & Initiated by the researcher. \\
         Google Play Console with Android Vitals &Interactive screenshots \& Vitals-scraper outputs &10\textsuperscript{1} & Beacon finding, drill down, across case comparisons, observation \& analysis. & Insight into the reporting effects of the Expo bug and the reporting provided for React-Native apps &  \\
         Sentry & Interactive screenshots & 10\textsuperscript{1} & Beacon finding, drill down, across \textit{tool} comparisons\footnotemark, observation \& analysis. & Insights into Sentry's reporting & Access continued until Sentry removed multi-account access from their free tier. \\
         Sentry automated emails &GMail & 10\textsuperscript{2} & Beacon finding, drill down, observation and analysis. & Insights into Sentry's reporting, dev practices, \& cross-platform reporting & \textit{ditto.} NB: they continue to send weekly reports by email. \\
         \bottomrule
    \end{tabular}
    \caption{LocalHalo: data sources}
    \label{tab:localhalo-data-sources}
\end{table}

\footnotetext{Pertinent details validated by email.}
\footnotetext{A specialisation of across case comparisons where the outputs of two mobile analytics tools were compared.}

\subsection{Local Halo: Intervention}
None: the Local Halo case study did not include a formal intervention.

\subsection{Local Halo: Outcomes for the company}
The nature of the case study (being remote from the development team, with access limited to two of the mobile analytics services, and correspondence being occasional) meant the outcomes for the company are hard to ascertain. It is clear the founders saw the value in choosing two particular mobile analytics services to help them manage the business and the service their provided to their users. 

Their main focus was to try to grow the business. Ultimately that was not successful, a topic discussed briefly in the next paragraph. Nonetheless, the analytics tools continue to report on the app and the website and continue to provide some insights into the reliability of the app and usage of the app.

Note: The founders of local halo indicated, as of October 2020~\footnote{Via their respective LinkedIn profiles: \url{https://www.linkedin.com/in/jamesroutledge/} and \url{https://www.linkedin.com/in/andriymarin/}.}, they are no longer actively involved in this project. And there are confirming indications in sentry.io as the app has not been updated in over a year, in contrast to the many updates they made previously. So even though the app remains online and available, and the analytics continue to report, there is no one actively maintaining the app or dealing with the errors being reported in the analytics.

\subsection{Local Halo: Contributions to the research and where they are located in the rest of this thesis}
TBC

\subsection*{Local Halo: Wish list}
{\small
\begin{itemize}
    \itemsep0em
    \item I wish we could find out how actively the development team were reading, reviewing and addressing crashes being reported. However, as the project no longer appears to be active that's unlikely to happen.
    \item I'd also appreciate ongoing access to both Sentry and Google Play Console with Android Vitals for the app, again I doubt this is viable given the project is defunct.
    \item NB: the material in this section needs editing and condensing there's a bit too much info and too much repetition. Once I add content to the next 3 chapters for Local Halo I should have a better idea of what doesn't belong here.
    \item Find a place to discuss the penetration of react-native in Android apps on Google Play, see \url{https://www.appbrain.com/stats/libraries/details/react_native/react-native}.
\end{itemize}
}

\clearpage

%===================================================================

\section{App-centric: Moodspace}~\label{case-study-overview-moodspace} % The title is used throughout this file, do a search and replace.
Moodspace is an Android app aimed at improving mental health through various exercises incorporated into the app. %The app was listed as one of the 25 best mental health apps~\footnote{\url{https://www.psycom.net/25-best-mental-health-apps}}. 
% The app features in various peer-reviewed papers, however they lack critical depth of the app e.g. in https://doi.org/10.1145/3411764.3445500 
%  -- ``A systematic review of cognitive behavioral therapy and behavioral activation apps for depression" (PLoS 2016)
% -- https://www.ncbi.nlm.nih.gov/pmc/articles/PMC8529472/ (and the DOI https://mhealth.jmir.org/2021/10/e26712 )
% -- The following paper explains the needle-in-a-haystack challenge of establishing the evidence for this and similar apps https://www.ncbi.nlm.nih.gov/pmc/articles/PMC7588098/
% It also features online e.g. in https://www.healthfoundry.org/covid-19-response 
It was released in 2019, with over 150K downloads by early 2020~\citep{objectbox2020_moodspace_interview}. Ian Alexander, the interviewee, was the software developer, co-founder, and runner of the company~\citep{objectbox2020_moodspace_interview} so he combined technical and operational responsibilities. He is an experienced app developer and also trained as a chemical engineer.

{\renewcommand{\arraystretch}{0.8}% Tighter
\begin{table}[htbp!]
    \centering
    \small
    \setlength{\tabcolsep}{1pt}
    \begin{tabular}{lp{9cm}}
       % Question &Answer  \\
       \toprule
       Website &\url{https://moodspace.org/} \\
       Google Play Home & \url{https://play.google.com/store/apps/details?id=boundless.moodgym} \\
       Founded & 2019 \\
       Business Domain & Health \\
       Business type & Online, in-app purchases. \\
       Technologies  & Android \\
       & ObjectBox ORM database \\
       Source code  &Closed and not available for research \\
       Analytics used by team & Firebase Analytics, Firebase Crashlytics, Google Play Console \\
       Development Practices & Sole software developer \\
       \midrule
       User base & 10,000's for the Android app \\
       Installations & 100,000's for the Android app \\
       \midrule
       Research methods &Email interview \& follow-up discussions. \\
       Analytics collected &Google Play Console with Android Vitals \\
       Research software & None applicable. \\
       Additional data collected &Exodus Privacy project reports. \\
       Active period & June 2019. \\
       \bottomrule
    \end{tabular}
    \caption{Case Study key facts: Moodspace}
    \label{tab:blank_case_study_anaytics_overview}
\end{table}
}

\subsection{Moodspace: Background - How the case study came about}
A practitioner recommended me and my research to the co-founder and CTO of Moodspace. He was willing to answer questions and also provide screenshots from Google Play Console and Android Vitals.

\subsection{Moodspace: Development microcosm}
Ian chose to use an in-memory ORM database, objectbox. In his words: \emph{``The app doesn't use any API, so all the data's stored in very fast ORM databases like object-box (and uses memory caching). This enables the app to be mostly synchronous, which hugely cuts down on complexity of code. i.e. no need to handle loading, errors, or concurrency. This is a big benefit! And cuts down on errors significantly, with no real impact on performance for users. To illustrate that it has little impact on users, I use firebase performance to run a trace on some methods that call the ORM/cache - it's peak duration is 40ms while the majority of calls take 3-6 ms."} (email correspondence, lightly edited to fix typos and improve readability)\improvement{Relocate this to the methodology chapter if I provide other edited extracts from emails.}.

At the time of the interview the team had developed a complete replacement for the Android app which was due to be launched in a couple of months (around August 2019).

\subsection{Moodspace: Experiences of using mobile analytics}
The app included both Firebase Analytics and Firebase Crashlytics. The discussion covered Firebase Crashlytics and Google Play Console with Android Vitals, Firebase Analytics was not mentioned by Ian. Details of their use are covered in the \nameref{chapter-analytics-in-use} chapter.

The current privacy policy for the Moodspace app confirms they actively use mobile analytics in the app: 

{\small
``To make the app work well at all we collect the following anonymous data:
    \begin{itemize}
        \itemsep0em
        \item Crash reports: If you've never seen the app crashing, it's because as soon as one happens, we get a crash report. A little red light flashes in our office, a loud siren blares, and we release a fix right away. It's quite annoying actually.
        \item Analytics: We assume you're going to use the app a certain way. We're almost always wrong, and you often surprise us. Analytics lets us see how people like you actually use the app, so we can make improvements to the right places. Analytics can use the Google Advertising ID to identify you. This doesn't tell us anything about you (it's just some numbers and letters), but if you really want to trick us you can reset your Google Advertising ID at any time. Go to your device Settings > Google > Ads."
    \end{itemize}~\citep{moodspace2021_privacy_policy}
}

\subsection{Moodspace: Data collected and methods used for collection}
The main source of data are emails from Ian, the CTO, the emails included extracts of screenshots from their Google Play Console with Android Vitals. These were supplemented with some additional research online using grey literature and grey data; Table~\ref{tab:moodspace-data-sources} provides additional details of the data sources used.

\begin{table}
    \centering
    \footnotesize
    \tabcolsep=0.12cm
    \begin{tabular}{p{2.3cm}p{2.1cm}rp{2.4cm}p{2.8cm}p{3.2cm}}
        Data Source & Records & Volumes & Analysis method & Contribution & Remarks \\
        \toprule
         email conversations & GMail & 10\textsuperscript{1} & Ask the app devs. & Effectively the interview, albeit using emails.  & We ended up simply using emails rather than arranging a synchronous call.  \\
         Google Play Console with Android Vitals &Interactive screenshots & 6 & Ask the app devs. & Evidence, and allows for comparisons. &  \\
         Twitter &Tweets & 3 & Grey Data & Triangulation & The app was made fully free in response to the Covid-19 pandemic. \\
         via Google Search &Peer reviewed publications & 3 & Secondary research. &Additional context &The app has been studied in various peer-reviewed papers. \\ 
         Exodus Privacy Project & Online reports & 4 & Grey Data & Details of mobile analytics integrated into the app & Their 4 snapshots indicate a variety of mobile analytics have been incorporated~\footnotemark. \\
         \bottomrule
    \end{tabular}
    \caption{Moodspace: data sources}
    \label{tab:moodspace-data-sources}
\end{table}

\footnotetext{\url{https://reports.exodus-privacy.eu.org/en/reports/search/boundless.moodgym/}}


\subsection{Moodspace: Intervention}
None: the Moodspace case study did not include a formal intervention.

\subsection{Moodspace: Outcomes for the company}
The startup was subsequently able to raise a round of funding and grow to six people. The project later returned to be a side-project which is maintained and updated~\citep{alexander2021_linkedin_profile}. % Some details of the funding also available on twitter ``Thank you for so much support on our @Crowdcube campaign! More than 50 investors and £46k raised. You can still invest here: " https://twitter.com/MoodSpaceApp/status/1237753829174710272?s=20 http://crowdcube.com/boundlesslabs (protected behind an intrusive account creation process which I abandoned).
The app has a strong positive User Experience rating of 4.18/5.00~\footnote{\url{https://onemindpsyberguide.org/apps/moodspace/}}.

% The app was made free of charge on 17 March 2020 https://twitter.com/MoodSpaceApp/status/1239942338434215941 and https://www.healthfoundry.org/covid-19-response 


\subsection{Moodspace: Contributions to the research and where they are located in the rest of this thesis}
TBC\pending{Add forward links when the relevant material has been included.}.

\subsection*{Moodspace: Wish list}
{\small
\begin{itemize}
    \item I emailed Ian in April 2021, it'd be useful to receive a reply, not least so he can cross-check what I have included in the thesis.
    \item Add results from analysing the binaries of the apps with exodus-privacy.
\end{itemize}
}

\clearpage

%===================================================================

\section{App-centric: Moonpig}~\label{case-study-overview-moonpig}
Moonpig is an e-commerce business in Europe that sells greeting cards and related gifts online; they operate in the United Kingdom, USA and Australia. They have a highly rated mobile app, with overall ratings of 4.8/5 in Google Play and the Apple App Store. Note some portions of this case study were published in \citep{harty_better_android_apps_using_android_vitals}.

{\renewcommand{\arraystretch}{0.8}% Tighter
\begin{table}[htbp!]
    \centering
    \small
    \setlength{\tabcolsep}{1pt}
    \begin{tabular}{lp{9cm}}
       % Question &Answer  \\
       \toprule
       Website &\url{https://www.moonpig.com/uk/} \\
       Google Play Home & \url{https://play.google.com/store/apps/developer?id=Moonpig.com} \\
       Founded &2000 \\
       Business Domain & Greeting cards and gifts \\
       Business type & e-commerce \\
       Technologies  & Native apps, Robospice, \\
       & AWS, GraphQL, nodeJS, \\
       & Commercetools, ContentStack, ... \\
       Source code  &Closed and not available for research \\
       Analytics used by team &Firebase, Google Play Console \\
       Development Practices & High performance engineering, ATDD, \\
         & micro-services architecture \\
       \midrule
       User base &100,000's for the Android app\\
       Installations &1,000,000+ for the Android app\\
       \midrule
       Research methods &In person interviews, email discussions, remote testing \\
       Analytics collected &Google Play Console with Android Vitals \\
       Research software &They used Vitals-Scraper, otherwise none applicable.\\
       Additional data collected &Interview notes and emails \\
       Active period &June 2019 to July 2019, with updates in Oct 2019 and Feb 2020. \\
       \bottomrule
    \end{tabular}
    \caption{Case Study key facts: Moonpig}
    \label{tab:moonpig_anaytics_overview}
\end{table}
}

\subsection{Moonpig: Background - How the case study came about}
One of the developers of the Android app learned about this research and offered to provide their insights into their use of mobile analytics for their Android app. Both the head of engineering and communication manager approved him doing so and gave permission for the material to be used.

\subsection{Moonpig: Development microcosm}
The engineering organisation consisted of various teams, including a team for the Android app. At the time of the case study, the Android app combined several generations of their architecture and used various third-party libraries. One of these third-party libraries, Robospice, led to a higher crash rate for their Android app on newer releases of Android. The details are discussed in the \nameref{chapter-analytics-in-use} chapter.\pending{Add forward link once I've incorporated the material into that chapter.}

Their software engineering team have been actively involved in encouraging the wider software engineering community to learn and practice good software development practices, for example by hosting Coding DoJos~\footnote{Historical examples available online on twitter \url{https://mobile.twitter.com/moonpigtech} and in a \href{https://www.codurance.com/publications/newsletters/2020-02-13-newsletter}{Codurance newsletter} from February 2020, for example.}. They practiced similar software development practices when developing their production software.

\subsection{Moonpig: Experiences of using mobile analytics}
The development team use Firebase Crashlytics and Google Analytics for diagnostics in addition to the information available in Android Vitals; and estimated they used Android Vitals approximately 30\% of their time to identify flaws and issues related to their Android app.
%
They also incorporated in-app analytics by using Firebase Analytics which recorded analytics related to how the users use the mobile apps and whether errors or other problems occurred while the app was being used. 

Their Android app had one of the highest stability scores, details are in the \nameref{chapter-analytics-in-use} chapter.\pending{Add forward link once I've incorporated the material into that chapter.}

\subsection{Moonpig: Data collected and methods used for collection}
The data was collected though a mix of in-person and online working sessions together with various email conversations. Screenshots and other materials were emailed by the developer to the researcher. 

The developer also ran Vitals Scraper to help evaluate whether it worked beyond the local use as part of various case studies. They sent the JSON results containing the crash clusters collected by Vitals Scraper.


%%%%%%%%%%%%%%%%%%%%%%%%%%%%%%%%%%
%%%%%%%%%%%%%%%%%%%%%%%%%%%%%%%%%%
% Julian to continue from here!!!!
%%%%%%%%%%%%%%%%%%%%%%%%%%%%%%%%%%
%%%%%%%%%%%%%%%%%%%%%%%%%%%%%%%%%%

\begin{table}
    \centering
    \footnotesize
    \tabcolsep=0.12cm
    \begin{tabular}{p{2.4cm}p{2.4cm}rp{2.4cm}p{3cm}p{2.5cm}}
        Data Source & Records & Volumes & Analysis method & Contribution & Remarks \\
        \toprule
          & contemporaneous notes\footnotemark & -1 & &  & 1  \\
         email conversations & GMail & 10\textsuperscript{1} & &  &  \\
         Google Play Console with Android Vitals &Interactive screenshots \& Vitals-scraper outputs &10\textsuperscript{1} & &  &  \\
         \bottomrule
    \end{tabular}
    \caption{Moonpig: data sources}
    \label{tab:moonpig-data-sources}
\end{table}



\subsection{Moonpig: Intervention}
None: the Moonpig case study did not include a formal intervention.

\subsection{Moonpig: Outcomes for the company}
For over two years since the case study started, in June 2019, the Android app has been highly rated by end users in Google Play. According to AppBrain it is in the top 1\% of apps by rating, and in the top 5\% in both ratings (46,000) and downloads (estimated at 3,000,000)~\citep{appbrain_moonpig}.

The company was able to `go public' in February 2021 by listing on the London Stock Exchange.  Their success online through their website and their mobile apps were cited in their IPO~\footnote{Initial Public Offering - listing a company on a stock market.} Prospectus~\citep[p.92]{moonpig2021_ipo_prospectus}.

\subsection{Moonpig: Contributions to the research and where they are located in the rest of this thesis}
TBC\pending{Add forward links when the relevant material has been included.}.

\subsection*{Moonpig: Wish list}
{\small
\begin{itemize}
    \item The chapter and related material will be reviewed by the lead contributor to this case study.
    \item Add results from analysing the binaries of the apps with exodus-privacy.
\end{itemize}
}
\clearpage

%===================================================================

\section{App-centric: Smartnavi}~\label{case-study-overview-smartnavi}\improvement{This is an additional contribution that was easy to add now there's a consistent structure to introduce the case studies.} 
SmartNavi is an unusual opensource project that replaces GPS for navigation when the users are walking. It uses significantly less power than using a true GPS provider. Google promoted the project as an Android experiment \url{https://experiments.withgoogle.com/smartnavi}.

{\renewcommand{\arraystretch}{0.8}% Tighter
\begin{table}[htbp!]
    \centering
    \small
    \setlength{\tabcolsep}{1pt}
    \begin{tabular}{lp{9cm}}
       % Question &Answer  \\
       \toprule
       Website &\url{https://smartnavi.app/home} \\
       Google Play Home & \url{https://play.google.com/store/apps/details?id=com.ilm.sandwich} \\
       Founded & 2014 \\
       Business Domain & Maps \& Navigation \\
       Business type & None, a student project that grew. \\
       Technologies  & Android \\
       & Background Service records steps and direction. \\
       Source code  & Opensource \url{https://github.com/Phantast/smartnavi} \\
       Analytics used by team & Fabric Crashlytics, Firebase Analytics, Google Play Console \\
       Development Practices & Main developer who accepts pull requests. \\
       \midrule
       User base & 10,000's for the Android app \\
       Installations & 50,000+ for the Android app \\
       \midrule
       Research methods &Online interview, email discussions, etc. \\
       Analytics collected &Google Play Console with Android Vitals \\
       Research software & None applicable. \\
       Additional data collected &Interview notes and emails. \\
       Active period & July 2020 \\
       \bottomrule
    \end{tabular}
    \caption{Case Study key facts: Smartnavi}
    \label{tab:smartnavi_anaytics_overview}
\end{table}
}

\subsection{Smartnavi: Background - How the case study came about}
The app was one of the opensource codebases investigated as part of my collaborative research on logging using Google Firebase Analytics. I noticed an issue which I reported to the project's developer and I submitted a pull request to address this issue~\footnote{\url{https://github.com/Phantast/smartnavi/pull/25}} which he accepted and merged. He agreed to be interviewed for this research.

\subsection{Smartnavi: Development microcosm}
The project's creator developed the app as part of his bachelors and masters degrees in Germany. He continued to develop and maintain it afterwards. The project does not have any automated tests, he tests the app interactively.

\subsection{Smartnavi: Experiences of using mobile analytics}
Mobile Analytics are used for several purposes including A/B experiments using Firebase Analytics and Crashlytics for crash reporting. Several years ago (probably 2016) he spent a lot of time where he actively focused on fixing crashes being reported in Fabric Crashlytics (since superseded by Firebase Crashlytics). The few crashes that remain are ones either too impractical to investigate (e.g. on a few unbranded low-end Android devices he has no access to) or in a third-party library that only occurs again on a few unusual devices.

\subsection{Smartnavi: Data collected and methods used for collection}
Contemporaneous notes and email discussion.

\subsection{Smartnavi: Intervention}
None: the Smartnavi case study did not include a formal intervention.

\subsection{Smartnavi: Outcomes for the company}
Bugs found by mobile analytics were fixed in the app including at least one raised by an Android user who wrote a 1-star review. The usability of the app was improved through the use of Firebase Analytics. 

\subsection{Smartnavi: Contributions to the research and where they are located in the rest of this thesis}
TBC.

\subsection*{Smartnavi: Wish list}
{\small
\begin{itemize}
    \itemsep0em
    \item Ask Christian for more info and any links related to fixing Crashlytics reported crashes.
    \item Also request screenshots (or readonly access) to Google Play Console with Android Vitals and Crashlytics.
    \item Add results from analysing the binaries of the apps with exodus-privacy.
\end{itemize}
}

\clearpage

%===================================================================

\section{App-centric: Kiwix}~\label{case-study-overview-kiwix}
The Kiwix project started as a way to make Wikipedia available offline, globally~\citep{sutherland2014_wikimedia_on_kelson}. The team wrote software and implemented systems to do so and have been working closely with the WikiMedia Foundation for many years. They also make vast amounts of other content available, including StackOverflow for example. Innumerable teams, projects, and people use Kiwix in various guises.

The project has multiple opensource projects including software tools that download content from various sources including Wikipedia, web servers that serve content, and a wide range of desktop and mobile apps. I have been a part-time volunteer with the Kiwix project since 2014. I have contributed to several of these opensource projects including the Android app where I have helped with automated testing and with continuous builds, amongst other areas.

{\renewcommand{\arraystretch}{0.8}% Tighter
\begin{table}[htbp!]
    \centering
    \small
    \setlength{\tabcolsep}{1pt}
    \begin{tabular}{lp{9cm}}
       % Question &Answer  \\
       \toprule
       Website &\url{https://www.kiwix.org/en/} \\
       Google Play Home & \url{https://play.google.com/store/apps/dev?id=9116215767541857492} \\
       Founded & 2007 \\
       Business Domain & Education \\
       Business type & Not for profit Association \\
       Technologies  & Native platform apps \\
       & And associated software tools. \\
       Source code  & Opensource \url{https://github.com/kiwix} \\
       Analytics used by team & \textit{none}\footnotemark, Google Play Console \\
       Development Practices & core developers combined with part-time volunteers. \\
       \midrule
       User base & 100,000's for the Android app \\
       Installations & 1,000,000+ for the Android app \\
       \midrule
       Research methods &Embedded volunteer part-time developer\footnotemark. \\
       Analytics collected &Google Play Console with Android Vitals \\
       Research software & None applicable. \\
       Additional data collected &Emails from Google Play Console, development artefacts. \\
       Active period & March 2019 - March 2020 \\
       \bottomrule
    \end{tabular}
    \caption{Case Study key facts: Kiwix}
    \label{tab:kiwix_anaytics_overview}
\end{table}
}

\footnotetext{The policy of the Kiwix project is not to integrate any mobile analytics into the apps in order to protect the safety and privacy of users, however they do use the anonymous platform analytics provided by app stores.}
\footnotetext{I have been a part-time volunteer contributor to the Kiwix project since 2014.}

\subsection{Kiwix: Background - How the case study came about}
The reliability of the Android app, as reported by Android Vitals, had been a concern for the core project leaders and by early 2019 was a major concern yet the active developers for the Android app had not been able to materially improve the reliability. We agreed it would be worth trying to focus on improving the reliability and that the planned project wide hackathon in Stockholm would be a great opportunity to do so. In parallel the project had funding to pay for an experienced Android lead developer several days a week and one of his objectives would be to improve the reliability of the app. 

\subsection{Kiwix: Development microcosm}
The codebase and development artefacts are all opensource, much of the communications is also publicly accessible, e.g. using GitHub issues, IRC and Slack~\footnote{\url{https://wiki.kiwix.org/wiki/Communication}}. In addition there are emails and informal discussions, and so on.

The project uses free-to-use services, for instance GitHub (and previously SourceForge) for the codebases. The Continuous Build service was Travis-CI at the time of the case study (since replaced by GitHub Actions) and the project had a free to use account on the BitBar device testing farm~\footnote{\url{https://bitbar.com/}}. The code base included a mix of unit tests and automated app tests. In short, the project used relatively well honed and complete tools and practices to manage their codebase, software contributions, perform code quality checks and run the automated tests on both virtual and actual Android devices.

At the start of the case study there were two parallel releases in progress, the production 2.x release and a planned 3.0 release.

One of the many benefits of the project’s openness is the visibility into the developers who have developed and maintained the source code \url{https://github.com/kiwix/kiwix-android/graphs/contributors}. Many of the contributors joined as volunteers through Google Summer of Code~\citep{google_summer_of_code} or Google Code-in \url{https://codein.withgoogle.com/archive/}~\footnote{Google Code-in was shutdown and the history archived by Google in 2020.}, and several of these became core contributors for a year or more, and some of these now work for leading technology businesses. There have been occasional contributions from Google software developers who volunteer their time.

\subsection{Kiwix: Experiences of using mobile analytics}
The project leaders have consistently chosen \emph{not} to incorporate any analytics in their applications in order to protect the users of Kiwix software. Nonetheless they were willing to use analytics provided by Google Play which are collected by Android rather than by the app. The expectation is that users who are willing to let Android collect this data are unlikely to be at risk from using Kiwix.

\subsection{Kiwix: Data collected and methods used for collection}
The development artefacts including the codebase \url{https://github.com/kiwix/kiwix-android} and the issues \url{https://github.com/kiwix/kiwix-android/issues} are both public and freely available. They include the history of the source code and of issues raised pertaining to the case study. 

The analytics artefacts were collected interactively and by using vitals-scraper. Various contemporaneous notes were made during the case study and hackathon.  

\subsection{Kiwix: Intervention}
The primary intervention was for several of the team to address several of the most prevalent crashes during the Kiwix hackathon. Follow on bug fixes, based on mobile analytics outputs, led to further improvements in the stability of the main Android app. When the custom apps were refreshed using the improved codebase their stability also increased.

\subsection{Kiwix: Outcomes for the project}
The project was able to significantly reduce the measured crash rates~\footnote{Some of the apps do not have many users and the crashes were too few for Google Play to report on them. This topic is discussed in the \nameref{chapter-tools-and-their-artefacts} chapter.} for all the shipping Android apps despite some additional complications related to Android app bundles. 

\subsection{Kiwix: Contributions to the research and where they are located in the rest of this thesis}
TBC.

\subsection*{Kiwix: Wish list}
{\small
\begin{itemize}
    \itemsep0em
    \item I've requested some information on who had access to Google Play Console for the Kiwix apps. This would be useful for the analytics-in-use chapter. Ditto the same info for Catrobat, now I'm able to contact them again.
    \item Well there are many things I'd wish we'd done during the case study however the past is impractical to change. At some point I'll aim to reengage with the project and see whether we can address the stability issues that have emerged in 2021.
    \item Add results from analysing the binaries of the apps with exodus-privacy.
\end{itemize}
}

\clearpage

%===================================================================

\section{Catrobat}~\label{case-study-overview-catrobat}
The Catrobat project was created and is actively developed by a team in the Graz University of Technology, Austria~\footnote{\url{https://www.tugraz.at/en/home/}}. It consists of the flagship Pocket Code app, several custom branded derivatives, and the increasingly popular Pocket Paint app which emerged from the Pocket Code app where it remains as a subset of the overall Pocket Code's functionality. 

{\renewcommand{\arraystretch}{0.8}% Tighter
\begin{table}[htbp!]
    \centering
    \small
    \setlength{\tabcolsep}{1pt}
    \begin{tabular}{lp{11cm}}
       % Question &Answer  \\
       \toprule
       Website &\url{https://catrobat.org/} \\
       Google Play Home & \url{https://play.google.com/store/apps/developer?id=Catrobat} \\
       Founded & 2010 \\
       Business Domain & Education \& Visual programming. \\
       Business type & Not for profit association. \\
       Technologies  & Android \\
       & Jenkins CI~\url{https://jenkins.catrob.at/job/Catroid/}  \\
       & JIRA~\url{https://jira.catrob.at/} \\
       Source code  & Opensource \url{https://github.com/Catrobat} \\
       Analytics used by team & Fabric Crashlytics, Google Play Console \\
       Development Practices & Sophisticated~\footnotemark \\
       \midrule
       User base & 100,000's for the Android app \\
       Installations & 1,000,000+ for the Android app \\
       \midrule
       Research methods &Hackathon, online interviews, email discussions, etc. \\
       Analytics collected &Fabric Crashlytics, Google Play Console with Android Vitals \\
       Research software & None applicable. \\
       Additional data collected &Interview notes and emails \\
       Active period & November 2019 to March 2020 \\
       \bottomrule
    \end{tabular}
    \caption{Case Study key facts: Catrobat}
    \label{tab:blank_case_study_anaytics_overview}
\end{table}
}

\footnotetext{Sophisticated in this context is explained in \secref{catrobat-development-microcosm}}

\subsection{Catrobat: Background - How the case study came about}
A PhD student who was part of the Catrobat project discovered my research at the MobileSOFT 2019 conference. Their flagship Pocket Code app had a persistently high, chronic crash rate and my early research for the Kiwix project seemed worth evaluating in case it could help them reduce the high crash rate. The case study included two main events, 1) a hackathon in November 2019 and 2) participation in a pre-conference workshop in Poland in February 2020. We agreed on a hackathon for a couple of reasons: Kiwix found them beneficial and productive, and the Catrobat team wanted to have a short, unusual and interesting way to try out the concept of using mobile analytics outputs to improve reliability that would also appeal to their developers. 

Their project leads selected Pocket Code as the app we would use for the field experiment as it had the higher crash rate and was also a significantly more complex app than their other core app Pocket Paint which was relatively self-contained and simple in terms of both functionality and codebase.



\subsection{Catrobat: development microcosm}~\label{catrobat-development-microcosm}
The development microcosm was \textbf{sophisticated} and one of the most mature in terms of opensource mobile app ecosystems~\footnote{I have worked in opensource for 15 years, including at Google, eBay, and other organisations so I say this based on my professional experience.}. For example: the project has automated tests. In an admittedly small sample only 9 of 19 opensource Android app projects had any automated tests, analysed by~\citet{silva2016_an_analysis_of_automated_tests_for_mobile_android_apps} (40.6\% of 1000 projects have automated tests~\citep[p. ]{cruz2019_guess_what_test_your_app}); the project has automated tests and CI/CD, only 14.7\% of 1000 opensource Android apps do so~\citep[p. ]{cruz2019_guess_what_test_your_app}~\footnote{Interestingly they did not evaluate either Pocket Code or Pocket Paint. They also discounted `self-hosted' CI including Jenkins which Catrobat uses extensively.}.  \textbf{TODO} check the code coverage for Catrobat.\emph{``only 19 are actually promoting full test coverage with coverage tracking services"}~\citep[p. ]{cruz2019_guess_what_test_your_app}. Of the 1000 projects\emph{``147 apps with both CI/CD and tests"}~\citep[p. ]{cruz2019_guess_what_test_your_app}. The project uses code quality tools and aims for zero warnings from these tools. They also developed their own custom test automation framework and integrated Fabric Crashlytics into their flagship Pocket Code Android app.

% Possibly also cite: 10.1109/ICSME.2017.47 10.1109/ICST.2015.7102609 10.1109/QRS-C.2019.00064 For now I'll keep writing!

The Android app in this case study is an extremely and unusually well researched and properly developed app and codebase. The project started in 2010, has had over 1,300 contributors, 4 million downloads, and 350 thousand active users, and is used in 180+ nations in 60+ languages~\citep{catrobat_project}. There are at least 216 contributors for the Android Pocket Code app~\citep{github_catroid}.

Many perceived good practices were and are assiduously applied on an ongoing basis, for instance:~\href{https://github.com/Catrobat/Catroid}{Test-Driven Development, Clean Code}~\citep{catrobat_first_steps_into}, a documented consistent~\href{https://github.com/Catrobat/Catroid/wiki/Workflow}{workflow} and \href{https://github.com/Catrobat/Catroid/wiki/Creating-a-pull-request}{Pull Requests}, and \href{https://jenkins.catrob.at/job/Catroid/}{Continuous Integration}. The codebase is far more complex than the Kiwix Android apps and the app is significantly richer in terms of the features and functionality~\citep{mueller2019_pocketcode}.

\subsection{Catrobat: Experiences of using mobile analytics}
The project had already incorporated Fabric Crashlytics into the Pocket Code Android app. %TODO confirm when and revise accordingly.
For their apps in Google Play they also had Google Play Console with Android Vitals. They did not appear to use either source of mobile analytics materially in their software development practices, however they were aware of the ongoing high crash rate for the Pocket Code app. %TODO check through their JIRA history for signs of them using either mobile analytics tool as a source of issues that they wanted to address.

\subsection{Catrobat: data collected and methods used for collection}
The project team provided access to Google Play Console for all their Android apps and also to their Fabric Crashlytics account. They also provided access to their JIRA and Jenkins systems (read access to both system is public). Their codebases are all public and available as opensource. The majority of the data for this case study is public, in JIRA tickets and in the codebase for Pocket Code. Outputs from the two mobile analytics tools were captured interactively and using vitals scraper for Google Play Console with Android Vitals. There are some email communications and similarly various handwritten field notes in notebooks.

\subsection{Catrobat: Intervention}
The key intervention was organise a weekend hackathon with an open invitation for any of the extended development team to participate. During day 1 of the hackathon, after informal introductions and a discussion about the aims of the hackathon, the next task was to create tickets in JIRA for the top 10 crash clusters and the top 10 ANR clusters as reported by Android Vitals. These were reported in JIRA during the first hour of the hackathon. The participants, in ones or twos, selected one of these tickets and worked on it. They then selected another ticket and worked on that one. They continued for approximately 5 hours until late afternoon that day. The event closed with a communal meal at a local pizzaria. The participants chose not to continue with day 2 of the hackathon (which was on the Sunday), instead they preferred to work on the issues during the normal working week (Monday to Friday). Several of them did so and continued to work on various tickets raised in the hackathon. The project team made two related releases of the Pocket Code Android app, with cumulative fixes in these releases.

\subsection{Catrobat: Outcomes for the project}
A group of six members of the Catrobat development team were able to usefully address various causes of the most prevalent failures of the Pocket Code app in production. 

\subsection{Catrobat: Contributions to the research and where they are located in the rest of this thesis}
TBC\pending{Add forward links when the relevant material has been included.}.

\subsection*{Catrobat: Wish list}
{\small
\begin{itemize}
    \itemsep0em
    \item Ask why Pocket Paint is on F-Droid~\url{https://f-droid.org/en/packages/org.catrobat.paintroid/} but not Pocket Code.
    \item It'd be helpful to re-establish communications with the project team in order to follow up on the results of the hackathon and on their current practices. I can also analyse their issues database to see whether they're actively using mobile analytics.
    \item Check whether they use code coverage measures and if so, what the numbers are.
    \item Extension work: it'd be interesting to apply \url{https://luiscruz.github.io/android_test_inspector/} to the project's apps.
    \item \textbf{Where should I write up the hackathon?} here? or in the next few chapters?
    \item Add results from analysing the binaries of the apps with exodus-privacy.
\end{itemize}
}  % end \small

\clearpage

%===================================================================

\section{App-centric: C1}~\label{case-study-overview-C1}
% A couple of sentences to introduce them

{\renewcommand{\arraystretch}{0.8}% Tighter
\begin{table}[htbp!]
    \centering
    \small
    \setlength{\tabcolsep}{1pt}
    \begin{tabular}{lp{9cm}}
       % Question &Answer  \\
       \toprule
       Website &\textit{Confidential} \\
       Google Play Home & \textit{Confidential} \\
       Founded & \textit{Confidential} \\
       Business Domain & A major corporation \\
       Business type & A public company \\
       Technologies  & \textit{Confidential} \\
       Source code  &Closed and not available for research \\
       Analytics used by team & Microsoft App Center, other commercial products, and Google Play Console \\
       Development Practices & Multiple teams working on the Android app. \\
       \midrule
       User base & 1,000,000's for the Android app \\
       Installations & 1,000,000's for the Android app \\
       \midrule
       Research methods &Consultant. \\
       Analytics collected &App Center, Google Play Console with Android Vitals \\
       Research software & None applicable. \\
       Additional data collected &Additional Analytics and logs, details are confidential. \\
       Active period & Q4 2020 - Q2 2021 \\
       \bottomrule
    \end{tabular}
    \caption{Case Study key facts: C1}
    \label{tab:blank_case_study_anaytics_overview}
\end{table}
}

\subsection{C1: Background - How the case study came about}
The researcher accepted a consulting engagement with the corporation and was asked to assist one of their key projects. This project included an Android app, online APIs developed by the larger project team, and other apps, systems and services. It also incorporated other internal systems, APIs, and services provided by other development teams. 

Owing to confidentially and other contractual obligations details have been removed from this case study any examples are independently able to be corroborated. 

\subsection{C1: Development microcosm}
The overall project team comprised over 100 people working directly on the product. Developers worked in a matrixed organisation~\citep[describes matrix organisations in detail]{stuckenbruck1979_the_matrix_organization}. Multiple groups of developers worked on the Android app. 

\subsection{C1: Experiences of using mobile analytics}
The project included multiple mobile analytics services integrated into their apps, including the Android app. However the development teams seldom appeared to use them proactively. 


\subsection{C1: Data collected and methods used for collection}
The data was collected contemporaneously during the consulting engagement and subsequently. Details cannot be provided here.

\subsection{C1: Intervention}

\subsection{C1: Outcomes for the company}

\subsection{C1: Contributions to the research and where they are located in the rest of this thesis}

\subsection*{C1: Wish list}
{\small

\begin{itemize}
    \item NB: I cannot publish the results of analysing the app binary with exodus-privacy as doing so may leak additional clues about the app.
\end{itemize}
}

\clearpage

%===================================================================

\section{Summary of the overview of the case studies}~\label{case-study-overview-summary}
TBC.\pending{To complete once the rest of the contents have been added to this chapter.}

