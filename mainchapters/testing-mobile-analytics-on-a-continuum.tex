\subsection{Testing mobile analytics tools on a continuum}~\label{section-testing-mobile-analytics-tools-on-a-continuum}

\buzzwords{Translucence, various levels of opacity, white+grey+white box testing}

 The research includes three methods of testing in order to evaluate aspects of the mobile analytics tools.\improvement{FYI this subsection is a w-i-p to explore whether the following material is clearer and more faithful to the research than the current local app experiments and FOSS contributions. If it is, this section will be woven into the rest of the methodology chapter. The catalyst was a suggestion by Arosha on \nth{8} Dec 2021.} 
 They are all forms of micro-experiment. The methods of testing are loosely based on familiar concepts of black-box, white-box, and grey-box testing where there are various levels, or depths, of opacity in the system under test.

The tests include: testing the developer support channels, testing the process and experience of providing contributions to several opensource projects owned by mobile analytics tool providers, and testing mobile analytics tools by treating them as a black-box system under test (SUT) though the development of several small-scale Android apps. These small-scale apps were used to exercise the app store's pre-launch facilities and several of the in-app mobile analytics tools.\pending{This needs connecting to the material that follows.} 

Add a table: mapping the methods to the granularity, show the connection to the colour of box and how they connect to the object of the testing.\pending{Add a table showing the mapping}


\newthought{Developer support channels}


\newthought{Mini, local app experiments} 
Essentially the same story as it the local app experiments method.

\newthought{Testing contribution processes} 
Code contributions to opensource packages of commercial mobile analytics projects.
%
Essentially the same material as in the FOSS contributions method. The hypothesis was that some companies make their proprietary code available as opensource but do not actually accept contributions. The experiments involved creating small improvements to one or two aspects of their opensource projects and then creating pull requests to discover their acceptance process and what the results would be in terms of them accepting the pull requests.

\newthought{Working with the Iteratively product}
This experiment emerged from earlier collaborations with the founders of Iteratively. They offered free access to their product. The experiment incorporated their product into a small Android app where the analytics was routed by their SDK to Amplitude. The experiment provided a translucent view into their SDK and its behaviour in terms of the integration with Amplitude and also the support for custom implementations.

\newthought{Collaboration with Google is also a form of testing}
In some ways the collaboration with Google's engineering team was a form of testing to see whether they would firstly accept the contributions as valid and of merit, and then whether they would improve their product to address flaws reported to them during the collaboration. Google... TBC.

%%%%%%%%%%%%%%%%%% See also %%%%%%%%%%%%%%%%%%%%%%
% https://en.wikipedia.org/wiki/Transparency_meter
% https://en.wikipedia.org/wiki/Opacity_(optics)