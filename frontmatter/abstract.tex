% File: abstract.tex
% Author: V?ctor Bre?a-Medina
% Description: Contains the text for thesis abstract
%
% UoB guidelines:
%
% Each copy must include an abstract or summary of the dissertation in not
% more than 300 words, on one side of A4, which should be single-spaced in a
% font size in the range 10 to 12. If the dissertation is in a language other
% than English, an abstract in that language and an abstract in English must
% be included.

% Improving Application Quality using Mobile Analytics

%% A Good Abstract succinctly answers these four topics:
%% \begin{itemize}
%%    \item What was the purpose of your research?
%%    \item How was your research designed?
%%    \item What were your findings?
%%    \item What were your conclusions?
%% \end{itemize}

%%% Feedback from Yijun 15 July 2022, to be applied when I've made more progress on the body of the thesis.
%%% After the 2nd paragraph, I would like to see some text about "What's before the study" and after the 3rd paragraph, some text about "what's after the study". At the moment, the phrases used on the 4th paragraph such as "some", "may be", "potential" gives reader an impression that it is an unfinished study. I would like you to consider rephrasing them into firm sentences, even if "may be" is used, they are used as a firm answer that there is no "yes", nor "no" answers to the questions.
%%% If you are uncertain, you can always write a new draft and pass it on me to see. Btw, abstract does not have to be restricted to 1 page. If you really intend it to be 1 page, please cut the other wordings so that some "conclusive" findings are highlighted and visible.


\chapter*{Abstract}
\label{the-abstract}
\begin{SingleSpace}

The purpose of this research is to learn how mobile analytics can help
real-world developers improve the quality of their apps efficiently and effectively. The research also considers the effects of mobile analytics in terms of the artefacts developed and maintained by the development team and also researches the key characteristics of a range of mobile analytics tools and services. %The success of their apps may depend, at least in part, on how their apps are rated using externally determined metrics established by an app store.

Research Design: the research takes a developer-oriented perspective of using two complementary sources of data: 1) platform-level analytics, using Android Vitals as the primary analytics tool, and 2) in-app analytics.
%
Action research techniques included roles of embedded developer, guide, and observer across different projects I was involved in. Hackathons were used to experiment with the speed and ability to find and address issues reported in the analytics tools. 
%
The research is intended to facilitate ease of future research and reproducibility,~\emph{e.g.} by using  open-source projects as the code, bug reports, \emph{etc.} are all published and available. Their apps have a combined active user base of over 3,000,000 users. Many of these apps use a mainstream crash analytics library which was used to complement and contrast the results provided in the primary analytics tool.
%
This research was complemented by collaborating with professional developers who provided additional examples and results.

Using mobile analytics helped to reduce crash rates markedly, quickly and effectively by applying techniques described here.
Various limitations and flaws were found in the analytics tools, these provide cause for concern as they may affect the app's placement in the app store, revenues, and also make some issues in the apps harder to identify, prioritise, and fix. We identified ways to compensate for many of these and developed open-source software to facilitate additional analysis. Flaws and bugs were reported to the Android Vitals team at Google who acknowledged they would fix several of them.
%
Several bugs were hard to reproduce, partly as Google deliberately hide some pertinent details from the data they gather. Nonetheless the developers were able to ameliorate or fix the bugs for some issues even when they were not able to reproduce them. 

Android Vitals shows the potential of how the combination of an app store and platform could be used to improve the quality of apps without users needing to actively participate. Some crashes were hard to reproduce and may be impractical to find before the app is released to end users. Developers can determine comparative improvements in their releases, such as whether they fixed a bug, by using Android Vitals and similar analytics tools; \textit{i.e.} mobile analytics may help teams to determine whether they have improved the quality of their app \textit{even with flaws and limitations in the mobile analytics}.
\end{SingleSpace}
\newpage
