% File: abstract.tex
% Author: V?ctor Bre?a-Medina
% Description: Contains the text for thesis abstract
%
% UoB guidelines:
%
% Each copy must include an abstract or summary of the dissertation in not
% more than 300 words, on one side of A4, which should be single-spaced in a
% font size in the range 10 to 12. If the dissertation is in a language other
% than English, an abstract in that language and an abstract in English must
% be included.

% Mobile Application Testing and Quality Improvement using Software Analytics


%% A Good Abstract succinctly answers these four topics:
%% \begin{itemize}
%%    \item What was the purpose of your research?
%%    \item How was your research designed?
%%    \item What were your findings?
%%    \item What were your conclusions?
%% \end{itemize}

\chapter*{Abstract}
\begin{SingleSpace}
% \textbf{What was the purpose of your research?} 
\initial{T}his research explores the potential of using two complementary sources of data to improve the testing of Android apps. Google provides the first source, known as Android Vitals, using data gathered from Android devices of  users who opt-in to providing various performance data. Google claims \textit{"Exhibiting bad behavior in vitals will negatively affect the user experience in your app and is likely to result in bad ratings and poor discoverability on the Play Store"}% https://developer.android.com/topic/performance/vitals
, therefore testing for these behaviors can help teams address bad behaviors to reduce the adverse effects on users and on the discoverability of the app.
% Possibly remove the second source from the mini-thesis, discuss.
The second source is gathered by the app, either directly or using a third-party library. Mobile analytics is a  well-established approach to gathering data from apps and may be a viable approach to provide this service. Logging is another mature approach but seldom used by mobile apps. This research aims to provide insights into commonalities and differences between logging and mobile analytics as implementations for this second source of data.
The improvements to testing include several distinct aspects: to help us review and assess the effects of our testing of apps now in production, to aid in bug localisation and reproduction, and to generate ideas and insights for future testing.
% investigates aspects of using data generated by software 

% \textbf{How was your research designed?}
% From the perspective of within the development team.
The research uses several mature opensource Android apps with active install bases of over 300,000 to evaluate Android Vitals, these apps do not collect any in-app data to protect the privacy of their users. Physical Android devices, selected by analysing data obtained from Android Vitals, were used to evaluate bug localisation and reproduction. We created software to 'scrape' data from Android Vitals in order to obtain the relevant data from the 1,000's of crash traces; Google do not provide an API.

% Possibly comment-out the following paragraph depending on how hermetic we want this mini-thesis to be...
The research was designed to incorporate mobile analytics libraries to one or more Android apps. Practical difficulties in finding a suitable existing project led to the development of a small prototype opensource Android app. 

% \textbf{What were your findings?} 
Several of the more popular Android apps had unusually high crash rates and crashes were more frequent on newer releases of Android. % Secondary finding: nonetheless the apps had high ratings in Google Play.
Android Vitals gathers non-personally identifiable information for Android apps. They also only provide reports and related data once volumes are sufficient to pass Google's thresholds to protect the privacy of uses, which means Android Vitals is less useful for apps with lower userbases. % TBD whether we can quantify the thresholds for when Google chooses to display various reports.
Android Vitals is in active development by Google with changes and features occurring several times during the research; I discovered bugs in Android Vitals and reported them to Google who acknowledged they would fix several of them; (the discussions are ongoing and Google are very interested in this aspect of my research).
Several bugs were hard to reproduce, partly as Google deliberately remove some pertinent details from the crash reports they gather. The data and reports was useful to devise new tests for the apps.

% \textbf{What were your conclusions?}
Android Vitals shows the potential of how the combination of their app store and platform could be used to help evaluate the effectiveness of testing of the apps being used by end-users, nonetheless some crashes were hard to reproduce and may be impractical to find before the app is released to end users. Developers may be able to determine comparative improvements in their releases, such as whether they have fixed a bug, by using Android Vitals; \textit{i.e.} scores and reports based on Android Vitals may help teams to determine whether they have improved the quality of their Android app \textit{even if they are not able to test as much as they might wish}.
\end{SingleSpace}
\newpage

\section{Advice from reviewers}
\marian{For the mini-viva: 
Create a clear problem statement to frame it, lots of signposting, and a strong table of contents for the structure. Aim for 20 pages. Show critical depth in the literature review. Use rigour in the methodology, and show the path to the conclusions.}


Articulate RQ's in the introduction to set the frame for the rest of the work.

What I want to know, what I need to do to discover it (approach), what I did.


Big questions and focal q's for what I need to know. Fine to refine the questions as I write.

\clearpage

\section{Publications}
\begin{enumerate}
    \item 	Giovanna Avellis, Julian Harty, Yijun Yu. "Towards mobile twin peaks for app development". In MOBILESoft '17 Proceedings of the 4th International Conference on Mobile Software Engineering and Systems. 2017. \href{https://doi.org/10.1109/MOBILESoft.2017.10}{DOI 10.1109/MOBILESoft.2017.10}~\cite{avellis_harty_yu_towards_mobile_twin_peaks}
    
    \textbf{My contribution:} I co-authored the work and presented the work at the conference.
    \item Julian Harty. "Google Play Console: Insightful Development using Android Vitals and Pre-Launch Reports". In MOBILESoft '19 Proceedings of the 6th International Conference on Mobile Software Engineering and Systems. 2019. \href{https://doi.org/10.1109/MOBILESoft.2019.00019}{DOI: 10.1109/MOBILESoft.2019.00019} ~\cite{harty_google_play_console_insightful_development_using_android_vitals_and_pre_launch_reports}
    
    \textbf{My contribution:} I wrote and presented the paper and a poster at the conference.
    
    \item Julian Harty, Matthias Müller. "Better Android Apps using Android Vitals".  WAMA 2019: Proceedings of the 3rd ACM SIGSOFT International Workshop on App Market Analytics August 2019 Pages 26–32 \href{https://doi.org/10.1145/3340496.3342761}{DOI 10.1145/3340496.3342761}~\cite{harty_better_android_apps_using_android_vitals}
    
    \textbf{My contribution:} I wrote most of the paper and did most of the research. Joseph Reeve developed the Vitals Scraper software with my input. Matthias Müller contributed the section on Catrobat.
    
    \item ICST 2020 Doctoral Symposium
    
    \item TAIC-PART 2020 fast abstract
    
    \item TAIC-PART 2020 Joint paper
    
    \item MOBILESoft 2020 SRC
    
\end{enumerate}
% How to create numbered lists https://www.overleaf.com/learn/latex/Lists

\subsection{Books}
\begin{enumerate}
    \setcounter{enumi}{8}
    \item Julian Harty, Antoine Aymer. "The Mobile Analytics Playbook: A Practical Guide to Better Testing". Hewlett Packard Enterprise, 2015."
    ISBN: 978-0-9970694-0-2 (PDF) 978-0-9970694-1-9 (Print)
    
    \textbf{My contribution:} I wrote most of the book and did much of the background research.
    
    \item "The Mobile Developer's Guide to the Galaxy" series of books, from \nth{6} to \nth{18} Editions. A widely distributed and read book that introduces the many and various challenges a developer of mobile apps needs to consider. The series is a collaborative project with multiple authors and editors where the book is published in print and electronically. The content is released under a creative commons license. 
    
    \textbf{My contributions:} I variously co-edited the book, wrote and revised various chapters both solely and with other authors. My main chapters include: Testing, Mobile Analytics, and Collecting and Understanding User Feedback. The current edition is freely available online at~\url{https://www.open-xchange.com/resources/mobile-developers-guide-to-the-galaxy/}
\end{enumerate}
\clearpage