\section{Publications}
My various public contributions follow in this section.

\subsection{Peer-reviewed papers}
\begin{enumerate}
    \item 	Giovanna Avellis, Julian Harty, Yijun Yu. \emph{``Towards mobile twin peaks for app development"}. In MOBILESoft '17 Proceedings of the 4th International Conference on Mobile Software Engineering and Systems. 2017. \href{https://doi.org/10.1109/MOBILESoft.2017.10}{DOI 10.1109/MOBILESoft.2017.10}~\cite{avellis_harty_yu_towards_mobile_twin_peaks}
    

    \item Julian Harty. \emph{``Google Play Console: Insightful Development using Android Vitals and Pre-Launch Reports"}. In MOBILESoft '19 Proceedings of the 6th International Conference on Mobile Software Engineering and Systems. 2019. \href{https://doi.org/10.1109/MOBILESoft.2019.00019}{DOI: 10.1109/MOBILESoft.2019.00019} ~\cite{harty_google_play_console_insightful_development_using_android_vitals_and_pre_launch_reports}
    
    \item Julian Harty, Matthias Müller. \emph{``Better Android Apps using Android Vitals"}.  WAMA 2019: Proceedings of the 3rd ACM SIGSOFT International Workshop on App Market Analytics August 2019 Pages 26–32 \href{https://doi.org/10.1145/3340496.3342761}{DOI 10.1145/3340496.3342761}~\cite{harty_better_android_apps_using_android_vitals}
    
    \textbf{My contribution:} I wrote most of the paper and did most of the research. Matthias Müller contributed the section on Catrobat. Note: Joseph Reeve developed the Vitals Scraper software with my input.
    
    \item Isabel Evans, Chris Porter, Mark Micallef, Julian Harty. \emph{``Stuck In Limbo With Magical Solutions: The Testers’ Lived Experiences of Tools and Automation"}. HUCAPP 2020. 
    
    \textbf{My contribution:} I helped inspire the research and contributed to revising and improving the paper.
    
    \item Julian Harty. ICST 2020 - Doctoral Symposium \emph{``How Can Software Testing be Improved by Analytics to Deliver Better Apps?"}.
    
    \item Julian Harty. TAIC-PART 2020 \emph{``Fast Abstract: Data Dynamics for Testing Systems"}.
    
    \item Isabel Evans, Chris Porter, Mark Micallef, Julian Harty. TAIC-PART 2020.~\emph{``Test Tools: an illusion of usability?"}.
    
    \item Julian Harty. MOBILESoft 2020 - Student Research Competition.~\emph{``Improving App Quality Despite Flawed Mobile Analytics"}.
    
    \item Julian Harty. ICGSE 2020.~\emph{``Designing Engineering Onboarding for 60+ Nationalities"}. (Nominated for best paper.)

\end{enumerate}
% How to create numbered lists https://www.overleaf.com/learn/latex/Lists

\subsection{Private reports}
\begin{enumerate}

    \setcounter{enumi}{9}
    \item Julian Harty. Written for Google Engineering Team for Google Play and Android Vitals~\emph{``Report on flaws discovered in Google Play Console and Android Vitals"}.

\end{enumerate}

\subsection{Books}
\begin{enumerate}
    \setcounter{enumi}{10}
    \item Julian Harty, Antoine Aymer. \emph{``The Mobile Analytics Playbook: A Practical Guide to Better Testing"}. Hewlett Packard Enterprise, 2015."
    ISBN: 978-0-9970694-0-2 (PDF) 978-0-9970694-1-9 (Print)

    \item \emph{``The Mobile Developer's Guide to the Galaxy"} series of books, from \nth{6} to \nth{18} Editions. A widely distributed and read book that introduces the many and various challenges a developer of mobile apps needs to consider. The series is a collaborative project with multiple authors and editors where the book is published in print and electronically. The content is released under a creative commons license. 
    
    \textbf{My contributions:} I variously co-edited the book, wrote and revised various chapters both solely and with other authors. My main chapters include: Testing, Mobile Analytics, and Collecting and Understanding User Feedback. The current edition is freely available online at~\url{https://www.open-xchange.com/resources/mobile-developers-guide-to-the-galaxy/}
\end{enumerate}

\subsection{Software}
Please see the ~\hyperlink{software-contributions-chapter}{\emph{\nameref{software-contributions-chapter}}} chapter in the appendices for details. 
\clearpage
