\section{Glossary with Abbreviations}

\begin{table}
\resizebox{\textwidth}{!}{%
\footnotesize
\begin{tabular}{@{}>{\bfseries}p{5.3cm}p{9.2cm}@{}}


Analogue Feedback & Here, used to identify feedback from human sources. Generally this feedback is intentional \emph{i.e.} the person chooses to provide the feedback. The feedback may vary massively from one instance to another.\\

Android Vitals & \\

ANR & Abbreviation for \emph{Application Not Responding} \\

Application~\mbox{Not Responding} & A term originated by Google to identify when an Android app stops responding to users for over 5 seconds. \\

Developer account & For the purposes of this research we use Google's concept of a Google Play Developer Account, described variously in~\cite{google_play_how_to_use_the_play_console, google_play_launch_checklist}. Developers need to register for an account, pay a one-time fee, and agree to abide by various policies, terms, and conditions. Someone who has a developer account can invite other people to share aspects of their account.\\

Digital Feedback & Here used to identify feedback from running software, generated automatically. The feedback is structured and the structure generally consistent.\\
Google Play Console & \\

EULA & End User License Agreement. \\

Event demographics & \emph{`Percentage of events triggered by each age group and gender.'} (Source: Google Firebase Analytics tooltip).\\

GDPR &\href{https://gdpr-info.eu/}{General Data Protection Regulation}. \\ 

Platform &Here, a platform includes an operating system, related software resources (including APIs and services), and an ecosystem including an app store. Google Android and Apple's iOS are both mobile platforms. They run on end-user devices and capture aspects of the software that runs on those devices. Platform services can monitor apps from the time they are installed until they are removed. \\

QoE & Abbreviation for \emph{Quality of Experience} \\

Quality of Experience & Oft used in Mobile Telecommunications to measure network transmission characteristics. Here used so we can more easily identify and consider the quality of experiences such as \emph{User Experience}. TBD whether to focus on perceived experience as perceived by end users of an app.\\

Service Provider &An organisation, often commercial, which includes software, and online services (and the people who provide these) that offers developers optional facilities such as in-app- analytics, crash reporting, messaging, feedback mechanisms, \emph{etc.} \\

Stability &A software quality identified initially by HP as part of their FunDex. It was subsequently used by Google to identify and measure two related indications of software failures: crashes and freezes. \\ %SHOULD_DO add references to cite both sources. 

UX & Abbreviation for User Experience \\
\end{tabular}}
\caption{Glossary with Abbreviations}
\end{table}

% Thanks to https://tex.stackexchange.com/questions/503784/how-to-arrange-tabular-in-alphabetical-order for the table formatting.
% and to get it to fit on a page https://tex.stackexchange.com/questions/27097/changing-the-font-size-in-a-table
% And the following to help me stop "Application Not Responding" from being split mid-word: https://texfaq.org/FAQ-wdnohyph
% https://tex.stackexchange.com/questions/121832/breaking-words-at-the-end-of-line/121835
%  https://tex.stackexchange.com/questions/380613/hyphenat-not-hyphenating-words-with-hyphens

\clearpage