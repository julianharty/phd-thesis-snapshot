\section{Glossary with Abbreviations}
\label{the-glossary}
\newcounter{headercounter}
\newcommand{\boxedheader}[1]{%
  \stepcounter{headercounter}%
  \begin{tikzpicture}[remember picture]
  \path[thick,rounded corners=1mm,draw=gray,fill=gray!40!white,overlay]
    (-0.2,-0.25\baselineskip) rectangle ([xshift=2mm,yshift=0.75\baselineskip]pic cs:tableright\roman{headercounter});
  \end{tikzpicture}%
  #1
  \tikzmark{tableright\roman{headercounter}}\\%
}

\begin{comment}
% This is hidden as it is a hack that doesn't work well however it does set the heading box to full width unlike the one below. However I've spent far more time on this than it justifies at this stage.

\newtcolorbox{mybox}[1][]{%
    enhanced jigsaw,
    boxrule=0.5pt,
    left=0pt,right=0pt,top=0mm,bottom=0mm, 
    colback = gray!20,
    height=2cm,
    arc=1mm,
}

\newcolumntype{R}{>{\raggedleft\arraybackslash}p{1cm}}
\newcolumntype{C}[1]{>{\centering\arraybackslash}p{#1}}

\begin{mybox}
\begin{longtabu} to \linewidth {@{\,}XZ@{}}
    \textbf{Term}  & \textbf{Description and remarks} 
\end{longtabu}%
\end{mybox}%
\end{comment}

\newcolumntype{Z}{>{\raggedright\arraybackslash}p{10cm}}

\begin{longtabu} to \linewidth {@{\,}X[1]Z@{}} %\linewidth {@{}Xcr@{}}
\boxedheader{\textbf{Term} & \textbf{Description and remarks} \extracolsep{\fill} } \\
Analogue Feedback & Here, used to identify feedback from human sources. Generally this feedback is intentional \emph{i.e.} the person chooses to provide the feedback. The feedback may vary massively from one instance to another. \\

Android Vitals~\label{glossary_android_vitals} & \\

ANR & Abbreviation for \emph{Application Not Responding}. \\

API &Application Programming Interface. \\
Application~\mbox{Not Responding} & A term originated by Google to identify when an Android app stops responding to users for over 5 seconds~\citep{google_play_view_crashes_and_ANR_errors}. \\

Crash Analytics &Analysis of application crashes collected automatically by software. Identify groupings and patterns in the crashes to provide developers with the opportunity to debug and fix them without needing to spend time reproducing the problem. (Paraphrased from~\citep{ibm_mobile_foundation_7_1_app_crash_analytics}. \\

Developer account & For the purposes of this research we use Google's concept of a Google Play Developer Account, described variously in~\cite{google_play_how_to_use_the_play_console, google_play_launch_checklist}. Developers need to register for an account, pay a one-time fee, and agree to abide by various policies, terms, and conditions. Someone who has a developer account can invite other people to share aspects of their account. \\

Digital Feedback & Here used to identify feedback from running software, generated automatically. The feedback is structured and the structure generally consistent. \\
Google Play Console & A Web-based application which is the primary user interface for developers who release Android apps in the Google Play Store. Google also provides mobile apps that provide a subset of the capbabilities of Google Play Console. \\

Error & ``An \textbf{error} is a system state that may cause a failure."~\citep{abreu2007_on_the_accuracy_of_spectrum_based_fault_localization}, based on~\citep{avizienis2004_basic_concepts_and_taxonomy}.\\

EULA & End User License Agreement. \\

Event demographics & \emph{`Percentage of events triggered by each age group and gender.'} (Source: Google Firebase Analytics tooltip).\\

Explicit Feedback &For example, in the form of reviews and comments~\citep{maalej2016_towards_data_driven_requirements_engineering}. \\

Failure & ``A \textbf{failure} is an event that occurs when delivered service deviates from correct service. "~\citep{abreu2007_on_the_accuracy_of_spectrum_based_fault_localization}, based on~\citep{avizienis2004_basic_concepts_and_taxonomy}.\\

Failure Repositories &A central location in which data pertaining to failures of software is stored and managed. Notes: Definition based on a service, ~\href{https://languages.oup.com/google-dictionary-en/}{Oxford Languages}, jointly provided by Google and Oxford University Press. Mentioned but not defined in various sources including:~\citep{maalej2016_towards_data_driven_requirements_engineering}, and exemplified in~\citep{cfdr_usenix}.\\

Fault & ``A fault is the cause of an error in the system."~\citep{abreu2007_on_the_accuracy_of_spectrum_based_fault_localization}, based on~\citep{avizienis2004_basic_concepts_and_taxonomy}.\\

FunDex & A score devised by Hewlett-Packard (HP) as part of their AppPulse Mobile product offering. \emph{``The score starts at 100, but drops with each problem the app has ..."}~\citep{hall2015_HP_courts_developers_with_tools_for_monitoring_mobile_apps}. It combines scores for UI performance, stability, and resource usage. \\

GDPR &\href{https://gdpr-info.eu/}{General Data Protection Regulation}.\\ 

Implicit Feedback &Automatically collected information about software usage~\citep{maalej2016_towards_data_driven_requirements_engineering}.\\

Interaction screen &Interaction screen conveys the screen as both an output and input device (rather than the term touch screen)~\footnote{This term is originated by Prof. Arosha K. Bandara, thank you.}. \\

Mobile App Analytics Tools & ``Mobile app analytics tools collect and report on in-app data pertaining to the operation of the mobile app and the behavior of users within the app, as well as aggregate market data on apps across public app stores"~\citep{gartner2015_market_guide_for_mobile_app_analytics}. \\

Operational Analytics & ``Operational analytics: Provides visibility into the availability and performance of mobile apps in relation to device, network, server and other technology factors. Operational analytics are essential to capture and fix unexpected app behavior (such as crashes, bugs, errors and latency) that can lead to user frustration and abandonment of the app."~\citep{gartner_what_is_mobile_app_analytics_software} \\

PII &Personally Identifiable Information. \emph{``Personally Identifiable Information; Any representation of information that permits the identity of an individual to whom the information applies to be reasonably inferred by either direct or indirect means."}~\citep{nist_pii}\\ 

Platform &Here, a platform includes an operating system, related software resources (including APIs and services), and an ecosystem including an app store. Google Android and Apple's iOS are both mobile platforms. They run on end-user devices and capture aspects of the software that runs on those devices. Platform services can monitor apps from the time they are installed until they are removed. \\

Mobile Analytics Policy~\label{glossary-mobile-analytics-policy} &Defines the `rules of engagement' when incorporating mobile analytics as part of the development, maintenance and where appropriate the operation of a mobile app. \\

Mobile Analytics Strategy~\label{glossary-mobile-analytics-strategy} &Describes how the team intend to work within their mobile analytics policy to achieve the aims and objectives of using mobile analytics. \\

QoE & Abbreviation for \emph{Quality of Experience}. \\

Quality of Experience & Oft used in Mobile Telecommunications to measure network transmission characteristics. Here used so we can more easily identify and consider the quality of experiences such as \emph{User Experience}. TBD whether to focus on perceived experience as perceived by end users of an app. \\

Risk & ``A risk is an unwanted event that has negative consequences."~\citep{pfleeger2000_risky_business} \\

Service Provider &An organisation, often commercial, which includes software, and online services (and the people who provide these) that offers developers optional facilities such as in-app- analytics, crash reporting, messaging, feedback mechanisms, \emph{etc.} \\

Stability &A software quality identified initially by HP as part of their FunDex~\citep{calleosoftware_AppPulseMobile}. The same term was subsequently used by Google to identify and measure two related indications of software failures: crashes and freezes. \\ %SHOULD_DO add references to cite both sources. 

UX & Abbreviation for User Experience. \\

\caption{Glossary with Abbreviations} \\
\end{longtabu}


%\end{table}

% Thanks to https://tex.stackexchange.com/questions/503784/how-to-arrange-tabular-in-alphabetical-order for the table formatting.
% and to get it to fit on a page https://tex.stackexchange.com/questions/27097/changing-the-font-size-in-a-table
% And the following to help me stop "Application Not Responding" from being split mid-word: https://texfaq.org/FAQ-wdnohyph
% https://tex.stackexchange.com/questions/121832/breaking-words-at-the-end-of-line/121835
%  https://tex.stackexchange.com/questions/380613/hyphenat-not-hyphenating-words-with-hyphens

% Replaced the previous structure with longtabu, a package that's caused some furore in the TeX community to enable the glossary to run to several pages
% https://tex.stackexchange.com/questions/201419/longtabu-rounded-box-in-the-header provided the main example
% I reached the above from https://tex.stackexchange.com/questions/201387/how-to-combine-mdframed-and-tabu however I abandoned the mdframed eventually as my table was longer than a page.
% https://sites.google.com/site/simonthelwall/home/latex provided some useful additional notes and has lots of other useful examples, including working with citations and tufte document templates.

% Possible alternatives
% If I can get my head around the following:
% https://tex.stackexchange.com/questions/501097/split-table-on-two-pages 

% Or try applying the following which will add column headings
% https://tex.stackexchange.com/a/456976/88466 

\clearpage