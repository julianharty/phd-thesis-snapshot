\section{A major corporation case study}

\textbf{Context} Acceptable use of PII. Fast growth of the team and very rapid initial growth of use of the service. High crash rates led to abandonment by users. 

\textbf{Relevance}
A wealthy corporation with large engineering teams can choose from any of the commercial mobile analytics tools and/or develop their own. They need to deliver software at scale and that scales. They have far larger professional software development teams than any of the other case-studies and face different challenges. They already use multiple mobile analytics tools, their challenge is to harness them consistently and productively to improve the `health' of their software and service to help staunch their loss of users because of software failures. In parallel they aim to launch new features that need to work for millions of users and serve both business-to-business and business-to-consumer users.

% A major corporation with millions of paying customers launch a free service in their main territory. Take up is rapid. 

\subsection{Methodology}
Consultant and advisor to the team with access to the key commercial sources of client-side analytics and access to the source code of the Android app and related library project.

\textbf{Priorities} data pipelines, ability to query across the many data sources without silos, 

\textbf{Some example issues} multiple analytics tools and data sources incorporated into the software and services, and yet extremely high ANR rates, very high crash rates. DNS error, how the error was discovered. 

\subsection{Lessons learned and/or confirmed}
Not all users of clients update, and a buggy release may leave a large stain on the statistics for the app. As the development team size increases so does the tendency to entropy in the implementation and use of embedded analytics. Furthermore, the team need to consistently check and apply the results of the many and various analytics services if they are to manage and improve the reliability and performance of their software.

The tools need to scale to billions of events per day.

Android Vitals provides a unique source of information, on ANR errors. Microsoft Windows does not offer equivalent reporting to the app developers. %TBD what Apple provides.

Blind spots exist, as do inconsistencies in the various end-to-end data collection and reporting in each analytics offering. 

