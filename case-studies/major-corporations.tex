\section{Corporate engineering case studies}
\label{section-corporate-engineering-case-studies}
\textbf{Relevance}
Brand name corporations with large engineering teams can choose from any of the commercial mobile analytics tools and/or develop their own. They need to deliver software at scale and that scales. They have far larger professional software development teams than any of the other case-studies and face different challenges. They already use multiple mobile analytics tools, their challenge is to harness them consistently and productively to improve the `health' of their software and service to help staunch their loss of users because of software failures. In parallel they aim to launch new features that need to work for millions of users and serve both business-to-business and business-to-consumer users.


\subsection{A major technological corporation}
\textbf{Context} Acceptable use of PII. Fast growth of the team and very rapid initial growth of use of the service. High crash rates led to abandonment by users. 


% A major corporation with millions of paying customers launch a free service in their main territory. Take up is rapid. 

\subsection{Methodology}
Consultant and advisor to the team with access to the key commercial sources of client-side analytics and access to the source code of the Android app and related library project. Worked with one of the product engineering teams for a fast growing product in the covid-19 era of 2020 and beyond.

\textbf{Priorities} data pipelines, ability to query across the many data sources without silos, 

\textbf{Some example issues} multiple analytics tools and data sources incorporated into the software and services, and yet extremely high ANR rates, very high crash rates. DNS error, how the error was discovered. 

\subsection{Lessons learned and/or confirmed}
Not all users of clients update, and a buggy release may leave a large stain on the statistics for the app. As the development team size increases so does the tendency to entropy in the implementation and use of embedded analytics. Furthermore, the team need to consistently check and apply the results of the many and various analytics services if they are to manage and improve the reliability and performance of their software.

The tools need to scale to billions of events per day.

Android Vitals provides a unique source of information, on ANR errors. Microsoft Windows does not offer equivalent reporting to the app developers. %TBD what Apple provides.

Blind spots exist, as do inconsistencies in the various end-to-end data collection and reporting in each analytics offering. 



\subsection{Uber}
MUST-DO write up notes from 14-Jan-2021.

Cite various recent public articles and videos on Uber's engineering practices and team size.


\subsection{Dynamics of sizes of development teams}
Development teams range from one person part-time to many people who work full time on particular mobile apps. A twitter survey that finished on \nth{2} January 2021 had 818 votes; the results are shown in Table~\ref{tab:gergelyorosz_twitter_team_size_survey_02_jan_2021}. %Source \url{https://twitter.com/GergelyOrosz/status/1345288831029956610}. 
The author, Gergley Orosz, asked anyone with more than 20 mobile engineers to add their company name in a reply. At least 22 people did so, they follow in approximate~\footnote{As the contents of answers varied an absolute order is hard to establish.} descending order.

\begin{table}[htbp!]
    \centering
    \begin{tabular}{c|c}
        Mobile app team size &Percent  \\
        Up to 5 devs &53.5\% \\
        5-20         &25.9\% \\
        More than 20 &8.9\% \\
        More than 50 &11.6\% \\
    \end{tabular}
    \caption{Native app mobile development team size~\citep{gergelyorosz2021_twitter_mobile_app_poll}}
    \label{tab:gergelyorosz_twitter_team_size_survey_02_jan_2021}
\end{table}

In addition to the survey respondents with teams Companies with more than 20 engineers include: Uber (100+), % Why is the Uber App So LArge?? https://www.youtube.com/watch?v=zmeCYiD0hnE 
WayFair (100+), Gojek Indonesia (80+ on Android, iOS similar), % 50M+ downloads, 3M6 ratings, a super-app https://play.google.com/store/apps/details?id=com.gojek.app 
Twitter (70+ for iOS, 70+ for Android), Rappi (70 android, 60 iOS), iFood (50+ Android consumer app), Android music app (50+ per mobile platform), Target (80), BumbleEng (40 Android devs, similar for iOS), Xing (30+ for Android), Skyscanner (`~30 per platform'), Garmin (about 30 per platform, excluding those working on internal libraries), `>50 Mercado Libre | nasdaq: Meli', Just Eat (20+ per platform), hotels.com have several subteams - Expedia estimated at 50 per platform, `many native devs at Walmart', unnamed enterprise app 40 Android app developers, 80\% via agencies, `At 
@Falabella\_India we are 29 engineers working on Falabella ecommerce app. 12 are Android Engineers, 13 are iOS Engineers and 4 are QA Engineer.', N26 (`more than 20 ios/android engineers'), 19 in total at thecodeside.com, SoundCloud (`under 20 if you count per platform')~\citep{gergelyorosz2021_twitter_mobile_app_poll}.
%%%%
% COULD-DO cross-reference the development teams with their apps, possibly in tabular format to see if there's a pattern between team size and approximate userbase of the app. Note unless the actual active user counts are available they would need to be estimated based on the installation buckets Google Play Console provides. As I've discovered these counts vary significantly so we'd need to guesstimate and provide error-bars. A bubble chart might be useful.

\subsection{i1}

Crash rates increased inadvertently through changes in the in house networking code paradoxically intended to improve the code quality of the networking related code. One of the changes added a network interceptor (a standard piece of functionality provided by the OkHttp library) to log networking calls and responses. This code inadvertently only worked when the networking response had a `happy path' response. The revised code was code-reviewed and tested as usual. When the app was released with the revised networking code and the rollout into the userbase increased the release management section of Google Play Console reported an increase in crashes, however the rollout was not paused or halted at the time, possibly the reports hadn't been noticed by many in the team at the time? By the time this release was installed on the majority of the userbase the daily crash rate had climbed  to around 10\%, over double the crash rate of previous recent releases.

MUST-DO describe how the networking issues were addressed, automated tests, modified code, new release, steady and ongoing improvement in the measured crash rate. COULD-DO also discuss the distinct yet related changes to the networking code in the separate API codebase.

The experience of the networking crashes are similar to those reported in an Industry report from Apteligent in September 2016 where they found 20\% of crashes were correlated with a network issue at the time. The authors stated the causation figure was closer to 7\% of all crashes on iOS and Android [were related to network-related code]~\citep{apteligent2016_data_report_network_crash_edition}. %See also https://www.slideshare.net/apteligent/apteligent-choosing-the-right-sdks-to-optimize-app-performance-68423213 which restates the crash rate. It also had useful tips on assessing an SDK and the provider of the SDK.

FWIW More information on similar refactoring available in `Legacy Project Refactoring: Handling API Errors with Retrofit2 and RxJava' \url{https://medium.datadriveninvestor.com/legacy-project-refactoring-handling-api-errors-with-retrofit2-and-rxjava-c9f8b23c7673}


\subsection{Some observations in common}
For the larger corporations where the multiple engineering teams combine to develop, maintain and operate mobile apps multiple sources of analytics are used. These often include a mix of external commercial services together with at least one proprietary logging and analytics service. In these examples the organisations see the benefits of integrating their various analytical sub-systems and invest significantly to integrate and streamline the various sources to enable and facilitate consistent reporting across products, platforms, and apps.
