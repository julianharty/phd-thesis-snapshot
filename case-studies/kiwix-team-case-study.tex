\section{Kiwix Android Apps}

\subsection{Introduction of Kiwix Android Case Study}
Relies solely on analytics and reports provided by the platform. We chose the most sophisticated and complex of the Android apps, which also had the highest Crash rate at the time. By applying what the team learned about crashes reported in Android Vitals the team was able to reduce the crash rate of this app several fold. When the improved codebase was used to refresh various custom apps their crash rates also decreased several fold.


\subsection{Case study of working with the Kiwix team}
As reported in \cite{harty_google_play_console_insightful_development_using_android_vitals_and_pre_launch_reports} and \cite{harty_better_android_apps_using_android_vitals} the Kiwix Android app had a very high overall crash rate caused by several significant flaws in the app. The project team released version 2.5 of the main Kiwix app in July 2019. As figure \ref{fig:kiwix_crash_rate_drops_v2_5} shows, the crash rate decreased significantly as version 2.5. In the last 30 days the crash rate was 1.87\% down from 5.07\% in February 2019.

One of the major changes in version 2.5 was the replacement of the in-house download utility with the default Android Download Manager\cite{kiwix_release_2_5_0}. The in-house version was a major source of crashes, and the replacement obviated a class of crashes, however it did so at a price in terms of functionality and usability. The in-house download utility allowed users to pause and resume downloads, and it would complete failed partial downloads. Users also received updates on the progress of the downloads, important when they often took many minutes or even hours or days in some cases (such as for multi-GB downloads over poor, slow, unreliable connections on low-end devices).
\begin{figure}
    \centering
    \includegraphics[width=\textwidth]{images/android-vitals-screenshots/kiwix-crash-rate-drops-with-v2_5.png}
    \caption{Kiwix Crash Rate Drops with V2.5 Release}
    \label{fig:kiwix_crash_rate_drops_v2_5}
\end{figure}

Following initial discussions about the crashes being reported in Android Vitals for version 2.5.0 of the Kiwix application, we collaborated on a week-long hackathon in Stockholm in August 2019. There, the developers ended up fixing some of the causes of the most frequent crashes with a surprisingly small amount of code of under 25 lines (including 10 lines of text added to the release log)\footnote{\url{https://github.com/kiwix/kiwix-android/pull/1388}}.

Several developers for the Kiwix project, the lead developer in particular, have been actively reviewing crashes reported by Android Vitals, filing issues, and addressing the causes of the crashes in order to reduce the crash rate and improve the app's stability. Evidence issues that mention crash: 133 closed, 6 open. \footnote{\url{https://github.com/kiwix/kiwix-android/issues?utf8=\%E2\%9C\%93&q=is\%3Aissue+crash}}
%TODO in a longer work, analyse each issue to identify the source of the crash.

Several releases later, each with various changes and improvements aimed at fixing causes of crashes the crash rate was materially lower than when we started, at the time of writing the overall crash rate for the last 7 days is 0.54\% which is inflated because the rash rate for the previous release (3.1.2) spiked at 1.38\%, compared to 0.18\% for release (3.0.5 -  the last production release) and 0.25\% for the recently released fix (3.1.3).

\subsection{Summary of Kiwix Android Case Study}
TBC