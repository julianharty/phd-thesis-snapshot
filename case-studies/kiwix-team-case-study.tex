\section{Kiwix Android Apps}
\label{section-kiwix-case-study}
\subsection{Introduction of Kiwix Android Case Study}
Relies solely on analytics and reports provided by the platform. We chose the most sophisticated and complex of the Android apps, which also had the highest Crash rate at the time. By applying what the team learned about crashes reported in Android Vitals the team was able to reduce the crash rate of this app several fold. When the improved codebase was used to refresh various custom apps their crash rates also decreased several fold.


\subsection{Case study of working with the Kiwix team}
As reported in \cite{harty_google_play_console_insightful_development_using_android_vitals_and_pre_launch_reports} and \cite{harty_better_android_apps_using_android_vitals} the Kiwix Android app had a very high overall crash rate caused by several significant flaws in the app. The project team released version 2.5 of the main Kiwix app in July 2019. As figure \ref{fig:kiwix_crash_rate_drops_v2_5} shows, the crash rate decreased significantly as version 2.5. In the last 30 days the crash rate was 1.87\% down from 5.07\% in February 2019.

One of the major changes in version 2.5 was the replacement of the in-house download utility with the default Android Download Manager\cite{kiwix_release_2_5_0}. The in-house version was a major source of crashes, and the replacement obviated a class of crashes, however it did so at a price in terms of functionality and usability. The in-house download utility allowed users to pause and resume downloads, and it would complete failed partial downloads. Users also received updates on the progress of the downloads, important when they often took many minutes or even hours or days in some cases (such as for multi-GB downloads over poor, slow, unreliable connections on low-end devices).

\begin{figure}[htbp!]
    \centering
    \includegraphics[width=\textwidth]{images/android-vitals-screenshots/kiwix-crash-rate-drops-with-v2_5.png}
    \caption{Kiwix Crash Rate Drops with V2.5 Release}
    \label{fig:kiwix_crash_rate_drops_v2_5}
\end{figure}

Following initial discussions about the crashes being reported in Android Vitals for version 2.5.0 of the Kiwix application, we collaborated on a week-long hackathon in Stockholm in August 2019. There, the developers ended up fixing some of the causes of the most frequent crashes with a surprisingly small amount of code of under 25 lines (including 10 lines of text added to the release log)\footnote{\url{https://github.com/kiwix/kiwix-android/pull/1388}}.

Several developers for the Kiwix project, the lead developer in particular, have been actively reviewing crashes reported by Android Vitals, filing issues, and addressing the causes of the crashes in order to reduce the crash rate and improve the app's stability. Evidence issues that mention crash: 133 closed, 6 open. \footnote{\url{https://github.com/kiwix/kiwix-android/issues?utf8=\%E2\%9C\%93&q=is\%3Aissue+crash}}
%TODO in a longer work, analyse each issue to identify the source of the crash.

Several releases later, each with various changes and improvements aimed at fixing causes of crashes the crash rate was materially lower than when we started, at the time of writing the overall crash rate for the last 7 days is 0.54\% which is inflated because the rash rate for the previous release (3.1.2) spiked at 1.38\%, compared to 0.18\% for release (3.0.5 -  the last production release) and 0.25\% for the recently released fix (3.1.3).

\subsection{Examples of real-time crashes}
Each of these examples exemplifies at least one characteristic of the reports that are provided by Android Vitals in Google Play.

\subsubsection{EsxRenderBucket::AddUnbucketedEntries(...)}
This crash is one of the most frequent crashes reported in early October 2020 and has been occurring on an ongoing basis according to Android Vitals for the current release of the Chemistry \& Physics simulations app (release 2020-04). Through analysis of the reports this crash only affects one release of the app (release 5200950) and it does not occur on the other three releases (6200950, 4200950, 3200950). It occurs on multiple manufacturer's device models, and on Android 9.0, 8.1, and 8.0). 

This crash is a native crash and mainly occurs within the context of \texttt{/system/app/Chrome/Chrome.apk} \textit{the web browser app created by Google!} The Kiwix apps rely on an embedded web browser, which is generally Google's Chrome browser, to render (\emph{i.e.} display) the content to the user\footnote{It also appears for other variants of the Android Chrome browser on some devices e.g. \texttt{/data/app/com.android.chrome-bAmCl9DcPfmqf3oKL54Efg==/base.apk (offset 0xbe7000)} and also the Android WebView component~\texttt{/data/app/com.google.android.webview-9ShSu\_81V02zu4ENrAjvJA==/lib/arm/libwebviewchromium.so} (also created by Google).}.

\begin{listing}[H]
\caption{Crash Cluster: EsxRenderBucket::AddUnbucketedEntries} \label{code:crash_cluster_add_unbucketed_entries}
\tiny
\begin{minted}{cpp}
*** *** *** *** *** *** *** *** *** *** *** *** *** *** *** ***
pid: 0, tid: 0 >>> org.kiwix.kiwixcustomphet <<<

backtrace:
  #00  pc 00000000001535a0  /vendor/lib/egl/libGLESv2_adreno.so (EsxRenderBucket::AddUnbucketedEntries(EsxCmdBufType, unsigned int)+132)
  #01  pc 0000000000152b17  /vendor/lib/egl/libGLESv2_adreno.so (EsxRenderBucket::BucketRenderingCmds(EsxRenderBucketParams*)+740)
  #02  pc 0000000000186a6d  /vendor/lib/egl/libGLESv2_adreno.so (EsxContext::BucketRenderingCmds(int)+712)
  #03  pc 00000000000e6987  /vendor/lib/egl/libGLESv2_adreno.so (EsxContext::BindDrawFramebuffer(EsxFramebufferObject*)+178)
  #04  pc 00000000000b6a5d  /vendor/lib/egl/libGLESv2_adreno.so (EsxContext::GlBindFramebuffer(unsigned int, unsigned int)+332)
  #05  pc 0000000001b8c659  /system/app/Chrome/Chrome.apk (offset 0x80c000)
\end{minted}

\end{listing}

Of the 53 crash clusters reported over the last 60 days for all Android versions and version 5200950 of the app, installed from Google Play, 16 of the 53 crash clusters are for this crash.

Searching local logs, generated using the opensource software~\texttt{vitals-scraper} that we created as part of this research we can see the same crash cluster has occurred in some, not all, of the Kiwix applications. The command used to find the files that contain this crash cluster is:~\texttt{grep -c EsxRenderBucket::AddUnbucketedEntries * | sort -t ':' -k 2 -g}. This returns a list of the files, sorted by the number of matches for the string found in each of the files. Here are the entries with at least one match.

\begin{listing}[H]
\caption{Logs that include crash cluster: EsxRenderBucket::AddUnbucketedEntries} \label{code:vitals_scraper_logs_add_unbucketed_entries}
\footnotesize
\begin{minted}{text}

android-crash-clusters-org.kiwix.kiwixcustomphet_1572958874833.json:2
android-crash-clusters-org.kiwix.kiwixmobile_1599898794048.json:2
phet-1-day-android-crash-clusters_1569599996989.json:2
android-crash-clusters-org.kiwix.kiwixcustomphet_1572966654935.json:3
android-crash-clusters-org.kiwix.kiwixmobile_1577913956806.json:3
android-crash-clusters-org.kiwix.kiwixcustomphet_1574380641173.json:4
android-crash-clusters-org.kiwix.kiwixcustomphet_1572976000060.json:5
android-crash-clusters-org.kiwix.kiwixcustomphet_1577913523667.json:5
android-crash-clusters-org.kiwix.kiwixcustomphet_1601883786819.json:5
wikimed-60-days-android-crash-clusters_1568705009571.json:6
phet-7-days-android-crash-clusters_1569484818005.json:13
android-crash-clusters-org.kiwix.kiwixcustomphet_1573403158401.json:14
android-crash-clusters-org.kiwix.kiwixcustomphet_1599898464809.json:14
phet-60-days-android-crash-clusters_1568703927627.json:18
android-crash-clusters-org.kiwix.kiwixcustomphet_1572903426940.json:21
phet-60-days-android-crash-clusters_1565933377493.json:21
android-crash-clusters-org.kiwix.kiwixcustomphet_1572812538185.json:25
\end{minted}

\end{listing}

From these results the crash occurs most often in the Physics \& Chemistry simulation custom app (these include the phrase `phet'\footnote{`phet' is the term used for the source of the contents used in this custom app, i.e. the source of the various Chemistry and Physics simulations, written in HTML5. They are extremely rich in terms of their content and dynamic rendering as they provide interactive, dynamic simulations.} as part of the filename. It also occurred relatively infrequently in two other of the apps: five times in the core Kiwix app, and six times in the Wikipedia in English app. The core Kiwix app can be used with the same contents as the project bundles in the custom apps, so some of the crashes~\emph{might} be for the same content, we don't know enough from the stack trace to determine the contents. the reasons for the error in the custom WikiMed app are not known at this stage. 

Searching online, using Google Search, for \texttt{EsxRenderBucket::AddUnbucketedEntries} finds a similar stack trace occurs with the Unity SDK and it appears to be related to a particular chipset:
\begin{itemize}
    \item \href{https://developer.qualcomm.com/forum/qdn-forums/software/adreno-gpu-sdk/67924}{Forums - Help with crash in {\footnotesize libGLESv2\_adreno.so (EsxRenderBucket::AddUnbucketedEntries)}} 2020
    % \item \href{https://github.com/flutter/flutter/issues/38676}{/system/vendor/lib/egl/libGLESv2_adreno.so #38676} 2019
    % \item \href{https://stackoverflow.com/questions/29728931/libglesv2-adreno-so-game-crash-in-galaxy-note-4-and-lollipop-5-0}{libGLESv2_adreno.so game crash in Galaxy Note 4 and Lollipop 5.0} 2015
    \item \href{https://forum.unity.com/threads/unity-2019-android-build-crashes-on-devices-using-adreno-506-gpu.712229/}{Unity 2019 Android build crashes on devices using Adreno 506 GPU}. This bug report includes several different method names where the crash occurs. Various developers report the issue, \href{https://forum.unity.com/members/waldgeist.1371619/}{Waldgeist} reporting the one with this method name.
\end{itemize}

The bug appears hard for developers to reproduce and from the app developer's perspective it happens in software they cannot fix themselves. 

As Ogien reports in~\href{https://forum.unity.com/threads/unity-2017-2-crashes-vs-5-6-2f1.511995/}{Unity 2017.2 Crashes vs 5.6.2f1} they may be able to identify correlations (in this example using a screenshot from Android Vitals, I believe, as shown in Figure~\ref{fig:unity-2017-2-android-vitals-annotated-graph}). The Unity support team state the bug may have been fixed and the developer promised to try the new release and report back at the time, in 2018, however they have not done so online at least\footnote{This user was online more recently, including \nth{24} September 2020.} so that issue has an indeterminate result from a research perspective.

\begin{figure}[htbp!]
    \centering
    \includegraphics[width=12cm]{images/unity-forum/unity-2017-2-android-vitals.jpg}
    \caption{Unity 2017.2 Android-Vitals annotated graph (\url{https://forum.unity.com/threads/unity-2017-2-crashes-vs-5-6-2f1.511995/}}
    \label{fig:unity-2017-2-android-vitals-annotated-graph}
\end{figure}

MUST-DO Discuss the augmented crash stack trace utility, how it works, and why we did~\emph{not} use it in Google Play apps.


\subsection{Lessons learned from this case study}

As~\citep{kidwell2015_toward_fault_taxonomy_application_of_software_analytics} notes, previous research by~\citep{weider1998_software_fault_prevention_in_coding_and_RCA} nearly half the faults were introduced during coding and \emph{``...many of the faults were preventable"}. These results were borne out in the Kiwix Android case study where some of the most frequent crashes were null pointer errors in the Java code. For the Kiwix Android project one of the longer term challenges was the youth of many of the volunteer contributors including some of the development leads who were often teenagers and pre-undergraduate level software developers who wouldn't have the expertise expected of professional software developers\footnote{(They often joined via Google Code-in~\citep{google_code_in_archive} or Google Summer of Code~\citep{google_summer_of_code}).}. While there may well be training techniques and software tools, including Android Lint, that may have been able to find some of the causes of the crashes reported by Android Vitals it's unlikely that these volunteers would choose to use those tools or want to undergo training. And as interviews with developers demonstrated the perceived effort of dealing with static analysis reports and the volume of false positives mean developers don't use static analysis tools very often to find bugs~\citep{johnson2013_why_dont_devs_use_static_analysis}.

\subsection{Summary of Kiwix Android Case Study}
TBC