\section{Greentech Apps}

\section{Introduction to the Greentech Apps case study}
The aims and objectives of this case study include:

\begin{itemize}
    \item \textbf{A linear increase (+1)} : validation the methods described in my research are repeatable and scale to additional apps beyond the previous case studies.
    \item \textbf{Additional examples of characteristics of Google Play Console (+1)} :
    \item \textbf{A closed source case study (?)} : The previous case studies were all with opensource codebases so the code was available for bug investigation purposes. For this case study the source code and build processes are treated as a black box (it may potentially become a grey box case study if the development team share details of their engineering practices, etc.)
\end{itemize}

\subsection{Background to the Greentech Apps case study}
A set of Android apps developed and provided by Greentech Apps Foundation. They are described as modern Islamic Applications, according to their website \url{https://gtaf.org/}. The project encourages voluntary contributions, for instance to provide translations~\url{https://greentech.oneskyapp.com/collaboration/}. Their apps are popular, and well regarded. % MUST_DO add data on usage and ratings.

At the start of the case study (June 2020) the team had ten Android apps published in Google Play~\url{https://play.google.com/store/apps/dev?id=7665838187257770408}.

\subsection{Summary of the Greentech Apps case study}
TODO complete this section, reflecting the topics raised in the introduction to the case study.