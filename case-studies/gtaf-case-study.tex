\section{Greentech Apps}

\subsection{Introduction to the Greentech Apps case study}
The aims and objectives of this case study include:

\begin{itemize}
    \item \textbf{A linear increase (+1)} : validation the methods described in my research are repeatable and scale to additional apps beyond the previous case studies.
    \item \textbf{Additional examples of characteristics of Google Play Console (+1)} :
    \item \textbf{A closed source case study (?)} : The previous case studies were all with opensource codebases so the code was available for bug investigation purposes. For this case study the source code and build processes are treated as a black box (it may potentially become a grey box case study if the development team share details of their engineering practices, etc.)
\end{itemize}

\subsection{Background to the Greentech Apps case study}
A set of Android apps developed and provided by Greentech Apps Foundation. They are described as modern Islamic Applications, according to their website \url{https://gtaf.org/}. The project encourages voluntary contributions, for instance to provide translations~\url{https://greentech.oneskyapp.com/collaboration/}. Their apps are popular, and well regarded. % MUST_DO add data on usage and ratings.
The project started in 2016 and the development team was predominantly volunteers until around 2018 when the first engineer was employed. In April 2020 the development team added two part-time paid developers and there are plans to grow the employed team including someone with UI and UX expertise. The team are funded through voluntary donations.

At the start of the case study (June 2020) the team had ten Android apps published in Google Play~\url{https://play.google.com/store/apps/dev?id=7665838187257770408}. Of these apps, three are their core apps, and a couple are overdue an engineering revamp.  Two analytics tools are used for their core apps: Firebase Crashlytics and the default combination of Google Play Console with Android Vitals.

Their development team maintain their issues lists in public, the source code is private. They use a variety of languages and frameworks to develop their apps.

\textbf{MUST-DO} check which analytics libraries are embedded in which of their apps. 

\href{https://play.google.com/store/apps/details?id=com.greentech.quran}{Al Quran (Tafsir \& by Word)} includes \href{https://reports.exodus-privacy.eu.org/en/reports/com.greentech.quran/latest/}{Google Crashlytics, Google Firebase Analytics}.

\href{https://play.google.com/store/apps/details?id=com.greentech.hadith}{	
Hadith Collection (All in one)} includes~\href{https://reports.exodus-privacy.eu.org/en/reports/77502/}{Google Firebase Analytics}.

\href{https://play.google.com/store/apps/details?id=com.greentech.hisnulmuslim}{	
Dua \& Zikr (Hisnul Muslim)} includes~\href{https://reports.exodus-privacy.eu.org/en/reports/54714/}{Google Firebase Analytics}. This app is available in two primary languages, English and Bangla (the mother tongue of where the core development team live, in Bangladesh). The \href{https://play.google.com/store/apps/details?id=com.greentech.hisnulmuslimbn}{Bangla} version of the app is several times more popular than the English release. Unlike the English version, the Bangla version of the app includes~\href{https://reports.exodus-privacy.eu.org/en/reports/146430/}{Google Crashlytics, Google Firebase Analytics}.


\subsection{Development microcosm}
Who, how, where source and bug tracking take place. 

When do teams decide to fix which bugs, and what influences their decision making process?

NPE's and IndexOOBExceptions vs. IllegalStateException and native crashes.

\subsection{Applying analytics to the development practices}

Check Android Vitals approximately once a week, Firebase more frequently. Differences noticed in their reports, however the focus is on the crashes reported in Firebase as they contain more contextual detail. ANRs seldom checked, considered to be less impactful on users and lower frequencies.

\subsubsection{Worked examples}
These worked examples are taken from Android apps developed and maintained by the Greentech team. They are taken on the \nth{7} and \nth{8} September 2020. They exemplify various aspects of the~\href{section-select-aggregate-scope-analyse-triage-and-prioritise}{\MakeLowercase{\emph{\nameref{section-select-aggregate-scope-analyse-triage-and-prioritise}}}} section. % SHOULD-DO find out why the section name still has an initial capital letter.



\subsubsection{Preserving the failure clusters}




\subsection{Summary of the Greentech Apps case study}
TODO complete this section, reflecting the topics raised in the introduction to the case study.