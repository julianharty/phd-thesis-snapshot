\section{Research in logging practices}
\textit{This section is based on joint research post the NII Shonan meeting~\citep{nii_shonan_workshop_152}} where a group of researchers agreed to investigate the use of logging in mobile applications.

Early research explored ways developers of opensource Android apps use local logging, a complementary and oft used approach intended to help developers learn more about how their app behaves locally at run-time. 

\begin{itemize}
    \item Our research.
    \item The opensource tools we created to facilitate the testing and analysis of logging by Android developers.
    \item Explorations in methods to improve the effectiveness of logging and the analysis of log messages generated by apps.
\end{itemize}

MUST-DO write up our post-shonan paper.

\url{https://github.com/mobileanalyticslogs/mobileanalyticslogging/}


\subsection{Designing logging}
Unstructured logging can serve immediate needs, for instance to trace code execution or display the value of a variable at run-time. The resulting entries into a log file have limited value in terms of longer term analysis and they may also be harder to identify, filter, and lack relevant content for such analysis.

In the domain of logging both business and research consider logging design important and valuable. 

Implementation choices: 

\subsection{Testing logging}
TBC

\subsection{Tacit/default use of analytics}
50 of the 107 codebases studied ... expand with contents from the collaborative research post Shonan.

