\section{Research in logging practices}

Logging is an integral part of the software development process, ranging from sometimes seemingly random print statements to using fully featured software libraries and sophisticated software systems to transfer, replicate, store and analyze log data. Logs are also important developer aids for diagnosing problems and errors, with many app developers collect crash logs for their app from end-user devices - some even ask users for permission to send crash logs when problems occur.  This use of logging is particularly important for mobile software development, where developers need to understand the difference between between effects of the runtime environment, such as poor connectivity, from behaviours of the app when handling the various runtime conditions.

After covering various aspects of logging in proprietary apps and codebases, this section addresses three topics with a common focus on logging and logging tools available in opensource projects. It covers two aspects, with a subsequent discussion on the design and testing of logging:

\begin{enumerate}
    \item Analysis of third-party opensource Android apps that incorporate Firebase Analytics.
    \item Tools we created to facilitate the testing and analysis of logging by Android developers.
    \item Discussion in methods to improve the effectiveness of logging and the analysis of log messages generated by apps.
\end{enumerate}

When developers incorporate mobile analytics in their apps, what do they use them for? 

In earlier case studies in this thesis several commercial developers shared their practices informally. This case study complements their insights by analysing 107 opensource Android projects that use Google's Firebase Analytics (the most popular and prevalent in-app analytics tool) to answer two research questions:
\begin{enumerate}
    \item What are the characteristics of logging practices with mobile analytics?
    \item What do developers log with mobile analytics?
\end{enumerate}

This research was performed jointly with an international group of researchers, all-bar-one, met at the~\textit{\nth{152 }NII Shonan meeting~\citep{nii_shonan_workshop_152}} where a group of us agreed to investigate the use of logging in mobile applications as a follow-up activity. We wrote and submitted our first paper in October 2020 where I am the first author. The paper is currently under review. The materials are available as an opensource project at \url{https://github.com/mobileanalyticslogs/mobileanalyticslogging/}.



Early research explored ways developers of opensource Android apps use local logging, a complementary and oft used approach intended to help developers learn more about how their app behaves locally at run-time. 



MUST-DO continue to write up our post-shonan paper.



\subsubsection{Tacit/default use of analytics}
50 of the 107 codebases studied ... expand with contents from the collaborative research post Shonan.

\subsection{Tools to facilitate capture and testing of logging}

\begin{itemize}
    \item \href{https://github.com/ISNIT0/log-searcher}{\textbf{Log Searcher}}:
    \item \href{https://github.com/ISNIT0/logcat-filter}{\textbf{Logcat Filter}}:
    \item \href{https://github.com/ISNIT0/log-complexity-comparison}{\textbf{Log Complexity Comparison}}:
    \item \href{https://github.com/ISNIT0/AndroidLogAssert}{\textbf{Android Log Assert}}:
    \item \href{https://github.com/ISNIT0/AndroidCrashDummy}{\textbf{Android Crash Dummy}}:
\end{itemize}



\subsection{Discussion on practical aspects for the design, incorporation, and testing of logging} \label{apx:practical-aspects-for-design-and-incorporation-of-logging}
%This appendix introduces various practical aspects of incorporation of mobile analytics that are not necessary to understand the overall approach. The intention is to help those who would be actively involved in the concepts and approach described in the core thesis.
This introduces germane aspects of the design and incorporation of logging...



\subsection{Designing logging}
Unstructured logging can serve immediate needs, for instance to trace code execution or display the value of a variable at run-time. The resulting entries into a log file have limited value in terms of longer term analysis and they may also be harder to identify, filter, and lack relevant content for such analysis.

In the domain of logging both business and research consider logging design important and valuable. 

Storage and transmission of log entries. Sematext provides opensource libraries for Android~\citep{github2020_sematext_logsene_android} and iOS.

Controllable logging: Developers can optionally provide facilities to control the amount of logging that is performed by the app. The well-established K-9 Mail Android app includes optional debug logging that users can activate to help developers diagnose problems and errors~\citep{github2020_k9mail_logging_errors}. A good example of this feature being used is issue 2705 "Deleted mail from inbox is doubled in trash directory" on their GitHub site where various users contributed logs and additional information~\citep{github2017_k9mail_issue_2705}. Out of interest, the issue took 20 months to address and multiple contributions before the development team determined the cause, according to the updates in the issue. The project team uses this facility extensively; by \nth{23} December 2020 they had 11 open and 128 closed issues that included the keyword~\href{https://github.com/k9mail/k-9/issues?utf8=\%E2\%9C\%93\&q=is\%3Aissue\%20is\%3Aopen\%20loggingerrors\%20}{loggingerrors}.
% Note: in 2017 the project has 30 open issues and 39 closed, which reference the LoggingErrors wiki page. https://github.com/k9mail/k-9/issues?utf8=%E2%9C%93&q=is%3Aissue%20is%3Aopen%20loggingerrors%20 Some issues are waiting for logging to be provided, others include the requested logcat output.

Materials to incorporate on designing logging:
\begin{itemize}
    \item Retrace from Stackify: Logging meets monitoring. Full lifecycle APM.~\url{https://stackify.com/}
    \begin{itemize}
        \item \url{https://docs.stackify.com/docs/error-and-logs} Logging Rate, Top Errors, Recent Errors, Top Error Chart, Error Rates in dev, test and production.
        \item Application Performance: Slow Pages, Performance Breakdown, Slow Query, Server Performance, Satisfaction Breakdown.~\url{https://docs.stackify.com/docs/application-performance-widgets}
        \item Monitoring and Metrics, App Score:~\href{https://docs.stackify.com/docs/app-score-widgets}{https:// docs.stackify.com/docs/app-score-widgets}.
        \item Centralized Logging: 
        \item Filters, Fields and Tags, and Querying logs:~\url{https://docs.stackify.com/docs/logs-dashboard}. Monitoring logs. 
    \end{itemize}Control Which Logs are Sent to Stackify; ~\texttt{Enrich.WithProperty};~\texttt{stackifyLogger.ForContext}~\url{https://docs.stackify.com/docs/errors-and-logs-serilog}. \emph{``Our .NET libraries automatically handle batching, compressing, queuing, throttling, and error retry logic for uploading your application logs."}\url{https://docs.stackify.com/docs/errors-and-logs-net-supported-frameworks}. \url{https://docs.stackify.com/docs/troubleshoot-errors-and-logs-net-configurations} 4 steps to troubleshooting logging to Stackify's systems. And possibly use the App Score somewhere else in the thesis?~\url{https://docs.stackify.com/docs/appscore}
\end{itemize}

Implementation choices: 

\subsection{Testing logging}
TBC
