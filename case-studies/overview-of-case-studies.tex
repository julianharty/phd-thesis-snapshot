\chapter{Case Studies}
\section{Overview of Case Studies}
The case studies provide three distinct perspectives of applying Mobile Analytics to Android apps. Android apps were selected to enable the use of Google Play Console reports and analytics, including Android Vitals; nonetheless one of the case studies chose to also include analytics in their iOS app so this will also be covered briefly.

\begin{table}[th]
    \centering
    \begin{tabular}{l|l|r|l}
      Case Study &Role of Researcher &Apps \\
      \hline
       \href{https://play.google.com/store/apps/dev?id=9116215767541857492&hl=en_GB}{Kiwix}  &Embedded &18 \\
       \href{https://play.google.com/store/apps/developer?id=Catrobat&hl=en_GB}{Catrobat} &Guide &6 \\
       \href{https://play.google.com/store/apps/developer?id=Moonpig.com&hl=en_GB}{Moonpig.com} &Observed &1 \\
       \href{https://play.google.com/store/apps/details?id=boundless.moodgym&hl=en_GB}{Moodspace app} &Interviewed &1 \\
       \href{https://play.google.com/store/apps/details?id=com.localhalo.app&hl=en_GB}{Local Halo app} &Observed &1 \\
    \end{tabular}
    \caption{Project teams and Commercial apps in the case studies}
    \label{tab:case_studies}
\end{table}

For the Kiwix case study, the researcher has been an intrinsic long-term member of the diffuse project team, able to work directly with the code-base and collaborate directly with the developers and ancillary members of the project team. 

For the Catrobat case study, the researcher advised and assisted the project team to apply mobile analytics to their larger, older, and less reliable app: \emph{Pocket Code}. The researcher helped lead a one-day hackathon and otherwise interacted through a bug reporting tool, JIRA, discussions and using shared documents. Pocket Code also included a crash-reporting library which allowed cross-tool comparisons of reports, analytics and data. During the research, the reporting platform for the crash-reporting was migrated to a newer service which provided further insights and comparisons across and between the various mobile analytics tools.

For the commercial app teams (Moonpig, Moodspace, and LocalHalo), the researcher corresponded with one of the development team for each of the commercial apps and received either direct access to their analytics tools (LocalHalo), or was provided with snapshots (Moonpig and Moodspace). Permission was granted by their respective organisations for their contributions to be used for research purposes.

\section{Kiwix Android Apps}
TBC

\section{Catrobat Android Apps}
TBC

\subsubsection{Catrobat iOS App}
TBC
\section{Field Reports from Commercial App Developers}
TBC

\subsection{Validity considerations}
In absolute terms, my research covers a minuscule percentage of all the apps available in Google Play, roughly 1 in 100,000. So these results may not apply to all the apps, or potentially even a majority of them. And yet, the results have consistently indicated that when development teams pay attention to stability metrics they are able to materially improve the reliability of their mobile apps even though their apps range across several app store categories and range in userbase from under 1,000 active users to over 160,000. These apps are spread across 6 of the 7 groups of downloads identified in AppBrain's `Download distribution of Android apps'~\cite{appbrain_download_statistics_june_2019} and similarly 5 of the 7 groupings representing over 94\% of the distribution of downloads in Google Play according to Wang ~\emph{et al} (2018)~\cite{wang2018beyond}.

\textbf{MUST-DO} answer the following question: What exists in the literature, common practices, vs what I was able to achieve. \emph{From a question raised by Alistair Willis, OU, 30 April 2020.}

\subsubsection{How many developers are enough to ask?}
On of the key considerations for research is adequacy in terms of coverage. For my research there are several types of coverage, including: development teams, user-bases for the various apps, software tech stacks used (in terms of programming languages, analytics libraries, etc.), application domains, and so on. 

c.f. Krug is a well respected Usability guru whose work is inherently practical in nature. In the first edition of his~\emph{``Don't make me Think"} book he discusses ways to obtain practical results even with short timescales and few resources. In terms of obtaining value the author indicated that 3 to 4 people were capable of delivering more relevant feedback by involving them over time, (Chapter 9 in ~\cite{krug2000dont_make_me_think}). In terms of selecting the candidates his recommendation was to worry less about selecting 'representative users', instead\emph{``Recruit loosely, and grade on a curve."} (Chapter 9 in ~\cite{krug2000dont_make_me_think})~\footnote{Note: Krug made several chapters, including this one available online when the second edition of the book was published. I have copies of all three versions of the book and of these chapters as PDF files.}.

My research included working with two mature project teams and developers of three commercial apps. It is also based on work I did in industry that predates the PhD research, unfortunately I am not able to provide details of those projects in my thesis. 

\section{Summary of Case Studies}
The case studies includes a useful range of Android apps developed by independent teams using a variety of programming languages, mindsets, objectives, and constraints. In each team they learned to actively focus on stability metrics as reported in various technology-facing analytics tools, and the developers continue to see the merit of doing so on an ongoing basis.