\chapter*{List of  corrections}~\label{ch-list-of-corrections}
\addcontentsline{toc}{chapter}{List of corrections}
\chaptermark{List of corrections}
%\setcounter{section}{-1} % So the rest of the sections align with the comments of the examiners:

\section*{Report to candidate}
\addcontentsline{toc}{section}{Report to candidate}
These items are repeated \emph{verbatim} from the post-viva report, followed by a short sentence in italics that links to the response for that item. Each item is then addressed in order as separate sections in this meta-chapter.

\textbf{Guidance to the student}
\begin{enumerate}
    \item The RQs need to be revisited and made more focused on the actual research carried out. The terms used in framing the work also need to be clear and defined. These RQs then need to motivated and frame all subsequent chapters. \emph{Addressed in \href{corrections-rqs}{Revision of the Research Questions}}.
    \item Discuss the specific contributions to knowledge that the work makes as these are not really distilled down or presented explicitly.
    \item Give an explanation of the case study and action research methodology and the mechanics of the research methods that were actually implemented at each case study company/project.
    \item Clarify exactly what research was done at each case study and why. For example, exactly who was interviewed at each CS, and how were interviews designed and conducted as well as the data collected analysed and used.
    \item Clarify the data that was collected at each case study and how it was analysed and how it was validated, with illustrative data provided throughout.
    \item More information is needed to clarify the research process i.e., to provide a clear link between the data collected, the findings and the conclusions drawn.
    \item Throughout the thesis thining of material is necessary. Only material directly relevant to the RQs should be included. Only findings with clear evidence should be presented. Remove all informal asides; in particular, reconsider Chapter 2 and Chapter 9. Ensure that the findings and contributions are discussed in relation to the literature in Chapter 9.
    \item Review the consistency and rigour of the language and terminology used throughout; for example the discussion of the software process throughout.
    \item Provide a rationale for the fishbone diagrams and fully explain the basis of the themes.
    \item Provide more specific information on the literature review method used; for example, specific search criteria, tags and keywords.
    \item Carefully distinguish between what the data is providing evidence of, what the literature is saying and the informal views/experiences of the author.
    \item Provide more detailed in the Appendices of the data and the data analysis done in relation to the RQs.
\end{enumerate}

\section{Revision of the Research Questions}~\label{corrections-rqs}
\emph{``The RQs need to be revisited and made more focused on the actual research carried out. The terms used in framing the work also need to be clear and defined. These RQs then need to motivated and frame all subsequent chapters.''}

\subsection{Focus the RQs on the actual research carried out}
The Research Questions are in \secref{section-research-questions}

\begin{quote}
  \emph{How can applying analytics improve software development and software testing for mobile apps in practice?}~\href{overall-research-question}{Overall research question}
\end{quote}

I believe the research covered ways and effects  of using \emph{mobile} analytics to improve software development and the resulting artefacts.

The research touches on \textbf{software testing} mainly in terms of prior art and in practices of the development teams. Automated tests were developed during the Commercial, \index{C1}, app-centric case study to reproduce a major crash that caused a spike in the crash rate and they measured the improvement in the crash rate using mobile analytics when the underlying fix was released. Mobile Analytics complemented automated tests and interactive testing by providing objective measurements of the stability of the apps. The research did not encompass many improvements in software testing practices, nonwtheless it identified that Google uses usage analytics to determine which locales to use in their automated `robo' testing as part of providing pre-launch reports.

\subsection{Clearly define the terms used in framing the work}
Terms to define:  
\begin{itemize}
    \item Mobile analytics
    \item Software testing
    \item Software development
    \item In practice
\end{itemize}
