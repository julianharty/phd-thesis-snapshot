\chapter*{List of  corrections}~\label{ch-list-of-corrections}
\addcontentsline{toc}{chapter}{List of corrections}
\chaptermark{List of corrections}
%\setcounter{section}{-1} % So the rest of the sections align with the comments of the examiners:

\section*{Report to candidate}
\addcontentsline{toc}{section}{Report to candidate}
These items are repeated \emph{verbatim} from the post-viva report, followed by a short sentence in italics that links to the response for that item. Each item is then addressed in order as separate sections in this meta-chapter.

\textbf{Guidance to the student}
\begin{enumerate}
    \item The RQs need to be revisited and made more focused on the actual research carried out. The terms used in framing the work also need to be clear and defined. These RQs then need to motivated and frame all subsequent chapters. \emph{Addressed in \href{corrections-rqs}{Revision of the Research Questions}}.
    \item Discuss the specific contributions to knowledge that the work makes as these are not really distilled down or presented explicitly.
    \item Give an explanation of the case study and action research methodology and the mechanics of the research methods that were actually implemented at each case study company/project.
    \item Clarify exactly what research was done at each case study and why. For example, exactly who was interviewed at each CS, and how were interviews designed and conducted as well as the data collected analysed and used.
    \item Clarify the data that was collected at each case study and how it was analysed and how it was validated, with illustrative data provided throughout.
    \item More information is needed to clarify the research process i.e., to provide a clear link between the data collected, the findings and the conclusions drawn.
    \item Throughout the thesis thining of material is necessary. Only material directly relevant to the RQs should be included. Only findings with clear evidence should be presented. Remove all informal asides; in particular, reconsider Chapter 2 and Chapter 9. Ensure that the findings and contributions are discussed in relation to the literature in Chapter 9.
    \item Review the consistency and rigour of the language and terminology used throughout; for example the discussion of the software process throughout.
    \item Provide a rationale for the fishbone diagrams and fully explain the basis of the themes.
    \item Provide more specific information on the literature review method used; for example, specific search criteria, tags and keywords.
    \item Carefully distinguish between what the data is providing evidence of, what the literature is saying and the informal views/experiences of the author.
    \item Provide more detailed in the Appendices of the data and the data analysis done in relation to the RQs.
\end{enumerate}

\section{Revision of the Research Questions}~\label{corrections-rqs}
\emph{``The RQs need to be revisited and made more focused on the actual research carried out. The terms used in framing the work also need to be clear and defined. These RQs then need to motivated and frame all subsequent chapters.''}

\subsection{Focus the RQs on the actual research carried out}
The Research Questions are in \secref{section-research-questions}

\begin{kaobox}[frametitle=The original RQ]
\begin{quote}
  \emph{How can applying analytics improve software development and software testing for mobile apps in practice?}~\href{overall-research-question}{Overall research question}
\end{quote}    
\end{kaobox}

I believe the research covered ways and effects  of using \emph{mobile} analytics to improve software development and the resulting artefacts.

The research touches on \textbf{software testing} mainly in terms of prior art and in practices of the development teams. Automated tests were developed during the Commercial, \myindex{C1}, app-centric case study to reproduce a major crash that caused a spike in the crash rate and they measured the improvement in the crash rate using mobile analytics when the underlying fix was released. Mobile Analytics complemented automated tests and interactive testing by providing objective measurements of the stability of the apps. The research did not encompass many improvements in software testing practices, nonetheless it identified that Google uses usage analytics to determine which locales to use in their automated `robo'\index{Robo} testing as part of providing pre-launch reports.

\subsection{Clearly define the terms used in framing the work}
Terms to define:  
\begin{itemize}
    \item Applying: the use of mobile analytics in order to effect improvements
    \item Mobile analytics: Analytics where the data is collected by software running on mobile smartphone-based devices pertaining to the app's qualities-in-use. This research focused on analytics collected pertaining to the stability of the app, where stability includes the reliability of the app.
    \item Software testing: Note I need to also add index entries for software testing. Decide if it's to be Testing->Software in the index.
    \item Software development: which includes tasks performed by the software developers such as bug reporting and bug tracking. 
    \item In practice: the key scope of measurement focuses on the efficacy in real-world projects.
\end{itemize}

\subsection{6 perspectives RQs}
(Repeated from \ref{rq-leads-to-six-perspectives}.)

\begin{kaobox}[frametitle=The six perspectives from Chapter 1]
\begin{enumerate}
    %\setlength\itemsep{-0.5em} %\itemsep0em
    \item [1a] What do app developers say they do? (understand the \emph{status quo} from their perspective).\index{Research questions}
    \item [2a] What's possible in terms of improving their processes, their practices?)
    \item [1b] What does their source code (and other available development artefacts) tell us about their use of mobile analytics? (\emph{i.e.} to understand current behaviours in terms of the code that's implemented.
    \item [2b] What's possible in terms of improving the product (and particularly the mobile app) through the application/use of mobile analytics?
    \item [1c] What do we learn about various current mobile analytics tools?
    \item [2c] What improvements are possible for mobile analytics tools based on what was learned in the various case studies?
\end{enumerate}    
\end{kaobox}

These could be tidied up if needed as they are somewhat vague, particularly 
 for 1a and 2a as they do not specify mobile analytics. Note the related figure is specific.


\section{Specify my research contributions}
\emph{``Discuss the specific contributions to knowledge that the work makes as these are not really distilled down or presented explicitly.''}

\textbf{What did we know before my research? and what do we now know?} - frame using the 3 verticals.

What we knew before my research:
Software analytics was a recently established field of research - Microsoft used usage data to measure and improve Windows Software. A longitudinal study using custom opensource mobile analytics, called Insight, demonstrated ways in-app mobile analytics was able to help developers of two popular apps to improve the performance of their mobile apps.

Ratings and reviews provided information about problems and failures in mobile apps.

\subsection{My contributions to knowledge}
The research contributes to the understanding of tools and information seldom available to research - of professional app developers, their artefacts, and of professional mobile analytics tools and services. 

\newthought{Processes}: 
My research contributes knowledge on the approaches various app development teams apply when they use mobile analytics including the selection, integration of code and services, and their application of mobile analytics to detect, identify, and address errors and failures reported by mobile analytics. It builds on prior research, for example, on Insight, and confirms their findings. It contributes new knowledge in the adoption platform-level and commercial in-app mobile analytics, including a) usage patterns by development teams ranging from individual  developers, small teams and large,  sharded teams, and b) public opensource projects, hybrid projects that combined private and proprietary practices, through to a development team at a major corporation.

Some of the findings were surprising in terms of the patterns of use and in the efficacy of using mobile analytics to achieve significant improvements.

Development teams who embedded mobile analytics into their ongoing, core practices, were able to achieve highly reliable and stable apps. 

\newthought{Artefacts}: 
The research extended prior art in studies of opensource mobile app codebases, with a focus on the use of the most popular product offering: Firebase Analytics. It also contributes insights from proprietary, commercial codebases and issue tracking artefacts.

\newthought{[Mobile  Analytics] Tools}:
The research identified characteristics of a wide range of mobile analytics tools that serve Android app developers in particular. It also found and  presents a range of flaws found in professionally-developed mobile analytics tools, including several of the most-used mobile analytics offerings.

The research contributes material relevant to professional app developers and to the developers of mobile analytics. Several of the tool development organisations including Amplitude, Google, and Iteratively actively sought insights and updates from my research.

Improvements were identified in all three areas and some of these were implemented during the research. 

Note: In addition to specifying my research contributions add explicit summaries of my contributions that have already been published.


\section{Make the details of the methodology explicit for each case study}
\emph{``Give an explanation of the case study and action research methodology and the mechanics of the research methods that were actually implemented at each case study company/project.''}

Highlight the methodological constructs:
\begin{itemize}
    \item Case Studies: Action Research.
    \item Case Studies: Interview-based using semi-structured techniques.
    \item Analysis of source code and issues databases.
\end{itemize}

Chapter 5 includes a summary of the methodology and the mechanics. I'm not sure how to adequately these changes without writing lots of text. Perhaps a couple of paragraphs.

\section{}

\section{}

\section{}

\section{}

\section{}

\section{}

\section{}

\section{}

\section{}

\section{Additional Improvements to the thesis}
There are numerous minor opportunities to improve the thesis in terms of readability. These include copy-editing of the contents, tidying up loose-ends, replacing poor-quality figures with clearer ones, and using side-citations wherever practical (some \verb+\cite{}+'s were left in the thesis unnecessarily. I'd also like to add epigraphs to the chapters that lack them.

The examiners also indicated they would appreciate a list of my publications related to the research contained in my thesis,  so I suggest I reinstate this list.