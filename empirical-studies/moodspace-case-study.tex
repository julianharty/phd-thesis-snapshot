\section{Moodspace}~\label{study-moodspace}
Moodspace is an Android app aimed at improving mental health through various exercises incorporated into the app. It was released in 2019, with over 150K downloads by early 2020~\citep{objectbox2020_moodspace_interview}. Ian Alexander, the interviewee, was the software developer, co-founder, and runner of the company~\citep{objectbox2020_moodspace_interview} so he combines technical and operational responsibilities. He is an experienced app developer and also trained as a chemical engineer.  % https://www.linkedin.com/in/ian-alexander-01353340/

MUST-DO populate the commented-out table in the latex source and un-comment it.
\begin{comment}

{\renewcommand{\arraystretch}{0.8}% Tighter
\begin{table}[htbp!]
    \centering
    \small
    \setlength{\tabcolsep}{1pt}
    \begin{tabular}{ll}
       % Question &Answer  \\
       \toprule
       Website &\url{https://www.example.com/} \\
       Founded & _date_ \\
       Business Domain & _domain_ \\
       Business type & _type_ \\
       Technologies  & _tech_ \\
       & _continue_if_needed \\
       Source code  &Closed and not available for research \\
       Analytics used by team & _other_ma_, Google Play Console \\
       Development Practices & _dev_practices_ \\
       \midrule
       User base & 00,000's for the Android app \\
       Installations & 000,000+ for the Android app \\
       \midrule
       Research methods &In person interviews, email discussions, etc. \\
       Analytics collected &Google Play Console with Android Vitals \\
       Research software & None applicable? \\
       Additional data collected &Interview notes and emails \\
       Active period & \\
       \bottomrule
    \end{tabular}
    \caption{Case Study key facts: Moodspace:}
    \label{tab:blank_case_study_anaytics_overview}
\end{table}
}

\end{comment}


\subsection{Moodspace: Background - How the case study came about}



\subsection{Moodspace: development microcosm}
Discuss app complexity - offline only no online interaction. Simple design and elegant fast implementation.

\subsection{Moodspace: Experiences of using mobile analytics}

(\nth{13} June 2019)
% Source https://mail.google.com/mail/u/0/?q=kiwix+crash+rate+#search/kiwix+crash+rate/KtbxLzGHgCVznXxsGDjVgJGLFwmmGXxvxq
I love Android vitals - especially \textbf{core vitals}. Seeing your app in comparison to apps of peers provides some great motivation to step up your game. The only issue I have with core vitals is that I can't see them all! We are by no means a big app, so don't have enough data to meet googles standard for anonymised results, so results for most of the core vitals are hidden. I don't quite understand why this should be the case as the headline figure of your apps performance surely doesn't have to rely on anonymous data? Whereas the drilled down details of a core vital should be anonymous, so maybe the details view could just be blocked instead of hiding the entire core vital? To give context, MoodSpace has at least 4k monthly users, so there must be plenty of apps which get little or no use form core vitals, simply from them being hidden.

As for the other tools Google provides:
\begin{itemize}
    \item \textbf{ANR \& crashes} - I usually use crashlytics, so never really use this tab. the reason being, that the play console used to be very unreliable for crashes. But by the looks of it, it seems to have improved a lot, and pretty much matches the crashes I see in crashlytics now. Looking at it now, I actually noticed an ANR which could probably be fixed!
    \item \textbf{Pre launch reports} - I don't usually use this. Although this tool provides a nice safety net, it's quite basic, so any bugs that would cause the pre launch tests to fail have been found pre uploading through some quick manual testing. I actually ended up ignoring most pre launch reports when accessibility problems used to trigger failures, as we don't really have the resources to handle accessibility for now. But this seems to have changed in the last few months and accessibility problems don't cause a host of errors - maybe a way to ignore issues would be useful? In fact, pre-launch reports currently don't run on my app and fail with an incompatible apk error, not too sure why thats the case...
    \item \textbf{App signing} - very useful! and always use this since it was added.
    \item \textbf{Seperate release tracks} - love them! Especially since the addition of the internal track. The only issue was that I couldn't easily distribute debug versions of the app from the play store, and had to use a 3rd party tool to achieve that. Although Google have recently added Internal app sharing which should remedy that problem - however, I haven't figured out how to integrate that into our continuous integration process quite yet.
    \item \textbf{App bundles} - I'm still trying to integrate this, but as our new apps going to be heavily illustrated, this should cut our apk size significantly
\end{itemize}

As for several things I think are missing:

\begin{itemize}
    \item A gradle plugin to integrate play store uploading into CI processes. I currently use a 3rd party plugin to do this, but it would feel a little more secure if it came from Google.
    \item Top line core vitals figures even if you don't have enough users!
    \item Someway for testers to download old apks from either internal app sharing, or the internal release track.
\end{itemize}

I've attached the ANR and crash rates for the app. As I say, I usually use Firebase crashlytics, so don't really fix crash issues from Play store data.

You're welcome to use any data or comments in your research! If you do use it, please send over a copy.

(\nth{15} June 2019)
I think there's two things which has helped keep the app quite stable:
\begin{itemize}
    \item The app has the benefit that it's been around for quite a while without any major features being added. So most updates have been small and incremental, which has gradually increased it's stability. (This may change when the new, big update drops...)
    \item The app doesn't use any api, so all datas stored in very fast ORM databases like object-box (and uses memory caching). This enables the app to be mostly synchronous, which hugely cuts down on complexity of code. i.e. no need to handle loading, errors, or concurrency. This is a bit benefit! And cuts down on errors significantly, with no real impact on performance for users. To illustrate that it has little impact on users, I use firebase performance to run a trace on some methods that call the ORM/cache - it's peak duration is 40ms while the majority of calls take 3-6 ms.
\end{itemize} 

Crashlytics only covers the crash report of Android vitals, so unfortunately there's no way to get things like battery usage of ANR reports unless Google makes those reports available :(. In terms of crashes, I'd always prefer Crashlytics to Android vitals, simply because there are added features like non-fatal reporting and logs which can make surfacing the cause of errors much easier (but do take need added effort to integrate compared to android vitals).

There's been a change of branding of the app development organisation to \href{https://play.google.com/store/apps/developer?id=Chachi+Productions&hl=en_GB&gl=US}{Chachi Productions}. \url{https://www.appbrain.com/dev/Chachi+Productions/}

\subsection{Moodspace: data collected and methods used for collection}

\subsection{Moodspace: Research findings and results from the Case Study}

\subsection{Moodspace: Outcomes for the company}

\subsection{Moodspace: Discussion}

\subsection{Moodspace: Contributions to the research and where they are located in the rest of this thesis}


\dotfill 

\julian{The following content needs integrating above.}

\subsection{Moodspace prior material TODO integrate}



\newthought{Notes to integrate into the case study}
\begin{itemize}
    \item \url{https://www.psycom.net/25-best-mental-health-apps} Helps set the context for these apps.
    \item \url{https://objectbox.io/moodspace-mobile-app-use-case/} An interview with the founder Ian Alexander on the technological choices, particularly objectbox as the database. He'd mentioned the performance was lightning fast and one reason the app was performant and reliable. 
    \item ``Built MoodSpace, a digital platform empowering everyone to take control of their mental health. MoodSpace began as a side project which later took on a round of funding and with a team of 6 supported a user base of 300k. Although no longer pursued as a business we continue to maintain the project and release regular updates to 10s of thousands of active users. The most recent iteration of the app was built with Kotlin, clean architecture, MVVM, Data binding, Gitlab CI, Coroutines/Flow, and ObjectBox and was architected to enable the use of Kotlin Multiplatform to share ~60\% of the codebase between platforms."~\url{https://www.linkedin.com/in/ian-alexander-01353340/}
    \item 4.18/5.00 User Experience rating \url{https://onemindpsyberguide.org/apps/moodspace/}
    \item ``To make the app work well at all we collect the following anonymous data:
    \begin{itemize}
        \item Crash reports: If you've never seen the app crashing, it's because as soon as one happens, we get a crash report. A little red light flashes in our office, a loud siren blares, and we release a fix right away. It's quite annoying actually.
        \item Analytics: We assume you're going to use the app a certain way. We're almost always wrong, and you often surprise us. Analytics lets us see how people like you actually use the app, so we can make improvements to the right places. Analytics can use the Google Advertising ID to identify you. This doesn't tell us anything about you (it's just some numbers and letters), but if you really want to trick us you can reset your Google Advertising ID at any time. Go to your device Settings > Google > Ads."
    \end{itemize}~\citep{moodspace2021_privacy_policy}
\end{itemize}
