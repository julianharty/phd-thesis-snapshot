\section{Greentech Apps}~\label{section-greentech-apps}

\subsection{Introduction to the Greentech Apps case study}
The aims and objectives of this case study include:

\begin{itemize}
    \item \textbf{A linear increase (+1)} : validation the methods described in my research are repeatable and scale to additional apps beyond the previous case studies.
    \item \textbf{Additional examples of characteristics of Google Play Console (+1)} :
    \item \textbf{A closed source case study (?)} : The previous case studies were all with opensource codebases so the code was available for bug investigation purposes. For this case study the source code and build processes are treated as a black box (it may potentially become a grey box case study if the development team share details of their engineering practices, etc.)
\end{itemize}

\subsection{Background to the Greentech Apps case study}
A set of Android apps developed and provided by Greentech Apps Foundation. They are described as modern Islamic Applications, according to their website \url{https://gtaf.org/}. The project encourages voluntary contributions, for instance to provide translations~\url{https://greentech.oneskyapp.com/collaboration/}. Their apps are popular, and well regarded. % MUST_DO add data on usage and ratings.


The project started in 2016 with the aim of enabling people to learn the Quran in the local language - Bangla - in Bangladesh. The project was started by a self-taught Android developer and his cousin Yemin, at the time an undergraduate student in computer science, who is now employed by the project in a hybrid role of software developer and project manager. The development team was predominantly volunteers until around 2018 when the first engineer was employed. In April 2020 the development team added two part-time paid developers and there are plans to grow the employed team including someone with UI and UX expertise. The team are funded through voluntary donations. From a research perspective my contacts at the project are Yemin and a fellow PhD student Riasat who also volunteers for the project.

At the start of the case study (June 2020) the team had ten Android apps published in Google Play~\url{https://play.google.com/store/apps/dev?id=7665838187257770408}. Of these apps, three are their core apps, and a couple are overdue an engineering revamp.  Two analytics tools are used for their core apps: Firebase Crashlytics and the default combination of Google Play Console with Android Vitals.

Their development team maintain their issues lists in public (\url{https://gitlab.com/greentech/}), the source code is private. They use a variety of languages and frameworks to develop their apps include React Native for at least one app (\href{https://play.google.com/store/apps/details?id=com.taskinator}{Taskinator}) and Java for the majority, they aim to use Flutter~\cite{flutter_dev_site} in future.

Exodus Privacy~\cite{exodus_privacy_project}, a non profit organization, was used to determine which analytics libraries are embedded in which of their apps as the source code was not available for these apps.

\textbf{SHOULD-DO} reformat the following as a table, or similar. Add the rest of their apps for completeness.

\href{https://play.google.com/store/apps/details?id=com.greentech.quran}{Al Quran (Tafsir \& by Word)} includes \href{https://reports.exodus-privacy.eu.org/en/reports/com.greentech.quran/latest/}{Google Crashlytics, Google Firebase Analytics}.

\href{https://play.google.com/store/apps/details?id=com.greentech.hadith}{	
Hadith Collection (All in one)} includes~\href{https://reports.exodus-privacy.eu.org/en/reports/77502/}{Google Firebase Analytics}.

\href{https://play.google.com/store/apps/details?id=com.greentech.hisnulmuslim}{	
Dua \& Zikr (Hisnul Muslim)} includes~\href{https://reports.exodus-privacy.eu.org/en/reports/54714/}{Google Firebase Analytics}. This app is available in two primary languages, English and Bangla (the mother tongue of where the core development team live, in Bangladesh). The \href{https://play.google.com/store/apps/details?id=com.greentech.hisnulmuslimbn}{Bangla} version of the app is several times more popular than the English release. Unlike the English version, the Bangla version of the app includes~\href{https://reports.exodus-privacy.eu.org/en/reports/146430/}{Google Crashlytics, Google Firebase Analytics}.

\href{https://play.google.com/store/apps/details?id=com.taskinator}{Taskinator} uses~\href{https://reports.exodus-privacy.eu.org/en/reports/146496/}{Firebase Analytics, OneSignal}.

Additional analysis by the researcher in July 2020 is available online in a Google Document that has been shared with two core members of the project team amongst others~\url{https://docs.google.com/document/d/1Ghu0UBXNrwOB3HYbgsPQ-OJV-fPmsP68lSLVRfcc4aA/edit?usp=sharing}.

\subsection{Development microcosm}
Who, how, where source and bug tracking take place. 

When do teams decide to fix which bugs, and what influences their decision making process?

NPE's and IndexOOBExceptions vs. IllegalStateException and native crashes.

The team have used Firebase TestLab to test some of the apps occasionally and the Robo testing performed automatically by the test lab has triggered various crashes in the apps being tested. One such example was where an app was missing a `resource'. The team fixed the build by adding the missing resource but did not explicitly retest the app afterwards in Firebase. 

\subsection{Applying analytics to the development practices}

The development team check Android Vitals approximately once a week, and Firebase more frequently as the team decided the crash reports in Firebase are more actionable. Perhaps unsurprisingly they check more often after new releases of their apps looking for any new bugs arising in the new release as it rolls out across the user population.



Differences noticed in their reports, however the focus is on the crashes reported in Firebase as they contain more contextual detail. ANRs seldom checked, considered to be less impactful on users and lower frequencies.

\subsubsection{Worked examples}
\textbf{This section needs redoing as the discussion with Yemin indicates they use Firebase predominantly}.
These worked examples are taken from Android apps developed and maintained by the Greentech team. They are taken on the \nth{7} and \nth{8} September 2020. They exemplify various aspects of the~\href{section-select-aggregate-scope-analyse-triage-and-prioritise}{\MakeLowercase{\emph{\nameref{section-select-aggregate-scope-analyse-triage-and-prioritise}}}} section. % SHOULD-DO find out why the section name still has an initial capital letter.



\subsubsection{Preserving the failure clusters}
Details of the crashes are recorded in individual issues on the respective project's GitLab site. The issue is cross-referenced in a note in the crash cluster reported in Firebase.

\subsection{Priorities of the project team}
Throughout 2020 and until April 2021 the team are focusing on bug-fixes which include fixing the causes of crashes in the apps. Three of the apps (four as one app is released as two distinct binaries) are in ongoing active development (\href{https://play.google.com/store/apps/details?id=com.greentech.quran}{Al Quran},~\href{https://play.google.com/store/apps/details?id=com.greentech.hadith}{Hadith Collection}, and~\href{https://play.google.com/store/apps/details?id=com.greentech.hisnulmuslim}{Dua \& Zikr}, which is also released separately in Bangla~\href{https://play.google.com/store/apps/details?id=com.greentech.hisnulmuslimbn}{{Dua and Zikr (Hisnul Muslim)}}~\emph{in Bengali}); and they plan to revamp two more of the apps (\href{https://play.google.com/store/apps/details?id=com.greentech.islamicquiz}{(Islamic Quiz)} and~\href{https://play.google.com/store/apps/details?id=com.greentech.salatbn}{Meaningful prayers (salat)}~\textit{in Bengali}, which was called salat in our interview).

% SHOULD-DO add support somehow for Bengali Unicode so I can replace the Anglicised app names with the correct ones. The issues are a mismatch between the compiler I'm using (pdfLatex) vs. the recommended packages to use
% e.g. 
% https://tex.stackexchange.com/questions/285507/how-can-i-use-bengali-script-in-an-english-document
% https://tex.stackexchange.com/questions/99606/how-to-write-bengali-in-latex
% https://www.overleaf.com/learn/latex/Multilingual_typesetting_on_Overleaf_using_babel_and_fontspec
% https://www.overleaf.com/learn/latex/international_language_support#Babel
% %%%
% https://www.researchgate.net/post/How_to_write_Bengali_unicode_text_in_English_document_in_Latex
% https://www.overleaf.com/learn/latex/arabic


\subsection{Summary of the Greentech Apps case study}
TODO complete this section, reflecting the topics raised in the introduction to the case study.