\section{Red Thread for the Empirical Studies}
\label{section-empirical-studies-red-thread}

\subsection*{Notes I need to apply}
Each case study needs to be formatted consistently so that the reader can find and compare any of them with any of the others, and establish patterns and connections as they're reading them, if indeed there are intentional patterns, orderings, and so on (beyond chronological).

``Empirical research methods in software engineering"~\citep{Wohlin2003_empirical_research_methods_in_software_engineering} introduces four research methods for empirical research in software engineering: 1) controlled experiments, case studies, surveys, post-mortem analyses. Each of these has been used during the research to some extent.

A controlled experiment formed the backbone of the initial Kiwix Android app case study, then this case study applied the results of the experiment to determine whether similar improvements were achievable across the set of the project's custom Android apps. % They were...

There are a variety of in-depth case studies for specific apps within projects. These were augmented by obtaining the experiences of developers of additional Android apps who were surveyed (interviewed) on a common theme of their use and experiences with using mobile analytics services. 

After the case studies were completed, a form of post-mortem analysis identifies adverse effects of entropy returning when development teams stop paying active attention to addressing issues reported through mobile analytics.


The research combines various forms of empirical studies. These include:
\begin{itemize}
    \itemsep0em
    \item Primary Research: Various \textbf{app case studies} each centered around a development team who are responsible for one or more related Android apps
    \item Secondary Research: Investigating the logging practices of developers of opensource Android apps that incorporate Google's Firebase Analytics service
    \item Primary Research: small experiments that were not appropriate in the main app case studies.
    \item Primary Research: interviews with app developers on their use of mobile analytics - indirect (secondary?) experiences of the services and tools.
    \item Primary Research: collaborations with two mobile analytics engineering teams: Google and Iteratively. 
    \item Secondary Research: Literature Review.
\end{itemize}

\subsection{Structure of the app case studies}\label{structure-of-the-app-case-studies}
A consistent structure helps: tell the story, provide the evidence and coherence for each of the app case studies, and demonstrate the rigour needed establish the credibility of the research. It also facilitates comparisons across the results and findings.

The current structure consists of six headings:
\begin{enumerate}
    \itemsep0em
    \item Introduction: to highlight key features of case, similarities and differences w.r.t. other cases;
    \item Context:  the terms of the engagement, product/project overview, Developer characteristics, tools, methods, key challenges for product/-project;
    \item Analytics intervention: describe what was done during the case study in terms of using analytics. Include the data trail and the methods used for the research aspects of the case study;
    \item Findings: reporting on the data that was collected;
    \item Results/Outcomes: in terms of any improvements and changes to the stability/reliability and to the engineering practices;
    \item Discussion: including reflection on the contributions of this case study to the research questions. Also, explore what these outcomes mean for the use of analytics in mobile software more generally.
\end{enumerate}


\subsection{Dimensions of the app case studies}
The following list is a proposed set of properties each case study would include:
{\small
\begin{itemize}
    \itemsep0em
    \item Role of the researcher?
    \item What are the focal points in this case study?
    \item Development practices of the project?
    \item Analytics sources: External, external + crash reporting, external + commercial internal, external, internal, and proprietary?
    \item Engagement level of the team?
    \item Privacy concerns?
    \item What opportunities did the case study present?
    \item What were the objectives (when I started the case study) compare and contrast with what actually happened? 
    \item What were the \textbf{F}indings and \textbf{I}nsights gleaned from each?
    \item What were the limitations of the tools that restricted the abilities to effect improvements? What were the limitations of the engineering practices that limited the improvements?
\end{itemize}
}

\begin{landscape} % Rotates the table into a landscape page in the generate PDF.
\definecolor{Gray}{gray}{0.9}
\begin{table}
    \setlength\extrarowheight{3pt} % provide a bit more vertical whitespace
    \captionsetup{size=footnotesize}
    \centering
    \tiny
    \tabcolsep=0.06cm
    %\rowcolors{1}{green}{pink}
    \begin{tabular}{lrrlllllllll}
        \mycolumnheading{1.9cm}{Case Study} &\mycolumnheading{0.9cm}{Apps} &\mycolumnheading{1.1cm}{Active Installs} &\mycolumnheading{1.6cm}{Role of Researcher} &\mycolumnheading{2.0cm}{Main Research Focus} &\mycolumnheading{2.0cm}{Development Practices} &\mycolumnheading{1.5cm}{Mobile Analytics} &\mycolumnheading{3.5cm}{Case Study Objectives} &\mycolumnheading{1.4cm}{Privacy} &\mycolumnheading{2.5cm}{Opportunities} &\mycolumnheading{3.2cm}{Findings} \\
        \toprule
        \rowcolor{Gray}
        Catrobat &2  &70.9K  &Coach         &Experiment &Sophisticated  &Crashlytics    &M.A. vs. Clean Code          &Strong        &Opensource          &Immediate improvements \\
                 &   &190K   &Observer      &Control     &\textit{ditto} &              &Control for above           &\textit{ditto} &\textit{ditto}      &N/A \\ 
        
        \midrule
        \rowcolor{Gray}
        C1       &1  &1M+   &Consultant     &at Scale   &Laminar        &Multiple       &Stability, Ways of Working   &Known        &Large-scale   &Rich \\
        \midrule
        GTAF     &11 &1.1M  &Observer       &Priorities &Team+          &Miscellaneous  &Accurate local language apps &Strong       &Distinct view &Their priorities \\
        \midrule
        \rowcolor{Gray}
        Kiwix    &1  &150K  &Embedded       &P-o-C      &Team+          &Android Vitals &Suppress crash rate          &V.Strong     &Open, \nth{1} case study &It works!\\
        \textit{-"-} WikiMed (EN) &1  &58.9K &Observer       &Control    &\textit{ditto} &\textit{ditto} &Control for above            &\textit{ditto} &\textit{ditto} &\textit{ditto} \\
        \rowcolor{Gray}
        \textit{-"-} Custom apps &16 &222K  &Observer       &Scaling    &\textit{ditto} &\textit{ditto} &Measure scaling              &\textit{ditto} &\textit{ditto} &\textit{ditto} \\
        \midrule         
        LocalHalo &1 &1.1K  &Observer       &Startup    &Cross-platform &Sentry.io      &New business view            &Unknown   &React-Native app &\\
        \midrule
        \rowcolor{Gray}
        Moodspace &1 &19.2K &Observer       &Startup    &1 core dev.      &Crashlytics    &New business view            &Unknown   &            &Feedback on M.A.\\
        \midrule
        Moonpig  &1  &138K  &Observer       &Firebase   &Clean Code     &Firebase       &Leading edge practices       &Known     &Firebase insights &Insightful \\ 
        \midrule
        \rowcolor{Gray}
        Big C's  &10\textsuperscript{1} &10\textsuperscript{7} &Observer &Multi-teams &N/A &N/A      &Large corporates         &Unknown   &Big picture &   \\
        \midrule
        Analytics tools &10\textsuperscript{6} &10\textsuperscript{9} &Various &Trustworthiness &Various &Various &Improve the tools &Commercial &Bleeding edge &Flaws in tools \&services \\
    \end{tabular}
    \caption{Overview of App Case Studies}
    \label{tab:overview_of_app_case_studies}
\end{table}
\end{landscape}
%%%% Notes on compressing tables
% https://tex.stackexchange.com/questions/10766/how-to-make-really-wide-tables-narrower
% https://stackoverflow.com/questions/2563498/making-latex-tables-smaller
% https://en.wikibooks.org/wiki/LaTeX/Tables#Resize_tables 
% Smaller text finally worked after applying the tips from https://tex.stackexchange.com/a/56011/88466
% Row colors https://texblog.org/2011/04/19/highlight-table-rowscolumns-with-color/
% Rotate table https://tex.stackexchange.com/questions/370393/how-to-rotate-the-large-table-and-caption/370394
% Add a footnote https://tex.stackexchange.com/a/66641/88466
% Improvements in the formatting of the generated table https://tex.stackexchange.com/a/327977/88466

\begin{comment}
MUST-DO
The monster table needs a column for what each study contributes to my thesis.
Any other pertinent information to add ?
What viewpoints did each case study provide?
Merge Research Focii and case study objectives vs. the objectives of the case study team.
The long view is useful to highlight of various case studies. Repeated and long-term engagement and access to analytics and/or code.
Mini-table for each case study, include the active study period, follow over ...
The role of the case study.
\end{comment}

\newthought{Joe's suggestions}
\textit{Note: these will be merged into the rest of the thesis, here as a reminder until that's done.}

Joe is an industrial colleague. He proposed each case study could be formatted as a mini-paper with an abstract per case study.
{\small
\begin{itemize}
    \itemsep0em
    \item Which of the research questions does it answer?
    \item What’s the context? Mobile app? Web? Reporting tools? Library?
    \item What’s the company? Org structure? Communication tools?
    \item What tools were used? MS App Center, Android Vitals, etc. Crashlytics, Testing Frameworks, Collaboration tools,
    \item Team structure?
    \item How many users?
    \item Do the devs have access to the tools? how integrated into the development practices?
    \item Future work for each case study.
\end{itemize}
}
There are some overlaps and similar topics between the suggested dimensions of each case study and Joe's proposed list. TODO These two lists need to be harmonised, non-essential topics may be pruned from the combined list.


\subsubsection{Revised set of Marian's comments from our call on \nth{1} Sept 2021}

\newthought{The map is vital.}

There new tables aim to provide a partial map for the examiners so they don't get lost or side-tracked. They also help to clarify the thinking and make the case studies more consistent and coherent.

\begin{enumerate}
    \item \ref{tab:empirical-studies-research-perspective} \nameref{tab:empirical-studies-research-perspective}
    \item \ref{tab:empirical-studies-their-characteristics} \nameref{tab:empirical-studies-their-characteristics}
    \item \ref{tab:empirical-studies-findings-and-results} \nameref{tab:empirical-studies-findings-and-results}
\end{enumerate}

Notes: The previous table \ref{tab:overview_of_app_case_studies} \nameref{tab:overview_of_app_case_studies} may eventually be retired and removed from this red-thread. Another table, on the analytics tools, is also worth me creating.

Ideally a drawn map would complement these tables.


\newthought{Demonstrate competence as a researcher}
You need to be more precise in how you report the methodology, e.g., the case studies actually use mixed methods (not just the expansion). A methodology combines reasoning with the method. You need to convey rigour within the constraints of access to case studies.

At some point, you need a 'map' of the case studies:  these need to include the: case, methods, purpose...  This should give the reader an overview of what each contributes and where they overlap.


\newthought{Explain the challenges of anyone being permitted by teams to access their sensitive data, the opportunistic access and rigour applied in the case studies.}

Maximise my ability to make systematic use of what was pragmatically available: Opportunistic access, then I worked systematically and rigorously. Demonstrating the demands of rigour is vital. Explicit reporting of what I had and the efforts I made to handle the data systematically and be vigilant for bias. This helps avoid the case studies being considered as anecdotal. Amplify with the dialogues I had as part of the teams.

Start with what data did I collect and how I collected it. The case studies with similar sources of data can be compared which multiply the contribution.

Aim to explain what are the issues that arose in each case study?
A clear rationale needs to be provided. 

It's wise for the thesis to demonstrate there's a really clear audit trail and justification for the conclusions.

\begin{itemize}
    \item Characterise the empirical studies.
    \item Make explicit what data was collected and for what purpose. 
    \item Separate insights into a separate table. 
\end{itemize}

Quality of reporting of the empirical work is key

\newthought{Be Orthogonal and separate concerns}

MUST-DO Separate the reporting from the discussion.



\begin{table}
    \centering
    \tabcolsep=0.06cm
    \footnotesize
    \begin{tabular}{lcrcll}\toprule
    Case Study               & Apps                 & Reach & Project context & Mobile Analytics Tools\textsuperscript{*}  &Dev. practices  \\
    \midrule
    Kiwix                    &                    1 &  150K & Wikipedia, FOSS            & GPC with AV &          Team+ \\ 
     \textit{-"-}            &                    1 & 58.9K & \textit{-"-}   & GPC with AV & \textit{ditto} \\
     \textit{-"-}            &                   16 &  222K & \textit{-"-}   & GPC with AV & \textit{ditto} \\
    Catrobat                 &                    1 & 70.9K & Coding, FOSS            & Crashlytics &  Sophisticated \\
     \textit{-"-}            &                    1 &  190K & \textit{-"-}   & GPC with AV & \textit{ditto} \\
    C1                       &                    1 &  1M+  & Mission-critical &    Multiple &        Laminar \\
    GTAF                     &                   11 &  1.1M & Another app category             &Miscellaneous &          Team+ \\
    LocalHalo                &                    1 &  1.1K & Risky early startup &   Sentry.io & Cross-platform \\
    Moodspace                &                    1 & 19.2K & Medical    & Crashlytics &        Startup \\
    Moonpig                  &                    1 & 138K  & E-commerce &    Firebase &     Clean Code \\
    Analytics tool providers &10\textsuperscript{6} &10\textsuperscript{9} & No &  Several &        Various \\
    \bottomrule
    \end{tabular}
    \caption[Case Studies: their characteristics]{Case Studies: their characteristics \\ {\tiny * All the apps used GPC with AV; one of the tool providers \emph{is} GPC with AV \\ FOSS = Free and Open Source Software}}
    \label{tab:empirical-studies-their-characteristics}
\end{table}

To consider: project context, the drivers for the project, and what's noteworthy about the case study to help the reader?

Each of the in-dept case studies had their own objectives from participating in the research, and a common need from their perspective was to try and reduce the reported crash rate for at least one of their Android apps.

\begin{landscape} % Rotates the table into a landscape page in the generate PDF
\begin{table}
    \centering
    \tabcolsep=0.06cm
    \tiny
    \begin{tabular}{lllll}\toprule
    Case Study                  &Evidence    &Results               &Main Findings             &Insights  \\
    \midrule
    Kiwix                       &Code and Analytics results &3x improvement        &Targeted bug fixes v.effective &Visibility drove action \\ 
     \textit{-"-}               &Baseline         &stable                &                     & \\
     \textit{-"-}               &Code and Concrete results &reduced crash-rates   &The approach scaled  &Lead dev. took ownership \\
     \midrule
    Catrobat                    &Corroboration, crashlytics &2x improvement     &Immediate improvements &Ongoing ownership is vital \\
     \textit{-"-}               &Baseline         &stable                &                     & \\
     \midrule
    C1                          &Concrete results &Multi-faceted improvements &OKRs achieved & Rich, multi-faceted \\
    GTAF                        &Corroboration    &                      &Historical perspective &Their priorities  \\
    LocalHalo                   &Add'l M.A. tool  &                      &Sentry.io &Limitations of GPC with AV for hybrid apps \\
    Moodspace                   &Corroboration    &                      &Effectiveness of good app design &Feedback on M.A. \\
    Moonpig                     &Corroboration    &                      &Effectiveness of engaged, motovated teams &Insightful \\
    Analytics tool providers    &Their priorities &Improvements made to their tools & Flaws in tools \& services &This research is v.valuable \\
    \bottomrule
    \end{tabular}
    \caption{Case Studies: findings and results}
    \label{tab:empirical-studies-findings-and-results}
\end{table}
\end{landscape}

\begin{comment}
MUST-DO
Disambiguate Contributions into Evidence, Insights, and the Distinct Opportunity/Perspectives it provided. Impact on the teams.
Consider an Insights table, mapping back on to the case studies. 
Insights in practices of the development teams. 
Try creating an evidence table. 
\end{comment}

\subsection{Classifications of Mobile Analytics tools}
\textit{Note: the following list is now duplicated in \href{classifications-of-mobile-analytics-tools-used-in-this-research}{\nameref{classifications-of-mobile-analytics-tools-used-in-this-research}}. As that matures, I'll remove the following and replace it with something more succinct.}

The case studies also facilitated the study of various mobile analytics tools. A map of these tools and the case studies they were used in may help the reader (and the author). Here are suggested topics to consider for each of these tools.
\begin{itemize}
    \itemsep0em
    \item To consider: which projects was the tool used in/relevant to?
    \item What was the tool used for?
    \item What insights did we glean from using the tool?
    \item If anythings made the tool distinctive, what were those things and why are they distinctive in the context of developers using mobile analytics?
    \item Type of the analytics and where in the application stack it collects the analytics.
    \item APIs and optional capabilities:
    \begin{itemize}
        \item Application APIs: that can used in the app to customise what is sent
        \item Reporting APIs: that can be used to obtain the data and reports, etc. 
    \end{itemize}
    \item Customisation of reporting, if any.
    \item Data output options (see the appendix on Data Sources for the current known set of output options).
    \item Platform(s) supported.
    \item Some notes on data retention facilities and limitations.
\end{itemize}
\textbf{Vitally} how the tool(s) were used in each case study.



\subsection{Contributions of the empirical studies to the research question}
For ease of reference, the main research question is:
\begin{quote}
    \emph{How can applying analytics improve software development and software testing for mobile apps in practice?}    
\end{quote}
Sub-questions focus on sources, value and impact. 
\begin{landscape} % Rotates the table into a landscape page in the generate PDF
\begin{table}
    \centering
    \tabcolsep=0.06cm
    \tiny
    \begin{tabular}{lllll}\toprule
    Case Study                  &Results                &Sources                &Value             &Impact within the case study  \\
    \midrule
    Kiwix                       &3x improvement         &GPC with AV            &Crash rates improved &  \\ 
     \textit{-"-}               &stable                 &GPC with AV            &Crash rates improved &  \\
     \textit{-"-}               &Reduced crash-rates    &GPC with AV            &Crash rates improved & Dev. lead took ownership. \\
     \midrule
    Catrobat                    &2x improvement         &GPC with AV, Fabric Crashlytics &Crash rates improved &Led to dev team visiting Poland. \\
     \textit{-"-}               &stable                &   &                  & \\
     \midrule
    C1                          &Multi-faceted improvements &In-house, Google Analytics, App Center &OKRs achieved &Sporadic \\
    GTAF                        &                      &TBC &High crash rates addressed &Pareto-like application \\
    LocalHalo                   &                      &sentry.io &Critical failures are fixed & \\
    Moodspace                   &                      & & & \\
    Moonpig                     &                      &Firebase & & \\
    Analytics tool providers    &Improvements made to their tools &GPC with AV, iterative.ly  &This research is v.valuable &Demonstrated by results \\
    \bottomrule
    \end{tabular}
    \caption{Empirical Studies: contributions to research questions}
    \label{tab:empirical-studies-contributions-to-research-questions}
\end{table}
\end{landscape}