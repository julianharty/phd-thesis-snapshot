%\section{Methodology, ethics, and repeatability of case studies} \label{section-methodology-ethics-and-repeatability-of-case-studies}


\section{Research ethics for the case studies}
\label{section-research-ethics-for-the-case-studies}

This section discusses the research ethics for both the app-centric and tool-centric case studies.

\begin{comment}
from Arosha: [please] provide an overview of:
(1) General ethical considerations in empirical software engineering research that uses the methods you've used (interviews, ethnography, data repository analysis, etc)
(2) Explain how these ethical issues applied to your case studies and why
(3) Steps you took to mitigate the ethical issues relevant to your cases. This would link back into the different phases of your research - i.e. steps taken at engagement phase, active case study phase, post-hoc analysis phase and documentation phase.
\end{comment}



\newthought{General ethical considerations in Empirical Studies in Software Engineering}

An article jointly written by Singer and Vinson and published in 2002 was one of the first aimed at establishing guidelines and procedures for empirical studies in software engineering (\href{glossary-esse}{ESSE}). 
%
They proposed~\emph{``four core ethical concepts...: informed consent, scientific value, beneficence and confidentiality"}~\citep[p.1178]{singer2002_ethical_issues_in_empirical_studies_of_software_engineering} and they provided various examples of ethical issues in empirical studies of software engineering. These four core ethical concepts are still relevant nearly 20 years on and they were applied to my research as follows in the rest of this section.


%\julian{TODO complete the following: I addressed these in turn (as follows sub-sections) Relocate my current material into these 4 topics, else put it aside for now.} 

\newthought{How ethical issues applied to my case studies and why}

The case studies included people working with software together with the software artefacts they created, therefore there were, and are, ethical considerations that apply to these case studies. 
% The majority of the case studies presented in this research include other people. 
It is right and proper to ensure the developers and their respective organisations are willing to support the research directly - by participating - and/or indirectly - for instance by providing access to systems, tools, bug tracking systems, and so on. Some organisations require non disclosure agreements, and - in practice - they all choose what access to provide to what, to whom, and for how long. 

\subsection{Informed consent}
Consent was obtained from the development team leaders and development team members who participated; and consent was also obtained from their respective organisation as appropriate. In every case they were explicitly aware of the research from the outset of discussions.

As mentioned earlier in the previous section (starting on page~\pageref{section-research-ethics-for-the-case-studies}), \textit{de-facto} consent was also given in terms of the access provided to the tools and artefacts that teams and their organisations provide during the case study. They may also place constraints on aspects of the use of the materials obtained, \textit{i.e.} consent may be fine-grained and also context dependent.

Tacit consent was also given by individuals in the development teams as their participation was core to many aspects of the case study. Tacit consent draws on the trust that is established with the people who manage the project and with the people who participate in the case study. They needed to individually and as a group see value in the work in order to actively participate in the case study.

\newthought{Forms of consent and approvals}

Organisations often require approval from one or more senior representatives in the organisation, these may include head of development, and one or more of their legal, marketing, commercial, and risk departments. 

For startups, one person may hold multiple roles, for instance in small startup teams they may be the CTO while also being an active developer of the software, \emph{and} the person responsible for operations, support and customer service. This was the case for two of the case studies presented in this research (LocalHalo and Iteratively), and in Moodspace the CTO was also the main developer of the app. In terms of obtaining permission, startups tend to be easy to work with if they agree to support the research as one or two people can quickly decide to support the research and provide access to whatever materials they are willing to share. 

They may not have time or patience to read or sign formal agreements, however an email summary of any verbal agreement helps to sum up that agreement and provide them the opportunity to confirm, clarify, or reject the contents of the agreement. These agreements are best addressed in the engagement phase of a case study, however occasionally they [also] occur in later phases for instance if the scope of the study increases and/or additional mobile analytics tools are introduced.


\subsection{Scientific value}
The importance and relevance of the research topic, to seek ways to improve the quality of mobile apps and the processes used when developing those apps have been confirmed from multiple sources including academia and industry. Mobile analytics are also seen to be important as evidenced by the practices of app developers who include mobile analytics in over 75\% of all Android apps in Google Play and by the ongoing development of a multitude of mobile analytics tools and related services. This research aims to contribute scientific value through applying the methodology to contribute to knowledge about the practices, the tools, and the outcomes of using mobile analytics tools as part of development practices. 

As the apps run on end users' mobile devices the risks of the data collection and the use of that data are also considered throughout the research.

The research is situated in real world projects who develop and publish real world apps, and it incorporates aspects of internal, external and ecological validity (a topic discussed in the methodological validity section starting on page \pageref{methodology-threats-to-validity-section} and in the discussion on validity starting on page \pageref{discussion-threats-to-validity-section}).

Permission was obtained from the app development teams to share findings with the mobile analytics development teams and vice-versa as and where appropriate (it was not applicable in all the cases).


\subsection{Beneficence}
\textbf{Beneficence}: aims to maximise the overall benefits for all the stakeholders and harm none. This includes the people who participate in the development teams, their organisations (where applicable), and the end users of their mobile app(s). 

\begin{itemize}
    \item For the app developers, they had the opportunity to use mobile analytics as they wished (including not using it) which offered them an additional technique and for some new skills.
    \item For the app developers' organisations, they had ongoing issues with the stability of their mobile apps and they would be able to benefit by having various stability issues considered and potentially addressed which would improve the stability and therefore the quality of their mobile app(s).
    \item For the tool developers and their respective organisations they had the opportunity to receive insights into the behaviours and into the use of their tools. They were able to freely use the material that was shared with them to improve their mobile analytics tools and related services.
    \item For the end users, they did not contribute any additional information nor did they need to do anything beyond what they chose to do in terms of using and upgrading the respective app on their devices. Improvements in the quality of the app had the potential to improve their user experience when using the app, through the app being more reliable and performant.
\end{itemize}

The case studies also involved development teams and their relationship with their organisation. 
As \citet[p.2]{robinson2019_applying_endosymbiosis_theory_tourism_and_its_young_workers} observe: \emph{``Business, or work, ecosystems are a community of interacting organisations and individuals (or groups) – or the organisms of the commercial world"}. Their research was into the relationships of tourism and its young workers, and the possibility for exploitation in either or both directions. The challenges in their domain may apply to this type of case study based research therefore the research was performed with due consideration of the potential adverse effects for any of the parties involved in the case study. 

A concrete example of how the consideration was applied is where the researcher held one-to-one working sessions with various developers in several of the app-centric case studies to discuss ways that using mobile analytics could help the developer to work more efficaciously and in ways that reduced the risk of developers being `blamed' for causing or neglecting stability issues in the app as they now have the opportunity to address these issues before their effects become widespread or public.

\newthought{Beneficence: the need for beneficial relationships}

This research aimed and aims to provide benefit all the participants and for any relationships to be mutually beneficial.

Mutualism, commensalism, parasitism, predation and competition are five types of symbiotic relationship. % ``There are five main symbiotic relationships: mutualism, commensalism, predation, parasitism, and competition."  Symbiosis: The Art of Living Together https://www.nationalgeographic.org/article/symbiosis-art-living-together/ 
Of these the last three may produce adverse outcomes for at least one participant, therefore the research aimed to limit working relationships to those that would provide mutual benefits for all the stakeholders.

In computer science research that involves organisations and live projects the type or types of symbiotic relationship(s) are another key consideration. The candidate projects and their organisations need to be confident that if they participate as case studies in research that they will not suffer in the relationship. If they see mutual benefits of the research they may be more willing to actively participate. 


\subsection{Confidentiality}
\textbf{Confidentiality}: protect the confidentiality of participants and also protect the confidentiality of information provided and/or gleaned during the case study unless a) the work is in the public domain, b) permission has been granted to make the information public \textit{e.g.} as part of this research.


\subsection{Steps taken to mitigate the relevant ethical issues in the app case studies}
For the hackathons the participants were volunteers who were already working on the respective opensource project. They freely volunteered to participate and their contributions in terms of development artefacts were (and are) public. Ethical approval obtained from the Open University for the workshop at the TestFest 2020 Conference in Poland. The ethics approval was requested as the workshop was intended to include personal, voluntary contributions of how the participants thought and felt about the use and efficacy of mobile analytics for software testing of the respective mobile apps. The workshop participants gave express consent for their contributions and any additional artefacts to be made public as part of the research. They were volunteers \textit{who wanted to actively participate in the workshop for their own intrinsic reasons - to learn and practice new skills, concepts, and mobile analytics tools.}

The participants were briefed and gave their permission either individually or on behalf of their organisations to use the material they freely provided. Several have reviewed my research and provided constructive feedback which has been applied. 
Note: It has not always been practical to reach all the participants, for instance some are no longer reachable.
%
Participants are anonymous with a couple of exceptions: 1) when the information is already public, 2) where they were happy to be identified in public as contributors to this research.  

The research did not involve other human subjects (such as the end users), the data is related to apps and how the app was used and how the app performed, humans were not and are not the subject of the research. No PII information was collected by the analytics tools used.

The majority of specific findings in this thesis are for opensource, freely available, apps without any restrictions on sharing the findings of the performance of the apps. 

The commercial project is subject to various professional legal agreements that include confidentiality and intellectual property considerations. Recreated, anonymous examples are used to protect the confidentiality of the the organisation, the development team, the app, and the artefacts.

For the interviews prior consent was requested and freely given. For the smaller organisations the participants were decision makers for the project and able to act on behalf of the project \textit{i.e.}, they did not need any additional approval. For the larger organisations, representatives of the organisation sought and obtained legal approval for the work covered in the respective case study.

Some of the opensource projects that form part of this research received and accepted pull requests from the researcher, these were freely given and freely received and have no known monetary value.

Note: during the case studies I was also a member of three relevant professional bodies: the British Computer Society (BCS), the IEEE, and the ACM, and worked to abide by their respective codes of conduct~\citep{bcs_code_of_conduct_2021, ieee_and_acm_code_1999on}.


\subsection{Steps taken to mitigate the relevant ethical issues in the app case studies}
For the mobile analytics tool providers they choose not to ask for confidentiality agreements or for any of the findings to be withheld from publication. 

With the engineering team at Google the respective managers and the overall director of Google Play at the time were explicitly aware of the research and were happy to openly discuss and debate the findings that emerged from the app-centric case studies. They read and reviewed research publications in order to check the findings, they suggested some improvements  They requested and received a report of all the various findings with examples of flaws found by this research with the aim of helping them to consider understanding and addressing them. 

With the Google engineering team we also discussed their various policies and terms of service that apply for app developers in terms of their applicability to the research being considered. Finally, we explored whether they would be willing to share confidential material subject to signing a confidentiality agreement; this did not come to fruition and therefore no confidential material was received from Google.

The co-founders of Iteratively (the CEO and CTO) confirmed on multiple occasions they were happy for any of the discussions and materials they had provided could be used freely for research purposes and freely used generally. They volunteered and provided additional materials, for instance from their market research. They have also provided permission to work directly with one of their development team to answer any ongoing questions, topics, and concerns.

Despite not having confidentiality agreements in place various details that emerged from working with these organisations have been kept private where I consider it appropriate to do so, for instance some information was provided off-the-record and/or informally which was not approved to be made public. 

Both organisations have been offered the opportunity to read this thesis so they can suggest corrections and/or offer additional insights, etc.

\newthought{Agency of participants and their organisations}

Another consideration is the concept of `agency' that the organisations and the relevant people are free to choose whether they wished to participate in the research. Some candidates declined to participate in the research on behalf of their project or organisation for various reasons. A common reason was lack of time on their part, another was that some candidates perceived the research would not be acceptable to their organisation, for instance owing to confidentiality or business risk.

The participants choose their model of engagement, this means the research needs to be adapted to their engagement model, availability, and ways of working. The researcher may need to bridge between and/or mediate between the academic research ways of working and those practiced in industry, and here in the domain of mobile app development. In particular the researcher needs to uphold the expectations of both academia and industry, this may be easier for someone who has sufficient experience and competence in both ecosysystems.


\begin{comment}
\begin{itemize}
    \item Ethics review for Workshop in Poland (and then for various reasons the contents of the workshop were not viable because of the effects of COVID-19.
    \item No other human subjects, the data related to apps and how the app is used and performs, humans are not the subject of the research.
    \item Opensource, freely available apps without any restrictions on sharing the findings of the performance of the apps. No PII information collected by the analytics tools used.
    \item Semi-structured interviews with various individuals in their professional and/or project capacities.
\end{itemize}
\end{comment}



\begin{comment}
TODO papers to consider discussing in the ethics section include: 
\begin{itemize}
    \item ``The human is the loop: new directions for visual analytics"~\citep{endert2014_the_human_in_the_loop_new_directions_for_visual_analytics}
    \item ``Not All Trust Is Created Equal: Dispositional and History-Based Trust in Human-Automation Interactions"~\citep{merritt2008_not_all_trust_is_created_equal_etc}.
    \item \emph{``Symbiosis
Symbiosis refers to the partnership (usually long-term) that is established between two or more organisms. In microbiology, symbiotic relationships are often established between a microorganism and its host, and the partnership can be mutualistic or parasitic."} \url{https://www.nature.com/subjects/symbiosis}
\end{itemize}
\end{comment}


\section{Repeatability}
\textbf{Expand on:} What's hard to repeat (and why), aims to improve and demonstrate repeatability of the practices applied in this research.
\textbf{Absolutely key is what another researcher would do.}

One objective is to make the \emph{post-hoc} analysis repeatable, where other researchers can perform the analysis and obtain similar results; therefore this section explains various patterns of analysis and there are various worked examples provided in the individual case studies.

\isabel{suggests repeatability is part of good research ethics.}
\isabel{Research as a political statement c.f. her transfer report.}

\begin{comment}
TODO papers to consider discussing here include: 
\begin{itemize}
    \item ``R3: repeatability, reproducibility and rigor"~\citep{vivek2012_r3_repeatability_reproducibility_and_rigor}

\end{itemize}
\end{comment}

\clearpage