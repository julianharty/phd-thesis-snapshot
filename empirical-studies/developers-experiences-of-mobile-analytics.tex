\section{Field Reports from Commercial App Developers}~\label{section-field-reports-case-studies}
The research includes field reports from developers of three distinct commercial Android apps:~\href{study-moonpig}{\nameref{study-moonpig}}, \href{study-moodspace}{\nameref{study-moodspace}}, and \href{study-localhalo}{\nameref{study-localhalo},}; one from a charitable foundation:~\href{study-greentech-apps}{\nameref{study-greentech-apps}}; and SmartNavi~\href{study-smartnavi}{\nameref{study-smartnavi}}. Each uses mobile analytics in their regular software development practices and at least one interview per project actively develops the software used in the app.

The interviewees agreed to provide their perspectives and experiences of using mobile analytics as part of their software development practices and gave permission for whatever they provided to be freely used and shared. They also obtained their organisation's permission and approval if they were not able to act on behalf of their organisation. With the exception of SmartNavi, their source code was not made available - SmartNavi is an opensource project and the codebase is public on GitHub.com.

Each person was interviewed at least once, either face-to-face or online \textbf{TODO} check interview details for MoodSpace. They also provided at least one set of their analytics reports, one - LocalHalo - provided read access to the two mobile analytics tools they used as part of their development practices. 

\begin{figure}
    \centering
    \includegraphics[width=15cm]{images/my/developer-interviews-six-perspectives-2x3-matrix.png}
    \caption{Six Perspectives for the Developer Interviews}
    \label{fig:six-perspectives-developer-interviews}
\end{figure}

Figure~\ref{fig:six-perspectives-developer-interviews} indicates which of the six perspectives these interviews contribute to. The main perspective is, unsurprisingly, the developers' perspectives on their uses of mobile analytics, however they also contribute through the source code of the SmartNavi project, and in understanding the status quo of various analytics tools through their experiences and through the access LocalHalo in particular provided to the analytics tools they used.


\dotfill
\subsection{Key considerations and actions}
Table of what I have for each case, which cases are covered in each analysis.
TODO Report my analysis methods for these sources.

Do I do collective findings? or as distinct stories.

Cases vs data types, complete the cells with answers, not just ticks.


\subsection{Characteristics of the commercial apps}

\begin{itemize}
    \item Technologies
    \item Analytics available to the team
    \item Development priorities for their use of analytics
\end{itemize}

\clearpage
\subsection{Moonpig}~\label{study-moonpig}
Moonpig is an e-commerce business in Europe that sells greeting cards and related gifts online. They have a highly rated mobile app, with overall ratings of 4.8/5 in Google Play and the Apple App Store. Their software engineering team have been actively involved in encouraging the wider software engineering community to learn and practice good software development practices, for example by hosting Coding DoJos~\footnote{Historical examples available online on twitter \url{https://mobile.twitter.com/moonpigtech} and in a \href{https://www.codurance.com/publications/newsletters/2020-02-13-newsletter}{Codurance newsletter} from February 2020, for example.}.
{\renewcommand{\arraystretch}{0.8}% Tighter
\begin{table}[htbp!]
    \centering
    \small
    \setlength{\tabcolsep}{1pt}
    \begin{tabular}{ll}
       % Question &Answer  \\
       \toprule
       Website &\url{https://www.moonpig.com/uk/} \\
       Founded &2000 \\
       Business Domain & Greeting cards and gifts \\
       Business type & e-commerce \\
       Technologies  & Native apps, Robospice, \\
       & AWS, GraphQL, nodeJS, \\
       & Commercetools, ContentStack, ... \\
       Source code  &Closed and not available for research \\
       Analytics used by team &Firebase, Google Play Console \\
       Development Practices & High performance engineering, ATDD, \\
         & micro-services architecture \\
       \midrule
       User base &100,000's for the Android app\\
       Installations &1,000,000+ for the Android app\\
       \midrule
       Research methods &In person interviews, email discussions, remote testing \\
       Analytics collected &Google Play Console with Android Vitals \\
       Research software &They used Vitals-Scraper, otherwise none applicable.\\
       Additional data collected &Interview notes and emails \\
       Active period &June 2019 `\\
       \bottomrule
    \end{tabular}
    \caption{Case Study key facts: Moonpig}
    \label{tab:moonpig_anaytics_overview}
\end{table}
}

\begin{comment}
Sources: 
- 4.8 rating in Apple App Store https://apps.apple.com/app/id393730279
- 4.8 rating in Google Play, and install count  https://play.google.com/store/apps/details?id=com.commonagency.moonpig.uk
- ATDD https://medium.com/moonpigtech/the-android-testing-approach-part-1-atdd-6e26e3851c08
- High Performance Engineering https://medium.com/moonpigtech/working-remotely-as-a-high-performance-engineering-team-at-moonpig-957b267de1d4
- micro services (in 2020) https://medium.com/moonpigtech/introducing-the-moonpig-engineering-blog-c3fde37f06bd
- Robospice - email discussions and Android Vitals crash reports
\end{comment}

\begin{figure}
    \centering
    \includegraphics[width=13cm]{images/android-vitals-screenshots/moonpig/real-time-crashes-Screenshot-2019-06-10-at-15.42.34.png}
    \caption{Android Vitals: Moonpig snapshot of top crashes \nth{10} June 2019}
    \label{fig:av-moonpig-top-real-time-crashes-10-jun-2019}
\end{figure}

In Figure~\ref{fig:av-moonpig-top-real-time-crashes-10-jun-2019} Android Vitals shows the most frequent crash clusters for the production releases of the Moonpig Android app. The majority of crashes were an \texttt{IllegalStateException} in a third-party library SpiceManager, part of the RoboSpice opensource project https://github.com/stephanenicolas/robospice . 

This crash is an excellent example of how changes and new developments in the ecosystem can render what was reliable working software into software that is no longer fit for purpose, \emph{i.e.} RoboSpice was developed in 2012 to help developers simplify coding of asynchronous networking requests \url{https://github.com/stephanenicolas/robospice} and the library worked really well at the time and for several years afterwards. It became popular as a result and was used widely by many teams, including at Moonpig. However, as the Google Android platform morphed some of the changes to Android were incompatible with RoboSpice. And with Android Oreo (Release 8.0) changes to the way background services worked broke the functionality of the library sufficiently that the project was archived by the creator and project owner \url{https://github.com/stephanenicolas/robospice/issues/467} 

For apps that used RoboSpice the crash rate increased on the newer releases of Android, for example Figure~\ref{fig:av-moonpig-crash-rate-groupings} shows the crash rates by Android version were lowest on Android 7, higher for Android 8 and 8.1, and significantly higher again for Android 9. The teams needed to allocate time and energy to finding a suitable alternative to RoboSpice and then implement and test the new approach. Android Vitals helped provide an indication of the effects of the crashed and therefore provided evidence the development team could use to assess and prioritise the work.

\begin{figure}
    \centering
    \includegraphics[width=13cm]{images/android-vitals-screenshots/moonpig/Screenshot 2019-06-10 at 15.41.23.png}
    \caption{Android Vitals: Moonpig various groupings of the crash rate \nth{10} June 2019}
    \label{fig:av-moonpig-crash-rate-groupings}
\end{figure}


\clearpage
\subsection{Moodspace}~\label{study-moodspace}
Moodspace is an Android app aimed at improving mental health through various exercises incorporated into the app. It was released in 2019, with over 150K downloads by early 2020~\citep{objectbox2020_moodspace_interview}. Ian Alexander, the interviewee, was the software developer, co-founder, and runner of the company~\citep{objectbox2020_moodspace_interview} so he combines technical and operational responsibilities. He is an experienced app developer and also trained as a chemical engineer.  % https://www.linkedin.com/in/ian-alexander-01353340/

(\nth{13} June 2019)
% Source https://mail.google.com/mail/u/0/?q=kiwix+crash+rate+#search/kiwix+crash+rate/KtbxLzGHgCVznXxsGDjVgJGLFwmmGXxvxq
I love Android vitals - especially \textbf{core vitals}. Seeing your app in comparison to apps of peers provides some great motivation to step up your game. The only issue I have with core vitals is that I can't see them all! We are by no means a big app, so don't have enough data to meet googles standard for anonymised results, so results for most of the core vitals are hidden. I don't quite understand why this should be the case as the headline figure of your apps performance surely doesn't have to rely on anonymous data? Whereas the drilled down details of a core vital should be anonymous, so maybe the details view could just be blocked instead of hiding the entire core vital? To give context, MoodSpace has at least 4k monthly users, so there must be plenty of apps which get little or no use form core vitals, simply from them being hidden.

As for the other tools Google provides:
\begin{itemize}
    \item \textbf{ANR \& crashes} - I usually use crashlytics, so never really use this tab. the reason being, that the play console used to be very unreliable for crashes. But by the looks of it, it seems to have improved a lot, and pretty much matches the crashes I see in crashlytics now. Looking at it now, I actually noticed an ANR which could probably be fixed!
    \item \textbf{Pre launch reports} - I don't usually use this. Although this tool provides a nice safety net, it's quite basic, so any bugs that would cause the pre launch tests to fail have been found pre uploading through some quick manual testing. I actually ended up ignoring most pre launch reports when accessibility problems used to trigger failures, as we don't really have the resources to handle accessibility for now. But this seems to have changed in the last few months and accessibility problems don't cause a host of errors - maybe a way to ignore issues would be useful? In fact, pre-launch reports currently don't run on my app and fail with an incompatible apk error, not too sure why thats the case...
    \item \textbf{App signing} - very useful! and always use this since it was added.
    \item \textbf{Seperate release tracks} - love them! Especially since the addition of the internal track. The only issue was that I couldn't easily distribute debug versions of the app from the play store, and had to use a 3rd party tool to achieve that. Although Google have recently added Internal app sharing which should remedy that problem - however, I haven't figured out how to integrate that into our continuous integration process quite yet.
    \item \textbf{App bundles} - I'm still trying to integrate this, but as our new apps going to be heavily illustrated, this should cut our apk size significantly
\end{itemize}

As for several things I think are missing:

\begin{itemize}
    \item A gradle plugin to integrate play store uploading into CI processes. I currently use a 3rd party plugin to do this, but it would feel a little more secure if it came from Google.
    \item Top line core vitals figures even if you don't have enough users!
    \item Someway for testers to download old apks from either internal app sharing, or the internal release track.
\end{itemize}

I've attached the ANR and crash rates for the app. As I say, I usually use Firebase crashlytics, so don't really fix crash issues from Play store data.

You're welcome to use any data or comments in your research! If you do use it, please send over a copy.

(\nth{15} June 2019)
I think there's two things which has helped keep the app quite stable:
\begin{itemize}
    \item The app has the benefit that it's been around for quite a while without any major features being added. So most updates have been small and incremental, which has gradually increased it's stability. (This may change when the new, big update drops...)
    \item The app doesn't use any api, so all datas stored in very fast ORM databases like object-box (and uses memory caching). This enables the app to be mostly synchronous, which hugely cuts down on complexity of code. i.e. no need to handle loading, errors, or concurrency. This is a bit benefit! And cuts down on errors significantly, with no real impact on performance for users. To illustrate that it has little impact on users, I use firebase performance to run a trace on some methods that call the ORM/cache - it's peak duration is 40ms while the majority of calls take 3-6 ms.
\end{itemize} 

Crashlytics only covers the crash report of Android vitals, so unfortunately there's no way to get things like battery usage of ANR reports unless Google makes those reports available :(. In terms of crashes, I'd always prefer Crashlytics to Android vitals, simply because there are added features like non-fatal reporting and logs which can make surfacing the cause of errors much easier (but do take need added effort to integrate compared to android vitals).

There's been a change of branding of the app development organisation to \href{https://play.google.com/store/apps/developer?id=Chachi+Productions&hl=en_GB&gl=US}{Chachi Productions}. \url{https://www.appbrain.com/dev/Chachi+Productions/}


\newthought{Notes to integrate into the case study}
\begin{itemize}
    \item \url{https://www.psycom.net/25-best-mental-health-apps} Helps set the context for these apps.
    \item \url{https://objectbox.io/moodspace-mobile-app-use-case/} An interview with the founder Ian Alexander on the technological choices, particularly objectbox as the database. He'd mentioned the performance was lightning fast and one reason the app was performant and reliable. 
    \item ``Built MoodSpace, a digital platform empowering everyone to take control of their mental health. MoodSpace began as a side project which later took on a round of funding and with a team of 6 supported a user base of 300k. Although no longer pursued as a business we continue to maintain the project and release regular updates to 10s of thousands of active users. The most recent iteration of the app was built with Kotlin, clean architecture, MVVM, Data binding, Gitlab CI, Coroutines/Flow, and ObjectBox and was architected to enable the use of Kotlin Multiplatform to share ~60\% of the codebase between platforms."~\url{https://www.linkedin.com/in/ian-alexander-01353340/}
    \item 4.18/5.00 User Experience rating \url{https://onemindpsyberguide.org/apps/moodspace/}
    \item ``To make the app work well at all we collect the following anonymous data:
    \begin{itemize}
        \item Crash reports: If you've never seen the app crashing, it's because as soon as one happens, we get a crash report. A little red light flashes in our office, a loud siren blares, and we release a fix right away. It's quite annoying actually.
        \item Analytics: We assume you're going to use the app a certain way. We're almost always wrong, and you often surprise us. Analytics lets us see how people like you actually use the app, so we can make improvements to the right places. Analytics can use the Google Advertising ID to identify you. This doesn't tell us anything about you (it's just some numbers and letters), but if you really want to trick us you can reset your Google Advertising ID at any time. Go to your device Settings > Google > Ads."
    \end{itemize}~\citep{moodspace2021_privacy_policy}
\end{itemize}


\clearpage
\clearpage
\subsection{Local Halo}~\label{study-localhalo}
Table~\ref{tab:local_halo_anaytics_overview} provides a succinct overview of this case study

{\renewcommand{\arraystretch}{0.8}% Tighter
\begin{table}[htbp!]
    \centering
    \small
    \setlength{\tabcolsep}{1pt}
    \begin{tabular}{ll}
       Question &Answer  \\
       \toprule
       Website &\url{https://www.localhalo.com/} \\
       Founded &2018 \\
       Business Domain &Digital neighbourhood groups in UK.\\
       Business type &Startup, two co-founders: CEO and CTO \\
       Technologies  &React Native for cross-platform Android and iOS development \\
       Source code  &Closed and not available for research \\
       Analytics used by team &Sentry, Mixpanel, Google Play Console \\
       Development Practices &Developers in Ukraine, London, and Kazakhstan \\
       \midrule
       User base &7,000 registered users, 1k to 2k monthly active (Jan 2020) \\
       \midrule
       Research methods &Interview, email discussions, bug analysis, use of mobile analytics \\
       Analytics collected &Live access to: Sentry, Google Play Console with Android Vitals \\
       Research software &Vitals-Scraper used to preserve results \\
       Additional data collected &Interview notes and emails \\
       Active period &Jan 2020 to June 2020 \\
       \bottomrule
    \end{tabular}
    \caption{Case Study key facts: Localhalo}
    \label{tab:local_halo_anaytics_overview}
\end{table}
}

The aims and contributions of this case study include:
\begin{itemize}
    \item \textbf{A startup case study}: how a small startup with a distributed team chose to incorporate mobile analytics into their product and working practices.
    \item \textbf{Their perspectives on mobile analytics tools}: They used three mobile analytics tools, each for distinct purposes. 
    \item \textbf{Mobile Analytics for cross-platform apps}: Including the opportunity to research Sentry Mobile Analytics. 
\end{itemize}


\subsubsection{LocalHalo: Introduction}
LocalHalo was a startup based in London who made a social network for neighbours~\footnote{\url{https://ain.ua/en/2019/10/18/localhalo-raises-500k/}} with developers in Ukraine, London, and Kazakhstan.~\citep{karpenko2019_localhalo_a_social_network_for_neighbors}. 


\subsubsection{LocalHalo: Background - How the case study came about}
The case study was started through initial discussions with the CEO, James Routledge, in November 2019. He arranged for the CTO, Andriy Marin, to have a discussion to agree on how they could help and participate in this research. This discussion was an online call on \nth{17} Jan 2020 where we agreed on the scope, etc. Two pages of \emph{ad-hoc} handwritten notes were made during the call, and a summary was written up and emailed to them of what was agreed during the call. They responded and confirmed they data and findings couple be used as part of this research.

They confirmed and provided access to two of the three mobile analytics services they used, these two were Google Play Console with Android Vitals and Sentry~\footnote{\url{https://sentry.io/welcome/}} which was used to track technical issues with their app and website. We agreed the research would not have access to the third analytics service, mixpanel~\footnote{\url{https://mixpanel.com/}}, which they used for user behaviour and related analytics. This service included personally identifiable information of their users and would not be useful for this research. We also agreed no access would be provided to the source code as that was deemed sensitive. 

\subsubsection{LocalHalo: development microcosm}
The development team used the Expo development platform \url{https://expo.dev/} to create native apps that ran on Android and iOS apps. These apps were released on Google Play and Apple's App Store respectively. The CTO was actively involved in writing and maintaining the source code and was supported by developers in three locations, in the Ukraine, London, and Kazakhstan.~\citep{karpenko2019_localhalo_a_social_network_for_neighbors}. 

Little additional information was available in terms of their development or release practices for their mobile apps. One observation is Expo claims to automate the release process to the app stores so the LocalHalo development team may have relied and used the Expo service.

\subsubsection{LocalHalo: applying analytics to the development practice}
They seldom used Google Play Console with Android Vitals, it did not suit apps developed in React Native as the crashes did not actually crash the shell app which wraps the React Native app~\footnote{Note, other developers have asked about this behaviour, for instance on StackOverflow \href{https://stackoverflow.com/questions/66166824/native-crash-reporting-for-expo-deployed-to-android/}{Native crash reporting for Expo deployed to Android?}} 
(which is what appears to be monitored by Android). The shell app restarts the react native app automatically.

The development chose to use separate analytics services for user behaviour analytics (Mixpanel) and issues such as crashes and anomalies (Sentry). Presumably the respective SDKs were incorporated into the React Native source code~\footnote{Some of the errors reported by Sentry indicate the integration used TypeScript (which is supported in React Native, see \url{https://reactnative.dev/docs/typescript}. Mixpanel also provides an opensource wrapper that supports react native \url{https://github.com/mixpanel/mixpanel-react-native}.}~\footnote{Note: they also incorporated Sentry into their website and provided access to Sentry for their website, however the website is out of scope for this research and will not be considered further here.}. Only two people have access to Sentry: the CTO and me, therefore any other members of the development team would have indirect access, for instance via screenshots and/or bug reports raised by the CTO. 

They saw value in using analytics to improve business results, for instance for App Store Optimisation to improve the ranking of their app in the app stores. The CTO made a key observation during the interview \emph{``If you have lots of crashes you have zero chance of being promoted [by the app store]."}

In terms of user experience analytics the CTO also observed \emph{``To improve user retention you need to do both, eliminate the bad stuff [and] improve the good stuff [to] increase value"}.

The overall impression was the team had decided to incorporate mobile analytics to help them provide a reliable and valuable service for their current and hoped-for users where the team would address errors on an \emph{ad-hoc} basis.

\subsubsection{LocalHalo: data collected and methods used for collection}
\newthought{Emails from Sentry}
The majority of data was provided by email from Sentry~\footnote{Note: access has been available to Sentry Mobile Analytics since late January 2020 and has continued through to September 2021. Access to Google Play Console was provided from \nth{20} January 2020 for roughly six months.}. Some of these emails were for alerts~\footnote{\url{https://docs.sentry.io/product/alerts/}} configured in Sentry. There were two alerts: 1) Send a notification for new issues that continue for at least 30 minutes, and 2) Send a notification for new issues for issues seen by more than 10 users in one day. Sentry also sent weekly summary reports by email for the project from \nth{25} May 2020.

In the email reports there were approximately 100 alerts were reported in the five months of the case study. These include alerts for iOS devices, Android devices (4), % https://mail.google.com/mail/u/0/#advanced-search/subset=all&has=localhalo+react-mobile+android&within=6m&sizeoperator=s_sl&sizeunit=s_smb&date=2020%2F01%2F01&query=localhalo+react-mobile+android
and also some triggered on a developer's laptop. Five of the alerts are for ANRs. Note: a significant proportion of errors started being reported towards the end of the active case study and subsequently for native Java Exceptions. These include: RuntimeException (9), IllegalArgumentException (1), JSApplicationIllegalArgumentException (1), NoSuchFieldException (1), NullPointerException (1), and UnexpectedNativeTypeException (1); \emph{i.e.} a total of 14 alerts reported for Java exceptions between \nth{1} May 2020 and \nth{30} Jun 2020. % https://mail.google.com/mail/u/0/#advanced-search/subset=all&has=localhalo+react-mobile+android&within=6m&sizeoperator=s_sl&sizeunit=s_smb&date=2020%2F01%2F01&query=localhalo+react-mobile+android
These exceptions may map to those counted by Google Play Console's Dashboard, however given the paucity of information in Android Vitals this was not practical to verify.
%For the full set use https://mail.google.com/mail/u/0/#search/localhalo+react-mobile+java

Other errors included runtime errors with libraries used by the developers within the app, such as for Apollo GraphQL and Mixpanel, logic errors, and other programming mistakes. 

The weekly reports indicate the majority of detected issues were existing issues, for instance 90.1\% in the week 14 - 21 September 2020, as illustrated in Figure~\ref{fig:localhalo-sentry-weekly-report-21-sep-2020}.

\newthought{Interactive access to Sentry}

\newthought{Interactive access to Google Play Console with Android Vitals}

Vitals Scraper was also used to programmatically save visual reports and extract summary data, and crash and ANR clusters.

\newthought{email conversations}


\begin{wrapfigure}{R}{0.5\textwidth}
  \begin{center}
    \includegraphics[width=0.4\textwidth]{images/google-play-console/resized-25pct-appdashboardplace_555059634831.png}
  \end{center}
  \caption{Google Play Console Dashboard for LocalHalo Android app}
  \label{fig:gpc-dashboard-for-localhalo-android-resized}
\end{wrapfigure}



\subsubsection{LocalHalo: Findings and Results of the Case Study}
Android Vitals does report a count of crashes, however it does not provide any details of these crashes in Android Vitals, possibly as the overall volumes are low (low tens of crashes per month).


\subsubsection{LocalHalo: Outcomes for the company}


\subsubsection{LocalHalo: Discussion}


\subsubsection{LocalHalo: Contributions to the research and where they are located in the rest of this thesis}

\subsubsection{Earlier stuff follows and needs relocating above here.}



\subsubsection{Experiences of using mobile analytics}
Localhalo incorporate two analytics libraries into their cross-platform mobile application: sentry for crash reporting and mixpanel for business-oriented usage analytics. For their Android app they also have access to Google Play Console.

They experience numerous crashes reported by Sentry, which occur within the React Native runtime environment. Sentry provides email alerts to the development team together with summary reports, (Figure~\ref{fig:localhalo-sentry-weekly-report-21-sep-2020} is example for the period~\nth{14} to~\nth{21} September 2020), and online access to their analytics.

A release in March 2020 had a high crash rate for the production release of their Android app. The top crash cluster was for:

\texttt{java.lang.RuntimeExceptionhost.exp.exponent.experience.a\$b.run}. 

This was traced to a problem in the expo library they used in the app~\url{https://github.com/expo/expo/issues/5839}~\footnote{Expo is a very popular opensource platform for making universal native apps that run on Android, iOS, and the web \url{https://github.com/expo/expo}.}. In that issue several developers for different Android apps provide data from Google Play Console confirming they also receive similar crash clusters. The cause has not yet been definitively traced or addressed, however for the LocalHalo app the crashes stopped being reported once a new release of the Android app, release 1.3.0, was released around \nth{6} April 2020.

In this GitHub issue, 5839, one of the contributors reported their app's crash rate had gone from below 1\% to 10\%~\footnote{\url{https://github.com/expo/expo/issues/5839\#issuecomment-571271045}} and that users were writing terrible reviews. Another contributor observed that the stack trace in Android Vitals is not very useful, what's needed is the error message that appears in the device's log immediately before the stack trace~\footnote{\url{https://github.com/expo/expo/issues/5839\#issuecomment-583838280}}; \phantomsection
\textbf{insight} the stack trace is not always sufficient to understand the failure\label{insight-expo-stack-trace-not-sufficient-to-identify-the-failure}.

\begin{figure}[htbp!]
    \centering
    \includegraphics[width=9cm]{images/localhalo/sentry-weekly-report-21-Sep-2020.png}
    \caption{LocalHalo: Sentry weekly report 14 - 21 September 2020}
    \label{fig:localhalo-sentry-weekly-report-21-sep-2020}
\end{figure}

\textbf{Sentry}

MUST-DO Check in Sentry for the various types of error. How actively are the development team reading, reviewing and addressing crashes being reported? 

\begin{figure}[htbp!]
\centering
\begin{minipage}{.5\textwidth}
  \centering
  \includegraphics[width=.8\linewidth]{images/localhalo/sentry-weekly-report-16-mar-2020.png}
  \captionof*{figure}{\nth{16} -~\nth{22} March 2020}
  \label{fig:localhalo-sentry-weekly-report-16-mar-2020}
\end{minipage}%
\begin{minipage}{.5\textwidth}
  \centering
  \includegraphics[width=.8\linewidth]{images/localhalo/sentry-weekly-report-23-mar-2020.png}
  \captionof*{figure}{\nth{23} -~\nth{29} March 2020}
  \label{fig:localhalo-sentry-weekly-report-23-mar-2020}
\end{minipage}
    \caption{Missing data reported in Sentry}
    \label{fig:sentry-missing-data-march-2020}
\end{figure}

\begin{comment}


\begin{figure}[htbp!]
    \centering
    %\subfigure[]{\includegraphics[width=0.4\textwidth]{images/localhalo/sentry-weekly-report-16-mar-2020.png}\caption{\nth{16} -~\nth{22} March 2020}\label{localhalo-sentry-weekly-report-16-mar-2020}}
    \subfigure[]{\includegraphics[width=0.4\textwidth]{images/localhalo/sentry-weekly-report-23-mar-2020.png}%\caption{\nth{23} -~\nth{29} March 2020}%\label{localhalo-sentry-weekly-report-23-mar-2020}
    }
    \caption{Missing data reported in Sentry}
    \label{fig:sentry-missing-data-march-2020}
\end{figure}
\end{comment}

%TODO fix the layout of the above figures: try: https://tex.stackexchange.com/questions/37581/latex-figures-side-by-side/37597#37597
Data missing from reports from~\nth{17} March to~\nth{25} March 2020 which affected the statistics around the time of the crashes related to Expo. Figure~\ref{fig:sentry-missing-data-march-2020} illustrates the gap across the two weekly reports. 

SHOULD-DO Consider summarising the crash totals per week. 

\textbf{MUST-DO} write up what we learn from the localhalo case study. external libraries can adversely affect the reliability of apps. Even small development teams can and do use multiple mobile analytics libraries. What can developers learn from the various reports provided by Sentry? How does cross-platform development in react-native affect the app reliability? are there crashes that only occur on Android or iOS? What's the correlation between crashes reported in Android Vitals and Sentry?

Note: The founders of local halo indicated~\footnote{Via their respective LinkedIn profiles: \url{https://www.linkedin.com/in/jamesroutledge/} and \url{https://www.linkedin.com/in/andriymarin/}} they are no longer actively involved in this project, and there are confirming indications in sentry.io as the app has not been updated in over a year, in contrast to the many updates they made previously. So even though the app remains online and available, and the analytics continue to report, there is no one actively maintaining the app or dealing with the errors being reported in the analytics.

\begin{comment}
Extra information available includes:
\begin{itemize}
    \item Their local communities for \href{https://docs.google.com/spreadsheets/d/1iqOvNjRlHIpoRzd61BcBLVkSxGvbta6vrzH2Jgc50aY/htmlview#}{COVID-19 volunteer support communities}.
    \item A social network for neighbors \url{https://ain.ua/en/2019/10/18/localhalo-raises-500k/}
    \item Stack Overflow questions (unpopular and seldom answered) \url{https://stackoverflow.com/questions/tagged/react-native+expo+crash}, a more general search does better \url{https://stackoverflow.com/search?q=\%5Breact-native\%5D+\%5Bexpo\%5D+crash}
    \item Some interesting projects to add device information to react native apps \url{https://github.com/robinpowered/react-native-vitals} and \url{https://github.com/robinpowered/react-native-device-battery}
    \item A long article, with screenshots in Polish or a similar language on publishing react native expo apps to both app stores \url{https://pagepro.co/blog/publishing-expo-react-native-app-to-ios-and-android/}
    \item \href{https://gist.github.com/himgupta229/83b4fd61f294020258976e3edb10b16a}{GitHub GIST: JSON for ANR in Sentry}.
    \item Expo recommends Sentry and documents the integration \url{https://docs.expo.dev/guides/using-sentry/}.
\end{itemize}
\end{comment}


\clearpage
\subsection{SmartNavi}~\label{study-smartnavi}
SmartNavi is an unusual opensource project that replaces GPS for navigation when the users are walking. It uses significantly less power than using a true GPS provider 

\begin{itemize}
    \item SmartNavi project website \url{https://smartnavi.app/home}
    \item Google promoted the project as an Android experiment \url{https://experiments.withgoogle.com/smartnavi}
    \item The app was one of the opensource codebases investigated as part of my collaborative research on logging using Google Firebase Analytics.
    \item First interview: Google Hangouts on \nth{28} Jul 2020
\end{itemize}

\newthought{email summary from \nth{28} Jul 2020 for my co-authors of the logging paper}

The project uses crash reporting (to catch, report, and analyse crashes) and mobile analytics (to try to ascertain where users do and don't use the app). In terms of our coding his use mainly focuses on events generated when users interact with his app.
The app is mature (5+ years) and originally had analytics he developed himself using MySQL, etc. to record the events. He did this mainly to learn about the topic, he was a bachelors|masters student at the time.

Over the years he's used Firebase and Google Analytics mainly for tracking the popularity and use of the app's features. One of the key challenges for this app in his estimation is helping users who don't understand GPS, etc. why the app may be relevant to them. It saves battery life because of the ways it tracks movement, and is mainly intended for people moving at walking pace (rather than in high-speed vehicles).

He also uses Firebase tools to perform A/B experiments related to tutorials and their usability and effectiveness.
He spent a lot of time a year or so ago where he actively focused on fixing crashes being reported in Fabric Crashlytics (since superseded by Firebase Crashlytics). The few crashes that remain are ones either too impractical to investigate (e.g. on a few unbranded low-end Android devices he has no access to) or in a third-party library that only occurs again on a few unusual devices.
The app can run in the background and provide GPS services to other apps e.g. to Google Maps. This has some interesting challenges where Android on various devices and in various conditions may suspend some background activities automatically. If analytics could help track how the background activity is working (and when it's no longer running) that might be very helpful to him and enable him to improve the app's behaviours when run in the background.

He's not proactively working on the app currently. He may spend more time on it soon if he perceives there are new users/use cases for the app. Separately to our research I may help find one or two projects that would be motivating for him e.g. related to mapping Tanzania.

He does coach students at his old university in Germany who are studying marketing. They do various practical exercises to help them gain better marketing skills. 

We've agreed that there's little more to be gained from asking more about his current quiescent use of analytics since he only checks the analytics infrequently in Firebase (he checked it while we were talking, previously it was around 8 weeks ago he thought). He is happy for others to collaborate on the project, to use it, etc.

\nth{30} Jul 2020

He's not actively working on the app, usage is currently in decline - he reckons few new users understand the benefits of the app (walking navigation using less battery than apps that use GPS exclusively). He's waiting/hoping for a fresh burst of interest in the app which will catalyse him into working on the app's codebase again. 

\subsection{Summary of Field Reports}
Active and ongoing use. 

%%% Ad-hoc notes from meeting Jakob D. Moonpig
% Business changes including the company's IPO mean the company has cut back on information they're willing to share internally and externally (e.g. for my PhD research).
