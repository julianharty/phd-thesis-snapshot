%%%%%%%%%%%%%%%%%%
% This is a useful place to keep various notes from calls and meetings as I'm writing up. It's been started late in the game so many notes predate and aren't in this file.
%%%%%%%%%%%%%%%%%%

\dotfill
\nth{7} Sept 2021
Discussion online with Jakob D. from Moonpig on his transition to MyPulse starting 13 Sept 2021. 
Call 7 Sep 2021 last 15 - 20 mins re my research… Notes are offline as they're incomplete and need work.
 \dotfill
Isabel session \nth{9} Sept 2021

\dotfill

In person working session with Joe Reeve 10 Sept 2021

We ran through the two case studies he was involved with (Kiwix and Catrobat). He emailed me our joint notes.

\dotfill

Notes from call with Marian Petre 10 Sept 2021
40 mins 13:35 - 14:15 ish.
 
heartbeat of android vitals

What's clear is my case studies are deep and immersive rather than case studies that ask developers questions in a survey. Also the very nature of what I'm working with means the case studies are opportunistic. I had this much access and no more... This is design I used, because…

It's vital to be very clear about what I've done AND DIDN’T DO, the data I gathered and how, the methods , I NEED TO BE CLEAR ABOUT WHAT I DID, WHEN, AND WHY.

For immersive case studies, there’s a risk of bias - explain how I worked to reduce the risk and/or effects of bias e.g. by interviewing developers of other apps.

By documenting the case studies, their methods and data, it's easier to show how the case studies compare and to be able to justify why I can compare them.

Needs explicit audit trail. What was missing is the coherence across the case studies. Help Marian to know where to look for the methodology.

Same order of sections, with the same name, per case study wherever this applies, however don't use a rigid structure blindly.

Given the dimensions and characteristics of the case studies we're also able also to comment on the aggregate.


Accumulate lessons per case study.
Independent data that leads to the same conclusion
Collegation and/or Triangulation.

Being aware of the potential for bias.
Look for opportunities to discover contradictions as a corroboration. 
Some contradictions lead to insights.

 \dotfill
Notes from call with Arosha 10 Sept 2021
40 mins or so on Skype.

We discussed tactics based on my earlier discussion with Marian and agreed on me working on the case studies and identifying common threads that can help feed the methodology chapter, etc. Arosha will skim through the related works chapter before next Wednesday's meeting and triage the content: must keep, needs serious work or removal, the rest.

I've created a vignettes folder where I'll have a go at adding some small additional examples from external projects, etc. that help amplify the work in the core of the case studies. These are likely to end up in the discussion chapter. The first example will be ObjectBox issue 605. They may need to be sidebars and include some brief context setting together with an explanation of their connection to the rest of my research.

 \dotfill
Yijun Sun 12 Sept 2021 16:00 - 17:20 ish
We started with brief on what I'm working on in the thesis (Red-thread, related works, case studies).

SE in practice, Mark Harman co-chair ICSE 2022 - be useful to look at previously accepted papers to help find the methodology. 
MUST-DO I'll sample 5+ papers from 2021, \url{https://conf.researchr.org/track/icse-2021/icse-2021-Software-Engineering-in-Practice?#event-overview} Yijun found 3 likely candidates, here's the 5 I've picked to get me started with studying their methodologies, etc.:
\begin{itemize}
    \item \href{https://arxiv.org/abs/2010.09974}{Scalable Statistical Root Cause Analysis on App Telemetry}
    \item \href{https://arxiv.org/abs/2010.09977}{Industry-scale IR-based Bug Localization: A Perspective from Facebook}
    \item \href{https://github.com/MobileSE/AndroSea/blob/main/paper/CompatibilityIssues.pdf}{Identifying and Characterizing Silently-Evolved Methods in the Android API}
    \item \href{http://arxiv.org/abs/2102.09336}{FIXME: Enhance Software Reliability with Hybrid Approaches in Cloud}
    \item \href{https://www.win.tue.nl/~aserebre/ICSE2021SEIP.pdf}{An interview study of how developers use execution logs in embedded software engineering}
\end{itemize}

Several more may also be useful once I've made a first pass through these 5. There are also some potentially relevant and interesting papers presented in 2020 \url{https://conf.researchr.org/track/icse-2020/icse-2020-Software-Engineering-in-Practice?#event-overview} e.g. ``Automated Bug Reproduction from User Reviews for Android Applications", ``Automatic Abnormal Log Detection by Analyzing Log History for Providing Debugging Insight", ``Debugging Crashes using Continuous Contrast Set Mining"

- Case studies is one type of empirical studies
- Logging is also empirical (second hand) 
Suggestion: Change the name to empirical studies as the umbrella term, then I can use case-study as the default primary research method for the case studies.
- Experimental case studies were sometimes necessary and can fit into empirical studies. 
Action: I will do.

Question to answer: How did I find out all the prerequisites in the framework in the red thread were essential contributing factors? - I need to provide some evidence for each.
The framework is a very strong contribution and may be a good contribution to the theory. 
Use the thesis order to re-order the red thread and to decide what to migrate. The MVP. Try to set deadlines for the whole thesis and each chapter. The main problem currently is fragmentation of the thesis.

I need a systematic approach to the entire thesis e.g. for grouping and separating content. 

Input, process, output (c.f. minimise couplings between software modules).
Consider removing 2/3rds of the content. It's currently very hard for Yijun to tell what can be cut.

Highlight the takeaway message, how it's been formed and how to evaluate the takeaways.

Balance the evidence with the claims of evidence.

Try creating a slide per important message - sufficient for a short 5 to 10 minute presentation. Anything extraneous shouldn't be in the thesis. <= 7 bullets per slide. So sim for at most 30 or so bullets in total. Aim to show the main idea and the main process in the presentation for the viva.

Present the slides in the order the ideas are developed and presented in the thesis. We can then discuss what's missing and what's extraneous. Need to consider the bandwidth of the examiners. I can also work on this presentation with Isabel. 

Next call with Yijun will be on Friday with Arosha iff I've completed the draft slides for the viva (a surrogate red thread), otherwise will fix a separate call to suit his availability, ideally within the next 7 days to establish and maintain momentum.

\dotfill

Isabel Monday 09:35 - 10:40 (including some discussions about her research)

100 words per chapter after defragging the red thread chapter.
2 dimensional table of case studies and their empirical methods. Consider qualitative vs. quantitative aspects. 
Consider gutting the red thread.
Draft the presentation for Wednesday and if so Isabel to review it.

1/3 length c.f. my 5* workload exercise, where we then relax the constraint to doing the current workload in 3 of 5 days per week allowing 2 days per week for process and self improvements.

Appium Keynote next Saturday - Isabel's happy to review the slides.

Joe's thoughts on tools as pivot points. Suggestion: Interview Joe as a participant's perspective as an Industry contribution.
c.f. the argument that research needs to be Political and Persuasive. 

OT: I'll introduce Isabel to the Espresso book author (met via Twitter).

Isabel's experiences of writing Flash Fiction where they're set non overlapping word lengths e.g. 75 - 100 words and 151 - 500 words. These non overlapping ranges force authors to edit drafts that are in the dead zone so they fit one or the other acceptable ranges. The contributions are improved through this constraint.

We've agreed I'll have a go at creating a draft for the short presentation in the next few days. I'll start by using some 5x3" cards in 4 colours which I already have. 
\begin{itemize}
    \item Have 20 cards in one colour representing each of the slides. 
    \item Allow up to 3 supporting cards per slide.
    \item Aim for 3 bullet points per slide. 
\end{itemize}
45 mins into the call we changed focus to her research, etc.

Summary of my unordered actions:
\begin{itemize}
    \item Apply changes suggested by Yijun to use Empirical Studies as the umbrella term.
    \item Create the card version of my viva presentation.
    \item Defrag the red thread.
    \item Write 100 words per chapter.
    \item Possibly review the entire thesis seeking content to remove from the thesis.
    \item OT: Follow up with the Espresso book author and introduce him to Isabel.
\end{itemize}

\dotfill
Marian Tues 14 Sept 2021 13:00 ish for an hour or so
Lots of discussions, e.g. on Kahneman's thinking fast and slow. summarised as the tables and the figures are the heart of the thesis, and once these are clear the rest of the writing should become unblocked. This led to the following immediate actions for me:
\begin{enumerate}
    \item Clarify the tables, their headings and labels, and remove or decode my 'shorthand'. Aim to make the contents of the tables orthogonal and for the items in the cells to be comparable (rather than having disparate content in some cells e.g. `Sophisticated' which is an editorial statement.
    \item Explicitly document the contributions of each study to the research questions.
    \item (later on) map contributions by each study to the items in the framework.
\end{enumerate}

She reiterated a clear, explicit audit trail of the data and the contributions from each study is critical. We also agreed that although the work on game and mechanics is really interesting it's not currently underpinned by my research, at least not in the thesis (it may be in my head).

\dotfill
Arosha Wed 15 Sept 2021 11:12 - 11:44 ish 
He wasn't able to review the literature review in the last few days owing to other demands on his time.
I brought him up to speed with recent conversations with Yijun, Isabel, and Marian. 

My recent changes improve migrating content from the red thread document to the thesis, revising the tables, and adding some material on empirical research.

Getting to clarity is vital, we agree. How that's achieved - through tables, presentation slides, writing 100 words per chapter/case study, are all possible ways to do so. 
He agrees the material on the game and mechanics is currently at risk of being a hostage to fortune. 
We reviewed the case study red thread tables together, 
Arosha to review the tables in the red thread in greater depth and add questions and notes by Thursday end of day.
I'm meanwhile continuing work on the case studies.

\dotfill
Marian Wed 15 Sep 2021 13:06 - 14:13
We spend most of the time digging into the 5 tables for the case studies. Good progress has been made, more is needed. There are ad-hoc inline notes in the latex I will need to address. 

Action - I'll create a monster table in a spreadsheet and review it tomorrow with Marian.

In the a monster spreadsheet: Ask have I really captured what I did, the data, the importance of the findings, what are the relative priorities. Context, establish what I did, what I collected, the findings, the insights, and the role in the thesis. 

\dotfill
Marian Thursday 13:00 for about 20 mins ish.

Action: Get on with it! :)

\dotfill
Marian Friday \nth{17} Sep 15:00 for about 15 mins.

We'll spend time on Monday \nth{20} Sept diagramming the case studies. 
Action: I'm working on revising the Catrobat case study. The monster spreadsheet is on hold (as is the draft slide deck for the viva) as we're working on the tables which serve a similar purpose of helping me create and clarify the story.

\dotfill
Arosha and Yijun \nth{17} Sep 16:30 48 mins.
We reviewed this week's progress. 
We also discussed and revised the proposed headings for the embedded case studies e.g. Kiwix and Catrobat. The revised set are in the Catrobat case study where I'll try them out. They're also written up in \href{structure-of-the-app-case-studies}{\nameref{structure-of-the-app-case-studies}} now.

\dotfill
Joe Reeve, in person discussion \nth{20} Sep, around 90 minutes.
We discussed my diagrams I'd made over the weekend (in preparation for my Keynote at the Appium conference). Joe had a go at redrawing one of my rough sketches. 

\dotfill
Isabel Evans, Mon \nth{20} Sep, 10am ish, online about 30 mins.

Mainly a sync up.

\dotfill
Marian, in person discussion \nth{20} Sep around 100 minutes.

Marian asked me questions and drew diagrams with some of my words in them. I also showed her various ideas and notes I'd made when preparing for my keynote for the Appium conference.

She suggested a good target to aim for is to complete the first draft of the complete thesis in 9 weeks - 9 chapters/9 weeks.

\dotfill
Marian, \nth{21} Sep 13:05 for about 32 mins.

We chatted about ways to structure the case studies, there's a vast amount and probably too much to keep in a single chapter. One suggestion is to write up the individual case studies and put these in the appendix which would compress the narrative of the core thesis.

Another (complimentary) option is to write an introduction chapter to explain and set the various perspectives in context and then a chapter per perspective. This then led to a fresh insight on the perspectives - with a revised approach with 4 perspectives in a 2x2 matrix.

I was distracted by David Pride's good news. back to work now...

\dotfill
Arosha, \nth{21} Sep, 50 mins

From 4 to 6 perspectives.

\dotfill
Marian, \nth{21} Sep, 10 mins ish.
A quick sense check of using the 6 perspectives.

What I need to provide for each study:
Context and orientation; then drive from the [6] perspectives.
A suggested order: Status Quo slice, then what happens if we interfere?

\dotfill
Marian \nth{22} Sep, 45 mins

We discussed what's needed in terms of structuring and populating each developer interview case study. We're iterating on what to include and where to put the information in the thesis - what goes where... Marian will write up a proposed structure and send it to me, it's fine for me to tailor the structure it still needs to answer her needs to navigate and orientate while also establishing and demonstrating research rigour. 

Characterisation of the company: short and focused. How many apps, their domains, ...

Scale of the work and the company...

What's special about this case study, and where it fits in my research. What were my aims and objectives in undertaking this case study? what the opportunity was for me and how it contributed. 

Purpose in engaging with them, what data was accessible to me, what analysis it it fed into in the rest of the thesis. Joe's notes provide the front end, Marian needs the research back end too. 

Main text for each case study needs a real pithy summary 1 page per case study (the table I'd created partly provides this). 

Topics to include: Characterising their practice and context of use of analytics. My aims and objectives, what data I collected, what I did with it. Summary of findings, point to where they're discussed fully. 

The aims and purpose of this case study... works well for Marian. Look at Greentech. 

Where should the clusters of findings go?
Currently TBD whether to include them in their source case study or in a separate section that aggregates and analyses them.

She needs a relatively detailed mapping of where each case study contributes.

I'll continue writing long-hand for now to provide raw content to then refine and condense.

\dotfill
Marian \nth{23} Sep, 20 - 30 mins

Discussed the structure for the case studies, focusing on LocalHalo.

\dotfill
Joe \nth{24} Sep, breakfast discussion and review of progress

Ideas about extending our current experiment app to feed various mobile analytics services. 

\dotfill
Marian \nth{24} Sep, 12:30 20 mins

The case study for LocalHalo isn't bad. It could do with better formatting of the content to help tired readers. Similarly it'll be useful for me to establish a consistent naming for figures so the list of figures is easier to comprehend and use.

\dotfill
Arosha and Yijun \nth{24} Sep, 16:30 around 45 mins.

I made some notes inline in the case studies to help improve the clarity.

\dotfill
Isabel, Monday 27 Sep 2021, 09:40, 50 mins.
Next week we'll plan a review schedule. Call will be at 3pm next Monday to allow for post-Jazz festival blues.

\dotfill
Marian, 5 min call Monday 27 Sep, 13:05.

A quick sync up. I'll email Marian once I've finished the sections in the Moonpig case study.

\dotfill
Marian, 75 mins, Tues 28 Sep, 13:05..

We went into depth in what's still needed in the LocalHalo case study. I need to provide evidence for my approach and demonstrate where my findings came from in terms of the data that was gathered. Creating tables may help as they provide structure and need to be concise yet informative.

Marian's offered to spend a day debriefing me on my methods and writing them as notes to help elicit the approach I've been using in my analysis during the case studies and of the case studies.

We're aiming to get at least one clean case study - that's methodologically sound. Then it can be reviewed for whether it's clear and compelling.

There are various ad-hoc notes in the localhalo case study I made during the call.

\dotfill
Marian, 15 mins ish, Wed 29 Sep 13:40..

Sync up on my progress since yesterday's call. Mainly writing up the analysis of the automated emails and doing some additional sanity checks using an opensource project. Marian will review the various comments and notes embedded in this case study and update and revise them.

\dotfill
Arosha, 48 mins ish, Wed 29 Sep 14:32..

A wide-ranging discussion on establishing repeatability of the work if other researchers wish to do some similar research. This touched on trust relationships with individuals in organisations, credibility of the researcher with an understanding of their industry perspective, and so on. I'm planning to write up the approach in the methodology chapter once this has been written up for a couple of case studies.

We also discussed the LocalHalo case study and broadly where the resulting materials that underpin the case studies will end up (probably some will be within the thesis project but not part of the actual printed thesis).

Arosha will have a chat with Marian this week to sync up and seek ways to work together effectively. Next week Marian's away for the week and I'm unlikely to be available between Tuesday and Thursday so Arosha and I will catch up on Monday and Friday next week. 

\dotfill
Marian, 25 mins, Thu 30 Sep 13:05

I'm heading in the right direction with writing the methodology, ethics, etc. Marian will send a paper for review that clearly and competently establishes the credibility of their arguments and is able to create a theory (of change) effectively. She'll also write out what she wants out of my methodology. She's about to sync up with Arosha. 

\dotfill
Marian, 10 mins, Fri 01 Oct, 13:05

Quick sync up and update on progress. Keep going. We'll have a telephone sync up next Monday as we're both out and about.

\dotfill
Arosha and Yijun, around 80 mins across several technologies and failed calls, 16:30..

We focused on the methodology for the case studies, and discussed several case studies in this context. We also discussed where the CSV files might belong in future. They might not be appropriate to publish as long lived open data, TBD. Arosha also recommends I check in with one of my supervisors before spending hours importing data into the thesis, a useful reminder. 

Yijun provided a helpful sketch which I've added to the methodology... file temporarily, I need to create a revised version of it, it's an excellent catalyst to help me do so.

\dotfill
Marian, 10 mins phone call, Monday 11:30 ish \nth{4} October 2021

Marian's sent a rough draft of a data table by email. 
\emph{`First, crude blurt of an organising table for the methods overview.  This one is organised around data;  an alternative would be to organise around analyses.  The goal is to convey system applied to opportunity.'}

She needs to know what I did sooner rather than later. I need to be clear what I did! cleaned up with post hoc rationalisation to avoid reporting every nuance of the journey. Report the basis for my finding. Cleaning up the meander is acceptable, hiding data is not (thankfully I don't plan or need to do this).

\dotfill
Isabel, Monday 25 mins ish, 15:05 ish
A catch up on my progress and what I'd like Isabel to review - We agreed she'd skim read the new methodology section. 
Some additional suggestions that may help: EuroSTAR 2022 Conference theme of shaping testing - my work may apply and be worth presenting there.
Going even faster than DevOps means analytics is even more vital; cf. the article she emailed. 

Isabel's offering her availability in 2 hour chunks until the end of November to help with reviewing materials as and when it's fit to review.

\dotfill
Arosha, Mon 20 mins, 15:32 ish

He's provided some comments inline in the ethics section and suggested several useful and relevant papers to help frame my materials. He'll skim the related works chapter and do some background reading of papers that look to be key.

\dotfill
Marian, Friday an afternoon catch up around 15 mins 

As we've both been away, we caught up generally and discussed `tough love' to help improve my writing and defence of the research in future. We've scheduled 3 calls next week so far.

\dotfill
Arosha, Friday, 16:33 for just under an hour

We spent quite a bit of time discussing two new rough sketches I've added this afternoon (Figures \ref{fig:outputs_from_inputs_code_config} and \ref{fig:analytics-feedback-cycle}; note these may be short lived as they will be revised soon.). 

We also went into depth on repeatability aspects of the research, and touched on some of the related works and how this research is clearly novel and worth highlighting. I also provided a brief update of my progress in releasing an app in Google Play on a test track in order to preserve my developer account. 

Next call on Wednesday afternoon.

\dotfill
