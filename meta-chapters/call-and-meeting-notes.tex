%%%%%%%%%%%%%%%%%%
% This is a useful place to keep various notes from calls and meetings as I'm writing up. It's been started late in the game so many notes predate and aren't in this file.
%%%%%%%%%%%%%%%%%%

\hrulefill
\nth{7} Sept 2021
Discussion online with Jakob D. from Moonpig on his transition to MyPulse starting 13 Sept 2021. 
Call 7 Sep 2021 last 15 - 20 mins re my research… Notes are offline as they're incomplete and need work.
\hrulefill
Isabel session \nth{9} Sept 2021

\hrulefill

In person working session with Joe Reeve 10 Sept 2021

We ran through the two case studies he was involved with (Kiwix and Catrobat). He emailed me our joint notes.

\hrulefill

Notes from call with Marian Petre 10 Sept 2021
40 mins 13:35 - 14:15 ish.
 
heartbeat of android vitals

What's clear is my case studies are deep and immersive rather than case studies that ask developers questions in a survey. Also the very nature of what I'm working with means the case studies are opportunistic. I had this much access and no more... This is design I used, because…

It's vital to be very clear about what I've done AND DIDN’T DO, the data I gathered and how, the methods , I NEED TO BE CLEAR ABOUT WHAT I DID, WHEN, AND WHY.

For immersive case studies, there’s a risk of bias - explain how I worked to reduce the risk and/or effects of bias e.g. by interviewing developers of other apps.

By documenting the case studies, their methods and data, it's easier to show how the case studies compare and to be able to justify why I can compare them.

Needs explicit audit trail. What was missing is the coherence across the case studies. Help Marian to know where to look for the methodology.

Same order of sections, with the same name, per case study wherever this applies, however don't use a rigid structure blindly.

Given the dimensions and characteristics of the case studies we're also able also to comment on the aggregate.


Accumulate lessons per case study.
Independent data that leads to the same conclusion
Collegation and/or Triangulation.

Being aware of the potential for bias.
Look for opportunities to discover contradictions as a corroboration. 
Some contradictions lead to insights.

\hrulefill
Notes from call with Arosha 10 Sept 2021
40 mins or so on Skype.

We discussed tactics based on my earlier discussion with Marian and agreed on me working on the case studies and identifying common threads that can help feed the methodology chapter, etc. Arosha will skim through the related works chapter before next Wednesday's meeting and triage the content: must keep, needs serious work or removal, the rest.

I've created a vignettes folder where I'll have a go at adding some small additional examples from external projects, etc. that help amplify the work in the core of the case studies. These are likely to end up in the discussion chapter. The first example will be ObjectBox issue 605. They may need to be sidebars and include some brief context setting together with an explanation of their connection to the rest of my research.

\hfillrule
Yijun Sun 12 Sept 2021 16:00 - 17:20 ish
We started with brief on what I'm working on in the thesis (Red-thread, related works, case studies).

SE in practice, Mark Harman co-chair ICSE 2022 - be useful to look at previously accepted papers to help find the methodology. 
MUST-DO I'll sample 5+ papers from 2021, \url{https://conf.researchr.org/track/icse-2021/icse-2021-Software-Engineering-in-Practice?#event-overview} Yijun found 3 likely candidates, here's the 5 I've picked to get me started with studying their methodologies, etc.:
\begin{itemize}
    \item \href{https://arxiv.org/abs/2010.09974}{Scalable Statistical Root Cause Analysis on App Telemetry}
    \item \href{https://arxiv.org/abs/2010.09977}{Industry-scale IR-based Bug Localization: A Perspective from Facebook}
    \item \href{https://github.com/MobileSE/AndroSea/blob/main/paper/CompatibilityIssues.pdf}{Identifying and Characterizing Silently-Evolved Methods in the Android API}
    \item \href{http://arxiv.org/abs/2102.09336}{FIXME: Enhance Software Reliability with Hybrid Approaches in Cloud}
    \item \href{https://www.win.tue.nl/~aserebre/ICSE2021SEIP.pdf}{An interview study of how developers use execution logs in embedded software engineering}
\end{itemize}

Several more may also be useful once I've made a first pass through these 5. There are also some potentially relevant and interesting papers presented in 2020 \url{https://conf.researchr.org/track/icse-2020/icse-2020-Software-Engineering-in-Practice?#event-overview} e.g. ``Automated Bug Reproduction from User Reviews for Android Applications", ``Automatic Abnormal Log Detection by Analyzing Log History for Providing Debugging Insight", ``Debugging Crashes using Continuous Contrast Set Mining"

- Case studies is one type of empirical studies
- Logging is also empirical (second hand) 
Suggestion: Change the name to empirical studies as the umbrella term, then I can use case-study as the default primary research method for the case studies.
- Experimental case studies were sometimes necessary and can fit into empirical studies. 
Action: I will do.

Question to answer: How did I find out all the prerequisites in the framework in the red thread were essential contributing factors? - I need to provide some evidence for each.
The framework is a very strong contribution and may be a good contribution to the theory. 
Use the thesis order to re-order the red thread and to decide what to migrate. The MVP. Try to set deadlines for the whole thesis and each chapter. The main problem currently is fragmentation of the thesis.

I need a systematic approach to the entire thesis e.g. for grouping and separating content. 

Input, process, output (c.f. minimise couplings between software modules).
Consider removing 2/3rds of the content. It's currently very hard for Yijun to tell what can be cut.

Highlight the takeaway message, how it's been formed and how to evaluate the takeaways.

Balance the evidence with the claims of evidence.

Try creating a slide per important message - sufficient for a short 5 to 10 minute presentation. Anything extraneous shouldn't be in the thesis. <= 7 bullets per slide. So sim for at most 30 or so bullets in total. Aim to show the main idea and the main process in the presentation for the viva.

Present the slides in the order the ideas are developed and presented in the thesis. We can then discuss what's missing and what's extraneous. Need to consider the bandwidth of the examiners. I can also work on this presentation with Isabel. 

Next call with Yijun will be on Friday with Arosha iff I've completed the draft slides for the viva (a surrogate red thread), otherwise will fix a separate call to suit his availability, ideally within the next 7 days to establish and maintain momentum.

\hfillrule

Isabel Monday 09:35 - 10:40 (including some discussions about her research)

100 words per chapter after defragging the red thread chapter.
2 dimensional table of case studies and their empirical methods. Consider qualitative vs. quantitative aspects. 
Consider gutting the red thread.
Draft the presentation for Wednesday and if so Isabel to review it.

1/3 length c.f. my 5* workload exercise, where we then relax the constraint to doing the current workload in 3 of 5 days per week allowing 2 days per week for process and self improvements.

Appium Keynote next Saturday - Isabel's happy to review the slides.

Joe's thoughts on tools as pivot points. Suggestion: Interview Joe as a participant's perspective as an Industry contribution.
c.f. the argument that research needs to be Political and Persuasive. 

OT: I'll introduce Isabel to the Espresso book author (met via Twitter).

Isabel's experiences of writing Flash Fiction where they're set non overlapping word lengths e.g. 75 - 100 words and 151 - 500 words. These non overlapping ranges force authors to edit drafts that are in the dead zone so they fit one or the other acceptable ranges. The contributions are improved through this constraint.

We've agreed I'll have a go at creating a draft for the short presentation in the next few days. I'll start by using some 5x3" cards in 4 colours which I already have. 
\begin{itemize}
    \item Have 20 cards in one colour representing each of the slides. 
    \item Allow up to 3 supporting cards per slide.
    \item Aim for 3 bullet points per slide. 
\end{itemize}
45 mins into the call we changed focus to her research, etc.

Summary of my unordered actions:
\begin{itemize}
    \item Apply changes suggested by Yijun to use Empirical Studies as the umbrella term.
    \item Create the card version of my viva presentation.
    \item Defrag the red thread.
    \item Write 100 words per chapter.
    \item Possibly review the entire thesis seeking content to remove from the thesis.
    \item OT: Follow up with the Espresso book author and introduce him to Isabel.
\end{itemize}

\hfillrule


\hfillrule


\hfillrule


\hfillrule


\hfillrule



