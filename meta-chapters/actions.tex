\section{Actions}
\label{meta-chapter-actions}
This is my global combination of stuff I know I want to do and next steps.
\begin{itemize}
    \item Add all the planned chapters with at least 1 - 2 pages of content to cover the purpose of the chapter. Approx 8 paragraphs according to Arosha would be good.
    \item Incorporate raw material from my papers to date. For instance, figures, revised tables, key findings, \emph{etc.}
    \item Work on the Discussion and also the Introduction.
    \item Devise my test plan for the test apps (zipternet, android-crash-dummy, etc.). \emph{In progress 24-May-2020, see} \url{https://joedocs.com/julianharty/evaluating-gpc-reports} \emph{for my research on evaluating Google Play Console reports.}
    \item Work on my Related Works chapter.
    \item (Done) Fix as many warnings as practical. \emph{24-May-2020 0 warnings apart from the word count (see comment in latex in this file} \texttt{actions.tex}.
    \item Clean up and make my references coherent and complete.
    \item \akb{style point - the voice of the text jumps around from talking about 'the research' to 'my research', addressing the reader directly to the third person, etc.  Need to stick to a consistent style.}
    \begin{itemize}
        \item Done: Abstract, Background Chapter(s)
        \item Next: Introduction, Research Questions.
    \end{itemize}
    \item Review all the text that appears in the ToC LoF, LoT, etc. to make the document structure clear and professional. See~\url{https://tex.stackexchange.com/questions/296759/footnote-in-caption} for an example of creating short labels for captions.
    \item Revisit my rejected papers (and the accepted ones) for snippets and evidence worth incorporating in this thesis. For example,
    \begin{itemize}
        \item \href{https://www.overleaf.com/4666349717jtgjwrkchfkh}{Testing Veracity and Fidelity of Software Analytics}
    \end{itemize}

\end{itemize}

% COULD_DO To fix the word count errors, try debugging along the lines of:  https://tex.stackexchange.com/questions/286019/how-can-i-convince-texcount-that-my-use-of-newcolumntype-is-perfectly-valid-syn

\section{Using MoSCoW for TODOs}
I discovered that TODO's in the material are not sufficiently precise to enable me to work out the differences between what I \textbf{M}ust do, what I \textbf{S}hould do, what I \textbf{C}ould do, and what I \textbf{W}on't do as part of this thesis. The various items are identified in Table \ref{tab:moscow_for_todos}:

\begin{table}[htbp!]
    \centering
    \begin{tabular}{l|l}
       Identifier         &Action by end of thesis \\
       \hline
       \texttt{MUST-DO}   &Must Do  \\
       \texttt{SHOULD-DO} &Should Do \\
       \texttt{COULD-DO}  &Could Do \\
       \texttt{WONT-DO}   &Won't do as part of this thesis, possibly will do subsequently. \\
         &
    \end{tabular}
    \caption{MoSCoW Identifiers during my writing}
    \label{tab:moscow_for_todos}
\end{table}

\subsection{Miscellaneous COULD-DO's}
Here are various general tasks that I could do that might improve the thesis, others are embedded in the latex source as this is a recent addition.
\begin{enumerate}
    \item Somewhat late to the party, I've finally discovered latex foundations that seem clear to me \url{https://artofproblemsolving.com/wiki/index.php/LaTeX:Layout}

    \item Make figures live so people can click them in the PDF and be taken to primary text that references them (this doesn't always hold if there's several references to the same figure. See \url{https://tex.stackexchange.com/questions/54927/how-to-insert-an-image-that-also-acts-as-a-link} and \url{https://tex.stackexchange.com/questions/84921/href-not-working-with-image-but-ok-with-text} if I run into issues.

    \item Size the images more consistently, currently I use a variety of heuristics to size and locate them on the page. That's not elegant and wouldn't cope well if the page margins and.or sizes change.

    \item There are an insane number of examples of beautiful latex, many with sources, in \href{https://tex.stackexchange.com/questions/1319/showcase-of-beautiful-typography-done-in-tex-friends}{Showcase of beautiful typography done in TeX \& friends} for inspiration. One example is a very attractive thesis where the author has provided a README and the sources he used for his thesis \url{https://www.levbishop.org/thesis/source/}. The nomenclature and bibliography are both far more attractive than mine currently.

    \item Implement backrefs so that a) the thesis compiles without errors, b) the backrefs in the bibliography are well-phrased (rather than just having the page numbers). As ever, the memoir document class seems to complicate matters! Currently the code is commented out in \texttt{latex-configuration.tex} Refs:
    \begin{itemize}
        \item \url{https://tex.stackexchange.com/questions/410270/memoir-citation-in-margin} Interesting and somewhat beyond what I currently want to do - it uses sidenotes for citations.
        \item \url{https://tex.stackexchange.com/questions/98528/further-customize-color-of-hyperref-links} using a basic document class.
        \item \url{https://tex.stackexchange.com/questions/15971/bibliography-with-page-numbers} sums up several other examples online for
        \item \url{https://tex.stackexchange.com/questions/522491/using-tufte-book-document-class-with-bibliography-backref}
        \item backref-memoir.tex works in a skeletal memoir example \href{https://www.overleaf.com/project/612fb5e6f6cc44c10b56afa1}{My list of examples overleaf project}
        \item \url{https://latex.org/forum/viewtopic.php?t=4791} if I get stuck, the answer isn't very clear or helpful though...
    \end{itemize}

    \item Create a proper glossary, e.g. see \url{https://husseinbakri.org/essential-latex-bibtex-and-biblatex-tips-for-research-students/} which has lots of sensible general info on using latex etc. including the glossary package.

    \item Consider whether to use alphas for subsubsections as per the fine thesis on pair programming. How to do so is simple - see the latex for this item as the code is commented out % \renewcommand{\thesubsubsection}{\thesubsection.\alph{subsubsection}} % See https://latex.org/forum/viewtopic.php?t=5349

    \item Look at improving the formatting of epigraphs, particularly for the discussion chapter. It also has a footnotemark that doesn't appear as part of the epigraph.

    \item Revise the Introduction and the Case Studies introductory material to use the \uuse and \iuse labels rather than the 1a 2a style labels.

    \item Understand more about `effect size' vs statistical significance, see the discussion in ~\citep[p.135]{Ko2015_a_practical_guide_to_controlled_experiments_of_sw_eng_tools_with_human_participants}.

    \item Review and revise the metrics and related words in the abstract. Also remember to ask Clara.

    \item Consider whether to minimise the content related to the joint research on source code analysis for remote logging in order to simplify the story in the thesis. Of course, refer to it in the thesis so readers are aware of the work and its connection to the thesis.

    \item I'd like to increase the emphasis on the software testing implications. This helps connect the research performed with Isabel, etc. and similarly connects the work on Data Dynamics.

    \item Now that I do finally have access to CrashScope - try running it on old (and current) builds of the Kiwix apps, etc. to see if it can reproduce any of the known crashes.

    \item Revise and update the software contributions chapter. It's also referenced in the opening pages of the thesis so important it presents well.

    \item Of course, actually complete the dedication section!

    \item (Done) Clean up the Author's declaration.

    \item Add page numbers (as per the example used for Vectored Questioning) to the glossary to show where each term is used. This is also a useful aide to help me ensure each terms has been covered adequately.

    \item (Done) Fix the 'unknown' citations e.g. for OBB in the glossary.

    \item Revise the definition of risk.

    \item (Done: took the table from the red thread for the empirical studies.) Add either the table from the Case Studies or the one I created in Microsoft Word 

    \item Remove 'Figure 5.2: Visual Connections in my research'

    \item Migrate 'TODO: Structure used to describe the case studies' into the case study procedure section of the methodology. Review Figure 5.3 Yijun's WIP diagram.

    \item Integrate '5.3 Methodology, ethics, and repeatability of case studies' into the methodology. Ask where the research ethics content is best placed.

    \item Revise/ move the sections on data sources such as Characteristing the data as presented and Google Play Console. It could to with rethinking how to present the approaches used to obtaining, preprocessing, processing, analysing, and storing the various data sources. The introductory section is 4.2 called Data Sources.

    \item (Done) Integrate `Clearly the catgorisation ...' with `Note: Figure ...' in the Methodology section (4.3)

    \item Do have a go at sketching the research version of the sense-making process figure (4.3)

    \item (Done) Complete the Code Analysis subsection.

    \item (Done) Add an introduction to the Case Study Procedure (4.5)

    \item Flesh out the threats to validity section.

    \item Add a summary that also bridges to the next chapter on Case Studies.

    \item TBD :)

\end{enumerate}

\clearpage
\section{Call and Meeting Notes - not part of the actual thesis.}
%%%%%%%%%%%%%%%%%%
% This is a useful place to keep various notes from calls and meetings as I'm writing up. It's been started late in the game so many notes predate and aren't in this file.
%%%%%%%%%%%%%%%%%%

\dotfill
\nth{7} Sept 2021
Discussion online with Jakob D. from Moonpig on his transition to MyPulse starting 13 Sept 2021. 
Call 7 Sep 2021 last 15 - 20 mins re my research… Notes are offline as they're incomplete and need work.
 \dotfill
Isabel session \nth{9} Sept 2021

\dotfill

In person working session with Joe Reeve 10 Sept 2021

We ran through the two case studies he was involved with (Kiwix and Catrobat). He emailed me our joint notes.

\dotfill

Notes from call with Marian Petre 10 Sept 2021
40 mins 13:35 - 14:15 ish.
 
heartbeat of android vitals

What's clear is my case studies are deep and immersive rather than case studies that ask developers questions in a survey. Also the very nature of what I'm working with means the case studies are opportunistic. I had this much access and no more... This is design I used, because…

It's vital to be very clear about what I've done AND DIDN’T DO, the data I gathered and how, the methods , I NEED TO BE CLEAR ABOUT WHAT I DID, WHEN, AND WHY.

For immersive case studies, there’s a risk of bias - explain how I worked to reduce the risk and/or effects of bias e.g. by interviewing developers of other apps.

By documenting the case studies, their methods and data, it's easier to show how the case studies compare and to be able to justify why I can compare them.

Needs explicit audit trail. What was missing is the coherence across the case studies. Help Marian to know where to look for the methodology.

Same order of sections, with the same name, per case study wherever this applies, however don't use a rigid structure blindly.

Given the dimensions and characteristics of the case studies we're also able also to comment on the aggregate.


Accumulate lessons per case study.
Independent data that leads to the same conclusion
Collegation and/or Triangulation.

Being aware of the potential for bias.
Look for opportunities to discover contradictions as a corroboration. 
Some contradictions lead to insights.

 \dotfill
Notes from call with Arosha 10 Sept 2021
40 mins or so on Skype.

We discussed tactics based on my earlier discussion with Marian and agreed on me working on the case studies and identifying common threads that can help feed the methodology chapter, etc. Arosha will skim through the related works chapter before next Wednesday's meeting and triage the content: must keep, needs serious work or removal, the rest.

I've created a vignettes folder where I'll have a go at adding some small additional examples from external projects, etc. that help amplify the work in the core of the case studies. These are likely to end up in the discussion chapter. The first example will be ObjectBox issue 605. They may need to be sidebars and include some brief context setting together with an explanation of their connection to the rest of my research.

 \dotfill
Yijun Sun 12 Sept 2021 16:00 - 17:20 ish
We started with brief on what I'm working on in the thesis (Red-thread, related works, case studies).

SE in practice, Mark Harman co-chair ICSE 2022 - be useful to look at previously accepted papers to help find the methodology. 
MUST-DO I'll sample 5+ papers from 2021, \url{https://conf.researchr.org/track/icse-2021/icse-2021-Software-Engineering-in-Practice?#event-overview} Yijun found 3 likely candidates, here's the 5 I've picked to get me started with studying their methodologies, etc.:
\begin{itemize}
    \item \href{https://arxiv.org/abs/2010.09974}{Scalable Statistical Root Cause Analysis on App Telemetry}
    \item \href{https://arxiv.org/abs/2010.09977}{Industry-scale IR-based Bug Localization: A Perspective from Facebook}
    \item \href{https://github.com/MobileSE/AndroSea/blob/main/paper/CompatibilityIssues.pdf}{Identifying and Characterizing Silently-Evolved Methods in the Android API}
    \item \href{http://arxiv.org/abs/2102.09336}{FIXME: Enhance Software Reliability with Hybrid Approaches in Cloud}
    \item \href{https://www.win.tue.nl/~aserebre/ICSE2021SEIP.pdf}{An interview study of how developers use execution logs in embedded software engineering}
\end{itemize}

Several more may also be useful once I've made a first pass through these 5. There are also some potentially relevant and interesting papers presented in 2020 \url{https://conf.researchr.org/track/icse-2020/icse-2020-Software-Engineering-in-Practice?#event-overview} e.g. ``Automated Bug Reproduction from User Reviews for Android Applications", ``Automatic Abnormal Log Detection by Analyzing Log History for Providing Debugging Insight", ``Debugging Crashes using Continuous Contrast Set Mining"

- Case studies is one type of empirical studies
- Logging is also empirical (second hand) 
Suggestion: Change the name to empirical studies as the umbrella term, then I can use case-study as the default primary research method for the case studies.
- Experimental case studies were sometimes necessary and can fit into empirical studies. 
Action: I will do.

Question to answer: How did I find out all the prerequisites in the framework in the red thread were essential contributing factors? - I need to provide some evidence for each.
The framework is a very strong contribution and may be a good contribution to the theory. 
Use the thesis order to re-order the red thread and to decide what to migrate. The MVP. Try to set deadlines for the whole thesis and each chapter. The main problem currently is fragmentation of the thesis.

I need a systematic approach to the entire thesis e.g. for grouping and separating content. 

Input, process, output (c.f. minimise couplings between software modules).
Consider removing 2/3rds of the content. It's currently very hard for Yijun to tell what can be cut.

Highlight the takeaway message, how it's been formed and how to evaluate the takeaways.

Balance the evidence with the claims of evidence.

Try creating a slide per important message - sufficient for a short 5 to 10 minute presentation. Anything extraneous shouldn't be in the thesis. <= 7 bullets per slide. So sim for at most 30 or so bullets in total. Aim to show the main idea and the main process in the presentation for the viva.

Present the slides in the order the ideas are developed and presented in the thesis. We can then discuss what's missing and what's extraneous. Need to consider the bandwidth of the examiners. I can also work on this presentation with Isabel. 

Next call with Yijun will be on Friday with Arosha iff I've completed the draft slides for the viva (a surrogate red thread), otherwise will fix a separate call to suit his availability, ideally within the next 7 days to establish and maintain momentum.

\dotfill

Isabel Monday 09:35 - 10:40 (including some discussions about her research)

100 words per chapter after defragging the red thread chapter.
2 dimensional table of case studies and their empirical methods. Consider qualitative vs. quantitative aspects. 
Consider gutting the red thread.
Draft the presentation for Wednesday and if so Isabel to review it.

1/3 length c.f. my 5* workload exercise, where we then relax the constraint to doing the current workload in 3 of 5 days per week allowing 2 days per week for process and self improvements.

Appium Keynote next Saturday - Isabel's happy to review the slides.

Joe's thoughts on tools as pivot points. Suggestion: Interview Joe as a participant's perspective as an Industry contribution.
c.f. the argument that research needs to be Political and Persuasive. 

OT: I'll introduce Isabel to the Espresso book author (met via Twitter).

Isabel's experiences of writing Flash Fiction where they're set non overlapping word lengths e.g. 75 - 100 words and 151 - 500 words. These non overlapping ranges force authors to edit drafts that are in the dead zone so they fit one or the other acceptable ranges. The contributions are improved through this constraint.

We've agreed I'll have a go at creating a draft for the short presentation in the next few days. I'll start by using some 5x3" cards in 4 colours which I already have. 
\begin{itemize}
    \item Have 20 cards in one colour representing each of the slides. 
    \item Allow up to 3 supporting cards per slide.
    \item Aim for 3 bullet points per slide. 
\end{itemize}
45 mins into the call we changed focus to her research, etc.

Summary of my unordered actions:
\begin{itemize}
    \item Apply changes suggested by Yijun to use Empirical Studies as the umbrella term.
    \item Create the card version of my viva presentation.
    \item Defrag the red thread.
    \item Write 100 words per chapter.
    \item Possibly review the entire thesis seeking content to remove from the thesis.
    \item OT: Follow up with the Espresso book author and introduce him to Isabel.
\end{itemize}

\dotfill
Marian Tues 14 Sept 2021 13:00 ish for an hour or so
Lots of discussions, e.g. on Kahneman's thinking fast and slow. summarised as the tables and the figures are the heart of the thesis, and once these are clear the rest of the writing should become unblocked. This led to the following immediate actions for me:
\begin{enumerate}
    \item Clarify the tables, their headings and labels, and remove or decode my 'shorthand'. Aim to make the contents of the tables orthogonal and for the items in the cells to be comparable (rather than having disparate content in some cells e.g. `Sophisticated' which is an editorial statement.
    \item Explicitly document the contributions of each study to the research questions.
    \item (later on) map contributions by each study to the items in the framework.
\end{enumerate}

She reiterated a clear, explicit audit trail of the data and the contributions from each study is critical. We also agreed that although the work on game and mechanics is really interesting it's not currently underpinned by my research, at least not in the thesis (it may be in my head).

\dotfill
Arosha Wed 15 Sept 2021 11:12 - 11:44 ish 
He wasn't able to review the literature review in the last few days owing to other demands on his time.
I brought him up to speed with recent conversations with Yijun, Isabel, and Marian. 

My recent changes improve migrating content from the red thread document to the thesis, revising the tables, and adding some material on empirical research.

Getting to clarity is vital, we agree. How that's achieved - through tables, presentation slides, writing 100 words per chapter/case study, are all possible ways to do so. 
He agrees the material on the game and mechanics is currently at risk of being a hostage to fortune. 
We reviewed the case study red thread tables together, 
Arosha to review the tables in the red thread in greater depth and add questions and notes by Thursday end of day.
I'm meanwhile continuing work on the case studies.

\dotfill
Marian Wed 15 Sep 2021 13:06 - 14:13
We spend most of the time digging into the 5 tables for the case studies. Good progress has been made, more is needed. There are ad-hoc inline notes in the latex I will need to address. 

Action - I'll create a monster table in a spreadsheet and review it tomorrow with Marian.

In the a monster spreadsheet: Ask have I really captured what I did, the data, the importance of the findings, what are the relative priorities. Context, establish what I did, what I collected, the findings, the insights, and the role in the thesis. 

\dotfill
Marian Thursday 13:00 for about 20 mins ish.

Action: Get on with it! :)

\dotfill
Marian Friday \nth{17} Sep 15:00 for about 15 mins.

We'll spend time on Monday \nth{20} Sept diagramming the case studies. 
Action: I'm working on revising the Catrobat case study. The monster spreadsheet is on hold (as is the draft slide deck for the viva) as we're working on the tables which serve a similar purpose of helping me create and clarify the story.

\dotfill
Arosha and Yijun \nth{17} Sep 16:30 48 mins.
We reviewed this week's progress. 
We also discussed and revised the proposed headings for the embedded case studies e.g. Kiwix and Catrobat. The revised set are in the Catrobat case study where I'll try them out. They're also written up in \href{structure-of-the-app-case-studies}{\nameref{structure-of-the-app-case-studies}} now.

\dotfill
Joe Reeve, in person discussion \nth{20} Sep, around 90 minutes.
We discussed my diagrams I'd made over the weekend (in preparation for my Keynote at the Appium conference). Joe had a go at redrawing one of my rough sketches. 

\dotfill
Isabel Evans, Mon \nth{20} Sep, 10am ish, online about 30 mins.

Mainly a sync up.

\dotfill
Marian, in person discussion \nth{20} Sep around 100 minutes.

Marian asked me questions and drew diagrams with some of my words in them. I also showed her various ideas and notes I'd made when preparing for my keynote for the Appium conference.

She suggested a good target to aim for is to complete the first draft of the complete thesis in 9 weeks - 9 chapters/9 weeks.

\dotfill
Marian, \nth{21} Sep 13:05 for about 32 mins.

We chatted about ways to structure the case studies, there's a vast amount and probably too much to keep in a single chapter. One suggestion is to write up the individual case studies and put these in the appendix which would compress the narrative of the core thesis.

Another (complimentary) option is to write an introduction chapter to explain and set the various perspectives in context and then a chapter per perspective. This then led to a fresh insight on the perspectives - with a revised approach with 4 perspectives in a 2x2 matrix.

I was distracted by David Pride's good news. back to work now...

\dotfill
Arosha, \nth{21} Sep, 50 mins

From 4 to 6 perspectives.

\dotfill
Marian, \nth{21} Sep, 10 mins ish.
A quick sense check of using the 6 perspectives.

What I need to provide for each study:
Context and orientation; then drive from the [6] perspectives.
A suggested order: Status Quo slice, then what happens if we interfere?

\dotfill
Marian \nth{22} Sep, 45 mins

We discussed what's needed in terms of structuring and populating each developer interview case study. We're iterating on what to include and where to put the information in the thesis - what goes where... Marian will write up a proposed structure and send it to me, it's fine for me to tailor the structure it still needs to answer her needs to navigate and orientate while also establishing and demonstrating research rigour. 

Characterisation of the company: short and focused. How many apps, their domains, ...

Scale of the work and the company...

What's special about this case study, and where it fits in my research. What were my aims and objectives in undertaking this case study? what the opportunity was for me and how it contributed. 

Purpose in engaging with them, what data was accessible to me, what analysis it it fed into in the rest of the thesis. Joe's notes provide the front end, Marian needs the research back end too. 

Main text for each case study needs a real pithy summary 1 page per case study (the table I'd created partly provides this). 

Topics to include: Characterising their practice and context of use of analytics. My aims and objectives, what data I collected, what I did with it. Summary of findings, point to where they're discussed fully. 

The aims and purpose of this case study... works well for Marian. Look at Greentech. 

Where should the clusters of findings go?
Currently TBD whether to include them in their source case study or in a separate section that aggregates and analyses them.

She needs a relatively detailed mapping of where each case study contributes.

I'll continue writing long-hand for now to provide raw content to then refine and condense.

\dotfill
Marian \nth{23} Sep, 20 - 30 mins

Discussed the structure for the case studies, focusing on LocalHalo.

\dotfill
Joe \nth{24} Sep, breakfast discussion and review of progress

Ideas about extending our current experiment app to feed various mobile analytics services. 

\dotfill
Marian \nth{24} Sep, 12:30 20 mins

The case study for LocalHalo isn't bad. It could do with better formatting of the content to help tired readers. Similarly it'll be useful for me to establish a consistent naming for figures so the list of figures is easier to comprehend and use.

\dotfill
Arosha and Yijun \nth{24} Sep, 16:30 around 45 mins.

I made some notes inline in the case studies to help improve the clarity.

\dotfill
Isabel, Monday 27 Sep 2021, 09:40, 50 mins.
Next week we'll plan a review schedule. Call will be at 3pm next Monday to allow for post-Jazz festival blues.

\dotfill
Marian, 5 min call Monday 27 Sep, 13:05.

A quick sync up. I'll email Marian once I've finished the sections in the Moonpig case study.

\dotfill
Marian, 75 mins, Tues 28 Sep, 13:05..

We went into depth in what's still needed in the LocalHalo case study. I need to provide evidence for my approach and demonstrate where my findings came from in terms of the data that was gathered. Creating tables may help as they provide structure and need to be concise yet informative.

Marian's offered to spend a day debriefing me on my methods and writing them as notes to help elicit the approach I've been using in my analysis during the case studies and of the case studies.

We're aiming to get at least one clean case study - that's methodologically sound. Then it can be reviewed for whether it's clear and compelling.

There are various ad-hoc notes in the localhalo case study I made during the call.

\dotfill
Marian, 15 mins ish, Wed 29 Sep 13:40..

Sync up on my progress since yesterday's call. Mainly writing up the analysis of the automated emails and doing some additional sanity checks using an opensource project. Marian will review the various comments and notes embedded in this case study and update and revise them.

\dotfill
Arosha, 48 mins ish, Wed 29 Sep 14:32..

A wide-ranging discussion on establishing repeatability of the work if other researchers wish to do some similar research. This touched on trust relationships with individuals in organisations, credibility of the researcher with an understanding of their industry perspective, and so on. I'm planning to write up the approach in the methodology chapter once this has been written up for a couple of case studies.

We also discussed the LocalHalo case study and broadly where the resulting materials that underpin the case studies will end up (probably some will be within the thesis project but not part of the actual printed thesis).

Arosha will have a chat with Marian this week to sync up and seek ways to work together effectively. Next week Marian's away for the week and I'm unlikely to be available between Tuesday and Thursday so Arosha and I will catch up on Monday and Friday next week. 

\dotfill
Marian, 25 mins, Thu 30 Sep 13:05

I'm heading in the right direction with writing the methodology, ethics, etc. Marian will send a paper for review that clearly and competently establishes the credibility of their arguments and is able to create a theory (of change) effectively. She'll also write out what she wants out of my methodology. She's about to sync up with Arosha. 

\dotfill
Marian, 10 mins, Fri 01 Oct, 13:05

Quick sync up and update on progress. Keep going. We'll have a telephone sync up next Monday as we're both out and about.

\dotfill
Arosha and Yijun, around 80 mins across several technologies and failed calls, 16:30..

We focused on the methodology for the case studies, and discussed several case studies in this context. We also discussed where the CSV files might belong in future. They might not be appropriate to publish as long lived open data, TBD. Arosha also recommends I check in with one of my supervisors before spending hours importing data into the thesis, a useful reminder. 

Yijun provided a helpful sketch which I've added to the methodology... file temporarily, I need to create a revised version of it, it's an excellent catalyst to help me do so.

\dotfill
Marian, 10 mins phone call, Monday 11:30 ish \nth{4} October 2021

Marian's sent a rough draft of a data table by email. 
\emph{`First, crude blurt of an organising table for the methods overview.  This one is organised around data;  an alternative would be to organise around analyses.  The goal is to convey system applied to opportunity.'}

She needs to know what I did sooner rather than later. I need to be clear what I did! cleaned up with post hoc rationalisation to avoid reporting every nuance of the journey. Report the basis for my finding. Cleaning up the meander is acceptable, hiding data is not (thankfully I don't plan or need to do this).

\dotfill
Isabel, Monday 25 mins ish, 15:05 ish
A catch up on my progress and what I'd like Isabel to review - We agreed she'd skim read the new methodology section. 
Some additional suggestions that may help: EuroSTAR 2022 Conference theme of shaping testing - my work may apply and be worth presenting there.
Going even faster than DevOps means analytics is even more vital; cf. the article she emailed. 

Isabel's offering her availability in 2 hour chunks until the end of November to help with reviewing materials as and when it's fit to review.

\dotfill
Arosha, Mon 20 mins, 15:32 ish

He's provided some comments inline in the ethics section and suggested several useful and relevant papers to help frame my materials. He'll skim the related works chapter and do some background reading of papers that look to be key.

\dotfill
Marian, Friday an afternoon catch up around 15 mins 

As we've both been away, we caught up generally and discussed `tough love' to help improve my writing and defence of the research in future. We've scheduled 3 calls next week so far.

\dotfill
Arosha, Friday, 16:33 for just under an hour

We spent quite a bit of time discussing two new rough sketches I've added this afternoon (Figures \ref{fig:outputs_from_inputs_code_config} and \ref{fig:analytics-feedback-cycle}; note these may be short lived as they will be revised soon.). 

We also went into depth on repeatability aspects of the research, and touched on some of the related works and how this research is clearly novel and worth highlighting. I also provided a brief update of my progress in releasing an app in Google Play on a test track in order to preserve my developer account. 

Next call on Wednesday afternoon.

\dotfill
Isabel, Monday 11th Oct, 45 mins ish, 09:30 - 10:15 ish
Isabel on Repeatability 
My research is unlike repeatability of chemical experiments, etc. it's not a case of taking the same ingredients and performing the same steps to obtain the equivalent result. There are qualitative aspects. These  are complex systems that change continually c.f. rapidly changing viruses, 

“you cannot cross the same river twice…”


Understanding undercurrents.

Comparing oneself against the undercurrents c.f. vanity (insanity) metrics. 

For app developers, determining your sanity metrics vs vanity metrics.

Pam, ex IBM, village website. Jargon hiding information even from experienced IT people. c.f. research as a political act.

Influences on quality c.f. my figure 2 on a mobile analytics cycle.

\dotfill
Marian, Monday 11th Oct, 45 mins, 11:13 - 12:15 ish

We discussed the 2 recent diagrams:
- If there there are repeated patterns, people can usefully apply in terms of interpreting the analytics that's fine. (given the variability of the systems, ecosystems, etc. I need to establish what is able to be applied usefully for many apps).

- Forces at play, be consistent in the diagrams, explain the types of app specific data inputs. Using sub-figures may help show the connection between the two figures.

Superimpose the use of sources, in the case studies on these 2 figures to help the reader connect the diagrams with the research that was done.

I can ask Joe and Damien to review these diagrams. I can ask Silvia to help draw them.


Emergent examples of the application will be in the case studies. (p.34). Don't over anticipate.

Remove duplication from the rest of the thesis.  New material may obviate and/or replace previous content.

Consider integrating the methodology... section into the methodology chapter. 

Marian needs me to provide: the framing of the case studies, and the framing of the the research. These could be in a common chapter or separate - the location is less important.

Actions: I'll email Marian once I've written the notes for both figures.

\dotfill
Joe Reeve, Tue \nth{12} Oct 2021 09:30 around 15 mins.

We mainly discussed Iteratively and Amplitude analytics and that our current p-o-c implementation for crash reporting doesn't seem to always report the crashes. That's fine in terms of the experiments we're currently doing, we'll fix the crash reporting after some more immediate tests I want to perform as part of this research. He'll take a look at the 2 recent figures today and provide his perspective on them and how they can be improved.

\dotfill
Marian, Tue \nth{12} Oct, 11:30 65 mins.

Key to deal with potential bias of analysis in the moment and being open to contradictory evidence. c.f. high integrity SIL levels to mitigate against bias. Make clear to the reader, at what point the conclusions were drawn. Report corroboration and contradiction from later in a case study and from other case studies. Consider horizontal and vertical slicing of the case studies. Review my entire thesis - for the ones that is was necessary that I was in the moment, tell those case studies with a timeline. For the rest provide summative results. 

Vital, clear methodology, background of each case study, then order by importance in the thesis. Prioritise by the conclusions. And talk about what has been found to matter. One cannot include everything and that is my defence. It's about time to work backwards from my conclusions and findings. Separate the mess from the rigourous research. For the direct engagement, we need to understand my role and how I observed change. 

Data traceability from source case study to the findings and conclusions. 

c.f. research in HCI with user stories to establish the breadth of the research, then put these details aside to present the findings. Vital to present similarities and differences in each case study succinctly. 

As the analytics and ecosystem continually change and evolve some of the research and analysis needs to be in the moment;- 'seize the data' while it's available. There is a risk of bias in working in the moment so the research also needs \emph{post-hoc} analysis and verification. 

The research has yet to clear, the story and clarity have not yet emerged. We've agreed to get together to draw and to drill down into my research in the hope of finding the bedrock, the story, and how to tell it so the story is clear, concise and compelling.

A possible outline for my thesis
\begin{enumerate}
    \item Analytics in context, at the depth suitable for this research. Sieze the data, and post-hoc analysis. 
    \item Therefore the methodology chosen.
    \item The case studies (presented consistently in terms of structure and in summary form - the longer stories can be in appendices).
    \item Drawing together findings from the case studies.
    \item The discussions. 
\end{enumerate}

It's also timely to arrange a 3+ in-person working session around whiteboard and walls where we can place diagrams and tables, etc. to help the story emerge. 
Marian recommends I have a go at writing a new outline for the thesis. 

\dotfill
Marian, Wed \nth{13} Oct, 11:30 50 mins discussion

BT Openreach ate about 3 hours of my waking life since we spoke yesterday. Meanwhile I've been working on the IDot Android app for several hours and added some notes to the thesis about this app in parallel.

Re the PhD, we discussed the importance of a clear methodology that includes the research perspective. The example from Akshika's thesis took multiple iterations to develop it to the final, clear state it now has. Later in his thesis he also mapped the findings back to the methodology. As my research has many more cases than his did the resulting figures may be more complex. 

Actions: Marian will read what's in the methodology chapter currently and suggest improvements, etc. Meanwhile I'll blurt out some written notes on how potential biases were addressed in the research. We agreed it's worth leaving most of the current material in the thesis until we have a much clearer handle on the methodology, then it'll be time to clean up the many and various notes and comments. A chapter by chapter process may work well for the clean up work, TBD.

\dotfill
Arosha, \nth{13} Oct 2021, 14:35, 25 mins.

I asked to meet up on campus with Marian for an in-person sync up and whiteboarding session, Arosha will ask the relevant person at the OU what'd be involved to do so. Tuesday morning would work well, failing that Monday morning.

Arosha's added some comments to the relocated material in the methodology chapter meanwhile. I also updated Arosha on the work on the IDot Android app and my ongoing BT saga.

\dotfill
Marian \nth{14} Oct 13:00

We discussed what's missing from the methodology chapter and the need for me to populate various tables, starting with cases and the types of data they provide. Marian sent me a revised template for the next few tables I need to populate. (I completed the first one that evening).

\dotfill
Marian \nth{15} Oct 11:32 45 mins

Knowing the difference between weak and strong evidence e.g. in terms of statistical tests.
I need to show in the thesis clear evidence of everything I relied on in an example.
PhD needs to be systematic. 

A couple of good examples, this is what I saw in the data because of … so I checked … and that confirmed/disagreed what I hypothesised. 
Validation steps, I tested the conjecture and looked for contradictory evidence in the following ways, then I did the following… 
Show how I carried learnings and insights from one case study to another. Explain about how cross case studies also validate.
Provide concise examples of key processes. backed up by tables. Justify what I excluded and what I included, and why.
I need a complete stock take of what I have then I can prioritise. 

Flag contributions to the thesis.

Articulate the observations and how I checked and cross checked these.
Marian can help elicit this from me and draft it. 

It's premature to revise the estimate on when the complete first draft of my thesis will be ready, first we need clarity on the methods and the introduction to the case studies.

\dotfill
Arosha \nth{15} Oct 2021 16:30, 100 mins

For the first table (the one I've completed a first complete draft of): it's worth me abstracting away from specific details to make the tables clearer and more readable. One idea is to use Icons for the tables to abstract away from details e.g. an icon for source code, ditto for the interviews.

It might also be viable to create icons for/from project interventions: point, widespread, none.

Typology of the types of review in the collaborative sessions. e.g. who led the walkthrough and why.
Explain the direction the knowledge was transferred. - e.g. is the role of an expert useful to help improve the use of mobile analytics. 

I'll aim to update the first table by Tuesday morning so we can discuss it with Marian in person.

Vectored analysis which includes excluding fluff. Slicing and dicing. 
Quality Trends Analysis and contributory causes. \textbf{Put headline findings in the cells - aim to sum up in one word or a short phrase}. Explain the analysis succinctly. The analysis reveals various types of contributory causes e.g. chipset, release version, other factors. 

We chatted about James Noble - Post-modern programming - SO (Stack Overflow) driven development. 

Correlations between what developers say vs what the analytics ‘says’ - these may lead to recommendations to devs.
correlations between analytics sources. - analysis of the data to gain other insights. 

What, from a research p-o-v is used to draw out conclusions. 

Try identifying the recommendations to the devs (operational level)
(research analysis) to draw out insights of how analytics improves development practices. How does the operational stuff actually effect changes. Which levers to press? 
The lack of a quality owner is my claim, what evidence can I offer: emails, artefacts, android vitals. Impacts that speak to quality aspects. Changes to team, changes to practice, (what are the clues that have an impact on quality). Quote literature that states there’s a correlation. This is a confirmation of what other research has uncovered, and my work builds on their research. I can show objective evidence from the mobile analytics. Explain what I was looking for e.g. types of bugs, enumerate how I looked at the bug repositories. After gathering lots of examples, seek abstractions (as we did for table 1), map out what I looked for. 

try to describe the vectored analysis - the enumeration of what was found helps support the abstraction set. 

I sometimes looked for temporal relationships in the analytics reports e.g. where and when a failure occurs - when did it start, what affects when it occurs, etc. 

Aim to distinguish what’s helpful to the development team vs what's helpful to support the research conclusion. My primary goal is NOT to help improve the research practices, it’s to support my research claims and conclusions. I can write methodological reflections in the discussions. I need to first tell people what I did and how I was able to use what I did to substantiate my research contributions.  

\dotfill
Marian 18 Oct 2021 11:30 75 mins

We had various discussions during the call, mainly to help make sense of what I did in terms of my research methods:

Some interim steps in the discussion follow (not all of it was actually discussed, some are my notes to self) as they may be helpful:
\begin{itemize}
    \item Looking for Beacons
    \item Combining tools in order to achieve improvements
    \item An orderly analysis of the available analytics
    \item Describe the process I used to make sense of what was in the error reports. Needs understanding of the app context. 
    \item Pattern finding is what I do
    \item I’m going through a diagnostic process to identify things that matter and the consequences of things that matter.
    \item \julian{PFOD (one approach is to reduce demand) -note to self}
    \item Pattern Findings: then provide individual pattern methods: What I’m looking for in the data and how I go about it. 
    \item Move the Comparisons to a separate row in the table
What is the agent for the analysis e.g. is it human validates, statistically valid
    \item Pattern finding is Inductive Analysis i.e. Pattern Finding (data driven in order to find patterns).
\end{itemize}


Thematic Analysis linking the data with the same columns, the rows are the questions I ask in the steps of my analysis.
c.f. Inductive Analysis, seeking correlations between the patterns, Sanity Checking, Need to articulate how I did the inductive analysis. 

I need to articulate the steps I take - there are few methodological approaches I use.
Distinct dig-down steps: to understand the cause, to understand the problem, 

Identifying responses (what happened next? type questions)
Seeking further evidence either confirming or contradicting (for instance). We sometimes add data to a data source. 

We returned to the 6 perspectives which are vital to establish the map of the methodology chapter for Marian.

Contexualisation and Corroboration activities need a way to describe what I did as a research methods.

\julian{c.f. seeking oil reserves and the methods that would research the practices and techniques. -note to self}

Sometimes there’s feeding back into the practice. 

One view of my method:
\begin{enumerate}
    \item What emerges as a phenomenon of interest. 
    \item Sense-making of what’s being presented
    \item Comparisons between the reports
    \item Checking with other sources 
\end{enumerate}

Evaluation through action research (do they react on what is found in the analytics?) There’s an inherent bias where the researcher wants the developers to make a decision, etc. I have to seek similar outcomes that didn’t come from my interventions. \julian{Addendum, reflecting on these notes:} the evaluation is also through ``inaction research" - what happens if the development team \emph{do not act on what is found in the analytics?}

Articulate my Inductive Analysis. 

Provide thumbnails of what I did in a distilled form. I check to see whether I’ve really found something e.g. by reviewing similar errors. Routine example abstracted from specific filters I use. 
\begin{itemize}
    \item Capture my approach as my analysis method.
    \item c.f. definition of slicing and dicing
    \item de- and re-grouping 
\end{itemize}

I can partly quantify the effort needed to reap the benefits of using analytics. The outcomes are useful and valuable. 

The analysis depends and depended on the data that was available (some times the desired data is not available for whatever reasons). 

For some of my less well evidenced work \emph{iff} the rest of my thesis demonstrates rigour I may be permitted to add additional material e.g. as an essay in the further work section.  

\dotfill

In person meeting with Arosha and Marian, 2 hour session 10:00 ish Tue \nth{19} Oct 2021 in MK.

We discussed ways of describing my methodology. I did part of a demo of using Google Play Console with Android Vitals. Marian wrote lots of notes and then Arosha sketched a figure representing the first three parts of my actual methodology which Marian added to. My immediate task is to write up the methodology - which I've started doing.

\dotfill
Marian 60 mins, 11:30 ish Wed \nth{20} Oct.

We dug into the methodology and some immediate actions (noted in the methodology materials)

\dotfill
Marian, 60 mins, 12:00 ish, Thu \nth{21} Oct. after PG Forum

Followed up on epistemology from the PG Forum session Marian led. Also discussed recent additions to the Word doc and the table I'd added to the methodology. Agreed to meet at the OU on Friday morning for a working session.

\dotfill
Marian, 3+ hours at the OU Fri \nth{22} Oct.

Where we reviewed the contents of the table in more depth and the question and evidence notes I'd added to the Word doc. We then experimented with no end of groupings of the methods I'd listed; and we had various clarifying moments e.g. on what the term artefacts includes and what it excludes.  We ended up with a potentially useful and viable grouping with 3 circles grouped around feedback. We also worked through Catrobat and Kiwix and where the hackathons fitted as a method in the action research. The six perspectives were revised slightly to generalise them and make them more consistent. I've yet to apply the revisions and will do so asap.

\dotfill
Arosha, an hour, 16:35 ish, Fri \nth{22} Oct.

Reviewed the intervening work with Marian. Arosha clarified the methodology chapter needs to include all the methods including data collection and analysis. We briefly discussed administrative aspects of the PhD - progress reports and fees. I can't sort either out until the OU sends me details. 

Our next scheduled call is Wed \nth{3} Nov, followed by Fri \nth{5} Nov.

\dotfill
Isabel, 50 mins, 09:30, Mon \nth{25} Oct.

Catch-up on our respective progress. We discussed epistemology in our respective microcosms. She'll be available to read my work when it's fit to be read, which isn't now while we're neck-deep in sense-making and trying to identify my methodology. Next call in 2 weeks as I'm away for 6 days later this week and early next.

\dotfill
Marian, a quick 10 mins call, Mon \nth{25} Oct, 11:30

Marian suggested a change of focus for the next few days: for me to summarise the case studies based on the work of the last few weeks. Meanwhile she'll provide some notes on the methodology chapter. Next call tomorrow.
Introduction of cases:
\begin{enumerate}
    \item Organisation overview (what the org is/does)
    \item Overview of tools/apps/analytics they use (the detailed material goes into the subsequent analysis chapters)
    \item Researcher engagement
    \item Data collected
    \item (eventually) Links forward to where that data is used
\end{enumerate}

\dotfill
Marian, 50 mins, Tues \nth{26} Oct, 11:30

\dotfill
2 sessions with Marian on Wed \nth{3} and Thu \nth{4} Nov, and one with Arosha on \nth{3} Nov. These were mainly on other topics rather than the contents of the thesis, e.g. completing my progress report with Arosha.

\dotfill

Arosha, Fri \nth{5} Nov 2021, 70 mins, 10:10 ish. In person at OU MK.

We discussed the methodology chapter in depth. I need to explain each of the methods and provide brief, pertinent examples where they help the reader e.g. in types of beacons in mobile analytics such as a spike in a graph. Also mention missing beacons - those that would be expected to be found but do not appear. Wherever practical, show the roots/history of methods including where they've been established and made credible by other researchers. Then explain briefly any material changes e.g. how the use of beacons in existing research has been in source code, that mobile analytics tools differ from source code, and provide the examples mentioned above.

I need to include: 
\begin{itemize}
    \item Data sources - including how acquired and preserved. Mention how to subscribe to automated emails, etc.
    \item The analysis of the data sources.
    \item Map back to the six perspectives and to the research questions.
    \item Show explicitly the relevance of the data, the methods, the analysis, etc.
    \item Reflect on the limitations of all the above. Mention it's not possible to mitigate against every aspect, however some mitigations are necessary and useful and explain those I did use, and those I would wish to have used but wasn't able to do.
    \item Also, cover the biases and mitigations that were applied in the research.
\end{itemize}

We could use the six perspectives in the literature review to start grouping existing research and siting it within the perspectives. Then argue that each of the six is important \textit{and} the relationships between the understanding of the current practices (the 1's) and the potential improvements (the 2's) are key connections worth researching.

Aim to show that even when the good practices identified in existing research e.g. in Software Analytics are applied that there are special characteristics of mobile apps and mobile analytics that have not yet been studied \textit{and} they are worthy of being researched.

Measurement fits in the literature review chapter as part of establishing the grounds for software and mobile analytics. Discuss the history of what is known mainly from/in research i.e. things that contributed to academics body of knowledge in these topics e.g. in `understanding' mobile apps which is were the bulk of my literature review should focus. The aim is to establish my research questions as pertinent, appropriate, and useful to research.

\dotfill

Marian, Fri \nth{5} Nov 2021, around 20 mins, 15:05 ish.

Barbera Teasley, Schultz, Beacons in code. 
Literature Review frames the research. 6 perspectives emerge from the Literature Review.


Consider writing the intro to the case studies concurrently with the methodology.

TBD the structuring of the in depth the case studies.

\dotfill

Isabel, Mon \nth{8} Nov, 09:30 about 45 mins.
Mainly syncing up on our respective progress. Isabel will forward references for the following: 
WWII fairisle knitters were recruited by forerunners to GCHQ in WWII to help interpret purported knitting patterns being sent by post. In my view they used their insights in knitting to detect beacons in the `messages'. 

\begin{itemize}
    \item \url{https://www.popsci.com/story/diy/secret-code-messages-knitting/}
    \item \url{https://history.howstuffworks.com/world-war-ii/spies-codes-knitting.htm}
\end{itemize}

\dotfill

Marian, Mon \nth{8} Nov, 11:30 about 45 mins.

Spent quite a bit of time finding references and papers to cite in my methodology. We need to complete this chapter this week.

Programming without Computers book - maps to Isabel's examples on knitting.

A second call with Marian for about an hour at 13:15 ish. A working session on my methodology material. We pivoted the groupings of analysis methods. 
Marian also managed to track down one of the hard to locate papers for me, thank you.

\dotfill

For calls with Marian and Arosha today \nth{9} Nov 2021
I've incorporated the meeting notes here too. The call with Marian was for about 30 mins and with Arosha closer to an hour. We'll have another in-person meeting on Friday afternoon.

- Help me un-entangle me as the subject and me as the researcher. Purpose to further the research vs to further the product improvement. Sometimes the work was indistinguishable. Add a reflection on situated research in the discussion - cost-benefit trade-off in the two roles. Alignment of purposes enabled both the practical and the research helped both. Discuss in the viva. I abstracted from the immediate hands-on work to consider the research aspects. 
- Beacon-finding and sense-making appear to be endemic, how can I describe them within clean boundaries? Beacon finding in artefacts and in the analytics tools. Also a continuous practice, not just a starting point. Heavily informed by the feedback mechanisms. 
--- supplementary beacon finding in development artefacts. 
- Local App Experiments, for example, contribute to sense-making and sense-building (producing/building/making stuff to feed the senses) - how can I usefully illustrate these many-to-many relationships in the methods in ways that'll help my readers? systematic evaluation of the behaviour of the analytics. 

sense-making deals with ephemera in mobile analytics 

Characteristics of beacons - write up and explain succinctly. Provide a concrete example of where this beacon arises/arose. distinguish between beacons useful for research vs useful for development in practice. These are all things developers could do and might help them improve their practices and to improve the product. I used them in ways that also contributed to the research.  
Actions: 
\begin{itemize}
    \item For me to draw the equivalent of the new figure (4.3 currently) with a research overlay. I'll start by drawing the research version separately. 
    \item Continue revisiting the table 4.1. Consider adding commentary to sense-making and evaluation through action research sections.
    \item Write up notes for figure 4.2 and the additional figure with the research perspective. 
    \item Consider whether to add a `no more issues' exit to figure 4.3 - one case study (Moonpig) might have actually achieved this, whereas the rest of the case studies leave many issues unprocessed.
    \item Fill in gaps based on the rest of the notes here.
\end{itemize}

\dotfill 

Marian \nth{10} Nov 2021. 11:33 ish, about 50 mins.

Grounded Theory Method from the thesis, (and from Wikipedia). c.f. Thematic Analysis.

Replace 1a, 2b, etc, with the I's and U's.

The case studies chapter to prepare the readers:
Structure and mapping for the case studies in order to group the findings 
data, my role in the case and the case's role in the research, where it produces something on which I rely in the perspectives and findings. Some perspectives can be grouped if that helps tell the story.

\dotfill

Marian and Arosha, in person, \nth{12} Nov 2021, 13:20 ish for about 50 mins.

We mainly discussed the contents of the methodology chapter.

\dotfill

Isabel, \nth{15} Nov 2021, 09:35 ish, for about 40 mins. 

We compared our progress on our respective research. We've scheduled a working session on Thursday afternoon for 2 hours to walk \& work through the methodology chapter. Hopefully this will also be useful for Isabel.

\dotfill

Marian, \nth{15} Nov 2021, 17:05 ish, 30 - 40 mins. Mainly on the methodology.

\dotfill

Marian, \nth{15} Nov 2021, 11:30 30 mins, ditto. 

\clearpage
