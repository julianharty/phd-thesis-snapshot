\section{Actions}
\label{meta-chapter-actions}
This is my global combination of stuff I know I want to do and next steps.
\begin{itemize}
    \item Add all the planned chapters with at least 1 - 2 pages of content to cover the purpose of the chapter. Approx 8 paragraphs according to Arosha would be good.
    \item Incorporate raw material from my papers to date. For instance, figures, revised tables, key findings, \emph{etc.}
    \item Work on the Discussion and also the Introduction.
    \item Devise my test plan for the test apps (zipternet, android-crash-dummy, etc.). \emph{In progress 24-May-2020, see} \url{https://joedocs.com/julianharty/evaluating-gpc-reports} \emph{for my research on evaluating Google Play Console reports.}
    \item Work on my Related Works chapter.
    \item (Done) Fix as many warnings as practical. \emph{24-May-2020 0 warnings apart from the word count (see comment in latex in this file} \texttt{actions.tex}. 
    \item Clean up and make my references coherent and complete.
    \item \akb{style point - the voice of the text jumps around from talking about 'the research' to 'my research', addressing the reader directly to the third person, etc.  Need to stick to a consistent style.}
    \begin{itemize}
        \item Done: Abstract, Background Chapter(s)
        \item Next: Introduction, Research Questions.
    \end{itemize}
    \item Review all the text that appears in the ToC LoF, LoT, etc. to make the document structure clear and professional. See~\url{https://tex.stackexchange.com/questions/296759/footnote-in-caption} for an example of creating short labels for captions.
    \item Revisit my rejected papers (and the accepted ones) for snippets and evidence worth incorporating in this thesis. For example, 
    \begin{itemize}
        \item \href{https://www.overleaf.com/4666349717jtgjwrkchfkh}{Testing Veracity and Fidelity of Software Analytics}
    \end{itemize}
    
\end{itemize}

% COULD_DO To fix the word count errors, try debugging along the lines of:  https://tex.stackexchange.com/questions/286019/how-can-i-convince-texcount-that-my-use-of-newcolumntype-is-perfectly-valid-syn 

\section{Using MoSCoW for TODOs}
I discovered that TODO's in the material are not sufficiently precise to enable me to work out the differences between what I \textbf{M}ust do, what I \textbf{S}hould do, what I \textbf{C}ould do, and what I \textbf{W}on't do as part of this thesis. The various items are identified in Table \ref{tab:moscow_for_todos}:

\begin{table}[htbp!]
    \centering
    \begin{tabular}{l|l}
       Identifier         &Action by end of thesis \\
       \hline
       \texttt{MUST-DO}   &Must Do  \\
       \texttt{SHOULD-DO} &Should Do \\
       \texttt{COULD-DO}  &Could Do \\
       \texttt{WONT-DO}   &Won't do as part of this thesis, possibly will do subsequently. \\
         & 
    \end{tabular}
    \caption{MoSCoW Identifiers during my writing}
    \label{tab:moscow_for_todos}
\end{table}

\subsection{Miscellaneous COULD-DO's}
Here are various general tasks that I could do that might improve the thesis, others are embedded in the latex source as this is a recent addition.
\begin{enumerate}
    \item Somewhat late to the party, I've finally discovered latex foundations that seem clear to me \url{https://artofproblemsolving.com/wiki/index.php/LaTeX:Layout}
    
    \item Make figures live so people can click them in the PDF and be taken to primary text that references them (this doesn't always hold if there's several references to the same figure. See \url{https://tex.stackexchange.com/questions/54927/how-to-insert-an-image-that-also-acts-as-a-link} and \url{https://tex.stackexchange.com/questions/84921/href-not-working-with-image-but-ok-with-text} if I run into issues.
    
    \item Size the images more consistently, currently I use a variety of heuristics to size and locate them on the page. That's not elegant and wouldn't cope well if the page margins and.or sizes change.
    
    \item There are an insane number of examples of beautiful latex, many with sources, in \href{https://tex.stackexchange.com/questions/1319/showcase-of-beautiful-typography-done-in-tex-friends}{Showcase of beautiful typography done in TeX \& friends} for inspiration. One example is a very attractive thesis where the author has provided a README and the sources he used for his thesis \url{https://www.levbishop.org/thesis/source/}. The nomenclature and bibliography are both far more attractive than mine currently. 
    
    \item Implement backrefs so that a) the thesis compiles without errors, b) the backrefs in the bibliography are well-phrased (rather than just having the page numbers). As ever, the memoir document class seems to complicate matters! Currently the code is commented out in \texttt{latex-configuration.tex} Refs:
    \begin{itemize}
        \item \url{https://tex.stackexchange.com/questions/410270/memoir-citation-in-margin} Interesting and somewhat beyond what I currently want to do - it uses sidenotes for citations.
        \item \url{https://tex.stackexchange.com/questions/98528/further-customize-color-of-hyperref-links} using a basic document class.
        \item \url{https://tex.stackexchange.com/questions/15971/bibliography-with-page-numbers} sums up several other examples online for 
        \item \url{https://tex.stackexchange.com/questions/522491/using-tufte-book-document-class-with-bibliography-backref}
        \item backref-memoir.tex works in a skeletal memoir example \href{https://www.overleaf.com/project/612fb5e6f6cc44c10b56afa1}{My list of examples overleaf project}
        \item \url{https://latex.org/forum/viewtopic.php?t=4791} if I get stuck, the answer isn't very clear or helpful though...
    \end{itemize}
    
    \item Create a proper glossary, e.g. see \url{https://husseinbakri.org/essential-latex-bibtex-and-biblatex-tips-for-research-students/} which has lots of sensible general info on using latex etc. including the glossary package.
    
    \item Consider whether to use alphas for subsubsections as per the fine thesis on pair programming. How to do so is simple - see the latex for this item as the code is commented out % \renewcommand{\thesubsubsection}{\thesubsection.\alph{subsubsection}} % See https://latex.org/forum/viewtopic.php?t=5349
    
    \item Look at improving the formatting of epigraphs, particularly for the discussion chapter. It also has a footnotemark that doesn't appear as part of the epigraph.
    
    \item TBD :)
    
\end{enumerate}

\clearpage
\section{Call and Meeting Notes - not part of the actual thesis.}
%%%%%%%%%%%%%%%%%%
% This is a useful place to keep various notes from calls and meetings as I'm writing up. It's been started late in the game so many notes predate and aren't in this file.
%%%%%%%%%%%%%%%%%%

\hrulefill
\nth{7} Sept 2021
Discussion online with Jakob D. from Moonpig on his transition to MyPulse starting 13 Sept 2021. 
Call 7 Sep 2021 last 15 - 20 mins re my research… Notes are offline as they're incomplete and need work.
\hrulefill
In person working session with Joe Reeve 10 Sept 2021

We ran through the two case studies he was involved with (Kiwix and Catrobat). He emailed me our joint notes.

\hrulefill

Notes from call with Marian Petre 10 Sept 2021
40 mins 13:35 - 14:15 ish.
 
heartbeat of android vitals

What's clear is my case studies are deep and immersive rather than case studies that ask developers questions in a sruvey. Also the very nature of what I'm working with means the case studies are opportunistic. I had this much access and no more... This is design I used, because…

It's vital to be very clear about what I've done AND DIDN’T DO, the data I gathered and how, the methods , I NEED TO BE CLEAR ABOUT WHAT I DID, WHEN, AND WHY.

For immersive case studies, there’s a risk of bias - explain how I worked to reduce the risk and/or effects of bias e.g. by interviewing developers of other apps.

By documenting the case studies, their methods and data, it's easier to show how the case studies compare and to be able to justify why I can compare them.

Needs explicit audit trail. What was missing is the coherence across the case studies. Help Marian to know where to look for the methodology.

Same order of sections, with the same name, per case study wherever this applies, however don't use a rigid structure blindly.

Given the dimensions and characteristics of the case studies we're also able also to comment on the aggregate.


Accumulate lessons per case study.
Independent data that leads to the same conclusion
Collegation and/or Triangulation.

Being aware of the potential for bias.
Look for opportunities to discover contradictions as a corroboration. 
Some contradictions lead to insights.

\hrulefill

