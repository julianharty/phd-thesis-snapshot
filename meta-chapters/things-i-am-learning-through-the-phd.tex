\chapter{Things I'm learning through the PhD}
Writing the thesis and papers are essential measures of my journey and ultimate completion of a PhD; however they're not likely to be the only, or necessarily most important, measure of my PhD. I've been learning a mix of tools, techniques, and practices.

\begin{itemize}
    \item Latex
    \item R
    \item Kotlin
    \item Research practices, tools, and techniques
    \item Research writing and reading techniques
    \item Critical thinking
    \item Academic collaboration, alignment, cooperation
\end{itemize}{}

\section{latex}
\url{https://stackoverflow.com/questions/40329736/latex-document-word-statistics}

\section{Advice from reviewers}
\marian{For the mini-viva: 
Create a clear problem statement to frame it, lots of signposting, and a strong table of contents for the structure. Aim for 20 pages. Show critical depth in the literature review. Use rigour in the methodology, and show the path to the conclusions.}


Articulate RQ's in the introduction to set the frame for the rest of the work.

What I want to know, what I need to do to discover it (approach), what I did.

Big questions and focal q's for what I need to know. Fine to refine the questions as I write.

\clearpage

Note on 12-Jun-2022: I used miro's fishbone design to create the mappings of topics and themes for the three Findings chapters. Doing so has been helpful both in terms of me realising that I'd like a better way to illustrate the material and also to help me decide what material belongs in which chapter. Several items will be relocated from this chapter to other chapters, these are shown in the figure as free standing grey post-its. There are two links between topics, these are shown with directional arrows.