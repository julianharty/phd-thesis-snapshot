\section{Inventories}
\label{section_inventories}
There are two inventories:
\begin{enumerate}
    \item 1st inventory: chapters, what is their role, and how finished are they?
    \item 2nd inventory: what do I have, what's the purpose of it, and how ready is each chunk?
\end{enumerate}

Inventory 1: what's included the current thesis:
\begin{itemize}
    \item Front Matter: Necessary content for the thesis.
    \begin{itemize}
        \item \href{the-abstract}{Abstract}: Succinct summary to introduce the thesis. Beta.
        \item \href{section-publications}{Publications, etc}: Demonstrates aspects of this work is considered novel and acceptable by the research community. Beta, needs updating.
        \item \href{the-dedication}{Dedication, etc}: Standard practice. I'll write this properly shortly before submission. Alpha.
        \item \href{the-declaration}{Author's declaration}: Boilerplate text. Done AFAIK.
        \item \href{the-glossary}{Glossary}: Helps define my terms so the readers have a consistent reference. Ongoing work, recently improved, Alpha.
        \item Table of contents, etc: Essential and expected. Ongoing, automatically generated. Beta, probably could do with improving the formatting and colour choices. 
    \end{itemize}

    \item Meta-materials: to help get the thesis written. Any relevant content will be subsumed within the thesis, the rest excluded.
    \begin{itemize}
        \item \href{section_read_thread}{The Red Thread}: To help me clarify my story, decide what of the rest of my material is sufficiently relevant to include in the thesis. W-i-P.
        \item \href{section_inventories}{Inventories}: Marian encouraged me to write these so we can work out what belongs where and project manage the thesis to successful completion.
    \end{itemize}
    
    \item \href{chapter-introduction}{Introduction}: Sets the context for the reader for the rest of the thesis. This needs revamping, restructuring, and content aligning across the entire chapter. W-i-P. Includes the~\href{section-research-questions}{Research Questions}.
    \item \href{chapter-related-work}{Related Work}: Essential to situate my research in the context of the previous work published by others. a review of the relevant literature as perceived by me. Pre-alpha, to be revised in tandem with defining the contents of the rest of the thesis, particularly my research questions.
    \item \href{chapter-preparing-the-ground}{Preparing the ground}: This is a mix of stuff I already know that I doubt either my examiners or other readers will know much about. Some of it is from my brain either as part of, or distinct from my PhD. It's unlikely to fit the typical question-evidence-method expected for a thesis (or research in general). So perhaps it's the non-academic equivalent of the related works chapter? Some of it is supported by academic research. Beta.
    \item \href{chapter-applying-analytics-to-development-practices}{Applying analytics to development practices}: I realise the contents and the title don't align. I'll move some of the contents to another chapter e.g. on motivation and on the cross cutting concerns and add more appropriate content. Alpha as it needs significant revision, else beta.
    \item \href{chapter-code-needed}{Key software developed for this research}: indicates necessary and useful work/contributions outside the thesis and intended to ground that work as part of my research.
    \item \href{chapter-case-studies}{Case Studies}: The core of my empirical work is described in this chapter and some of the specific results e.g. improvements are illustrated here. Incomplete, lots of material to add from the recent large-scale industrial experience. Also the chapter needs restructuring and some of the smaller, older case studies can be compressed and distilled to keep the overall size of the thesis digestible for the readers.
    \begin{itemize}
        \item \href{section-overview-of-case-studies}{Overview of case studies}: Sets the context for this core chapter. Alpha, subject to revision in parallel with the work on the rest of this chapter.
        \item \href{section-research-ethics-for-the-case-studies}{Research ethics for the case studies}: To demonstrate I actively considered and addressed the research ethics. Beta.
        \item \href{section-kiwix-case-study}{Kiwix Android apps}: The first of the case studies (and currently first in this chapter) that illustrates how a team can use only the default platform analytics and improve the reliability of their apps many-fold. Beta, needs recent updates adding to complete the story.
        \item \href{section-catrobat-case-study}{Catrobat Android apps}: The second case study, where my role is as a coach for a mature project and team. We were able to make incisive improvements quickly and effectively. There's lots of raw content that needs distilling. Pre-alpha.
        \item \href{study-greentech-apps}{Greentech apps}: a minor case study of popular closed-source not-for-profit apps. Helps illuminate the priorities of the development team and where dealing with crashes fits. Beta.
        \item Field reports from commercial app developers: These people actually need to satisfy manifold often contradictory demands, three vignettes of how they use mobile analytics help to show that the approach is broadly applicable and relevant. Provides some background colour and depth to the thesis. Beta.
        \item \href{section-corporate-engineering-case-studies}{Corporate engineering case studies}: One hands-on for a complex app with 1M+ users, the other a broader piece of secondary and tertiary research to provide insights into the practices of ultra-large mobile app development teams. Incomplete, particularly for the hands-on case study. 
        \item \href{section-research-in-logging-practices}{Research in logging practices}: Supports research where developers proactively add logging to their code using mobile analytics libraries. The output of the logging augments the default results they receive from various sources including the platform level analytics. Pre-alpha. 
        \item \href{section-tool-providers-perspective}{A tool provider's perspective}: Again, adding some additional depth and breadth of the challenges as perceived by an innovative tool vendor and people they've interviewed. NB: I realise that the feedback and requests from the Google engineering team could also fit here.
        \item Some additional examples:
    \end{itemize}
    \item \href{chapter-evaluation}{Evaluation}: to add an overall perspective of the case studies and tie them together. Beta.
    \item \href{chapter-discussion}{Discussion}: build on the work so far in the thesis. Open up several topics for consideration. Beta.
    \item \href{chapter-conclusions}{Conclusions}: wrap up the main thesis. Beta.
    \item \href{chapter-future-work}{Future work}: introduces areas of research that could build on the research I've covered here. Beta.
    \item \href{the-bibliography-follows}{Bibliography}: Essential. I'm not very happy with the format or how they appear in citations. Also there are lots of currently unused entries. Alpha.
    \item Appendices: Additional, non-core material. Beta
    \begin{itemize}
        \item \href{app:thesis-countdown}{Thesis Countdown}: To keep me focused and motivated. To be removed pre-publication.
        \item \href{app:miscellaneous-topics}{Miscellaneous Topics}: currently 3 minor topics to help the reader. I'll probably move one and remove another of these topics. Alpha.
        \item \href{app:practical-aspects}{Practical aspects for the design and incorporation of mobile analytics}: some incomplete ideas, undetermined where they'd fit best in the thesis or whether to remove them; I think the topics are important, at least for some. Incomplete.
        \item \href{app:software-contributions}{Software Contributions}: pre-dates the chapter on key software and a superset; intended to illustrate the many and various related software contributions made during this thesis. Alpha.
        \item \href{section-case-studies-data-sources}{Data sources}: an early chapter, may be of relevance for future researchers on the practical aspects e.g. for reproducibility. Incomplete, mainly as I've my recent work includes several strong examples.
        \item \href{appendix-analytics-tools}{Analytics tools used in this research}: A sibling to the appendix on data sources. Helps summarise my use of analytics tools and services. Alpha.
        \item \href{appendix-on-mobile-analytics}{On mobile analytics}: A miscellany of thoughts and materials intended to help the reader. There's some duplication with other content, so needs to be repurposed.
        \item Actions: A temporary appendix to help reviewers interpret my various internal notes in the thesis. Will be removed pre-publication.
    \end{itemize}
    
\end{itemize}

Inventory 2: what (else) have I created that we could possibly use/adapt/include?
\begin{itemize}
    \item Appendices not in the main thesis: Odds and sods. Some are meta-discussions to help remind me of my progress and what I've learned as a necessary side-effect of the thesis.
    \item Case Studies not in the main thesis: Appachhi collaboration which fizzled out yet was germane at the time. I also have a bibliography for the Catrobat case study here. \href{section-some-examples}{Some examples} is an incomplete worked example of work we did to a) address the cause of a major crash that affected around 10\% of the users b) to improve the confidence that we had identified the fault and fixed it completely. 
    \item Draft Materials: These are all materials I wrote as part of the initial version of this thesis. With the exception of the fieldstones material, I've made few updates to these materials. There may yet be useful stuff worth extracting and incorporating into the actual thesis.
    \item Examples: these where inherited with the latex I used as the template for this thesis. I kept them for reference in terms of the way they used latex.
    \item Images: many of these have been incorporated into the thesis, not all. There will probably be some worth considering for inclusion. I've structured many but not all of them, again something to consider tidying up...
    \item Meta chapters: includes this content! Some may need migrating. Others e.g. the possible architectures contents were used to help me structure this thesis.
    \item References: includes 2 bibliography files, one for the core thesis, the other for an unused appendix on my PhD experience.
    \item utilities: Currently a shell script that helps me quantitatively measure the contents of the thesis.
    \item The root folder: has the latex configuration, the project README, and two files, one to construct the thesis the other intended to print a subset I'm actively focusing on.
\end{itemize}
