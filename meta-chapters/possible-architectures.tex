\chapter{Possible Architectures for my Thesis}
This chapter contains various ideas and recommendations on how to design a PhD thesis. 

\section{Stepping Stones to achieving your Doctorate}
\emph{This work is based on chapter 4 "Architecture of the Doctoral Thesis"}~\cite{trafford2008stepping}. They propose writing several flavours of possible layouts and structures of the thesis can help the thesis author (me) to hone their thinking and also improve the thesis they write. I'll give it a go... :)

Key concepts:
\begin{itemize}
    \item Use a visual strategy to assist the research.
    \item Consider several structures, and compare their advantages and disadvantages.
    \item \emph{c.f.} conceptual map
\end{itemize}

BTW: I found their chapter was weak on examples, I don't actually understand what they propose for a part/chapter approach despite reading the chapter several times.

\subsection{Answering the 10 questions}
These are in Example 4.1 in the book~\cite{trafford2008stepping} and reproduced below in Table ~\ref{tab:ten_questions_on_your_thesis}.

\begin{table}[ht]
    \centering
    \footnotesize
    \begin{tabular}{r|p{12cm}}
        \# &Question  \\
         \hline
    1 &What will be the continuous thread/theme/issue that runs through
your thesis? \\
    2 &Where does that thread/theme/issue originate? \\
    3 &What will your readers’ valued memories be after reading your
thesis? \\
    4 &What structure will your thesis have? \\
    5 &Will it have parts/chapters or just chapters and what sort of sections
will be used? \\
    6 &What are the provisional titles of its parts/chapters/sections? \\
    7 &What are the expected contributions that each part/chapter will
make to your thesis? \\
    8 &Can you express those contributions briefly in just a few lines? \\
    9 &What might be the relative word count of the parts/chapters or
sections? \\
    10 &How many appendices might be included and in which chapter
might each one appear for the first time in the thesis? \\
    \end{tabular}
    \caption{Ten questions, reproduced from exercise 4.1}
    \label{tab:ten_questions_on_your_thesis}
\end{table}

\emph{1). What will be the continuous thread/theme/issue that runs through
your thesis?} The continuous theme is that analytics can help us improve our products and potentially our processes (beyond simply looking at analytics reports).

\emph{2). Where does that thread/theme/issue originate?} The theme originates from my work in industry helping test and improve the quality of various extremely popular apps at Google, eBay, and other companies. I saw the potential of applying analytics to improve the understanding of how mobile apps were being used, in identifying various flaws and problems. I also saw that only a few of the teams that used analytics did do so wholeheartedly. While working with various teams across the industry we discovered flaws in the analytics tools, their reports, and the results of applying information from these reports.

\emph{3). What will your readers’ valued memories be after reading your
thesis?} I hope readers will enjoy reading my thesis, i.e. that it will be clear and easy to read and comprehend. If they're also encouraged to participate in co-developing and applying the concepts I'm proposing that would be even better. I believe there's significant potential in applying analytics to help us improve our craft and products. I'm unlikely to do so alone.

\emph{4). What structure will your thesis have?} I'm undecided, writing this is part of helping me work out the best structure. It'll probably have several case studies, of Kiwix and Catrobat, together with interviews of various developers. I'll also include chapters on developing tools, and the local testing using Joe's apps. Perhaps also a chapter on the effects of Google's policies on self-policing.

\emph{5). Will it have parts/chapters or just chapters and what sort of sections
will be used?} Again, I don't yet know - partly as I've not seen examples of both.

\emph{6). What are the provisional titles of its parts/chapters/sections?} Hmmm, certainly an introduction, a chapters that sets my research into context of other research (provisionally titled "related work"), some sort of discussion chapter, and a conclusion. I'd like to include a chapter that encourages readers to get involved. Perhaps it'd be worth providing a chapter that explains the analytics tools we're using, what and how they measure, known limitations, etc.

\emph{7). What are the expected contributions that each part/chapter will
make to your thesis?} For the case studies they'll provide evidence of the value of paying attention to analytics data and using it actively to identify issues and address them. The discussion will cover ways we can make progress even though there are flaws and gaps in the analytics and our application of it. In the discussion I can also critique flaws in my research practices e.g. on the sporadic nature of gathering the data. I expect this question is worth revisiting and iterating on as I make more progress designing my thesis.

\emph{8). Can you express those contributions briefly in just a few lines?} My research has established that development teams can use analytics data to effect improvements in the reliability of their apps. There is weak evidence (nonetheless positive in nature) related to improving testing practices. And I'm finally in the very early stages of providing evidence for the work I'd first proposed when I started my PhD i.e. that in-app analytics shows potential to help further.

\emph{9). What might be the relative word count of the parts/chapters or
sections?}
The discussion chapter seems like one that could expand massively. There is much to discuss, e.g. on the SWOT of using analytics, on power relationships and their effects on individuals and populations. 

\emph{10). How many appendices might be included and in which chapter
might each one appear for the first time in the thesis?}
Again this depends, if I place case-studies in the appendices then there'll be:
\begin{enumerate}
    \item Kiwix apps
    \item Catrobat apps
    \item Developer conversations and evidence
    \item Platform analytics (exemplified by Google Play Console \& Android Vitals)
    \item In-app analytics (particularly crash-recording analytics as that's the one I've most experience in)
\end{enumerate}
The case studies would be first mentioned in the introduction and then the method, etc. I expect the types of analytics would also be mentioned in the introduction, etc.

\section{Thesis maps}
The work of \emph{The thesis whisperer} (Inger Mewburn) recommends creating a \emph{thesis map} to help streamline the actual writing. She provides an example of her Thesis Map~\cite{thesis_map_example_inger_mewburn}. In a blog post in 2015 she claims students can write 10,000 words a day~\cite{how_to_write_10k_words_day}.

In her thesis map she answers 4 questions, I'll also answer here.

\vspace{3mm}
\emph{1. Thesis Statement: } applying mobile analytics helps improve the quality of the apps. 

In other words: usage data provides real-world feedback on the use and performance of mobile apps. Tools and services are available to automatically collect and analyse the usage data and provide reports on various aspects of quality. When a development team uses the information provided in these reports to identify and fix issues in their app and releases a newer version of the app to end users then they can improve the measured quality of the app. The feedback is available within hours to days depending on the underlying mechanisms used to collect and report on the data being collected and analysed.

\vspace{3mm}
\emph{2. This thesis contributes to knowledge by: }
analysing results for two mature organisations who develop various mobile apps. The research focused on the Android codebases for both of these projects. 

\begin{itemize}
    \item Describing how the key Google Play Console behaves and how the results can be used to understand various flaws and issues that lead to poor quality as measured in the field.

    \item Providing examples of software tools and utilities to help collect, preserve, and analyse reports within software used by potentially a million developers to manage several million Android apps for a user-population of several billion.

    \item Helping determine the necessary fidelity of analytics tools for several types of quality flaw. 

    \item Identifying and reporting issues in globally used analytics tools provided by one of the global leaders in software development.  
\end{itemize}
\vspace{3mm}
\emph{3. This research is important because: }
End users may be upset and adversely affected by poor quality software and may choose to ignore or reject that software. Developers generally wish to provide software that performs well where the quality is adequate, they do not want to be associated with poor quality software. My research has evaluated the application of a key freely available and free-of-charge platform-wide analytics offering provided by Google for their Android platform. It demonstrates developers can address reported quality issues \emph{despite numerous flaws in the analytics being provided by Google}. The quality of the revised apps improved, and the improvements are large, ranging from 3x to over 10x the measured reliability for crashes.

The data being collected for crashes and ANRs helps with bug identification, scoping and potentially bug reproduction. Developers can use the analytics reports to validate the effectiveness of the fixes they applied.

The relevant Google engineering teams confirmed various flaws I found and reported and requested a report of the various issues I discover. They confirm they wish to address the flaws.  

\vspace{3mm}
\emph{4. Key research question: }
TBD: in terms of the 'products', most of my research and results are in reducing the measured crash-rate of several Android apps. 

In terms of reports, most of the research and findings are for Google Play Console, including Android Vitals, with some additional analysis of Fabric and Firebase Crashlytics.

\vspace{3mm}
She then provides a chapter-by-chapter synopsis. I aim to do this shortly...

\section{Conceptual Frameworks}
Apparently a conceptual framework is an essential attribute of a successful thesis, and needs to be recognised and accepted as valid by the examiners. Hence, working out what a conceptual framework is, and which one I'm using, and how I'm using it are all vital.

Quotes:
\begin{itemize}
    \item \emph{"... most authors use the term [conceptual framework] to describe a specific function and set of relationships within the research process."}~\cite{leshem2007overlooking}.
    \item "a scaffold..."~\cite{leshem2007overlooking}.
    \item "...providing traceable connections between theoretical perspectives, research strategy and design, fieldwork and the conceptual significance of the evidence. Thus, the conceptual framework is a bridge between \emph{paradigms which explain the research issue} and \emph{the practice of investigating that issue.}"~\cite{leshem2007overlooking}.
    \item "... a device that makes sense of their data."~\cite{leshem2007overlooking}.
\end{itemize}

\emph{Editorial note: Further reading: }~\url{https://www.scribbr.com/dissertation/conceptual-framework/}

\subsection{Some of my concepts in my research}
Here are some key concepts and how they connect with other concepts.

\begin{itemize}
    \item Quality-in-Use matters to end users and may affect their "user experience", their perceptions of the software and provider of that software, and may also affect how much they use the software. Quality-of-Experience, which is used in several technology domains, might help us to focus on the relevant areas of how our software behaves and performs.
    \item Measurement needs source data, the data by default has a finite life, which may vary from almost instantaneous to weeks or potentially even longer indefinite periods if the data is recorded and preserved. Therefore whatever we measure has at least one data source. By learning about how the data is generated and made available to the measuring system we can understand what can and cannot be measured using a particular combination of source and measurement. 
\end{itemize}