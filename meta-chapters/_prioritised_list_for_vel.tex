\begin{enumerate}
    \item Improve the positioning and legibility of the side-by-side figures in chapter 8 (8.2, 8.3, 8.4, 8.6). I think I'd like them to take the full page's text width as each pair of figures and for whichever side to be aligned with the respective body text. 
    \item Improve the layout of table 7.3 \ref{tab:kiwix-evaluation-reductions-in-crash-rates}, which seems unnecessarily narrow currently.
    \item Table 7.2 isn't aligned with either side of the rest of the chapter's text; please could you align it with either the start or end of the text width of the chapter's text to suit whether it's on an even or odd page.
    \item improve the presentation of \Command{newthought} - see latex comments below for sources of the command I used.
    \item In chapter 5 many of the sections have two tables. Latex is currently putting both these on a separate page at the end of the section, I'd prefer to have a table + text on the first page, and a table + text on the second page of each section where this occurs, please. Also I found there was a tendency for one of the tables to slip onto a third page if either of the tables has lots of content. I'd prefer each section to be limited to two pages unless that's impractical because of the volume of text in that section e.g. For section 5.7 for Catrobat \ref{case-study-overview-catrobat}, there's too much content for it to fit on 2 pages AFAIK. 
    \item In Table \ref  the sideways text `Interventions' is too low and needs centering (similarly the earlier sideways text `Interviews' is also not centered but less obviously messy). Please can you improve the layout (and perhaps also stop the thin blue lines from intersecting with the word Interventions by starting the blue lines on the left margin of the Case Study column)?
    \item Figure 7.2 \ref{fig:55-crashes-WebViewFactory-MissingWebViewPackageException} might be too big for the page, if so please trim its footprint so it doesn't expand beyond the margins please.
    \item Longer URLs (actually \Command{href}'s) in the side margin sometimes run into the body text. A good example is in chapter 7 for the URL \url{https://github.com/Catrobat/Catroid/pull/2419/files} and earlier in the same chapter for URL \url{https://docs.sentry.io/platforms/android/configuration/integrations/okhttp/} - please could you tweak the necessary so they're unlikely to clash with the body text, yet remain readable. I'm guessing either adding more break characters, or slightly shrinking the presentation of the visible section of the \Command{href} could work. BTW If that's not easy to address I can hack the layout by adding a space in the visible text or convert the sidenote to a footnote.
    \item Sometimes the sidenotes overwrite each other partially, e.g. for \Command{begin{kaobox}[frametitle=Automated tests for logging]} in Chapter 7 (currently line 132 in the latex source) I have hand hacked the layout in the box to reduce the overlap. If you can easily and reliably reduce the ability of the overwritten text that'd be great.
    \item In Chapter 6 (in line 344 currently), the following \Command{texttt} content does not wrap on a period character. I can hand hack it, however if there's a way to cleanly enable line-breaks that'd be good: `\texttt{java.lang.IllegalArgumentException Parameter specified as non-null is null: method org.kiwix.kiwixmobile.core.utils.LanguageUtils.getCurrentLocale, parameter context}'
    \item In Chapter 6, figure \ref{fig:gp-review-denise-lavington-always-crashes} is currently on it's own page rather than being on the same page as the body text. Please can you find a way to stop it appearing on its own page?  
    \item Chapter 7 has a sidenote with the url \url{https://jira.catrob.at/browse/CATROID-379} (currently line 219 in the latex source of the chapter) where the text is widely spaced, perhaps raggedright would make the sidenote more attractive? Or maybe I should create a proper reference for the issue? Anyway, a way to improve the layout would be good.
    \item If practical in the limited time we have, find a way to have side notes with the equivalent citation appearance to how \Command{sidecite} looks.
    \item in Listing 2 \ref{listing:android_manifest_xml_for_sentry} in Chapter 7, there are two backslash characters with a red box around them. Ideally I'd like both the backslashes and the box to disappear and the rest of the formatting to be more or less as-is, where android:value aligns directly below the <meta-data... 
    \item Various \Command{table*} in Chapter 4 are centered on the page (probably by design), if they could be repositioned or resized so they align one vertical edge with the body text (or conversely if they use the space they have so that one vertical edge naturally aligns) please do so. 
\end{enumerate}

Separately, *** This is possibly the biggest ask, so at the bottom of the list as I don't want to lose sight of the other hopefully easier changes by having this as the blocking entry) Explore ways to auto-generate the content for the Links to findings. I'm trying creating custom indexes as per \url{https://en.wikibooks.org/wiki/LaTeX/Indexing}. \url{https://tex.stackexchange.com/a/124223/88466} looks close to what I'd like, albeit the mentions would be in the side-margin. \url{https://tex.stackexchange.com/a/78099/88466} also looks interesting. \url{https://tex.stackexchange.com/questions/393821/how-to-refer-to-indexed-words} \url{https://tex.stackexchange.com/questions/410362/distinguish-forward-cross-references-from-backward-cross-references}

%%%%%%%%%%%%%%% An attempt to reproduce the \newthought{} command from the tufte-book document.
%%%% Sources:
%%%%   https://github.com/Tufte-LaTeX/tufte-latex/blob/efb8c8e836890bad71fb5834acdd316ebde6db12/tufte-common.def#L779
%%%%   https://github.com/Tufte-LaTeX/tufte-latex/blob/efb8c8e836890bad71fb5834acdd316ebde6db12/tufte-common.def#L1527-L1535
%%%%   Improvement to vertical formatting https://tex.stackexchange.com/a/291748/88466 (via https://tex.stackexchange.com/questions/291746/tufte-latex-newthought-after-section)
%%%%   https://tex.stackexchange.com/questions/288472/bold-first-letter-combined-with-smallcaps 